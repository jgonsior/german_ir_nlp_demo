\documentclass[a4paper, 10pt]{article}
\usepackage[T1]{fontenc}
\usepackage[sfdefault]{AlegreyaSans} %% Option 'black' gives heavier bold face
\usepackage[a4paper, left=2cm, right=2cm, bottom=2cm, top=2cm]{geometry} % Adjust left and right margins
\usepackage{amsmath} % for \boxed

% Set the width of the lines around the box
\setlength{\fboxrule}{2pt}

\begin{document}

\begin{minipage}[t]{\textwidth}
    \vspace*{-1.5cm} % Move the content up by 0.5cm
    \begin{flushright}
        \hspace*{\fill} % Move the content to the right edge
        $\boxed{\textbf{\Huge\phantom{00}3\phantom{00}}}$ % Your content here with increased padding
    \end{flushright}
\end{minipage}

%SEITE0
\section*{\huge Schloss Hogwarts}
%ABSATZ
Schloss Hogwarts ist ein großes, siebenstöckige hohes Gebäude in Schottland, durch Magie unterstützt, mit vielen Türmen und Türmchen und sehr tiefen Kerken. Es wurde im späten Frühmittelalter (ca. 993) errichtet von einem magischen Architekten und den vier berühmtesten Zauberern dieses Zeitalters, den Hogwartsgründern: Godric Gryffindor, Helga Hufflepuff, Rowena Ravenclaw und Salazar Slytherin. Das Schloss beherbergt die Hogwartsschule für Hexerei und Zauberei, die als die beste Zauberschule in der Welt betrachtet wird.
\vspace{10pt}
\newline
Im Gebäude, in den Türmen, Türmchen und Mauertürmchen sowie den Kellergewölben befinden sich 142 Treppen. Im gesamten Schloss und seiner Umgebung herrscht Magie, sodass das Gebäude aufgrund seines Alters eine gewisse Empfindungsfähigkeit hat. So wurde das Schulleiterbüro nach der längeren Abwesenheit des Schulleiters für die unerwünschte Nachfolgerin im Amt 1996 versiegelt und nicht zugänglich, Treppen können ihre Richtung ändern, es bestehen Trickstufen, falsche Türen und Stockwerk übergreifende Abkürzungen.
\vspace{10pt}
\newline
Schüler, Lehrer, Angestellte und Hauselfen wohnen außer in den Ferien internatsmäßig im Schloss, das Schloss ist eines der Gebäude in Großbritannien mit den meisten Geistern.
\vspace{10pt}
\newline
Es wurde für Muggel unsichtbar gemacht. Muggel sehen an dessen Stelle nur eine Burgruine mit einem Schild, auf dem "Betreten auf eigene Gefahr" steht.
\vspace{10pt}
\newline
Im Zuge des zweiten Zaubererkrieges 1995-1998 wurde das Gebäude in der Schlacht von Hogwarts erheblich beschädigt. Es gab viele Todesopfer auf Seiten der Hogwarts-Verteidiger, doch schließlich siegten die Mitglieder vom Orden des Phönix, von Dumbledores Armee, die volljährigen Hogwarts-Schüler mit ihren menschlichen und Kreaturen- Verbündeten, und Harry Potter konnte seinen Erzfeind, Lord Voldemort in der großen Halle zur Strecke bringen.

\subsection*{\Large Gründung}
%ABSATZ
Die Hogwartsgründer waren befreundet, besonders Rowena Ravenclaw und Helga Hufflepuff waren schon länger persönlich befreundet. Die vier Hexen und Zauberer wollten eine Schule zur besten Ausbildung in Zauberei und Hexerei für die Kinder mit magischem Blut gründen. Nach einer beliebten geschichtlichen Annahme hat Rowena Ravenclaw sowohl den Ort wie den Namen für die Schule ausgesucht: In einem Traum hat ein Warzenschwein sie zu einer Klippe an einem See geführt. Es war auch Rowena Ravenclaw, die die Idee hatte, dass die Treppen und Flure sich bewegen.
\vspace{10pt}
\newline
Im Jahre um 993 wurde das Schloss Hogwarts gebaut. Alle vier Gründer leiteten ein nach ihnen benanntes Haus, wobei sie unterschiedliche Anforderungen an die Schulkinder stellten. Godric Gryffindor glaubte, dass jedes Kind, welches magische Eigenschaften vor dem elften Geburtstag aufwies, sollte an der Hogwarts-Schule aufgenommen werden. Gryffindor schätzte Tapferkeit und Heldenmut, das waren für ihn die hervorragenden Eigenschaften. Kinder mit diesen Eigenschaften wurden bevorzugt in sein Haus aufgenommen
\vspace{10pt}
\newline
Helga Hufflepuff war eine freundliche, warmherzige Hexe. Schüler in ihrem Hause sollten loyal, geduldig und fleißig sein. Durch Sie kamen die Hauselfen ins Schloss.
\vspace{10pt}
\newline
Rowena Ravenclaw war eine scharfsinnige, intelligente Frau, sie bevorzugte intelligente Hexen und Zauberer, so nahm sie die schlauesten und intelligentesten Kinder in ihr Haus auf.
\vspace{10pt}
\newline
Salazar Slytherin wollte nur Schüler ausbilden, die aus reinblütigen Hexen- und Zaubererfamilien stammten, seine Schüler und Schülerinnen sollten einfallsreich, entschlossen und ehrgeizig sein. Wegen seiner sturen Haltung zum Reinblut überwarf er sich überwiegend mit Gryffindor, aber auch mit den beiden anderen, so dass Slytherin das Schloss und die Schule verließ, jedoch noch vor seinem Weggang eine den anderen nicht bekannte, geheime unterirdische riesige Kammer erschuf, in die er einen Basilisken platzierte, der zur gegebenen Zeit von einem seiner Nachfahren mit Parsel befehligt werden sollte, die Schule von Muggelstämmigen und Halbblütern bereinigen sollte.
\subsection*{\Large Lage}
%ABSATZ
Das Schloss befindet sich in den Hochlagen Schottlands in einem Tal, umringt von Bergen, im Süden des Hauptgebäudes liegt der Großer See. Das Dorf Hogsmeade liegt am anderen Seeufer, eine Straße führt von dort zum Schloss. In Hogsmeade wohnen nur Hexen und Zauberer. Hogsmeade ist die Endstation des Hogwarts-Expresses, welches die Schüler nach den Ferien hierher fährt. In Hogsmeade bestehen Einkaufsmöglichkeiten oder Gaststättenbesuche für die Hogwartsbewohner. Nächstgelegene 
%SEITE1
Ortschaften sind Dufftown im Bezirk Moray und Achintee im Bezirk Lochaber. Die Eingangstore des Schlosses zeigen nach Westen. Von dort senkt sich das Gelände über Wiesen und Rasen. Weiter im Westen liegt ein großer Wald, genannt der verbotene Wald.
\subsection*{\Large Magie im Schloss}
%ABSATZ
Das Schloss wird durch Magie unterstützt, also Zustände, die nicht durch andere Möglichkeiten aufrecht erhalten oder hergestellt werden können. Ein gutes Beispiel dafür sind die sich spontan bewegenden Treppen, die eine der Hogwartsgründerinnen, Rowena Ravenclaw gerne im Schloss haben wollte. Das Schloss Hogwarts wird auch durch eine Vielzahl uralter Zauber geschützt. Dazu gehörte auch der Anti-Disapparier-Fluch (vor langer Zeit wurde dieser Fluch auf das Schloss Hogwarts angewendet, vermutlich vom Schulleiter der Hogwarts-Schule für Hexerei und Zauberei, um zu verhindern, dass Schüler unbemerkt aus dem Schloss abhauen können. Nur für den Schulleiter kann diesen Zauber außer Kraft setzen. Der Fluch wird vorübergehend in der Großen Halle aufgehoben, wenn Schüler im 6. Schuljahr am Fach Apparieren und Disapparieren teilnehmen). Das Schloss ist unortbar und verzaubert, so dass Muggel, die sich dem Schloss nähern, nur eine baufälilge Ruine sehen würden, mit einem Warnschild davor, wegen Gefahren nicht näher zu treten. Trotzdem wären die Muggel wohl in der Lage, dass Schloss in der natürlichen Form zu sehen, wenn sie in das Schloss eingeladen werden oder sonstwie Kenntnis vom Schloss gelangen. Das war 1932 der Fall, als Jacob Kowalski ins Schloss Hogwarts eingeladen wurde. Der magische Schutz um das Schloss schien auch erfahrende Schwarzmagier abzuweisen.
\vspace{10pt}
\newline
Der Hüter der Schlüssel und Ländereien von Hogwarts, Rubeus Hagrid, behauptete zu Harry Potter, das Hogwarts der sicherste Platz sei, sogar noch sicherer als die Gringotts Zaubererbank, wohl ein Grund, weswegen der Stein der Weisen zum erhöhten Schutz nach Hogwarts kam. Nachdem das Zaubereiministerium letztlich zugab, dass Lord Voldemort 1996 zurückgekehrt war, wurden die Schutzmaßnahmen um das Schloss Hogwarts weiter verstärkt. Viele davon bewirkte Albus Dumbledore, so wurden die Eintrittstore magisch verschlossen, die nur Hogwartslehrer aufschließen konnten, das Hineinfliegen in das Hogwartsgelände mit Flugbesen war nicht mehr möglich. Auch alle Geheimgänge, die angeblich unbekannt waren, erhielten verstärkten Schutzzauber. Zusätzlich wurden Mitglieder vom Orden des Phönix und Auroren als Wachposten eingesetzt, der Hausmeister Argus Filch überprüfte alle hereinkommenden und hinausgehenden Hogwarts-Schüler auf gefährliche und verbotene Gegenstände, so dass ein Eindringen unmöglich schien.
\vspace{10pt}
\newline
Draco Malfoy hatte allerdings eine Möglichkeit gefunden, Todesser mit Verbündeten ins Schloss zu schleusen: Draco Malfoy entdeckte ein Verschwindekabinett bei Borgin & Burke's in der Nokturngasse und dessen Gegenstück im Raum der Wünsche im siebten Stockwerk, nach deren Reparatur durch Draco konnten Feinde von der Nokturngasse ins Herz des Hogwarts-Schlosses gelangen. Das ein minderjähriger Schüler aus dem Schloss heraus das am sichersten geglaubten magischen Gebäude in der Zaubererwelt hatte durchbrechen können, hat die Zaubererschaft mächtig erschüttert. Es zeigte sich, dass es nirgends mehr vor Lord Voldemort sicher war.
\vspace{10pt}
\newline
Als Ergebnis der Magie, die sowohl durch den Unterricht der Schüler in Zauberei und Hexerei, durch die magischen Objekte im Schloss und die auf das Schloss liegende Verzauberungen besteht, lag eine magische Atmosphäre auf und um das Schloss, so dass technische, auf Elektronik und Elektrizität fußende Geräte nicht funktionierten. Mechanische Geräte wie Armbanduhren oder Kameras waren nicht betroffen. 
\vspace{10pt}
\newline
\subsubsection*{\large Erwähnenswerte Magie im Schloss}
\vspace{10pt}
%ABSATZ
\begin{itemize}
    \item Das gesamte Schloss ist von Magie durchzogen. Die Funktion elektrischer Geräte der Muggel ist hier gestört oder unwirksam.
    \item Das Schloss ist unortbar.
    \item Treppen ändern ihre Richtung, Trickstufen bilden sich aus. Es gibt Geheimgänge und Stockwerkübergreifende Abkürzungen.
    \item Treppen zu Mädchenschlafräumen verwehren Jungen den Zutritt.
    \item Klassenräume und Büros ändern ihre Lage im Schloss.
    \item Das Schulleiterbüro kann sich für ungeeignete Schulleiter und Schulleiterinnen mit einer Zugangssperre versiegeln.
    \item Ein Raum der Wünsche verändert sich nach den Bedürfnissen eines Wünschenden.
    \item Apparieren und Disapparieren ist hier nicht möglich, ebenso wenig kann nicht mit dem Flugbesen in das oder vom Hogwartsgelände geflogen werden.
%SEITE2
    \item Einige Räume und Hallen bilden ein Innenraumwetter ab.
    \item Muggel sehen von außen nur eine Ruine mit einem Warnschild "Nicht Betreten".
    \item Besteck, Geschirr, Schüsseln, Speisen und Getränke verschwinden von den Speiseaal-gespiegelten Tischen in der Küche zu den Tischen in der großen Halle.
    \item Wegen der dauerhaft geballten Anwesenheit von Jugendlichen im Schloss konnte schon vor Jahrhunderten ein landesweit einmalig bemerkenswerter Poltergeist erschaffen werden.
\end{itemize}
\vspace{10pt}

\subsection*{\Large Erdgeschoss}
\subsubsection*{\large Eingangshalle}
%ABSATZ
Am Haupteingang zum Schloss Hogwarts sind zwei große, schwere Flügeltüren aus Eichenholz. Durch diese kommt man in die Eingangshalle Die Eingangshalle befindet sich im Erdgeschoss. Sie hat eine Höhe wie eine Kathedrale. Gegenüber der Eingangstür ist eine breite Marmortreppe die zu den Obergeschossen führt. Zwei große Flügeltüren an der rechten Seite führen in die Große Halle.  Von hier gelangt man auch in das Tiefgeschoss, zur Küche und die Kerker nach unten und über die Große Treppe nach oben.
\subsubsection*{\large Empfangsraum}
%ABSATZ
Der Empfangsraum befindet sich in einem kleinen Vorgebäude vor der Eingangshalle. In diesem Raum ließ die stellvertretende Schulleiterin, Professor Minerva McGonagall den Neuankömmlingen zum ersten Schuljahr warten, bis die Zuordnungszeremonie und das Willkommensfest beginnt.
\subsubsection*{\large Viaduktvorhof}
%ABSATZ
Der Viaduktvorhof führte zur Eingangshalle. Dieser Hof ersetzte die Vorhalle zur Eingangshalle nach 1994.
\subsubsection*{\large Große Halle}
%ABSATZ
Die große Halle ist der Hauptversammlungsort der Schule. Hier wurde die gemeinsamen Mahlzeiten meist dreimal täglich serviert und gegessen, die Schüler erhielten mit Eulenpost ihre Post, hier wurden verschiedene Abschlussprüfungen durchgeführt, und große Veranstaltungen und Feste finden hier statt. Die Halle ist so groß, dass alle Schüler und Angestellte der Hogwarts-Schule und mögliche Gäste darin Platz hatten. Die Wände sind sehr hoch, die Hallendecke ist verzaubert, indem sie den Himmel und das Außenwetter darstellt.
\vspace{10pt}
\newline
In der großen Halle fand auch am 2. Mai 1998 die abschließende Schlacht von Hogwarts statt, bei der Harry Potter seinen Erzfeind Lord Voldemort endgültig besiegte.
\subsubsection*{\large Lehrerzimmer}
%ABSATZ
Das Lehrerzimmer ist ein länglicher, getäfelter Raum mit dunklen, hölzernen, zusammengewürfelten Stühlen. Im Raum steht ein Kleiderschrank, in dem einst ein Irrwicht lebte. Der Eingang zum Lehrerzimmer wurde von zwei sprechenden Steinfiguren bewacht.
\subsubsection*{\large Hausmeisterbüro}
%ABSATZ
Der Hausmeister Argus Filch hatte hier sein Büro. Es ist an die Eingangshalle angeschlossen und befindet sich im Erdgeschoss des Schlosses. Der Raum ist klein und fensterlos, von der Decke hängt eine Öllampe. Ein leichter Geruch von frittiertem Fisch hing in der Luft.
\subsubsection*{\large Zentralturm-Hof}
%ABSATZ
Das ist eine Säulengänge im Inneren des Schlossgebäudes. Im offenen Bereich war eine Rasenfläche, in deren Mitte wuchs über die Jahrhunderte ein riesiger Baum. Dort steht auch eine große Armillarsphäre aus Massiveisen.
%SEITE3
\subsubsection*{\large Abteilung für Verwandlung}
%ABSATZ
In diesem Flügel des Hogwarts Schlosses sind die Räume, die für den Unterricht für Verwandlung (Fach) vorgesehen sind. Die Abteilung für Verwandlung ist vermutlich um den Zentralen Innenhof gelegen, einschließlich Klassenraum 1 B.
\subsubsection*{\large Klassenraum 1 B#(Verwandlung-Klassenraum)}
%ABSATZ
In diesen Raum wurde das Fach Verwandlung unterrichtet. Es ist im Erdgeschoss gelegen und grenzt an den Zentralturm-Hof. Der Raum ist sehr groß, hat hohe Fenster, es ist mit vier Reihen Schreibtischen ausgestattet, sowie mit einigen Bücherregalen, verschließbaren schränken, zwei Schultafeln und einem Lehrertisch. Dieser Raum wurde auch von der Hauslehrerin Minerva McGonagall im Herbst 1994 benutzt, um die älteren Gryffindor-Schüler auf den Weihnachtsball beim Trimagischen Turnier mit praktischem Tanzunterricht vorzubereiten.
\subsubsection*{\large Klassenraum 11 (Firenze Wahrsagen)}
%ABSATZ
Klassenraum 11 befindet sich im Erdgeschoss des Schlosses. Es wurde vom Zentauren Firenze im Fach Wahrsagen benutzt. Der Raum wurde von Albus Dumbledore derart verzaubert, dass es dem natürlichen Lebensraum von Zentauren im verbotenen Wald nachempfunden wurde.
\subsubsection*{\large Viadukteingang}
%ABSATZ
Diese große Kammer verbindet den Viadukt mit dem ersten Stockwerk über eine breite Treppe, die zum Tapetenkorridor führt. Es gibt auch einen Bogengang, der abwärts der über die Zaubertränketreppe in den Kellerflur führt.
\subsubsection*{\large Viereckhof}
%ABSATZ
Dieser Hof ist im Schlossaußenbereich. Er verbindet den Viadukt mit dem Bootshaus. Der Hof ist gepflastert und ringsum von Mauern mit Zinnen umgeben. Oben auf den Mauern ist ein begehbarer Gang, der in Abständen in Zinnen mündet. Von dort ist der Blick in den Hof möglich.
\subsubsection*{\large Viadukt}
%ABSATZ
Diese Talbrücke besteht aus Stein und führt bis zum großen See. Diese Brücke überspannt eine tiefe Schlucht. Der Viadukt verbindet an einem Ende den Viaduktvorhof mit dem linken und rechten Viaduktturm am anderen Ende.
\subsubsection*{\large Parterre-Korridor}
%ABSATZ
Dieser Korridor verbindet das Hauptschloss mit dem Trainingsgelände über den Eingang zum Trainingsgelände

\subsubsection*{\large Andere Örtlichkeiten im Erdgeschoss}
\vspace{10pt}
%ABSATZ
\begin{itemize}
    \item Astronomieturm-Hof (Engl. Astronomy Tower courtyard)
    \item Bohnenbelohnungsraum (Engl. Bean Bonus Room) (Nicht-kanonischer Auftritt)
    \item Eingang zum Trainingsgelände
    \item Ersatzklassenraum für Verteidigung gegen die dunklen Künste
    \item Gesperrter Raum (Parterre) (Engl. Out of Bounds)
    \item Glockenturmflügel (Engl. Bell Tower Wing)
    \item Hängebrückevorhalle
    \item Hauspunkte-Zeremoniekammer
    \item Hofkorridor (Engl. Courtyard corridor)
    \item Kräuterkundekorridor (Engl. Herbology corridor)
    \item Kräuterkunde-Vorratsraum
    \item Korridor vor der Eingangshalle (Engl. Corridor off the Entrance Hall)
%SEITE4
    \item Lagerraum (Engl. Storeroom (ground floor))
    \item Lange Galerie (Engl. Long Gallery)
    \item Lehrerzimmerkorridor (Engl. Staffroom corridor)
    \item Maßwerkhalle (Engl. Tracery Hall)
    \item Portrait-Raum
    \item Schulhof (Engl. Courtyard)
    \item Zentralturm-Hof (Engl. Middle Courtyard)
\end{itemize}
\subsection*{\Large Die große Treppe}
%ABSATZ
Die große Treppe (  Engl.  Grand staircase ) ist eine gewaltige Treppenflucht im Inneren des Schlosses, sie reicht von den Kellergewölben bis in das siebte Stockwerk, so dass alle Stockwerke erreicht werden. Hunderte Portraits bedecken die Wände neben den Treppen, manche von ihnen verdecken geheime Gänge zu anderen Örtlichkeiten des Schlosses. Manche dieser Treppen ändern ihre Richtung.


\subsection*{\Large Erstes Stockwerk}
\subsubsection*{\large Klassenraum Geschichte der Zauberei}
%ABSATZ
In diesem Klassenraum wurde Geschichte der Zauberei (Fach) seit Jahrzehnten vom Geist, Professor Cuthbert Binns an der Hogwarts-Schule für Hexerei und Zauberei gelehrt. Er galt bei den meisten Hogwarts-Schülern als der langweiligste und der am meisten monotone Lehrer der Schule.
\subsubsection*{\large Klassenraum für Muggelkunde}
%ABSATZ
In diesem Klassenraum wurde Muggelkunde unterrichtet.
\subsubsection*{\large Krankenflügel}
%ABSATZ
Madam Poppy Pomfrey war die Leitende Heilerin des Krankenflügels (  Engl.  Hospital Wing ). Hogwarts-Schüler, denen im Rahmen ihres Unterrichts ein gesundheitliches Missgeschick passierte oder die in ihrer Freizeit oder beim Üben der Magie Verletzungen oder misslungene Zaubereien passierten, werden im Krankenflügel behandelt. Der Krankenflügel war gut ausgestattet um mit aller Arten von magischen und irdischen Verletzungen zurecht zu kommen, dazu zählten gebrochene Gliedmaßen und andere Knochen, Knochen nachwachsen zu lassen, Verbrennungen, Verätzungen zu heilen. oder Werwolf-Bisse zu behandeln.
\vspace{10pt}
\newline
Nur Personen mit sehr schweren Verletzungen und Fluchauswirkungen wurden in das St.-Mungo-Hospital für Magische Krankheiten und Verletzungen zur Weiterbehandlung eingewiesen, das traf zu, als Katie Bell im Winter 1996 die mit einem Fluch belegte Opal-Halskette berührte, an der sie fast gestorben wäre oder als Professor Minerva McGonagall von vier Schockzauber im Frühsommer 1996 gleichzeitig in die Brust getroffen wurde.
\subsubsection*{\large Poppy Pomfreys Büro}
%ABSATZ
Poppy Pomfreys Büro (  Engl.  Poppy Pomfrey's office ) ist ein großer Raum, der an den Krankenflügel anschließt. Er enhält einen Bücherschrank, einen Schreibtisch, zwei Kämmerchen, zwei Betten für Patienten und ein weiteres Bett, durch einen Vorhangvom Rest des Raumes getrennt.
\subsubsection*{\large Minerva McGonagalls Büro}
%ABSATZ
Das Büro der Hauslehrerin der Gryffindors (  Engl.  Minerva McGonagall's office ) lag an einem Korridor im ersten Stockwerk, welches über die große Treppe von der Eingangshalle erreicht werden kann. Bis 1998 hatte die Hauslehrerin von Gryffindor, Professor Minerva McGonagall dieses Büro. Dem Büro angeschlossen war ihr Schlafraum, welcher auch durch eine Geheimtür zugänglich war.
\subsubsection*{\large Steinbrücke}
%ABSATZ
Die Steinbrücke (  Engl.  Stone Bridge ) ist der benachbart zu der Hängebrücke. Die Steinbrücke verbindet den Erster Stock-Korridor des Steinbrückenturms mit dem Wandteppichkorridor im Viadukteingang
%SEITE5
\subsubsection*{\large Wandteppichkorridor}
%ABSATZ
Dieser Korridor im ersten Stockwerk verbindet den Viadukteingang zum Hauptkorridor im ersten Stockwerk. Die Wände sind mit Wandteppichen geschmückt, und darüber sind die Portraits aus der Zaubererwelt. In diesem Flur befindet sich auch der Zaubertrank-Aufbewahrungsraum.
\subsubsection*{\large Zaubertrank-Aufbewahrungsraum}
%ABSATZ
Dieser Aufbewahrungsraum (  Engl.  Potions Storeroom ) wird über den Wandteppichkorridor erreicht. Hier bewahrte Professor Severus Snape seine privaten Zaubertränke und Zaubertrankzutaten auf. Es ist einer von zwei bekannten Aufbewahrungsräumen für Zaubertränke im Schloss Hogwarts; dieses ist der private Aufbewahrungsraum, dann gibt es noch den Aufbewahrungsschrank für Hogwarts-Schüler. Unter anderem bewahrte Snape das Veritaserum und Zutaten für den Vielsafttrank dort auf.
\subsubsection*{\large Erster Stock Mädchentoilette}
%ABSATZ
Die Mädchentoilette (  Engl.  First-Floor Girl's Toilets ) befindet sich im ersten Stockwerk. Hier wurde Hermine Granger am 31. Oktober 1991 von einem Bergtroll angegriffen.
\subsubsection*{\large Zinnen des Viereckhofs}
%ABSATZ
um den gepflasterten Viereckhof sind Mauern, die in Abständen von Zinnen unterbrochen sind. Im Inneren sind die Zinnen mit Holzdielen ausgelegt, die in äußeren Laufstegen enden. Auf einer Seite der Zinnen befindet sich ein Glasfenster. Die Zinnen wurde von den Verteidigern von Hogwarts in der Schlacht von Hogwarts als erhöhte Aussichtsplatform benutzt, diese Zinnen wurden in der Schlacht zerstört.
\subsubsection*{\large Andere Örtlichkeiten im ersten Stockwerk}
\vspace{10pt}
%ABSATZ
\begin{itemize}
    \item Klassenraum für Arithmantik
    \item Klassenraum 1A
    \item Klassenraum 1C
    \item Klassenraum 1D
    \item Klassenraum 1E
    \item Haftzimmer
    \item Duellierraum (I)
    \item Duellierraum (II)
    \item Erster Stock-Korridor
    \item Passage der Regelverstöße
    \item Hintere Halle
\end{itemize}

\subsection*{\Large Zweites Stockwerk}

\subsubsection*{\large Arkaden im zweiten Stockwerk}
%ABSATZ
Im zweiten Stockwerk sind Arkaden (  Engl.  Second-floor arcade ) die als Vorraum zu einer Treppe in das dritte stockwerk dienen. Eine Ritterrüstung bewacht einen Geheimgang zu einem geheimen Aufbewahrungsraum in dieser Region. Hier war es offensichtlich ein beliebter Aufenthaltsort, da Hogwarts-Schüler von allen Häusern sich dort gerne in den Morgenstunden aufhielten.
%SEITE6
\subsubsection*{\large Badezimmer der Maulenden Myrte}
%ABSATZ
„Sie [Maulende Myrte] spukt in einer der Mädchentoilettenräume im ersten Stock“
\vspace{10pt}
\newline
— Hermine Granger erwähnt die Maulende Myrte
\vspace{10pt}
\newline
„Er [Harry] lief die Stufen zur Eingangshalle hoch. Doch hier war gewiss nichts mehr zu hören...Ron und Hermine dicht auf den Fersen rannte Harry die Marmortreppe zum ersten Stock hoch. Aus dem nächsten Stockwerk, aus weiter Ferne, hörte er die verblassende Stimme ... Er nahm drei Stufen der nächsten Treppe auf einmal und jagte durch alle Gänge im zweiten Stock. Erst als sie in den letzten, verlassenen Korridor eingebogen waren, hielten sie an.“
\vspace{10pt}
\newline
— Das Trio auf dem Weg zur Schrift an der Wand im zweiten Stock
\vspace{10pt}
\newline
Der Toiletten- und Waschraum für Mädchen im zweiten Stockwerk ist auch bekannt als das Badezimmer der Maulenden Myrte (  Engl.  Moaning Myrtle's Bathroom ) Dieser Raum wurde seit Jahren und Jahrzehnten nicht mehr benutzt, seit eine Hogwarts-Schülerin namens Myrte Warren 1943 von einem Basilisken auf Befehl von Tom Riddle, dem Erbe Salazar Slytherins, getötet wurde. Seitdem spukt sie als Geist in diesem Toilettenraum, weswegen die Mädchen der Schule diese Toiletten vermeiden.
\vspace{10pt}
\newline
In diesem Badezimmer befindet sich der Geheimeingang zur Kammer des Schreckens.

\subsubsection*{\large Büro des Lehrers für Verteidigung gegen die dunklen Künste}
%ABSATZ
Dieses Büro (  Engl.  Defence Against the Dark Arts Professor's office ) befindet sich im zweiten Stockwerk vom Schloss Hogwarts. Es ist mit einer Treppe zum Klassenraum Verteidigung gegen die dunklen Künste verbunden.

\subsubsection*{\large Geheimer Aufbewahrungsraum 2. Stockwerk}
%ABSATZ
Es gibt im zweiten Stockwerk einen geheimen Aufbewahrungsraum (  Engl.  Second-floor secret storeroom ) ein paar Stufen hinab hinter einem Geheimgang in den Arkaden, der von einer Ritterrüstung bewacht wird. Im Raum sind einzelne Möbel wie Kommoden, Truhen, Lehnsessel und Vasen, sowie Ersatzrahmen für Portraits und Kerzenhalter. Die Zimmerdecke ist niedrig. 1991 gab es dort einen heftigen Gnombefall.

\subsubsection*{\large Gesperrter Raum wegen Überflutungsgefahr}
%ABSATZ
Dieser Raum (  Engl.  No entry due to flooding - Mr. Filch ) geht von der Galerie ab. Er war immer außer Funktion und daher ständig verschlossen, denn es bestand immer Gefahr der Überflutung in diesem Raum.
\subsubsection*{\large Halle im zweiten Stock}
%ABSATZ
In dieser Halle (  Engl.  Second-floor hall ) sind einige Schaukästen ausgestellt, hier ist ein Kamin und ein einzelnes Portrait, welches einen geheimen Ort mit einer Truhe verdeckt. Die Zimmerdecke und die Fenster sind hoch, an der Rückwand ist eine Dreier-Fenstergruppe mit Buntglas. Der Raum hatte Zugang über einen Vorraum im zweiten Stockwerk. Der Kamin verdeckte eine geheime Abkürzung, die von den Weasley-Zwillingen 1991 benutzt wurde.
\subsubsection*{\large Klassenraum 31}
%ABSATZ
Im Klassenraum 31 (  Engl.  Class 31 ) im zweiten Stockwerk fand im Schuljahr 1991-1992 der Unterricht in Verteidigung gegen die dunklen Künste statt.
\subsubsection*{\large Korridor im zweiten Stockwerk}
%ABSATZ
Der Korridor im zweiten Stockwerk (  Engl.  Second-floor corridor ) erstreckt sich über beide Seiten vom Hogwarts-Schloss.
\subsubsection*{\large Wandelhalle}
%ABSATZ
Im zweiten Stock befindet sich eine Wandelhalle (  Engl.  Second-floor foyer ), sie ist mit Teppichboden ausgelegt und hat eine hohe Raumdecke. Vier Bogengänge führen zu weiteren Korridoren. In der Halle befanden sich Vasen, Schaukästen waren ausgestellt. Mit großen Fallgittern konnte der Raum abgesperrt werden. Der Poltergeeist Peeves hat Harry Potter 1991 dadurch eingesperrt. Hogwarts-Schüler aus allen Häusern hielten sich gene in den Morgenstunden in der Wandelhalle auf.
%SEITE7
\subsubsection*{\large Zauberkunst-Übungsraum}
%ABSATZ
Der Zauberkunst-Übungsraum (  Engl.  Leviosa Challenge ) befindet sich im zweiten Stockwerk. Dort befindet sich ein Hindernis-Parcour, der für praktische Übungen des Schwebe- und Lenkzaubers Wingardium Leviosa vorgesehen war. Der Hinderniskurs enthielt eine Anzahl von Anforderungsbelohnungen, die bei erfolgreichem Bestehen gesammelt werden konnten. Etliche Objekte waren auch nützlich, um den Zauberspruch zu üben. An einigen Blöcke sprießten Flügel, wenn der Schwebezauber an diesen verübt wurde und dadurch wurde der Schwebezustand unterstütztt. Im Schuljahr 1992-1993 ließ Professor Filius Flitwick seine Klasse diesen Parcour als Wiederauffrischungskurs vom letzten Jahr durchlaufen.
\subsubsection*{\large Zaubertrank-Aufbewahrungsraum}
%ABSATZ
Neben dem Zaubertrank-Aufbewahrungsraum im ersten Stockwerk, der die persönlichen Bestände von Professor Severus Snape enthielt, gibt es im zweiten Stockwerk einen für Hogwarts-Schüler zugänglichen Zaubertrank-Aufbewahrungsraum (  Engl.  Second-floor potions storeroom ). Es enthält etliche Regale mit Messbechern, Zaubertränken und Zaubertrankzutaten in Vorratsgläsern. Dieser Raum schien ein beliebter Treffpunkt für Hogwarts-Schüler aller Häuser in den Morgenstunden gewesen zu sein. Unter dem Teppich war ein Geheimraum.


\subsection*{\Large Drittes Stockwerk}

\subsubsection*{\large Bibliothek}
%ABSATZ
Die Hogwarts-Bibliothek liegt im dritten und vieten Stockwerk. Sie enthält zehntausende Bücher auf tausenden Regalen. Die Bücherei wurde geführt von Irma Pince. Hogwarts-Schüler können vor Ort Bücher lesen oder sich welche zu Studienzwecken oder zum Vergnügen ausleihen. Die Bibliothek schließt täglich abends um acht Uhr.
\subsubsection*{\large Geheimgang an der Statue der einäugigen Hexe}
%ABSATZ
Dieser Geheimgang an der Statue der einäugigen Hexe führe in einen unterirdischen Tunnel vom dritten Stockwerk im Schloss Hogwarts in den Keller des Süßwarenladens Honigtopf in Hogsmeade. Harry Potter lernte diesen Geheimgang erstmals kennen, als ihm Fred und George Weasley die Karte des Rumtreibers 1993 überließen.
\subsubsection*{\large Klassenraum 2E (Zauberkunstklassenraum)}
%ABSATZ
Im Klassenraum 2E (  Engl.  Classroom 2E ) wird im dritten Stockwerk vom Schloss Hogwarts der Zauberkunstunterricht durchgeführt. Der Raum ist mit vier langen Tischen ausgestattet, jeweils zwei Reihen an den Seiten, die wandnahe Tischreihen sind höher als zur Raummitte. Es gibt einen großen Kamin im Klassenzimmer und etliche Bücherstapel, auf denen Professor Flitwick als kleinwüchsiger Zauberer steht, wenn er die Schüler unterrichtet.
\subsubsection*{\large Klassenraum 3C (Verteidigung gegen die dunklen Künste)}
%ABSATZ
In diesem Klassenraum wurde das Fach Verteidigung gegen die dunklen Künste unterrichtet. Von der Decke hing ein eiserner Kronleuchter und ein Drachenskelett. An der Rückwand befand sich ein Projektor, der magisch betrieben wurde. Doppelschreibtische für die Schüler waren aufgereiht, genug für eine Doppelklasse. An den Seitenwänden befinden sich große Fenster, die bei Bedarf vedunkelt werden können. Vorne rechts zum Lehrer hin befand sich eine geschwungene Treppe mit massiven Geländer und oben ein Halbrundabsatz, vond dort gelangte der Lehrer in sein Büro.
\vspace{10pt}
\newline
Die Raumansicht wurde in den Jahren häufig geändert, vermutlich entsprechend den Bedürfnissen der Lehrkraft in diesem Fach. Professor Gilderoy Lockhart hatte 1992-1993 an den Wänden zahlriche Bilder und Portraits von sich selbst, bis hin zu einer großen Sammlung von Skeletten und Schrumpfköpfen 1986-1987 unter Professor Snape.
\subsubsection*{\large Korridor im dritten Stockwerk}
%ABSATZ
Im Schuljahr 1991-1992 war der Korridor zur rechten Seite des Schlosses gesperrt, denn dort befand sich der Zugang zu den Untergrundkammern, die durch lebensgefährliche Schutzmaßnahmen, wie den dreiköpfigen Riesenhund, einer todwürgenden Teufelsschlinge, einen gefährlichen Troll oder tötliches Gift gesichert wurden. Im siebten Hindernis wurde der Stein der Weisen versteckt. Der Schulleiter Albus Dumbledore warnte die Hogwarts-Schüler bei der Jahreseinführungsfeier eindringlich davor, diesen Teil des dritten Stockwerkes nicht zu betreten.
%SEITE8
\subsubsection*{\large Pokalzimmer}
%ABSATZ
Eine breite Treppe führt hinab vor vergitterte, verzierte Flügeltoren aus Metall, die bei Annäherung sich öffnen. Die Stufen setzen sich im Eingang über die gesamte Raumbreite fort. Es ist ein großes Gewölbe, an den seitlichen Wänden sind breite, vierkantige Säulen mit Holzverkleidung oder aus Holz, die sich oben im Bogen nach vier Seiten aufweiten und in der Kuppe auf die Bögen der Nachbarsäulen treffen. Die Säulen stehen auf gleichbreite Vitrinen mit gläsernen Fronten, in den Vitrinen sind dutzende Pokale und Ornamente unterschiedlicher Größe und Form aufgestellt. Die Vitrinen sind etwa menschenhoch.
\vspace{10pt}
\newline
In der Mitte des Raumes steht ein Tisch mit schwarzem, bodenlangem Tischtuch, auf und um den Tisch dicht an dicht gestellt auch hier dutzende Pokale und Kelche. Einige der Artefakte drehen sich oder schwingen. Abends und nachts schimmert ein bläuliches Licht durch die mittlere Vitrine und hinter einer der Säulen, von der Decke hängt eine vierkantige, nach unten schmaler werdende Leuchte mit Milchglas. Gegenüber dem Eingang sind vergitterte Sprossenfenster, dort befindet sich ein großer Kamin, davor stehen zwei Stühle. Auf dem Kaminsims stehen drei schwarz angemalte, möglicherweise metallartige, verzierte Schachteln.
\subsubsection*{\large Rüstungskorridor}
%ABSATZ
Der Rüstungskorridor (  Engl.  Armoury ) ist ein Korridor neben dem Pokalzimmer, in dem zahlreiche Ritterrüstungen vor den Fenstern ausgestellt sind.
\subsubsection*{\large Uhrenturmeingang}
%ABSATZ
Der Uhrenturmeingang (  Engl.  Clock Tower Entrance ) ist einer der drei Haupteingänge ins Schloss Hogwarts. Es besteht aus einem großen offenen, freien Raum Am Fuß des Uhrenturms, dort befinden sich zwei Treppen, die in die oberen Stockwerke des Turmes führen. Gegenüber vom Turmeingang befindet sich ein großes Fenster. Ein Portrait von Damara Dodderidge ist im Eingangsbereich aufgehängt, im Jahre 1995 hing hier auch das Portrait von Temeritus Shanks.
\vspace{10pt}
\newline
Das massive Uhrenpendel schwingt über diesem Eingang. Der Eingang kann mit einem Fallgitter verschlossen werden. Das Fallgitter kann mit einem Hebel hoch- und hinabgelassen werden. Die massive Doppeltür wird mit mehreren Riegeln verschlossen.
\subsubsection*{\large Zauberkunstkorridor}
%ABSATZ
Am Zauberkunstkorridor (  Engl.  Charms corridor ) liegen der Klassenraum 2E(Zauberkunst) und in der Nähe ein zweiter Klassenraum, der Korridor führt zum Korridor im dritten Stockwerk. Hinter einem Wandteppich ist ein Geheimgang, der in Richtung zum Pokalzimmer und zum Rüstungskorridor (drittes Stockwerk) führt. In der Nähe des Zauberkunstkorridors hängt ein Gemälde von betrunkenen Mönchen.

\subsubsection*{\large Sonstige Örtlichkeiten im dritten Stockwerk}
\vspace{10pt}
%ABSATZ
\begin{itemize}
    \item Klassenraum 3D
    \item Klassenraum für Ghulstudien
    \item Gunhilda von Gorsemoor Korridor
    \item Lagerraum (drittes Stockwerk)
    \item Raum der verlorenen Zauberstäbe
    \item Serpentinenkorridor
\end{itemize}

\subsection*{\Large Viertes Stockwerk}

\subsubsection*{\large Abstellraum für den Spiegel Nerhegeb}
%ABSATZ
In diesem nicht mehr benutzten Klassenraum (  Engl.  Disused classroom ) wurde der Spiegel Nerhegeb von September bis Dezember 1991 aufbewahrt. Dort sah Harry Potter im Spiegel sein Spiegelbild und seine gesamte, nicht mehr lebende Verwandschaft. Harry suchte diesen Raum mehrmals auf, um seine Verwandten im Spiegel wieder zu sehen. Nach diesen Ereignissen veranlasste der Schulleiter Albus Dumbledore, dass der Spiegel an einem zunächst unbekannten, geheimen Ort aufgestellt wurde.
%SEITE9
\subsubsection*{\large Badezimmer}
%ABSATZ
In diesem Badezimmer im vierten Stockwerk vom Schloss Hogwarts (  Engl.  Fourth-floor bathroom ) tauchte endlich der vermisste Slytherinschüler Graham Montague wieder auf, nachdem ihn Fred und George Weasley 1995 in ein Verschwindekabinett gestopft hatten. Montague wollte als Mitglied des neu eingerichteten Inquisitionskommandos den Weasleyzwillingen Hauspunkte abziehen, doch bevor er den Satz beenden konnte, hatten die beiden ihn schon ins Verschwindekabinett geschubst
\subsubsection*{\large Cuthbert Binns Büro}
%ABSATZ
Cuthbert Binns Büro (  Engl.  Cuthbert Binns's office ) liegt im vierten Stockwerk, und wird von Professor Cuthbert Binns, dem Lehrer im Fach Geschichte der Zauberei an der Hogwarts-Schule für Hexerei und Zauberei benutzt. Im Büro gibt es einen Kamin, zwei Schränke und Binns Schreibtisch sowie einen Tisch. Hinter einem Vorhang stand Binns Bett, obwohl es unwahrscheinlich ist, dass Binns das Bett überhaupt benutzen würde, seit dem er ein Geist geworden war.
\subsubsection*{\large Korridor im Krankenflügelturm}
%ABSATZ
Der Korridor verbindet den Krankenflügel zum vierten Stockwerk der Uhrenturmtreppe. Die Treppe im Krankenflügelturm ist neben dem Krankenflügel und führt zum Haupt-Korridor im vierten Stockwerk. Die Uhrenturmtreppe führt nach unten zum Uhrenturmeingang und nach oben zum Korridor im Krankenflügelturm (fünftes Stockwerk).
\subsubsection*{\large Korridor im vierten Stockwerk}
%ABSATZ
Der Korridor im vierten Stockwerk (  Engl.  Fourth-floor corridor ) kann durch den Studierbereich erreicht werden. Zum Korridor gehört auch ein Balkon. An den Wänden hängen zahlreiche Portraits von Hexen und Zauberern. Will man vom Westturm zum Gryffindorturm, kommt man durch diesen Korridor. Als Harry Potter einmal diesen Weg einschlug, wurde er vom Poltergeist Peeves aufgehalten, der ihm in diesem Korridor eine Vase überstülpen wollte.
\subsubsection*{\large Studierbereich (viertes Stockwerk)}
%ABSATZ
Der Studierbereich (  Engl.  Study Area ) ist eine Bibliothek im vierten Stockwerk in der Nähe vom Klassenraum 4F# (Geschichte der Zauberei). Hinter einem Bücherregal ist ein Geheimgang, der zum Korridor im siebten Stockwerk führt, gerade gegenüber vom Portrait Sir Cadogan. Ein weiterer Geheimgang befindet sich hinter einem Wandteppich, den Hermine Granger entdeckte und benutzte, als sie die Statue in Drachen-Verwandlung Draconifors benutzte. Im Studierbereich gibt es einen großen Kamin. Eine Tür führt zum Korridor im vierten Stockwerk
\subsubsection*{\large Verbotene Abteilung}
%ABSATZ
Die verbotene Abteilung ist ein Abschnitt der Hogwarts-Bibliothek im vierten Stockwerk. Sie wird mit einem Seil vom allgemein zugänglichen Teil abgesperrt, nur Studierende, die eine schriftliche Erlaubnis einer Lehrkraft vorweisen, hatten Zutritt in die verbotene Abteilung. Die Bücher in dieser Abteilung handelten typischerweise von den dunklen Künsten, oder von anderen Informationen, die nicht für die Allgemeinheit oder jüngeren Hogwarts-Schüler geeignet sind. Diese Bücher konnten von älteren Schülerinnen und Schülern für die Verteidigung gegen die dunklen Künste gelesen werden.
\subsubsection*{\large Sonstige Örtlichkeiten im vierten Stockwerk}
%ABSATZ
\vspace{10pt}
\begin{itemize}
    \item Alchemieräume (Alchemy Rooms)
    \item Amphitheater (The Old Theatre)
    \item Amulettsammlung (Amulet Collection)
    \item Balkon im vierten Stockwerk (Fourth-floor balcony)
    \item Bestückungsraum (Mounting Room)
    \item Brutpassage (Incubation passage)
    \item Dunkler Raum (The Dark Room)
    \item Eleazar Figs Büro (Eleazar Fig's office)
    \item Gästezimmer (viertes Stockwerk) (Guest Rooms)
    \item Großer Ofen (The Great Stove)
%SEITE10
    \item Herbarium (Herbarium)
    \item Klasse 67# (Theorie der Magie) (Class 67)
    \item Landkartenraum (Map Room)
    \item Museum (Museum)
    \item Musikraum (Music Room)
    \item Mykologieraum# (Pilzkunde) (Mycology)
    \item Raum der Mäuseplage (Room overrun with mice)
    \item Reptilienhaus (Reptile House)
    \item Skeletthalle (Skeleton Hall)
    \item Spiegelhalle (Mirror passage)
    \item Trockenräume (Drying Rooms)
    \item Waschraum (Fourth-floor washroom)
    \item Westbrücke (The West Bridge)
\end{itemize}

\subsection*{\Large Fünftes Stockwerk}

\subsubsection*{\large Kunstklassenraum}
%ABSATZ
Im Kunstklassenraum (  Engl.  Art Classroom ) wird sowohl Kunst in der Zaubererwelt wie auch der Muggelwelt an der Hogwarts-Schule für Hexerei und Zauberei gelehrt. Der Raum liegt im fünften Stockwerk vom Schloss Hogwarts, in dem auch die Klassenräume für Muggelkunde, Muggelmusik und Musik der magischen Welt gelegen sind. Im Kunstklassenraum sind sechs Staffeleien, drei Gemälde an der Wand, ein Schrank, ein Tisch und der Schreibtisch der Lehrkraft.

Dieser Klassenraum dürfte von den ehemaligen Professoren für Muggelkunde benutzt worden sein: Quirinus Quirrell, seine Nachfolgerinnen Charity Burbage und Alecto Carrow benutzt worden sein. Es ist allerdings höchst unwahrscheinlich, dass Alecto Carrow den Schülern das Malen und Zeichnen beigebracht hätte.
\subsubsection*{\large Musikklassenraum}
%ABSATZ
Der Musikklassenraum (  Engl.  Music Classroom ) befindet sich im fünften Stockwerk vom Schloss Hogwarts. Hier wird sowohl die Musik aus der Muggelwelt als auch der Zaubererwelt unterrichtet. In den 1990er Jahren war ein nicht bekannter männlicher Professor dafür zuständig.
\subsubsection*{\large Vertrauensschülerbad}
%ABSATZ
Das Vertrauensschülerbad ist ein besonderes Bad im fünften Stockwerk vom Schloss Hogwarts. Das Bad konnte nur von Vertrauensschüler und Vertrauensschülerinnen, Schulsprechern und Hogwarts Quidditch-Kapitänen genutzt werden. Es befand sich im fünften Stock, hinter der vierten Tür, links neben der Statue von Boris dem Bekloppten, die sich erst öffnete, wenn das richtige Passwort gesagt wurde.
\subsubsection*{\large Sonstige Örtlichkeiten im fünften Stockwerk}
\vspace{10pt}
%ABSATZ
\begin{itemize}
    \item Eiskorridor (Icy corridor)
    \item Geheimer Muggelraum (Secret Muggle Room)
    \item Halle (Fifth-floor Hall)
    \item Jungenbadezimmer
    \item Klassenraum 5B (Classroom 5B)
    \item Korridor im fünften Stockwerk (Fifth-floor corridor)
    \item Korridor im Krankenflügelturm (fünftes Stockwerk)
%SEITE11
    \item Klassenraum (Fifth-floor classroom)
    \item Lagerraum (Storeroom (fifth floor))
    \item Muggelstudienvorführraum (Muggle Studies showroom)
    \item Obere Halle (Upper Hall)
    \item Treppe im Ravenclawturm (Ravenclaw Tower staircase)
    \item Verschlossener Raum (Locked Room)
    \item Verwunschener Korridor (Cursed corridor)
    \item Verwunschenes Eisverlies (Vault of Ice)
    \item Weasley Schnäppchenvorräte (Weasley Supplies Bargain Basement)
    \item Weasley Süßwarenladen (Weasley Candy Emporium)
\end{itemize}

\subsection*{\Large Sechstes Stockwerk}
\subsubsection*{\large Horace Slughorns Bürom}
%ABSATZ
Horace Slughorns Büro befindet sich auf dem sechsten Stock von Schloss Hogwarts. Ziemlich geräumig enthielt dieses Büro einen Kamin mit zwei großen Sofas, einem runden Esstisch, groß genug, dass zehn Personen daran sitzen können und Zugang zu einem privaten Balkon. Es hatte genug Platz, um eine Slug-Club Weihnachtsfeier dort abzuhalten. Das Büro hat zwei Eingangstüren, von denen mindestens eine zum Korridor im sechsten Stockwerk führt.
\subsubsection*{\large Jungenbadezimmer}
%ABSATZ
Von diesem Badezimmer (  Engl.  Sixth-floor boys' bathroom ) führt eine Passage über die Marmortreppe auch in das siebte Stockwerk.[105
\subsubsection*{\large Klassenraum Studium alte Runen}
%ABSATZ
In dem Klassenraum 6A (  Engl.  Classroom 6A ) wurde das Studium der Alten Runen und Altertumswissenschaften unterrichtet. ES enthielt einige Schreibtische, einen Bücherschrank und einen Rednerpult für den Professor. Es ist neben Klassenraum 6B gelegen. Professor Bathsheda Babbling unterrichtet in dem Klassenraum 6A dieses Fach.
\subsubsection*{\large Korridor im sechsten Stockwerk}
%ABSATZ
Am Korridor im sechsten Stockwerk (  Engl.  Sixth-floor corridor ) vom Schloss Hogwarts liegt ein Büro eines Professors. Hier befindet sich auch der geheime Eingang zum Raum der Wünsche. Der Korridor gleicht dem Korridor im siebten Stock.
\vspace{10pt}
\newline
Der Korridor führt zur großen Slughorntreppe in den Nordturm und zur großen Treppe im Haupttreppenturm.
\subsubsection*{\large Ostflügel}
%ABSATZ
Der Ostflügel (  Engl.  East Wing ) ist eine große Halle. In der Mitte befindet sich eine große Feuerstelle, an den Wänden sind einige Portraits. Einige Ritterrüstungen bewachten den Raum nach Sonnenuntergang. Der Glanmore Peakes-Korridor im sechsten Stockwerk führt in den Ostflügel. In einem Raum am Ostflügel versteckten Fred und George Weasley die Karte des Rumtreibers.
\subsubsection*{\large Raum der Auszeichnungen}
%ABSATZ
Der Raum der Auszeichnungen (  Engl.  Room of Rewards ) ist ein geheimer Raum im Schloss. Dort werden die verschiedenen Auszeichnungen für Erfolge und Leistungen der Hogwarts-Schüler aufbewahrt. Über die Große Treppe durch das Portrait von Vindictus Viridian konnte der Raum erreicht werden.
\subsubsection*{\large Stillgelegtes Badezimmer}
%ABSATZ
Dieses Badezimmer (  Engl.  Disused sixth floor bathroom ) ist zugänglich durch ein Portrait im Leseraum) sowie durch eine Tür auf der anderen Schloss-Seite neben dem Glanmor Peakes-Korridor. Vor 1991 war dieses Bad unbenutzt, so konnten Fred und George Weasley dort ihr eigenes Geschäft für Gryffindor-Schüler einrichten. Zwei Jahre später verließen 1993 die Weasley-Zwillinge dieses Bad und beschlagnahmten ein weiteres Bad im siebten Stockwerk.
%SEITE12
\subsubsection*{\large Andere Örtlichkeiten im sechsten Stockwerk}
\vspace{10pt}
%ABSATZ
\begin{itemize}
    \item Argus Filch's Lagerraum
    \item Leerer Klassenraum
    \item Geheimgang (sechstes Stockwerk)
    \item Glanmore Peakes-Korridor
    \item Schreibbedarflager
    \item Westturmwehrgang
\end{itemize}
\subsection*{\Large Siebtes Stockwerk}
\subsubsection*{\large Arithmantikklassenraum}
%ABSATZ
Im Klassenraum 7A (  Engl.  Classroom 7A ) wurde das Fach Arithmantik unterrichtet. Es befindet sich im siebten Stockwerk vom Schloss Hogwarts am Serpentinenkorridor Dieser Klassenraum liegt neben dem Klassenraum 3C
\subsubsection*{\large Fette Dame-Korridor}
%ABSATZ
An diesem Korridor (  Engl.  Fat Lady's Corridor ) hing das Portrait der Fetten Dame. Dieses Portrait versteckte den Eingang zum Gryffindorturm. Der Korridor war mindestens eine Biegung entfernt von der großen Treppe. Im Schuljahr 1993-1994 haben Trolle hier zur Sicherheit Wache gehalten, um weitere Angriffe auf das Portrait vor dem aus Askaban ausgebrochenen Sirius Black zu schützen.
\subsubsection*{\large Halle der Verhexungen}
%ABSATZ
Die Halle der Verhexungen (  Engl.  Hall of Hexes ) ist eine Halle im siebten Stockwerk in der Nähe vom Raum der Wünsche.
\subsubsection*{\large Hauslehrer Ravenclaws Büro}
%ABSATZ
Das Büro vom Hauslehrer der Ravenclaws ist das dreizehntes Fenster rechts vom Westturm. Im Büro sind Portraits, ein Schreibtisch und Schlafquartiere. In diesem Büro wurde im Schuljahr 1993-1994 Sirius Black festgesetzt, wo er den Kuss des Dementors erwarten sollte. Er wurde von Harry Potter und Hermine Granger gerettet, so konnte Black mit dem Hippogreif Seidenschnabel entkommen.
\subsubsection*{\large Nordflügel}
%ABSATZ
Im Nordflügel (  Engl.  North Wing ) sind hunderte Bücherregale an den Wänden, verschiedene Tische und Möbel verteilen sich im Raum. Er dient als Leseraum und Studierbereich. Es gibt zwei Eingänge: ein Zugang vom Korridor im siebten Stock, der andere vom Fette Dame-Korridor. Im Raum steht eine große Statue vom Blutigen Baron.
\subsubsection*{\large Raum der Wünsche}
%ABSATZ
Der Raum der Wünsche (  Engl.  Room of Requirement ) ist ein Geheimraum, der demjenigen erscheint, der in dringendem Bedarf eines Raumes nach seinen Wünschen hat. Der Raum erscheint in der Form, die eine Hexe, ein Zauberer oder ein Hauself gerade in dem Moment benötigt. Der Raum ist vermutlich unortbar, da er nicht auf der Karte des Rumtreibers erscheint. Vielleicht haben die vier "Rumtreiber", die die Karte erstellt hatten, den Raum der Wünsche selbst nie entdeckt.
\subsubsection*{\large Unbenutztes Badezimmer}
%ABSATZ
Dieses Badezimmer (  Engl.  Disused seventh floor bathroom ) hat den Zugang durch ein Portrait im Lesezimmer, es wurde bis vor 1993 nicht benutzt. Fred und George Weasley verlegten ihr "Geschäft" vom unbenutztem Badezimmer im sechsten Stockwerk 1993 hierher.
\subsubsection*{\large Wahrsagen-Klassenraum}
%ABSATZ
Der Wahrsagen-Klassenraum (  Engl.  Divination Classroom ) befindet sich im Nordturm. Der Zugang ist über eine Wendeltreppe, die vom Wahrsagen-Korridor abgeht, durch eine runde Falltür. Es wurde beschrieben als eine Kreuzung zwischen einem Dachboden und einem altmodischen Teeladen.
%SEITE13
\subsubsection*{\large Lehrerin Wahrsagen-Büro}
%ABSATZ
In diesem Büro (  Engl.  Sybill Trelawney's office ) ist der Arbeitsplatz und der Wohnraum für den Professor in Fach Wahrsagen. Es ist erreichbar über eine kleine Wendeltreppe hinauf vom Wahrsagen-Klassenraum.
\subsubsection*{\large Korridor}
%ABSATZ
Zu diesem Korridor (  Engl.  Seventh-floor corridor ) gelangt man über die Treppe vom sechsten Stockwerk, der Korridor führt in die eine Roichtung zum Fette Dame-Korridor und zum Nordflügel und umrundet den Viereckhof. Eine Treppe führt zu den Kerkern, über die man auch in den Korridor im sechsten Stockwerk gelangen kann. Gegenüber dem Trollwandteppich befindet sich eine nackte Wand zwischen einem Fenster und einer menschengroßen Vase, dieser Wandabschnitt führt zum geheimen Raum der Wünsche.
\subsubsection*{\large Sonstige Örtlichkeiten im siebten Stockwerk}
\vspace{10pt}
%ABSATZ
\begin{itemize}
    \item Bohnenherausforderungsräume (Engl. Bean Callenge rooms) (Nicht-kanonischer Auftritt)
    \item Gefängniszelle an der Spitze des Dunklen Turms
    \item Gryffindorturm
    \item Haftfluchtweg (Engl. Detention Escape Route)
    \item Hogwarts-Dachkammer (Engl. Hogwarts attic)
    \item Klassenraum 7A (Arithmantik)
    \item Klassenraum 7B
    \item Klassenraum 7C
    \item Leseraum (Gryffindorturm)
    \item Penetrante Passage (Engl. Pungent Passage)
    \item Poltergeistpassage
    \item Porticus Olidus
    \item Ravenclawturm
    \item Rittersaal (Engl. Knight's Room)
    \item Runenkorridor
    \item Schlafraum Gryffindor-Jungen
    \item Schlafraum Gryffindor-Mädchen
    \item Treppe zum Mädchenschlafraum (Gryffindor)
    \item Vivarium
    \item Wahrsagen-Korridor
\end{itemize}

\subsection*{\Large Tiefparterre}
\subsubsection*{\large Bootshaus}
%ABSATZ
Das Bootshaus ist ein unterirdischer Hafen unter dem Schloss, wo die Boote aufbewahrt werden und wo sie andocken, wenn die Erstklässler an jedem 1. September Schloss Hogwarts erreichen. Bis 1995 hing ein Porträt von Percival Pratt an einem der Wände und verbarg eine Abkürzung zur großen Treppe. Es gibt einige Sparren in der Decke, die verwendet werden, um die kleinen Holzboote aufzubewahren. Möwen sind rund um den Bereich des Bootshauses zu sehen.
\vspace{10pt}
\newline
Nachdem Pratts Porträt aus dem Bootshaus genommen wurde, gab es nur noch drei Eingänge zu diesem Gebäude: Der Gang außerhalb des gepflasterten Viereckhofs, die Bootshaustreppe, die zum Viaduktvorhof und dem Empfangsraum führt, schließlich zum großen See. Harry Potter trifft Luna Lovegood hier, um zur Weihnachtsfeier des Slug-Clubs im Jahr 1996 zu gehen. Außerdem wurden Feuerwerkskörper rund um das Bootshaus gesetzt, um den Start der Party anzukündigen.
%SEITE14
\subsubsection*{\large Hufflepuff-Keller}
%ABSATZ
Im Hufflepuff-Keller (  Engl.  Hufflepuff Basement ) ist der Gemeinschaftsraum für Schülerinnen und Schüler von Hufflepuff. Der Eingang zu diesem Keller ist versteckt hinter einem Stapel Fässer. Eingang wurde gewährt, wenn ein bestimmter Rhythmus an den Fässern geklopft wird. De Eingang befindet sich in der Nähe der Hogwarts-Küche. Der Gemeinschaftsraum hat gelbe Wandbehänge und ist mit Armlehnsesseln voll gestellt. Unterirdische Tunnel führen in die Schlafräume, alle Türen sind kreisrund, wie die Fassdeckel.
\subsubsection*{\large Küchen}
%ABSATZ
Die Küchen von Hogwarts sind direkt unter der großen Halle in der Tiefparterre. Dorthin führt dieselbe Treppe wie zum Hufflepuff Gemeinschaftsraum. Tische sind in der Küche genauso angeordnet, wie die Tische in der großen Halle, die genau darüber sind. Geschirr, Besteck, Speisen und Getränke werden in den Küchentischen verteilt, sie erscheinen dann magisch auf den Tische ein Stockwerk höher vor den Essensteilnehmern. Über einhundert Hauselfen sind in der Küche beschäftigt. Auch Dobby (nach 1993), Winky (seit 1994) und Kreacher (seit 1996) sind in Hogwarts eingestellt.
\vspace{10pt}
\newline
Um Eingang in die Küche zu bekommen, muss im Gemälde eine Birne gekitzelt werden, die sich dann in einen Türknauf verändert.

\subsection*{\Large Keller und Kerker}
\subsubsection*{\large Büro des Zaubertrankmeisters#(Severus Snapes Büro)}
%ABSATZ
Das Büro (  Engl.  Potions Master's office ) wurde von den Zaubertrankmeistern im Schloss Hogwarts im 19. Jahrhundert von Aesop Sharp und von 1981 bis 1996 von Severus Snape benutzt. Das Büro ist in den Kellergewölben vom Schloss Hogwarts gelegen. Dort ist es düster und schwach beleuchtet. An den schattigen Wänden standen Regale mit großen, gläsernen Gefäßen, die mit schleimigen, ekelerregenden Sachen gefüllt waren. Es waren darin Teile von Pflanzen und Tieren, die in den Zaubertränken von verschiedenen Farben schwammen. Diese Sammlung von Professor Snape wuchs mit der Zeit - 1994 stellte Harry Potter mehr Gefäße als 1992 fest
\vspace{10pt}
\newline
Das Büro ist mit einem Kamin ausgestattet. In einer Ecke stand ein Schrank, in dem Professor Snape seinen persönlichen Bedarf an Zaubertränken und Zaubertrankzutaten aufbewahrte. Im Jahre 1997 stand im Büro auch ein Tisch, an dem Harry als Strafarbeit alte Akten sortieren sollte. Es ist nicht bekannt, ob dieser Tisch auch schon vorher im Büro gestanden hatte
\subsubsection*{\large Kammer des Schreckens}
%ABSATZ
„er stand am Ende einer langen, schwach beleuchteten Kammer, turmhohe Steinsäulen waren mit umschlungenen Schlangen verziert, die Säulen verloren sich oben in der Dunkelheit. Sie warfen lange, schwarze Schatten über die eigenartige, grünliche Düsternis, welches diesen Ort füllte.“
\vspace{10pt}
\newline
— Harry erkundet die Kammer
\vspace{10pt}
\newline
Die Kammer des Schreckens (  Engl.  Chamber of Secrets ) ist eine geheime Kammer "meilenweit" unter dem Hogwarts-Schloss, sogar unter dem großen See gelegen, die einst von Salazar Slytherin erbaut wurde. In den tausend Jahren konnte die Kammer von den gelehrtesten Zauberern und Hexen nie gefunden werden. Der Legende nach soll Salazar Slytherin die Kammer versiegelt haben, so dass keiner sie hätte öffnen können, bis sein wahrer Erbe zur Schule kommt. Im Schuljahr 1942/1943 wurde sie erstmals geöffnet, wobei eine muggelstämmige Schülerin durch ein dort drin lebendes Monster ums Leben kam. Das zweite Mal wurde sie erneut 1992 geöffnet, wobei mehrere Hogwartsbewohner durch ein in der Kammer lebendes Monster versteinert wurde. Am 2. Mai 1998 konnten Ronald Weasley und Hermine Granger erneut in die Kammer des Schreckens eindringen, dieses mal, um sich Giftzähne eines Basilisken zu entnehmen, mit denen können Horkruxe vernichtet werden.
\vspace{10pt}
\newline
Der Eingang befindet sich im Badezimmer der Maulenden Myrte an einem funktionslosen Waschbecken, nur durch eine winzige, kaum sichtbare Schlangenverzierung ein einem Wasserhahn erkennbar..
\subsubsection*{\large Kerker}
%ABSATZ
Die Kerker (  Engl.  Dungeons ) von Schloss Hogwarts befinden sich unter der Hogwarts-Schule für Hexerei und Zauberei und es ist dort kälter als im Hauptschloss. Der Zaubertränke-Klassenraum befindet sich in den Kerkern, ebenso der Eingang zum Slytherin-Gemeinschaftsraum und von dort deren Schlafräume.
%SEITE15
\subsubsection*{\large Kerkerkorridor}
%ABSATZ
Der Kerkerkorridor (  Engl.  Dungeon corridor ) ist ein langer, düsterer Steinkorridor, der von der schmalen Wendeltreppe erreicht wird, die vom Viadukteingang hinabführt, zu der Treppe hinunter zur glatten Steinmauer, welche den Eingang zum Slytherinkerker darstellt. Das Zaubertränke-Klassenraum, und möglicherweise Severus Snapes Büro, kann auf diesem Korridor gefunden werden. Die Slytherin-Korridore sind wahrscheinlich neben diesem Korridor.
\subsubsection*{\large Kerkerschrank}
%ABSATZ
Der Hogwarts-Keller (  Engl.  Dungeon Cupboard ) ist ein kleiner Schrank neben dem Kerkerkorridor, kurz vor der Kerkerhalle. Hier hatte Rubeus Hagrid die junge Acromantula Aragog in den 1940er Jahren versteckt.
\subsubsection*{\large Slytherin-Gemeinschaftsraum}
%ABSATZ
Der Gemeinschaftsraum der Slytherins (  Engl.  Slytherin Dungeon ) liegt hinter einer Wand der Kerker von Hogwarts. Eintritt ist nur möglich mit dem richtigen Kennwort an die Mauer gerichtet. Anschließend würde sich ein Gang zeigen, welcher zum Gemeinschaftsraum führt. Dieser hat eine niedrige Zimmerdecke, Inventar wie Sessel und die Lampen sind in grün gehalten. Im Raum stehen auch schwarze Ledersofas und ein Kamin. Der Kerkerraum erstreckt sich zum Teil unter den großen See, so dass die Slytherinschüler und Schülerinnen die Bewohner des Sees erblicken können.
\subsubsection*{\large Todestagsfeier-Halle}
%ABSATZ
Die Todestagsfeier-Halle (  Engl.  Deathday Party Hall )  ist einer der geräumigeren Kerker in Schloss Hogwarts. Der Raum ist erreichbar über die Kerkertreppe. Am 31. Oktober 1992 feierte in diesem Raum der Hausgeist von Gryffindor, Sir Nicholas de Mimsy-Porpington, seinen fünfhundertsten Todestag.
\subsubsection*{\large Untergrundkammern}
%ABSATZ
Die Untergrundkammern (  Engl.  Underground Chambers ) waren eine Reihe von Kellergewölben unterhalb des Schlosses Hogwarts. Sie dienten vom 1. August 1991 bis zum 4. Juni 1992 zum Schutz des Steins der Weisen. Es gab sieben Hindernisse, davon sind sechs in den Kellergewölben, jede enthielt ein magisches Hindernis oder eine Aufgabe, die den Stein beschützen sollten und einen Eindringling behindern oder am Weiterkommen hindern sollten.
\vspace{10pt}
\newline
Reihenfolge der sieben Räumlichkeiten mit ihren Schutzvorrichtungen
\vspace{10pt}
\newline
Dreiköpfiger Riesenhund: Der Korridor im dritten Stockwerk war im Schuljahr 1991-1992 in der rechten Seite vom Schloss gesperrt. Dort bewachte der Hund in einem verschlossenen Raum eine Falltür.
Teufelsschlinge fing die tief herabfallenden Personen auf, der Raum war dunkel und kühl, die Teufelsschlinge würgte jeden, der hineinfiel.
\vspace{10pt}
\newline
Untergrundkammer mit geflügelten Schlüsseln. Ein schräg abwärts führender Gang mit Steinmauern, an denen tropfendes Wasser zu hören war, führt zu einer Kammer, die beim Eintreten hell erleuchtet war. Diese Kammer hatte eine sehr hohe Decke. Der Raum war erfüllt von glitzernden, geflügelten Schlüsseln. An der anderen Seite der Kammer befand sich eine altmodische Holztür mit einem Schloss aus Silber, dieses ließ sich nicht am Türgriff oder mit einem Zauberspruch öffnen, der passende Schlüssel musste gefunden werden
\vspace{10pt}
\newline
Schachspielkerker: Diese Kammer war sehr dunkel, es war kaum etwas zu sehen. Es enthielt ein Schachspiel mit übermenschengroßen Schachfiguren. Der Raum konnte nur durchquert werden, indem man selbst als Schachfigur am Spiel teilnahm, mit dem Ziel, den gegnerischen Schachkönig entsprechend den Regeln des Zaubererschachs matt zu setzen.
\vspace{10pt}
\newline
Trollkerker: Diese Kammer beherbergte einen Bergtroll. Von diesem ging ein fauliger Geruch aus. Es hieß, es entstünde große Erleichterung, wenn man erst an ihm vorbei war.
Zaubertrankrätsel-Kerker: Sowie diese Kammer betreten wurde, wurde der Eingang durch Purpurfeuer versperrt, und der Ausgang auf der anderen Seite war von einem schwarzen Feuer unpassierbar. In der Raummitte stand ein Tisch mit sieben unbeschrifteten Flaschen von unterschiedlicher Größe und Form, die mit Flüssigkeiten gefüllt waren. Daneben lag ein Zettel mit einem Logikrätsel, dessen Lösung es einem ermöglichte, entweder zurück oder vor zu gehen. Drei Fläschchen enthielten Gift, zwei Nesselwein und zwei verschiedene Feuerschutztränke.
\vspace{10pt}
\newline
Untergrundkammer#(Nummer 7): In der letzten und tiefsten Kammer war der magische Spiegel NEHREGEB aufgestellt. Durch einen (nicht bekannten) Zauberspruch von Albus Dumbledore war der Stein der Weisen in magische Weise mit dem Spiegel verbunden und versteckt. Jemand, der den Stein finden wollte, ohne ihn auch benutzen zu wollen, konnte beim Anblick in den Spiegel sein Spiegelbild sehen, wie dieses den Stein besaß.
%SEITE16
\subsubsection*{\large Zaubertränke-Klassenraum}
%ABSATZ
Der Zaubertränke-Klassenraum war einer der großen Kerker in Schloss Hogwarts, benutzt für Zaubertrankunterricht. Er ist groß genug, dass zwanzig Schüler darin arbeiten können. Seine Wände sind mit Schränken und Regalen voll mit eingelegten Tieren in Gläsern gesäumt.] In einer Ecke des Raumes steht ein Becken, in welches sich eiskaltes Wasser aus dem Mund eines Wasserspeiers ergießt, während sich in einer anderen Ecke ein Aufbewahrungsschrank für Schülermaterialien befindet. Es gibt auch eine Tafel, auf welcher der Zaubertrankmeister die Unterrichtsanweisungen schreiben kann. In dem Raum war es sehr kalt, besonders im Winter, wenn die Schüler ihren eigenen Atem sehen konnten. Der Raum war üblicherweise quadratisch groß, mit großen Arbeitstischen und großen Fenstern. Ab dem Schuljahr 1995-1996 war er Raum eher oval geformt, und die Arbeitstische waren eher klein. Der Raum war 1996 vergrößert, so dass mehr Platz für Arbeitstische war.
\subsubsection*{\large Zaubertränkekeller}
%ABSATZ
Es befand sich ein Vorratskeller (  Engl.  Potions basement ) unter dem Zaubertränke-Klassenraum in Schloss Hogwarts. Er war über eine Falltür zugänglich und wurde verwendet, um Zaubertrankkessel, Zutaten und Phiolen unterzubringen.
\subsubsection*{\large Sonstige Örtlichkeiten in den Kerkern}
\vspace{10pt}
%ABSATZ
\begin{itemize}
    \item Eingangskerker
    \item Gang zum Schulhof
    \item Hebelräume
    \item Hieroglyphenhalle
    \item Kerker Fünf
    \item Kerkergrube
    \item Kerkerhalle
    \item Kerkerhöhle
    \item Kerkertreppe
    \item Porticus Circumscriptus
    \item Quidditch-Höhle
    \item Schneckengrube
    \item Zaubertrankmischraum
\end{itemize}

\subsection*{\Large Treppen}
%ABSATZ
Das Schloss hat einhundert-und-zweiundvierzig Treppen
\subsubsection*{\large Treppen zwischen Kerkern}
%ABSATZ
Kerkertreppe verbindet den Kerkerkorridor mit der Todestagsfeier-Halle.
\subsubsection*{\large Treppen von Kerkern aufwärts}
%ABSATZ
Eingangskerkertreppe führt vom Eingangskerker zum Viereckhof und befindet sich im Steinbrückenturm.
Quidditch-Treppe verläuft durch den Quidditch-Turm, oben gelangt man an der Mauer des Trainingsgeländes, unten ist das Quidditcheingangstor.
\vspace{10pt}
\newline
Slughorns Treppe ist eine enge Steintreppe, die die Eingangshalle mit dem Kerkerkorridor verbindet.
\vspace{10pt}
\newline
Große Slughorntreppe führt von den Kerkern zum Korridor im siebten Stock, es besteht auch Zugang zum Korridor im sechsten Stockwerk. Sie ist eine der längsten Treppen im Schloss.
%SEITE17
\subsubsection*{\large Treppen von Parterre aufwärts}
%ABSATZ
Kräuterkundetreppe: von Parterre des Gewächshausturms aufwärts.
\vspace{10pt}
\newline
Turris Magnus Treppe: Diese Treppe ist eine der längsten Treppen in Schloss Hogwarts. Sie befindet sich in Turris Magnus, beginnend im Parterre und heraufreichend bis zur Spitze des Turms.
\vspace{10pt}
\newline
Marmortreppe: von der Eingangshalle zum Erster Stock-Korridor.
\vspace{10pt}
\newline
Treppe zwischen Parterre und Korridor im zweiten Stock.
\vspace{10pt}
\newline
Treppe vom Hofkorridor zum Serpentinenkorridor,
\vspace{10pt}
\newline
Treppe zwischen den Korridoren im ersten und dem zweiten Stockwerk, sie beginnt in der Nähe des Zentralturm-Hofs und führt zum Klassenraum 2E#(Zauberkunst) und zur Hogwarts-Bibliothek.
\vspace{10pt}
\newline
Treppe zwischen den Korridoren in Parterre und dem zweiten Stockwerk#(Zauberkunsttreppe)
Hängebrücketreppe verbindet den Parterre-Korridor mit dem Korridor im dritten Stockwerk.
\subsubsection*{\large Treppen vom ersten Stockwerk aufwärts}
%ABSATZ
Treppe zwischen den Korridoren im ersten und dem zweiten Stockwerk.
\vspace{10pt}
\newline
Treppe zwischen dem Korridor im ersten Stockwerk und dem Serpentinenkorridor, unten befindet sich Minerva McGonagalls Büro.
\vspace{10pt}
\newline
Treppe zwischen Bibliotheks- und zweiten Stock-Korridor.
\subsubsection*{\large Treppen vom zweiten Stockwerk aufwärts}
%ABSATZ
Treppe zwischen der Arkade im zweiten Stockwerk und dem dritten Stockwerk
\subsubsection*{\large Treppen vom dritten Stockwerk aufwärts}
%ABSATZ
Uhrenturmtreppe: vom Uhrenturmeingang zum Korridor im Krankenflügelturm (fünftes Stockwerk), Zugang besteht auch zum Korridor im Krankenflügelturm (viertes Stockwerk).
\subsubsection*{\large Treppen vom vierten Stockwerk aufwärts}
%ABSATZ
Krankenflügelturmtreppe verbindet den Korridor im vierten Stockwerk mit dem Korridor im Krankenflügelturm (viertes Stockwerk).
Geheimtreppe vom vierten zum siebten Stockwerk vo Korridor im vierten Stockwerk zum Fette Dame-Korridor.
\subsubsection*{\large Treppen vom sechsten Stockwerk aufwärts}
%ABSATZ
Treppe zwischen dem sechten Stockwerk und dem Korridor im siebten Stockwerk
\subsubsection*{\large Treppen vom siebten Stockwerk aufwärts}
%ABSATZ
Wahrsagen-Wendeltreppe verbindet den Wahrsagen-Korridor zum Treppenabsatz zum Wahrsagen-Klassenraum.
Griffin-Wendeltreppe führt vom siebten Stockwerk zum Schulleiterbüro.
Grumpytreppe führt vom Korridor im siebten Stock in der Nähe vom Raum der Wünsche in ein Türmchen.
\subsubsection*{\large Treppen von unbekanntem Stockwerk aufwärts}
%ABSATZ
Astronomietreppe veläuft im Astronomieturm zum Astronomie-Klassenraum an der Turmspitze.
Ravenclawturmtreppe führt zum Gemeinschaftsraum. Sie könnte im fünften stockwerk beginnen.
Treppe zum Westturmwehrgang, eine enge, steinige Wendeltreppe.
Wirbeltreppe, die sind ähnlich wie die Große Treppe, die vermutlich ihre Richtung ändern kann. Sie ist irgendwo im nicht bekannten Bereich im Schloss gelegen.
%SEITE18

\subsection*{\Large Türme}
\subsubsection*{\large Astronomieturm}
%ABSATZ
Der Astronomieturm (  Engl.  Astronomy Tower ) ist der höchste Turm am Schloss. Oben ist er von einer Brüstung umgeben dort befindet sich ein Türmchen. Auf dieser Plattform wurde während des Astronomieunterrichts bei Professor Aurora Sinistra die Planeten und Sterne mit den persönlichen Teleskopen beobachtet. Das fand meistens gegen Mitternacht statt, wenn die Sterne am besten zu sehen sind. Der Turm wurde nur für die Unterrichtsstunden betreten. Das riesige Fenster des Turms kann vom Bootshaus aus gesehen werden.
\vspace{10pt}
\newline
Innerhalb des Turms liegt der Astronomiekorridor, der Leseraum, der [[Astronomie-Klassenraum]], die Treppe, die Astronomie-Abteilung und der Astronomieraum. Am 30. Juni 1997 wurde der Schulleiter Albus Dumbledore durch Severus Snape mit dem Todesfluch getötet. Allerdings wurde dieser Tod von den beiden Monate vorher vereinbart, um Draco Malfoy vor einem Mord zu verschonen und um den möglichen qualvollen Tod, den Dumbledore durch den fluchbelegten Horkrux, den Ring von Lord Voldemort, zu erwarten hatte.
\subsubsection*{\large Dunkler Turm}
%ABSATZ
Der dunkle Turm (  Engl.  Dark Tower ) ist ein schmaler, hoher Turm in der der Nähe des Zentralturm-Hof#(Verwandlungshofes). Der dunkle Turm wird als Gefängnis benutzt, er besitzt einige Gefängniszellen, die oberste Zelle befindet sich am Dach. Die Zellen sind nicht gegen Einbruch von außen durch Magie oder sonstwie geschützt. Der Turm hat sieben Stockwerke. An den Turmseiten sind vier Wasserspeier angebracht, die das Regenwasser vom Wehrgang weg von den Turmmauern ableiten.
\subsubsection*{\large Glockentürme}
%ABSATZ
Die Glockentürme (  Engl.  Bell Towers ) sind zwei quadratische Türme, die über dem Haupteingang der Gewächshäuser vom Schloss Hogwarts aufragen. Zwischen den beiden Türmen befinden sich hölzerne Doppeltüren, die in die Glockenturm-Eingangshalle entlang des Parterre-Korridors führen.
\subsubsection*{\large Gryffindorturm}
%ABSATZ
Der Gryffindorturm (  Engl.  Gryffindor Tower ) besteht aus dem Gemeinschaftsraum und den Schlafräumen für Jungen und Mädchen vom Hogwartshaus Gryffindor.
\vspace{10pt}
\newline
Im Allgemeinen sind die Räumlichkeiten in veschiedenen Rot- und Goldfarbtönen gehalten. Über dem Kaminsims hängt ein Gemälde eines Löwen (das Wappentier des Hauses). Der Gryffindorturm zählt zu den drei höchsten Türmen im Schloss Hogwarts neben dem Ravenclawturm und dem Astronomieturm.
\vspace{10pt}
\newline
Der Eingang zum Gemeinschaftsraum befindet sich hinter dem Portrait der fetten Dame beim großzügigen Teppenabsatz. Für den Eintritt wird nach Aufforderung der "Fetten Dame" das richtige Kennwort für die jeweilige Woche benötigt. Wer falsch antwortete, musste solange davor abwarten, bis jemand kam, der das Kennwort kannte.
\subsubsection*{\large Hauptreppenturm}
%ABSATZ
Der Haupttreppenturm (  Engl.  Marble Staircase Tower ) ist ein sehr hervorstechender Turm, der die Skyline in der Nähe der großen Halle und dem Viereckhof sichtbar prägt. Der Turm ist ausgesprochen mächtig, er reicht von den Kerkern bis zum siebten Stockwerk, und vielen Räumen in den Dachgeschossetagen, zu erkennen an den vielen Dachgeschossfenstern (Gaubenfenstern) im spitz zulaufendem Dach.
\subsubsection*{\large Außenansicht}
%ABSATZ
Der Turm ist rund mit regelmäßig etagenmäßg angeordneten Fenstern. Einige Fenster verlaufen schräg versetzt, vermutlich im Verlauf einer an der Außenwand entlang führenden Treppe. Der oberste Teil des Turms unter dem Dach ist vorstehend über dem Mauerwerk darunter mit vielen Mauerausparungen. Hier befinden sich nur wenige Fensteröffnungen, einreihig unterhalb des Dachvorsprungs. Das Dach ist ebenfalls über dem Mauerwerk hervorstehend. Das Dach hat vermutlich zwölf Flächen, oben zur Turmspitze spitz zulaufend, auf der Spitze eine Verzierung, möglicherweise in Form eines keltischen Kreuzes, vermutlich aus grün angelaufenem Kupfer.
\vspace{10pt}
\newline
An jeder zweiten Dachfläche befinden sich jeweils sieben Fenstergauben übereinander, hinter denen vermutlich unterschiedliche Räume vorhanden sind. Am Dach sind drei kleine Türmchen.
%SEITE19
\subsubsection*{\large Innen}
%ABSATZ
Den Hauptraum beansprucht die Große Treppe in einer rechteckigen Mitte des Turms. Zur großen Treppe führt die Marmortreppe, die von der Eingangshalle beginnt. Diese große Treppe reicht von den Kerker-Räumen bis zum siebten Stockwerk über mindestens acht Stockwerke. Aus den großen Fenstern kann der Viereckhof überblickt werden.
\subsubsection*{\large Krankenflügelturm}
%ABSATZ
Der Krankenflügelturm (  Engl.  Hospital tower ) hat eine halbovale Form. Der Krankenflügelturm liegt an einer Seite des Viereckhofs und steht am Ende des Korridors, der den Uhrenturm mit dem Rest des Schlosses verbindet. Im ersten Stockwerk befindet sich der Krankenflügel und das Büro der Krankenflügelleitung.Hinter einem Bücherschrank in diesem Büro befindet sich ein Geheimgang.
\vspace{10pt}
\newline
Eine Treppe führt zum Korridor im vierten Stockwerk und mit einer Verlängerung zur großen Treppe. Über dem Krankenflügel befindet sich im fünften Stockwerk das Vertrauensschülerbad
\subsubsection*{\large Nordturm}
%ABSATZ
Der Nordturm (  Engl.  North Tower ) ist auch bekant als der Wahrsagen-Turm, er ist durch den Wahrsagen-Korridor mit dem Rest des Schlosses verbunden. In diesem Turm ist der Wahrsagen-Klassenraum und das Büro der Wahragen-Lehrerin, die über eine Wendeltreppe erreicht werden. Das Portrait von Sir Cadogan hängt zu Beginn des Wahrsagen-Korridors.
\vspace{10pt}
\newline
Im Klassenraum 104 im Nordturm wurde Xylomantie (als Teilgebiet vom Wahrsagen (Fach)) unterrichtet, sowie Verteidigung gegen die dunklen Künste für die Erstklässler.
\subsubsection*{\large Ravenclawturm}
%ABSATZ
Der Ravenclawturm (  Engl.  Ravenclaw Tower ) beinhalted den Ravenclaw-Gemeinschaftsraum der Schüler vom Hogwartshaus Ravenclaw. Die meisten Räume sind in den Hausfarben blau und bronze. Der Gemeinschaftsraum ist der luftigste von allen vier Häusern.
\vspace{10pt}
\newline
Der Ravenclawturm zählt zu den drei höchsten Türmen am Schloss neben dem Gryffindorturm und dem Astronomieturm.
\subsubsection*{\large Schulleiterturm}
%ABSATZ
Im Schulleiterturm (  Engl.  Headmaster's Tower ) befindet sich das Schulleiterbüro und seine Unterkunft im Schloss Hogwarts. Diese Dreiertürmchen befinden sich am Dach des Westturms.
\vspace{10pt}
\newline
Das größte der drei kleinen Türmchen entspringt dem Spitzdach des Westturms. Das zweite Türmchen ist außen am ersten Türmchen etwas nach oben versetzt angebracht, an diesem wiederum ist außen das dritte Türmchen, ebenfalls außen, und leicht nach oben versetzt
\subsubsection*{\large Steinbrückenturm}
%ABSATZ
Der Steinbrückenturm (auch: Aussichtsturm) (  Engl.  Lookout Tower ) überschaut die Steinbrücke, der Turm erinnert stark an den Astronomieturm. Das erste Stockwerk im Turm verbindet über den Korridor im ersten Stock mit der Steinbrücke.
\vspace{10pt}
\newline
Am Viereckhof stehen auch der Haupttreppenturm (  Engl.  Marble Staircase Tower ), der Gryffindorturm und der Nordturm. Die Steinbrücke verbindet den Steinbrückenturm zum Viadukteingang im ersten Stockwerk. Es gibt auch eine Brückenverbindung zum Oktagonturm.
\vspace{10pt}
\newline
Eine Treppe führt vom Erdgeschoss dieses Turms zur Kerkerebene des Turms. Die Treppe wird über den Viereckhof erreicht und dient als Zugang zum Eingangskerker.
\subsubsection*{\large Turris Magnus}
%ABSATZ
Hogwarts-Turris Magnus (  Engl.  Hogwarts Turris Magnus ) (auch:Verteidigungsturm) ist einer der höchsten Türme von Schloss Hogwarts. Er befindet sich neben dem Viadukteingang, er überschaut das Trainingsgelände und die Hängebrücke. Dieser Turm hat eine halbkreisförmige Form.
\vspace{10pt}
\newline
Der Turm beherbergt Minerva McGonagalls Büro im ersten Stock, den Korridor im dritten Stockwerk und den Serpentinenkorridor, den Klassenraum 3C#(Verteidigung g.d.D.K) und Klassenraum 7A#(Arithmantik), Verlorene Zauberstäbe-Lager und dem Klassenraum für Ghulstudien im dritten Stock. Ein Treppenhaus verläuft durch die Mitte des Turms.
%SEITE20
\subsubsection*{\large Uhrenturm}
%ABSATZ
Der Uhrenturm (  Engl.  Clock Tower ) ist eines der älteren Gebäude vom Schloss Hogwarts. Im Turm ist eine antike Uhr mit verglastem Ziffernblatt. Das Uhrwerk ist im Turminneren. An der Turmbasis ist der Uhrenturmhof
\vspace{10pt}
\newline
Der Turm beginnt auf der Höhe des dritten Stockwerks vom Schloss, da es auf einem Hügel erbaut wurde. Der Eingang besteht aus einer großen Halle mit hölzernen Treppen rechts und links. Das Uhrenpendel des Uhrenturms kann im dritten Stockwerk langsam pendelnd gesehen werden. Der erste Treppenabsatz befindet sich im vierten Stockwerk, dort befindet sich sich das Ziffernblatt. Auf dieser Etage führt ein langer Korridor zum Eingang in den Krankenflügel. Der gleiche Korridor verbindet den Uhrenturm mit dem Rest des Schlosses.
\vspace{10pt}
\newline
Die Zweite Turmetage wird im fünften Stockwerk des Schlosses wiederum mit Holztreppen erreicht. Hier sind die Uhrenglocken und die Pendelfeder angebracht. Die vier goldenen und kupfernen Glocken Läuten zur vollen Stunde. Auch auf dieser Etage führt ein Korridor zum fünften Stockwerk des Schlosses.
\subsubsection*{\large Westturm}
%ABSATZ
Im obersten Stockwerk des Westturms (oder Eulereiturms) liegt die Eulerei. Im Winter wird es im Turm so kalt, dass sich in den oberen Stockwerken Eis bildet. Eine Wendeltreppe führt bis in die Spitze des Turmes hinauf.
\subsubsection*{\large Alle bekannten Türme}
\vspace{10pt}
%ABSATZ
\begin{itemize}
    \item Astronomieturm
    \item Bootshaus-Glockenturm (Bell Tower (Boathouse))
    \item Dunkler Turm (Dark Tower)
    \item Fakultätenturm (Faculty Tower)
    \item Glockentürme (Bell towers)
    \item Gründerturm (Founders' Tower)
    \item Gryffindorturm
    \item Hängebrückentürme (Suspension Bridge towers)
    \item Haupttreppenturm (Große Treppe) (Marble Staircase Tower)
    \item Hufflepuff-Duellierturm (Hufflepuff Duelling Turrets)
    \item Kleiner verschlossener Turm (Small Locked Tower)
    \item Kräuterkundeturm (Herbology Tower)
    \item Krankenflügelturm (Hospital tower)
    \item Mädchentoilettenturm (First floor girls' lavatory tower)
    \item Nordturm (North Tower)
    \item Nordwehrgang (North battlements)
    \item Obere-Hallen-Turm (Upper Hall tower)
    \item Oktagonturm (Octagon tower)
    \item Pfefferpott (Pepperpot)
    \item Quidditch-Turm (Quidditch tower)
    \item Quidditcheingangstor (Quidditch gate)
    \item Ravenclawturm (Ravenclaw Tower)
    \item Rechter Viaduktvorhofturm (Right Long Gallery tower)
    \item Schulleiterturm (Headmaster's Tower)
%SEITE21
    \item Steinbrückenturm (Lookout Tower)
    \item Südturm (South Tower)
    \item Trainingsgeländeturm (Training Grounds tower)
    \item Turris Magnus
    \item Turris Medius (Hogwarts Turris Medius)
    \item Viaduktvorhof-Torhaus (Viaduct Courtyard Gatehouses)
    \item Viereckhof-Wehrgang (The Quad battlements)
    \item Westturm (West Tower)
    \item Westturmwehrgang (West Tower battlements)
    \item Zentralturm (Central Tower)
\end{itemize}
\subsection*{\Large Hogwartsgelände}
\subsubsection*{\large Gewächshäuser und Gärten}
%ABSATZ
Es gab mindestens drei Gewächshäser, in denen Unterricht von der Lehrerin in Kräuterkunde erteilt wurde, der Zutritt zu Gewächshaus drei war für Hogwarts-Schüler verboten, es sei denn, die Kräuterkunde-Professorin hatte die entsprechende Aufsicht und Anleitung. In diesem Gewächshaus wuchsen gefährliche Pflanzen wie Venomosa Tentacula, Alraunen oder Teufelsschlinge.
\subsubsection*{\large Glockenturmhof}
%ABSATZ
Der Glockenturmhof ist ein Säulengang am Fuße der Glockentürme. Der scheint ein älterer Teil des Schlosses zu sein, denn die Bedachung und die Wände sind verfallen. Der Hof führt auch zur Hängebrücke#(Holzbrücke). Die Türme sind auf einem Hügel erbaut, insofern befindet sich der Hof in der Höhe des dritten Stockwerks. In der Hofmitte steht ein antiker Springbrunnen umgeben von Adlerstatuen. Im Hof wächst auch ein Birnenbaum.

\subsubsection*{\large Der große See}
%ABSATZ
Der große See, (  Engl.  Great Lake ) auch als der schwarze See (  Engl.  Black Lake ) bekannt, ist ein Süßwassersee im Süden vom Schloss Hogwarts. Das Schloss steht auf einer Klippe am Wasser. Der See ist etwa achthundert Meter im Durchmesser (  Engl.  half a mile ). Die Abwässer vom Schloss Hogwarts werden in den See gleitet. Der See wird von ungewöhnlichen Kreaturen bewohnt, einschließlich eine Siedlung von Wassermenschen, vielen Grindelohs und der Hogwarts-Riesenkrake.
\vspace{10pt}
\newline
Jedes Jahr zum 1. September brachte der Hogwarts-Express die Schüler zum neuen Schuljahr nach Hogsmeade, dort wurden die Neulinge vom Hüter der Schlüssel und Ländereien, Rubeus Hagrid (oder eine Vertretung) vom Bahnhof abgeholt. Hagrid führte die Erstklässler zum Seeufer zu den Hogwarts-Booten. Der Halbriese Hagrid besetzte ein Boot, die restliche Bootsflotte wurde je Boot mit drei bis vier Schülern besetzt. Mit einem Ruderzauber setzte sich die Bootsflotte in Bewegung und brachte die Schülerschar über den See zum Schloss Hogwarts.
\subsubsection*{\large Holzbrücke}
%ABSATZ
Die Holzbrücke (  Engl.  Wooden Bridge ) erschien baufällig, sie wurde auf hölzernen Pfeilern gestützt, die bis zur Schlucht, die sie überbrückte, reichten. Die Brücke war überdacht, welches durchhing und vermutlich mit einer Teerdachpappe bedeckt war. Die Brücke passte sich gut in die natürliche Landschaft ein.
\vspace{10pt}
\newline
Die Brücke verlief von der Basis am Uhrenturmhof zu einem kleinen, steinernen Aussichtspavillon mit Sitzbänken mit Blick auf den Steinkreis(Sonnenuhrgarten). Die Brücke sieht instabil aus, es ist daher anzunehmen, dass sie durch Magie sicher gezaubert wurde.

\subsubsection*{\large Haupteingangstor}
%ABSATZ
Zum Haupteingangstor (  Engl.  Entrance Gate ) vom Hogwartsgelände führt der kürzeste Weg vom Bahnhof von Hogsmeade. Durch dieses Tor kommen Besucher, sowie jeweils zu Schulbeginn am 1. September die Hogwarts-Schüler ab dem zweiten Schuljahr mit dutzenden Kutschen, die von Thestralen gezogen werden. Die zwei gewaltigen Flügeltore sind aus massiven Schmiedeeisen, an ihre Seite stehen auf zwei Säulen geflügelte Keilerstatuen. Wenige Meter vom Tor ist ein kleines Torhaus mit einem hohen Schornstein.
%SEITE22
\subsubsection*{\large Hütte des Wildhüters}
%ABSATZ
Die Hütte des Wildhüters (besser bekannt als Hagrids Hütte) (  Engl.  Hagrid's Hut ) befindet sich am Rande des verbotenen Waldes. Seit Jahrzehnten lebt hier der Hüter der Schlüssel und Ländereien, Rubeus Hagrid, der auch als Wildhüter eingesetzt ist. Seit dem Schuljahr 1994-1995 ist er auch Professor in Pflege Magischer Geschöpfe an der Hogwarts-Schule für Hexerei und Zauberei.
\vspace{10pt}
\newline
Die Hütte ist klein und aus Holz gebaut. Sie besteht aus einer Wohn-Schlaf-Kücheneinheit. Im Schuljahr 1991-1992 hingen an der Decke Schinken und Fasane. Über einer Feuerstelle hing meistens ein Kupferkessel, in dem Wasser zum Kochen aufgestzt wird. In einer Ecke steht ein massives großes Bett, bedeckt mit einer Patchworkdecke. Manchmal, wenn Hagrid seine persönliche Habe nicht bei sich hatte, wie seinen rosafarbenen Regenschirm (in dem sein Zauberstab versteckt ist), seinen Maulwurfsfellmantel oder seine Armbrust, wurden diese in der Hütte aufbewahrt. Weitere Artikel, die dort zu finden sind, sind Nahrungsmittelzutaten und Gerätschaften für die PFlege magischer Geschöpfe, und seit dem Schuljahr 1994-1995 auch für den Unterricht in Pflege Magischer Geschöpfe.
\subsubsection*{\large Quidditchfeld}
%ABSATZ
Hogwarts hat ihr eigenes Quidditchfeld, dort übt und trainiert jedes Hogwartshaus mit seiner Quidditch-Mannschaft und sie führen Wettkämpfe gegeneinander aus. Jedes Jahr werden sechs Spiele zwischen den Häusern ausgetragen, jedes Haus spielt gegen die anderen drei. Dazu kommen unzählige Trainingstunden für jede Mannschaft.
\vspace{10pt}
\newline
Hohe Gestelle um das Feld werden zu jedem Quidditch-Wettkampf unterschiedlich geschmückt. Jedes zweite Gestell wird in den Farben der einen Mannschaft verziert, und jede andere mit dem Farben der gegnerischen Mannschaft. Ganz oben in den Feldgestellen, auf Höhe der Torringe, befinden sich Tribünenartige Sitzreihen, die auch schräg überdacht sind. Vemutlich nehmen hier eingeladene Gäste und die Hogwartsmitarbeiter und Professoren Platz. Zwischen den hohen Gestellen sind die regulären Tribünen für die Hogwarts-Schüler.
\vspace{10pt}
\newline
An jeder Seite des Spielfeldes stehen jeweils drei hohe Torringe auf vergoldeten Pfählen.
\subsubsection*{\large Quidditcheingangstor}
%ABSATZ
Das Quidditcheingangstor (  Engl.  Quidditch gate ) ist der Eingang zum Quidditchfeld. Das Eingangstor wird von zwei turmartigen steinernen Gebäuden flankiert, die den Zuschauertürmen entsprechen, welche das Quidditchfeld umgeben. Innerhalb des Eingangstores sind etliche Trophäen und Ehrenschilde ausgestellt.
\subsubsection*{\large Trainingsgelände}
%ABSATZ
Im Trainingsgelände (  Engl.  Training grounds ) lernten die Erstklässler bei Madam Hooch Flugbesen zu benutzen. Das Gelände ist flach, der Rasen ist kurz gemäht. Das Gelände ist in der Nähe der Gewächshäuser. Die Glockentürme, der Turris Magnus, der Trainingsgeländeturm und das Zwischengebäude begrenzen das Trainingsgelände.
\subsubsection*{\large Der verbotene Wald}
%ABSATZ
Der verbotene Wald (  Engl.  Forbidden Forest ) wird von der Hogwarts-Schule für Hexerei und Zauberei so genannt, da es wegen der gefährlichen Kreaturen im Wald für Hogwarts-Schüler verboten ist, dort hineinzugehen. Ausnahmen waren nur Strafaufgaben unter Aufsicht des Wildhüters oder bei gelegentlichen Unterrichtsstunden in Pflege Magischer Geschöpfe, die dann im Wald stattfinden. Der Wald ist sonst auch bekannt als der dunkle oder der schwarze Wald. Er liegt an der Grenze des Schulgeländes der Hogwarts-Schule für Hexerei und Zauberei.
\vspace{10pt}
\newline
Im Wald leben dauerhaft oder zeitweilig magische, teilweise sehr gefährliche Kreaturen, dazu zählen Acromantulas Einhörner, Hippogreife, Thestrale und Zentauren. Zeitweilig auch Feuerkrabben, und zum Trimagischen Turnier wurden vier verschiedenartige Drachen importiert und im Wald versteckt gehalten.
\subsubsection*{\large Weißes Grabmal}
%ABSATZ
Das weiße Grabmal (  Engl.  White Tomb ) ist das einzige Grab auf dem Hogwartsgelände. Es liegt am Ufer des großen Sees, dort hat Albus Dumbledore seine letzte Ruhestätte. Er opferte freiwillig sein Leben im Dienste für den Orden des Phönix im Kampf gegen die Todesser von Lord Voldemort im Verlauf des zweiten Zaubererkrieges.
%SEITE23
\subsubsection*{\large Sonstige Örtlichkeiten im Hogwartsgelände}
\vspace{10pt}
%ABSATZ
\begin{itemize}
    \item Beet mit gefährlichem Gemüse (  Engl.  Dangerous vegetables patch )
    \item Besenschuppen (Broomshed)
    \item Bootshaus (Boathouse)
    \item Bootshaus-Glockenturm
    \item Bootshaustreppe (Boathouse steps)
    \item Bowtruckle-Insel (Bowtruckle Island)
    \item Buchenbaum am großen See (Beech tree on the Edge of the Black Lake)
    \item Eingangs-Torhaus (Entrance gatehouse)
    \item Eulerei (Owlery)
    \item Fußweg zum Eingang (Entrance footpath)
    \item Geheimgarten (Secret Garden)
    \item Geheimraum beim Trainingsgelände (Secret room near the Training Grounds)
    \item Hagrids Garten (Hagrid's Garden)
    \item Hagrids Kürbisbeet (Hogwarts gamekeeper's pumpkin patch)
    \item Hagrids Schuppen (Hagrid's Hut's shed)
    \item Hogwarts-Peitschende Weide (Hogwarts Whomping Willow)
    \item Horklumpfeld (Horklump Patch)
    \item Irrgarten zum Trimagischen Turnier (Triwizard Maze)
    \item Knallschotenbeet (Puffapod patch)
    \item Koppel (The Paddock)
    \item Kreaturenklasssenraum (Beast Classroom)
    \item Knuddelmuffgelände (Hogwarts Puffskein patch)
    \item Wassermenschensiedlung (Merpeople village)
    \item Pfad zur Eulerei (Path to Owlery)
    \item Pfad zur Peitschenden Weide (Path to the Whomping Willow)
    \item Pomona Sprouts Garten (Pomona Sprout's garden)
    \item Portraitnische (Grounds Portrait Room)
    \item Quidditcheingangstor (Quidditch Gate)
    \item Quidditch-Höhle (Quidditch cave)
    \item Quidditch-Trainingsfeld (Quidditch Training Pitch)
    \item Quidditch-Umkleideraum (Quidditch Changing Rooms)
    \item Quidditch-Viadukt (Hogwarts Quidditch pitch viaduct)
    \item Ravenclaw Duellarena (Ravenclaw Duelling Arena)
    \item Schlucht (Ravine)
    \item Steinstatuen-Tor (Gargoyle Gate)
    \item Steinkreis (Stone Circle)
%SEITE24
    \item Steinkreis-Torhaus (Gatehouse)
    \item Straße nach Hogsmeade (Road to Hogsmeade)
    \item Straße nach Hogwarts (Road to Hogwarts)
    \item Trainingsgelände (Training Grounds)
    \item Torhaus am Bootshaus (Hogwarts Gatehouse)
    \item Vergessene Wiese (Forgotten Grounds)
    \item Versunkenes Verlies (Sunken Vault)
    \item Viadukteingang (Entrance to the Viaduct)
    \item Wasserfall beim Hogwarts-Schloss (Waterfall by Hogwarts Castle)
    \item Weg vom Trainingsgelände zum Quidditchfeld (Training Grounds way to Quidditch Pitch)
    \item Wichtelbrunnen (Pixie water well)
    \item Wiese beim verbotenen Wald (Lawn by the Forbidden Forest)
\end{itemize}

\subsubsection*{\Large Unbekannte Örtlichkeiten}
\subsubsection*{\large Reparaturwerkstatt für Monsterbücher der Monster}
%ABSATZ
Diese Werkstatt wird in der Karte des Rumtreibers gezeigt. Hier wurde vermutlich beschädigte Exemplare der Monsterbücher der Monster repariert.
\subsubsection*{\large Kesselschrank}
%ABSATZ
Der Kesselschrank (  Engl.  Cauldron Cupboard ) ist ein Raum im Schloss Hogwarts. Vermutlich werden hier Zaubertrankkessel für das Unterrichtsfach Zaubertränke aufbewahrt. Sollte es so sein, gehört der Raum vermutlich dem Zaubertrankmeister.
\subsubsection*{\large Stinkbombenladen}
%ABSATZ
Der Stinkbombenladen (  Engl.  Stink Bomb Store ) ist ein Raum der auf der Karte des Rumtreibers aufgezeichnet ist. Es ist vermutlich ein nicht-genehmigter Laden, denn wenn ein Professor hier eintrat, rannten die Hogwarts-Schüler aus dem Raum. In diesem Raum wurde vermutlich Stinkbomben und Mistbomben verkauft.

\subsection*{\large Andere Örtlichkeiten unbekannter Lage}
\vspace{10pt}
%ABSATZ
\begin{itemize}
    \item Apothekarium-Abteilung (Apothecary department)
    \item Ædificium Oriens
    \item Geheimgang zu den Küchen (Undercover route to the Kitchens)
    \item Grindeloh-Lagune (The Grindelow Lagune)
    \item Horace Slughorn erstes Büro (Horace Slughorn's first office)
    \item Klassenraum Pflege magischer Geschöpfe (Care of Magical Creatures classroom)
    \item Lord Voldemorts letzte Hogwarts-Ruhestätte (Resting Chamber of Lord Voldemort)
    \item Nordostkorridor der unteren Kammern (Lower Chambers Corridor North-East)
    \item Poltergeistpassage (Poltergeist Passage) => 7. Stock
    \item Porticus Circumscriptus
    \item Porticus Imago
    \item Porticus Medius
%SEITE25
    \item Porticus Periculum
    \item Portraitkorridor der unteren Kammern (Lower Chamers Portrait Corridor)
    \item Prüfungsraum der Erstklässler (First year written exams classroom)
    \item Raum der Runen (Room of Runes)
    \item Raum der Verdammung (Room of Doom)
    \item Rolanda Hoochs Büro (Rolanda Hooch's office)
    \item Shop
    \item Streunerpassage (Prowling Passage)
    \item Studierhalle (Study Hall)
    \item Wäscherei (Hogwarts laundry)
    \item Vorraum des Verderbens (Vestibule of Mischief)
    \item Westkorridor der unteren Kammern (Lower Chambers Corridor West)
\end{itemize}

\subsection*{\Large Auftritte}
\vspace{10pt}
%ABSATZ
\begin{itemize}
    \item Harry Potter (Buchreihe)
    \item Harry Potter (Filmreihe)
    \item Harry Potter-Videospiele
    \item Harry Potter und das verwunschene Kind
    \item Harry Potter und das verwunschene Kind (Theaterstück)
    \item Phantastische Tierwesen und wo sie zu finden sind (Nur erwähnt)
    \item Phantastische Tierwesen und wo sie zu finden sind: Das Originaldrehbuch (Nur erwähnt)
    \item Phantastische Tierwesen: Grindelwalds Verbrechen (Film)
    \item Phantastische Tierwesen: Grindelwalds Verbrechen (Das Originaldrehbuch)
    \item Phantastische Tierwesen: Dumbledores Geheimnisse (Film)
    \item Phantastische Tierwesen: Dumbledores Geheimnisse - Das Originaldrehbuch
    \item Die Märchen von Beedle dem Barden (Nur erwähnt)
    \item Pottermore
    \item Wizarding World
    \item The Road to Hogwarts Sweepstakes
    \item Harry Potters offizielle Webseite
    \item Harry Potter: Ein Pop-Up Buch
    \item LEGO Harry Potter: Erschaffung einer Magischen Welt
    \item LEGO Harry Potter
    \item LEGO Creator: Harry Potter
    \item Creator: Harry Potter und die Kammer des Schreckens
    \item LEGO Dimensions
    \item LEGO Harry Potter: Jahre 1 - 4
%SEITE26
    \item LEGO Harry Potter: Jahre 5 - 7
    \item Harry Potter: Quidditch-Weltmeisterschaft
    \item Harry Potter für Kinect
    \item Wonderbook: Das Buch der Zaubersprüche
    \item Wonderbook: Das Buch der Zaubertränke
    \item Harry Potter Sammelkartenspiel
    \item Harry Potter: Die Welt der magischen Wesen
    \item Harry Potter: Die Welt der magischen Figuren
    \item The Wizarding World of Harry Potter
    \item Harry Potter: Hogwarts Mystery
    \item Harry Potter: Wizards Unite
    \item Harry Potter: Die Magie erwacht
    \item Harry Potter: Rätsel \& Zauber
    \item Hogwarts Legacy
%SEITE27
\end{itemize}

\end{document}



