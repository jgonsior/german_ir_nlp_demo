\documentclass[a4paper, 10pt]{article}
\usepackage[T1]{fontenc}
\usepackage[sfdefault]{AlegreyaSans} %% Option 'black' gives heavier bold face
\usepackage[a4paper, left=2cm, right=2cm, bottom=2cm, top=2cm, marginparsep=-7in]{geometry} % Adjust left and right margins
\usepackage{amsmath} % for \boxed

\marginpar{\raggedleft\small}


% Set the width of the lines around the box
\setlength{\fboxrule}{2pt}

\begin{document}

\begin{minipage}[t]{\textwidth}
    \vspace*{-1.5cm} % Move the content up by 0.5cm
    \begin{flushright}
        \hspace*{\fill} % Move the content to the right edge
        $\boxed{\textbf{\Huge\phantom{00}1\phantom{00}}}$ % Your content here with increased padding
    \end{flushright}
\end{minipage}



%SEITE0
\section*{\huge Albus Dumbledore}

%ABSATZ
Professor Albus Percival Wulfric Brian Dumbledore, (Orden des Merlin erster Klasse, Hexenmeister, Ganz Hohes Tier, Internationale Vereinigung der Zauberer) (* August 1881; † 30. Juni 1997) war ein englischer Halbblut-Zauberer, der Professor für Verteidigung gegen die dunklen Künste, später Professor für Verwandlung und später Direktor der Hogwarts-Schule für Hexerei und Zauberei. Professor Dumbledore war auch ein ganz hohes Tier der Internationalen Vereinigung der Zauberer (? –1995) und Leiter des Zaubergamot (? –1995; 1996–1997). Er war ein Halbblut-Zauberer, der als der größte Zauberer der Neuzeit angesehen wurde, vielleicht aller Zeiten. Er war der Sohn von Percival und Kendra Dumbledore und der ältere Bruder von Aberforth und Ariana. Sein Vater starb in Askaban, als Dumbledore jung war, während seine Mutter durch seine Schwester später versehentlich getötet wurde. Beim Drei-Wege-Duell mit Aberforth und seinem früheren Liebhaber Gellert Grindelwald verlor er auch Ariana. Von diesen Ereignissen im Sommer von 1899, konnte er sich nie ganz erholen. Seine frühen Verluste wirkten sich sehr auf ihn aus und machten ihm zu einem besseren und vorallem weiseren Menschen.
\vspace{10pt}
\newline
Obwohl Dumbledore zunächst nicht in der Lage war, direkt gegen Gellert Grindelwald vorzugehen, schickte er Newt Scamander nach New York, um den Donnervogel Frank freizulassen und später nach Paris, um Credence Barebone zu retten. Nach dem tragischen Tod von Leta Lestrange in Paris war Dumbledore entschlossen, dem Widerstand gegen Grindelwald, insbesondere Newt, Theseus, Tina, Jacob, Yusuf und Nagini, im globalen Zaubererkrieg hilfreiche Ratschläge zu geben. Urspürnglich beabsichtigte Grindelwald den Obscurial-Verwandten, Aurelius Dumbledore, als Waffe gegen Dumbledore einzusetzen.
\vspace{10pt}
\newline
Nach dem Zaubererkrieg und dem Ende der Größeren Wohl-Revolution wurde Dumbledore berühmt für seinen Sieg über Gellert Grindelwald, die Entdeckung der zwölf Verwendungszwecke von Drachenblut und seine Arbeiten zur Alchemie mit Nicolas Flamel.
\vspace{10pt}
\newline
Durch Dumbledore bildete sich ein Widerstand gegen den Aufstieg von Lord Voldemort. Er gründete den Orden des Phönix der während des Ersten Zaubererkrieg gegen Voldmort kämpfte. Nach der plötzlichen Rückkehr von Voldemort und dem Beginn des Zweiten Zaubererkrieges, aber vor dem Rücktritt von Cornelius Fudge bildete Dumbledore rasch den zweiten Orden des Phönix und führte ihn auch an. Er untersützte viele Schüler wie beispielsweise Jacobs Geschwisterkind und später das Trio von Harry, Ron und Hermine während ihrer ersten sechs Jahre in Hogwarts. Das Trio gründete auch die sogenannte Dumbledores Armee zu Ehren von ihm und seinem Können.
\vspace{10pt}
\newline
Aufgrund der Tatsache, dass er einen scharfen Verstand und legendäre Kraft hatte, wurde Dumbledore der einzige Zauberer, den Voldemort jemals fürchtete. Er war von 1945 bis zu seinem Tod der Besitzer und der Meister vom Elderstab und wurde von vielen als der beste Schulleiter von Hogwarts angesehen. Als er durch Vorlost Gaunts Rings sterben wollte, plante Dumbledore seinen eigenen Tod mit Severus Snape. Gemäß dem Plan wurde Dumbledore von Snape während der Schlacht auf dem Astronomieturm getötet.
\vspace{10pt}
\newline
Obwohl er zu diesem Zeitpunkt nicht mehr am Leben war, wurde Lord Voldemort durch Dumbledores Manipulationen endgültig besiegt und der Zaubererwelt ein dauerhafter Frieden wiedergegeben. Dumbledore ist der einzige Schulleiter, der auf dem Hogwartsgelände begraben wurde. Albus Dumbledores Porträt ist immer noch in Hogwarts. Harry Potter benannte später seinen zweiten Sohn Albus Severus Potter nach Dumbledore.

\subsection*{\Large Biographie}

\subsubsection*{\large Frühe Lebensjahre (1881-1892)}

%ABSATZ
Albus Percival Wulfric Brian Dumbledore wurde im Sommer 1881 in dem hauptsächlich von Zauberern geprägten Dorf Mould-on-the-Wold als Sohn von Kendra und Percival Dumbledore geboren. Er hatte einen jüngeren Bruder namens Aberforth Dumbledore und eine jüngere Schwester namens Ariana Dumbledore. Die frühen Jahre von Dumbledores Leben waren von einer Tragödie geprägt, als seine jüngere Schwester, Ariana Dumbledore, von einer Gruppe Muggeljungen angegriffen wurde, die sahen, wie sie zauberte, und Angst vor dem hatten, was sie sahen.
\vspace{10pt}
\newline
Ariana war geistig und psychisch von dem Ereignis schockiert und ihre magischen Kräfte haben sich stark verändert. Albus' Vater, Percival Dumbledore, war wütend und machte sich auf die Suche nach den Muggeljungen. Er fand die Muggelkinder und griff sie an, wie sie es bei seiner Tochter gemacht hatten. Dabei benutzte er die Zauberei und aus diesem Grund wurde Percival zu lebenslanger Haft in Askaban verurteilt, wo er später starb. Er verschwieg, warum er die Jungen angegriffen hatte, weil ihm sonst Ariana genommen worden wäre.
\vspace{10pt}
\newline
Um den anklagenden Augen der Zaubererwelt zu entkommen, zog Kendra Dumbledore mit ihrer jungen Familie in das Dorf Godric's Hollow, das wie Mould-on-the-Wold ein hauptsächlich von Zauberern und Hexen bewohntes Dorf war. Kendra wies alle
%SEITE1
ihre Nachbarn, mit Ausnahme von Bathilda Bagshot, zurück und zog es vor, in Ruhe gelassen zu werden. Albus wurde gezwungen, seine Schwester oder seinen Vater in der Öffentlichkeit nicht zu erwähnen.

\subsubsection*{\large Schulzeit in Hogwarts}

%ABSATZ
Dumbledore besuchte Hogwarts zum erstem Mal am 1. September 1892 und wurde in das Gryffindor-Haus eingeteilt. Dumbledores erstes Jahr wurde häufig von Gerüchten über seinen Vater überschattet. Viele von Albus Mitschülern glaubten fälschlicherweise, dass Albus genauso muggelfeindlich wie sein Vater war, da sein Vater nie die Wahrheit erzählt hatte. Einige lobten auch die Handlung seines Vaters, in der Hoffnung, mit Albus eine Freundschaft zu beginnen, jedoch scheiterte dies in vielen Fällen. Obwohl sich später in seinem Leben unter dem Einfluss seiner Liebe zu Gellert Grindelwald einige Vorurteile gegen Muggel entwickelten, erkannte er bald den Irrtum hinter dem Muggelhass.
\vspace{10pt}
\newline
An seinem ersten Tag in Hogwarts freundete er sich mit Elphias Doge an. Doge litt zu der Zeit unter den Folgen von Drachenpocken (grünliche Haut und Flecken), was die meisten davon abhielt, sich ihm zu nähern. Dumbledore zeigte jedoch seine ungewöhnliche Freundlichkeit und Bereitschaft, unter die Oberfläche zu schauen und innere Schönheit in den Menschen zu finden. Dumbledore wurde auch während seiner Hogwarts-Jahre sehr gut vernetzt und machte bemerkenswerte Freunde wie Nicolas Flamel, Bathilda Bagshot und Griselda Marchbanks, die ihn später daran erinnerten, dass er Dinge mit einem Zauberstab tat, die sie noch nie zuvor gesehen hatten. Während seiner Schulzeit veröffentlichte er mehrere seiner Arbeiten und schrieb Briefe mit dem Theoretiker Adalbert Schwahfel. Minerva McGonagall veröffentlichte diese Briefe nach dem Tod von Albus, um der Gesellschaft der Zauberer weiter zu helfen.
\vspace{10pt}
\newline
Während seiner Schulzeit gewann Dumbledore den Barnabus-Finkley-Preis für außergewöhnliche Zauberei, wurde der britische Jugendbeauftragte des Zaubergamots und erhielt die Goldmedaille für seinen bahnbrechenden Beitrag zur Internationalen Alchemie-Konferenz in Kairo.
\vspace{10pt}
\newline
In seinem dritten Jahr nahm Dumbledore vermutlich das Fach "Runen" und mindestens ein anderes Fach auf, von dem bekannt ist, dass es kein Wahrsagen war.
\vspace{10pt}
\newline
Während seines vierten Schuljahrs setzte Dumbledore "versehentlich" die Bettvorhänge in seinem Schlafsaal in Brand (obwohl er zugab, dass er sie ohnehin nie gemocht hatte).
\vspace{10pt}
\newline
Albus wurde in seinem fünften Jahr zum Vertrauensschüler ernannt und in seinem siebten Jahr zum Schulsprecher ernannt. Er galt als der brillanteste Schüler, der die Schule jemals besucht hatte.

\subsubsection*{\large Tragödie, Freundschaften und Liebesbeziehungen (1899)}
%ABSATZ
Nach seinem Abschluss in Hogwarts im Juni 1899 plante Dumbledore mit seinem Freund Doge die traditionelle "Weltreise". Am Vorabend der Abreise ereignete sich jedoch eine Tragödie, als Kendra Dumbledore durch einen Ausbruch von Arianas unkontrollierbarer Magie getötet wurde und Albus nun die Verantwortung für seine jüngeren Geschwister übernahm. Albus kehrte zu Godric's Hollow zurück, voller bitterer Verstimmung über seine Situation, fühlte sich gefangen und verstört. Er verbat Aberfoth die Schule zu verlassen und bestand darauf, dass er seine Ausbildung zuerst abschloss, bevor er sich um Ariana kümmerte. Das Glück schien Dumbledore jedoch endlich zu erreichen, als ein etwa gleichaltriger Ausländer namens Gellert Grindelwald in seinem kleinen Dorf ankam. Grindelwald lebte die nächsten zwei Monate mit seiner Großtante, der berühmten magischen Historikerin und Familienfreundin der Dumbledores, Bathilda Bagshot, zusammen, die die beiden kurz nachdem Grindelwald ihre Haustür überquert hatte, vorstellte. Erfreut darüber, dass der Neuankömmling genauso brillant war wie er selbst, kamen sich die beiden jungen Männer sofort näher und knüpften schnell eine enge Freundschaft, die durch die Enthüllung von Grindelwalds Forschungen über die Existenz und den Verbleib der sagenumwobenen Heiligtümer des Todes vereint wurde. Dumbledore war auch schnell fasziniert von Grindelwalds Vorstellungen von der Herrschaft der Zauberer, offensichtlich in Anbetracht dessen, dass Grindelwald eine unerwartete neue Chance bot, seine eigene Brillanz zu zeigen und aus dem örtlichen Elend herauszukommen. Während der zwei Monate, in denen Grindelwald in Godrics Hallow blieb, wurde die Freundschaft der beiden jungen Männer romantisch, und Albus ignorierte alle Bedenken, die er in Bezug auf Gellerts dunkle Absichten hatte, indem er sich selbst davon überzeugte, dass die Revolution, die sie einleiten wollten, wahrhaftig zum Wohle der Allgemeinheit war, und dass jeder verursachte Schaden tausendfach nach seinem Abschluss zurückgezahlt würde.
\vspace{10pt}
\newline
Als Albus' Bruder Aberforth sich jedoch von Albus' Plänen angeekelt fühlte und sich darüber Sorgen machte, dass er und Ariana für die Fahrt mitgeschleppt werden müssen, konfrontierte er ihn wütend mit seiner Vernachlässigung von Ariana in den letzten Wochen. Auch sagte er ihm, dass er seine Pläne nicht ausführen konnte, weil Ariana nicht im Stande war, diese Reise anzutreten. Dies brachte Albus zurück in die Realität, obwohl er diese Worte nicht hören wollte. Grindelwald war jedoch wütend über Aberforths Intervention und setzte den Cruciatus-Fluch in einem Anfall von Wut auf ihn ein. Um seinen Bruder zu verteidigen, kam es zwischen den drei Jungen im Teenageralter zu einem gewalttätigen Duell. Ariana selbst versuchte, in den Kampf einzugreifen, konnte es aber aufgrund ihrer unberechenbaren Fähigkeiten nicht und wurde im folgenden Chaos von einem verirrten Fluch
%SEITE2
getroffen und starb. Niemand wusste, wer sie versehentlich getötet hatte, aber Grindelwald floh sofort, da er rechtliche Konsequenzen fürchtete und ließ den trauernden Albus und wütenden Aberforth zurück. Aberforth ärgerte sich über Albus und sie hatten fortan eine angespannte Beziehung. Aberforth machte Albus für Arianas Tod verantwortlich und brach ihm bei Arianas Beerdigung die Nase. Aberforth wusste nie von der großen Schuld und dem großen Bedauern, das Albus für den Rest seines Lebens über seinen Anteil an Arianas Tod empfinden würde.


\subsubsection*{\large Frühe Karriere in Hogwarts (? –1925)}
%ABSATZ
Irgendwann nach Grindelwalds Flucht kehrte Dumbledore, der sich schon als Schüler zum Unterrichten hingezogen fühlte, schließlich an die Hogwarts-Schule für Hexerei und Zauberei zurück, zu der Zeit unter der Leitung von Professor Phineas Nigellus Black, um dort zu unterrichten. Belastet von dem geheimen Wissen über Grindelwalds bevorstehenden Aufstieg zur Macht und dem Wunsch, frühere Fehler zu korrigieren, fühlte er sich möglicherweise von der Aufgabe des Lehrers für Verteidigung gegen die dunklen Künste angezogen, eine Position, in der er sich auf Gefahren vorbereiten konnte, von denen er wusste, dass sie eines Tages kommen würden. Albus, der künftig unter dem Namen "Professor Dumbledore" für nachfolgende Generationen von Studenten bekannt sein sollte, wurde das Klassenzimmer 3C und ein Büro gewährt.
\vspace{10pt}
\newline
In den 1910er Jahren unterrichtete Dumbledore den Hufflepuffschüler Newton Scamander und die Slytherinschülerin Leta Lestrange. Zu seinem Unterricht gehörte es, einen Irrwicht zu verwandeln und den Zauberspruch zu lernen, um ihn verschwinden zu lassen. Als Newts Irrwicht sich in einen Schreibtisch verwandelte, fragte Dumbledore, warum er diese Form annahm: Newt gestand, dass seine größte Angst darin bestand, in einem Büro festzusitzen. Nachdem Newt den Irrwicht erfolgreich entfernt hatte, ermutigte Dumbledore Leta, es zu tun, und versicherte ihr, dass es keine Schande war, Angst vor etwas zu haben. Jeder hatte Angst vor etwas. Die Form von Letas Irrwicht überraschte Dumbledore und die gesamte Klasse: ihr Halbbruder Corvus Lestrange (V), wie er im Wasser ertrank. Das Ereignis erschütterte Leta zutiefst.
\vspace{10pt}
\newline
1913 gefährdete eines von Letas Experimenten mit einem Jarvey das Leben eines anderen Studenten. Newt beschloss, die Schuld auf sich zu nehmen und wurde von Schulleiter Phineas Nigellus Black zum Ausschluss der Schule verurteilt. Dumbledore legte ein gutes Wort für Newt ein jedoch brachte dies nichts, aber Dumbledore gelang es, Newt zumindest zu erlauben, seinen Zauberstab zu behalten.
\vspace{10pt}
\newline
Während seiner Zeit in Hogwarts entwickelte Dumbledore auch enge Beziehungen und Freundschaften mit Zauberern und Hexen auf der ganzen Welt. Zu diesen Freunden zählte er den französischen Alchemisten Nicolas Flamel und Ilvermornyprofessor Eulalie Hicks. Er pflegte engen Kontakt zu diesen Freunden in Form von verzauberten Büchern, die sie alle besaßen und die es ihnen ermöglichten, trotz der Entfernung zu sprechen. Das britische Zaubereiministerium, das Dumbledore wegen seiner früheren Beziehung zu Grindelwald bereits misstrauisch gegenüberstand, war dieser Beziehungen überdrüssig und glaubte, dass er diese Kontakte als Spione nutzte und beobachte Dumbledore und seine Verbündeten. Chef unter Dumbledores Rivalen im Ministerium war Torquil Travers, der Leiter der Abteilung für magische Strafverfolgung, der nie versuchte, seine Abneigung gegen den Hogwarts-Lehrer zu verbergen.

\subsubsection*{\large Globaler Zaubererkrieg (1926-1945)}
\subsubsection*{Newt Scamanders Besuch in New York}
%ABSATZ
Im Jahr 1926 erfuhr Dumbledore, dass in New York seltsame Ereignisse geschahen, die auch die Aufmerksamkeit der Internationale Vereinigung von Zauberern auf sich zogen. Diese Ereignisse waren die Folge, als die europäischen Zauberergemeinschaften einer Reihe von Angriffen ausgesetzt waren, die von Grindelwald und seinen Gefolgsleuten durchgeführt wurden.
\vspace{10pt}
\newline
Er wandte sich an seinen früheren Schüler Newt Scamander, der Magizoologe bei der Abteilung zur Führung und Aufsicht magischer Geschöpfe im Londoner Zaubereiministerium geworden war und auf Reisen durch die ganze Welt war, um magische Kreaturen zu finden und vor Zauberern zu retten, vor allem Jäger, die glaubten, dass diese Kreaturen eine Bedrohung für die Geheimhaltung der Zauberei darstellen würden. Dumbledore erzählte Newt von einem Donnervogel, der in Ägypten gefangen gehalten wurde, in dem Wissen, dass er nicht für eine solche Ungerechtigkeit eintreten und ihn nicht nur retten, sondern auch in seine Heimat Arizona zurückbringen würde. Newts Besuch in New York führte nicht nur zur Entdeckung von Credence Barebone, dem mächtigsten Obscurus, den es je gab (der einzige, der bis ins Erwachsenenalter überlebte), sondern auch zur Gefangennahme von Grindelwald, der sich als Percival Graves, einem hochrangigen Beamten des Magischen Kongress der Vereinigten Staaten von Amerika (MACUSA) verkleidet hatte, um den Obscurus zu finden. Dort geriet Newt jedoch in Schwierigkeiten, als einige der magischen Kreaturen aus seinem Koffer flüchteten und sich in Gebieten der Stadt verteilten, in denen hauptsächlich nicht-magische Menschen lebten.
\vspace{10pt}
\newline
Wegen seiner Rolle bei dem Vorfall in New York sowie wegen seiner Zurückhaltung und den Ausweichversuchen, wenn es darum ging, seinen Vorgesetzten im Ministerium Einzelheiten darüber mitzuteilen, warum er überhaupt in New York gewesen war, wurde Newt vom Ministerium das internationale Reisen untersagt. Ein Verbot, das sie angeboten hatten aufzuheben, wenn er gestand, dass Dumbledore ihn nach New York geschickt hatte. Obwohl er realisierte, dass Dumbledore ihn heimlich nach
%SEITE3
New York geschickt hatte, blieb Newt seinem alten Lehrer treu und weigerte sich, dessen Rolle in der Angelegenheit zu bekennen. Auch in späteren Jahren, als die Gerüchte von Rita Kimmkorn in einer Newt-Biographie namens Mann oder Monster - Die WAHRHEIT über Newt Scamander veröfffentlicht wurden, bestritt er weiterhin, dass Dumbledore für seine Reise nach New York verantwortlich war, als er eine neue Ausgabe von Phantastische Tierwesen und wo sie zu finden sind veröffentlichte.
\vspace{10pt}
\newline
Im selben Jahr wurde Dumbledore ein regelmäßiger Kolumnist für Verwandlung heute, und im Tagesprophet mit den Worten, er sei "entzückt" für die Zeitschrift zu schreiben zitiert.

\subsubsection*{Bewegung gegen Gellert Grindelwald}
%ABSATZ
Als Grindelwald 1927 geflohen war, traf sich Dumbledore mit Newt in London, um ihn um Hilfe zu bitten. Er schickte seinen ehemaligen Studenten nach Paris, um Credence Barebone zu finden, der scheinbar doch die Schlacht in New York überlebt hatte. Er gab Scamander auch die Adresse für ein 'sicheres Haus', in das er gehen kann, wenn er sich eine Zeit lang verstecken muss, welches sich dann als das Haus von seinem alten Freund Nicolas und seine Frau Perenelle Flamel herausstellte.
\vspace{10pt}
\newline
Die Gerüchte über Newts mögliche Reise nach Frankreich erreichten das britische Zaubereiministerium, und eine Delegation von Auroren, darunter Newts Bruder Theseus Scamander und Torquil Travers, wurde zum Schloss Hogwarts geschickt, um Dumbledore zu befragen. Travers verlangte von Dumbledore, gegen Grindelwald zu kämpfen, aber Albus lehnte es ab, was Travers zu diesem Zeitpunkt verärgerte. Deshalb erhielt Dumbledore einen Admonitor und durfte nicht weiter Verteidigung gegen die dunklen Künste unterrichten. Dumbledore wandte sich an Theseus, der als letzter das Klassenzimmer verließ, und warnte ihn, dass er Travers, wenn er ihm jemals vertraute, nicht zu Grindelwalds Kundgebung schicken sollte.
\vspace{10pt}
\newline
Kurz darauf traf er in einem Klassenzimmer Leta Lestrange, die ebenfalls in der Delegation war, aber sie trennte sich von der Gruppe der Auroren. Leta beschuldigte Albus, dass er der Einzige war, der sie in ihrer Jugend nicht bösartig fand und dass er sich in dieser Zeit geirrt habe. Dumbledore erzählte ihr, dass Gerüchte über ihren Bruder Corvus Lestrange, der ursprünglich Credence Barebone sein sollte, für sie schmerzhaft gewesen sein müssen, wofür die verärgerte Hexe seine Aussage unterbrach und hinzufügte, dass er sie nicht verstehen könne, weil er nie einen Bruder verloren hatte. Dumbledore antwortete nur, dass er vor Jahren seine Schwester Ariana verloren habe. Am Ende des Gesprächs riet Albus ihr, das Trauern loszuwerden, da es sein untrennbarer Begleiter seit dem Tod seiner Schwester war.
\vspace{10pt}
\newline
Eines Nachts ging Dumbledore in den Raum der Wünsche, wo der Spiegel Nerhegeb aufbewahrt wurde. Als erstes sah er zu seiner Freude noch alte Erinnerungen an sich und Grindelwald aus der Zeit, als ihre romantische Beziehung auf dem Höhepunkt war, die sie dazu brachte, sie mit dem Blutpakt in einer Scheune zu besiegeln und zu schwören, niemals gegeneinander zu kämpfen. Dumbledore sah Grindelwald auch aus der Gegenwart. Albus sehnte sich nach diesem Treffen und fürchtete sich gleichzeitig sehr davor, weil er wusste, dass er der Einzige war, der Grindelwald aufhalten konnte, und er würde sich der Wahrheit stellen müssen, wer Arianas Mörder ist.
\vspace{10pt}
\newline
Nach der Bekennung von Gellert Grindelwald wartete Dumbledore auf Newt in Hogwarts. Albus drückte seine Trauer über Letas Tod aus. Allerdings ist es Newts Niffler gelungen, Grindelwald die Phiole abzunehmen, in der das Blut von Gellert und Albus, durch deren Blutschwur aus früheren Jahren, versiegelt ist. Newt bringt Dumbledore die Phiole, welcher hofft, dass er sie zerstören kann, damit er gegen Grindelwald kämpfen kann und ihn besiegt.

\subsubsection*{Treffen mit Gellert Grindelwald}
%ABSATZ
Irgendwann zwischen 1927 und 1932 plante Dumbledore ein Treffen zwischen sich und seinem ehemaligen Liebhaber Gellert Grindelwald um über den Blutschwur und Gellerts Zaubererrevolution zu reden. Albus nahm einen Zug zur Londoner U-Bahn und fuhr hinauf zum Piccadilly Circus. Das Treffen fand in Londoner Café statt. Albus war zuerst da und bestellte für sich eine Tasse Tee. Grindelwald traf einige Zeit später ein. Während des Treffens besprachen die beiden ihre Vergangenheit, den gemeinsamen Blutschwur und Grindelwalds Pläne, die Albus als Wahnsinn bezeichnete. Außerdem gab Albus zu, dass er in seiner Jugend verliebt in Grindelwald gewesen war, und ihm deshalb anfangs bei der Zaubererrevolution hatte helfen wollen. Grindelwald erwiderte daraufhin, dass Albus die Welt genauso so sehr hatte verändern wollen wie er und, dass er die Welt der Muggel mit oder ohne Albus Hilfe zerstören würde.

\subsubsection*{Dumbledores erste "Armee"}
%ABSATZ
Da Albus wusste, dass die Gefahr durch Gellert Grindelwald immer größer wurde, beauftragte er Newt Scamander damit, eine Gruppe aus unerschrockenen Zauberern zusammenzustellen, während er gleichzeitig versuchte, einen Weg zu finden, den Blutpakt zu zerstören, den er und Gellert in seiner Jugend geschlossen hatten.
\vspace{10pt}
\newline
Irgendwann nach seinem Treffen mit Gellert Grindelwald traf sich Albus mit Newt und Theseus Scamander. In einem der Zimmer, vom Lokal Eberkopf in Hogsmeade, demonstrierte er den Scamander-Brüdern, dass sein Blutpakt ihn nicht nur daran hinderte, Grindelwald anzugreifen, sondern auch daran, an einen Angriff auf Gellert zu denken. Nachdem ihm der Gedanke,
%SEITE4
Grindelwald anzugreifen, durch den Kopf gegangen war, schoss die Blutpaktphiole in die Höhe und die Kette an der sie hing wickelte sich um Albus' Handgelenke und Hals und begann ihn zu strangulieren. Wenn Albus es nicht schaffen würde den verbotenen Gedanken aus seinem Kopf zu kriegen, würde der Blutpakt ihn immer weiter verletzen und schließlich töten. Zu diesem Zeitpunkt waren auf Albus Arm mehrere ältere Wunden sichtbar, die ebenso vom Blutpakt stammten. Dumbledore stoppte seine Demonstration und der Blutpakt lockerte sich. Er erklärte den Scamander-Brüdern, dass das Qilin, ein magisches Tierwesen, welches Grindelwald kürzlich gestohlen hatte, über besondere Fähigkeiten verfügte und Gellert aufgrund dieser Fähigkeiten Visionen der Zukunft sehen konnte, weshalb die Brüder sehr vorsichtig sein mussten.
\vspace{10pt}
\newline
Da Grindelwald die Zukunft vorhersehen kann, entwickelt Dumbledore mehrere sich überlappende Pläne um ihn zu verwirren. Außerdem gibt er jedem Mitglied der Gruppe einen besonderen Gegenstand. Theseus bekam eine Krawatte mit einer großen goldenen Anstecknadel, Jacob bekam einen schlangenförmigen Zauberstab, der aber keinen Kern hatte, also keine Macht; Bunty bekam eine Liste mit Anweisungen, die nur sie sehen konnte. In den Anweisungen befahl Dumbledore ihr, einige Kopien von Newts Koffer anzufertigen. Nachdem sie diese gelesen hatte, fielen die Anweisungen auseinander. Yusuf wurde angewiesen, nach Nurmengard zu gehen und Grindelwald auszuspionieren, während Lally, Newt, Jacob und Theseus nach Berlin geschickt wurden um Anton Vogel, dem amtierenden Vorsitzenden der internationalen Zauberervereinigung eine Nachricht von Albus zu überbringen. Newt richtet Anton Vogel aus, dass er tun solle was richtig und nicht was einfach ist. Vogel erkennt sofort, dass die Botschaft von Albus kommt und fragt Newt, ob dieser selbst vor Ort sei. Newt verneint und Vogel meint daraufhin, dass dies zu erwarten gewesen wäre, Albus hätte schließlich keinen Grund Hogwarts zu verlassen, wenn die Welt draußen brennt. Danach gibt Vogel bekannt, dasss Grindelwald, aus Mangel an Beweisen von sämtlichen Anklagepunkten freigesprochen werde. Dieser gibt prompt seine Kandidatur für Vogels Nachfolge als Vorsitzender bekannt.

\subsubsection*{Wahl des neuen Vorsitzenden der Zauberervereinigung 1932 und Auflösung des Blutpaktes}
%ABSATZ
In Bhutan soll der neue Vorsitzende der „Internationalen Vereinigung von Zauberern“ in einer Zeremonie bestimmt werden. Nach einer alten Tradition verneigt sich ein Qilin vor demjenigen, der reinen Herzens ist. Für die Zeremonie wird jenes Qilin eingesetzt, das zuvor von Grindelwalds Anhängern entführt wurde. Grindelwald hat das Tier getötet und durch ein nekromantisches Ritual wiederbelebt. Daher verbeugt sich das Qilin vor ihm, nicht jedoch vor anderen anwesenden Personen. Anton Vogel erklärt Grindelwald zum Gewinner der Wahl.
\vspace{10pt}
\newline
In seiner Siegesansprache erklärt Grindelwald den Muggeln den Krieg und belegt Jacob mit einem Cruciatus-Fluch. Sein Wahlbetrug wird aber von Credence und Newt offengelegt. Die Wahl muss daher mit einem neuen Qilin wiederholt werden. Dieses verbeught sich vor Albus Dumbledore. Dieser sagt er fühle sich geehrt, lehnt das Amt, aber ab. Kurz darauf versucht Grindewald Dumbledores Neffen Credence zu töten, sein Fluch trifft jedoch auf die Schutzzauber der Dumbledore-Brüder. Hierbei wird der Blutpakt zerstört, da weder Dumbledore noch Grindelwald daran dachten, sich gegenseitig anzugreifen; sondern nur gegenteilige Ziele verfolgten - Gellert wollte jemanden töten, Albus wollte jemanden verteidigen.
\vspace{10pt}
\newline
Es kommt zu einem kurzen, aber intensiven Kampf zwischen Albus Dumbledore und Gellert Grindelwald, der in einem Remis endet, da am Ende keiner der beiden in der Lage ist den anderen ernsthaft zu verletzen. Als Albus sich von ihm abwendet fragt Grindelwald: "Wer wird dich jetzt lieben Dumbledore?" Albus antwortet daraufhin: "Du bist ganz allein." Grindelwald verlässt Bhutan schließlich, ohne dass er von den anwesenden Zauberern, wegen des Wahlbetruges, festgenommen oder zur Rechenschaft gezogen werden kann. Später wurde Dumbledore Zeuge, wie Newt, Tina Goldstein, Theseus und Lally an der Hochzeit von Jacob und Queenie Goldstein in New York City teilnahmen. Albus bedankte sich bei Newt für seine Hilfe, woraufhin Newt antwortete, dass er ihm helfen würde, wenn Albus ihn noch einmal darum bittet.
\subsubsection*{Treffen mit Tom Riddle}
%ABSATZ
Dumbledore half auch bei der Rekrutierung von Studenten für Hogwarts. Der prominenteste unter den Studenten, die er rekrutierte, war Tom Riddle, den er 1938 im Wool's Waisenhaus kennenlernte. Allein Dumbledore war der Professor in Hogwarts, der von der natürlichen Ausstrahlung und Schlauheit von Tom Riddle nicht bezaubert war. In ihrer ersten Begegnung wurde Dumbledore sofort misstrauisch gegenüber seinen "offensichtlichen Instinkten für Grausamkeit, Geheimhaltung und Herrschaft" und beschloss, ihn während seines Studiums an der Schule genau im Auge zu behalten. Riddle wurde bewusst, dass er Dubmledore mit seiner charmanten Fassade nicht so leicht beeinflussen konnte, weshalb er begann Dumbledore zu fürchten und nicht zu mögen.
\vspace{10pt}
\newline
Irgendwann danach aß Dumbledore Bertie Botts Bohnen jeder Geschmacksrichtung und aß eine Bohne die nach Erbrochenem schmeckte. Seitdem hat er seine Vorliebe für diese verloren.
\subsubsection*{Professor der Verwandlung}
%ABSATZ
An einem unbekannten Punkt in oder vor 1943 wurde Dumbledore Professor für Verwandlung.
\vspace{10pt}
\newline
1943 wurden mehrere muggel-geborene Studenten auf mysteriöse Weise von einem Tier versteinert, das angeblich vom Erben
%SEITE5
Slytherins aus der mythischen Kammer des Schreckens entlassen wurde. Tatsächlich wurde das letzte Opfer des Tieres, Myrte Warren, getötet; dieser Akt schien wahrscheinlich die endgültige Schließung der Schule zu gewährleisten, was Riddle zwang, ins Waisenhaus zurückzukehren. Während Dumbledore zu Recht vermutete, dass Riddle hinter den Anschlägen stand, sorgte Riddle dafür, dass der Gryffindor-Student Rubeus Hagrid fälschlicherweise als Täter angesehen wurde.
\vspace{10pt}
\newline
Riddle war in der Lage, Hagrids gezähmte Acromantula, Aragog, als das Tier, das Studenten angreift, darzustellen. So wurde Aragog vertrieben, Hagrids Zauberstab zerbrochen und Hogwarts blieb offen. Dumbledore war nie in der Lage, harte Beweise zu enthüllen, die Riddle implizierten, aber er bewachte Riddle sehr genau und hinderte ihn daran, die Kammer des Schreckens während seiner verbleibenden Jahre in Hogwarts wieder zu öffnen. Er konnte den damaligen Schulleiter auch davon überzeugen, dass Hagrid als Wildhüter an der Schule bleiben durfte.
\subsubsection*{Niederlage von Gellert Grindelwald}
%ABSATZ
Als Dumbledore anfing junge Zauberer und Hexen zu erziehen, war Gellert Grindelwald immer noch auf freiem Fuß in Europa, baute seine Armee auf und hatte bereits eines der Heiligtümer des Todes gefunden, den Elderstab, den er benutzte, um den Kontinent weiter zu terrorisieren. Obwohl der Blutpakt nicht mehr vorhanden war, traute er sich immer noch nicht wirklich, sich seinem altem Freund entgegenzustellen, da er ihn weiterhin trotzdem liebte. Vor allem aber fürchtete Dumbledore Grindelwald. Dumbledore hatte vor allem Angst davor, dass Grindelwald ihm direkt sagen konnte, dass er selbst den Fluch geworfen hatte, der Ariana tötete. Nachdem Albus seine Verfolgung so lange wie möglich hinausgezögert hatte, konnte er 1945 auf dem Höhepunkt von Grindelwalds Macht nicht mehr guten Gewissens die Augen vor den vielen Gräueltaten Grindelwalds und seiner Anhänger verschließen, und so spürte er schließlich seinen alten Freund auf. Als er Grindelwald fand, fand das bekannteste Duell zwischen zwei mächtigen Zaubern statt. Dumbledore besiegte Grindelwald und lieferte ihn an die Regierung aus. Grindelwald wurde in der obersten Zelle von Nurmengard eingesperrt, während Dumbledore den Elderstab für sich behielt. Dumbledore wurde weltweit als Held gefeiert, der den Globalen Zauberkrieg beendete und Grindelwalds Revolution der Zaubererherrschaft über Muggel beendete, und wurde von Leonard Spencer-Moons Regierung mit dem Orden von Merlin (Erster Klasse) ausgezeichnet. Ebenfalls erhielt er seine eigene Schokofroschkarte.

\subsubsection*{\large Nach dem Krieg (1946 - 1969)}
\subsubsection*{Weitere Schulzeit}
%ABSATZ
Dumbledore unterrichtete nach diesem legendären Ereignis wieder mehrere andere Schüler, von denen einige ihm treu blieben, während andere zu Feinden wurden und sich mit Lord Voldemort verbündeten und ihm bei seinem Aufstieg halfen. Unter diesen Studenten waren die zukünftigen Orden des Phönix-Mitglieder Arthur Weasley und Molly Prewett, sowie die Todesser Bellatrix Lestrange und Lucius Malfoy. Während dieser Zeit wurde ihm mehrmals das Amt des Zaubereiministers angeboten, aber er lehnte das Angebot jedes Mal ab und erinnerte sich daran, wie sein Streben nach Macht in der Vergangenheit seine Schwäche gewesen war. Auch wollte er Hogwarts nicht verlassen.
\vspace{10pt}
\newline
Um diese Zeit traf er auf eine Hauselfe namens Hokey. Bevor sie starb, nahm er ihr eine Erinnerung ab, in der Tom Riddle sein Interesse an den magischen Antiquitäten der kürzlich verstorbenen Hepzibah Smith zeigte.
\subsubsection*{Ernennung zum Schulleiter von Hogwarts}
%ABSATZ
Etwa zur gleichen Zeit wurde Dumbledore vom Hogwarts Schulrat irgendwann zwischen März 1965 und März 1971 zum Schulleiter ernannt, als Nachfolger von Armando Dippet auf dieser Position. Es stimmt zwar, dass es eine Ehre und ein Privileg war, die Leitung von Hogwarts anvertraut zu bekommen, aber es bedeutete auch, dass von ihm fortan erwartet wurde, dass er die Verantwortung für all die schwierigen Entscheidungen, die mit der Führung kamen, treffen musste. Bei solchen Gelegenheiten trat Dumbledore in den Glockenturm, wo er die ruhige Nacht und die frische Luft hilfreich fand und entwickelte auch die Angewohnheit, viel in seinem neuem Büro herumzuschreiten, vor allem in Zeiten, in denen die Sicherheit der Burg zu Wünschen übrig ließ. Gelegentlich besuchten ihn auch Schüler, weil sie gegen Regeln verstoßen hatten oder seine Hilfe brauchten. Irgendwann nach seiner Ernennung begann Professor Dumbledore, den Studenten bei ihren Hausaufgaben zu helfen, wofür er sich immer am Freitag mit ihnen im Klassenzimmer 13E traf.
\subsubsection*{Lord Voldemorts Bitte}
%ABSATZ
Kurz nachdem Dumbledore Schulleiter wurde, wandte sich Tom Riddle, der zu diesem Zeitpunkt ausschließlich als Lord Voldemort bekannt war, an Dumbledore mit der gleichen Bitte, die bereits Armando Dippet abgelehnt hatte. Seine Bitte war es nach Hogwarts als Lehrer der Verteidigung gegen die dunklen Künste zurückkehren zu dürfen. Dumbledore, der von Voldemorts illegalen Aktivitäten wusste, seit er Hogwarts verließ, lehnte seinen Antrag ab. Dies führte dazu, dass Voldemort wütend die Position der Verteidigung gegen die dunklen Künste verfluchte und sich dadurch niemand danach länger als ein Jahr als Lehrer in diesem Fach halten konnte.
%SEITE6
\subsubsection*{Verteidigung von Remus Lupin}
%ABSATZ
Dumbledore erfuhr von dem jungen Remus Lupin, der vom Werwolf Fenrir Greyback gebissen worden war, und wie er von seinen Eltern fest gehalten wurde und magische Bildung erhielt, da sie befürchteten, dass Remus nirgendwo akzeptiert würde, obwohl er im Stande war, in Hogwarts als Schüler teilzunehmen. Dumbledore war der Meinung, dass Remus auch die Möglichkeit erhalten sollte, als Schüler in Hogwarts teilzunehmen. Dumbledore bereitete die Heulende Hütte als Ort für Remus' monatliche Verwandlung vor, mit der Peitschende Weide als Wächter. Nachdem er die Heulende Hütte vorbereitet hatte, ging er dann in das Haus der Familie Lupin und trotz der Bemühungen von Lyall Lupin und Hope Howell, ihn davon abzuhalten, brach er mühelos ihre Verteidigung und betrat unbemerkt das Haus. Bevor er mit den Eltern ins Gespräch trat, spielte er Koboldstein mit Remus. Seine Vorbereitung und seine Freundlichkeit sorgte schließlich doch dafür, dass Remus nach Hogwarts durfte. Remus war Dumbledore deshalb für immer dankbar, weil viele Schulleiter den Jungen einfach ignoriert hätten. Als Severus Snape dieses Geheimnis herausfand, verbot Dumbledore ihm, es an irgendjemanden weiterzugeben. Die Heulende Hütte wurde schließlich als Gefahr und verfluchte Hütte erklärt, nachdem man immer wieder das Geheule von dem Werwolf hörte.

\subsubsection*{\large Erster Zaubererkrieg (1970 - 1981)}
\subsubsection*{Die verwunschenen Verliese}
%ABSATZ
Zwar wagten sich Lord Voldemort und seine Anhänger aufgrund der Anwesenheit von Dumbledore nicht mehr in die Nähe von Hogwarts, jedoch begaben sich wieder mehrere Studenten auf die Suche den mysteriösen verwunschenen Verliesen, um dort verschiedene Schätze zu finden. Es war nicht das erste Mal in der Geschichte von Hogwarts, dass ein Schüler nach den Gewölben suchte. Bereits die ehemalige Gryffindor-Schülerin Patricia Rakepick glaubte fest an die Existenz der Verliese und versuchte das Schulpersonal vor diesen zu warnen. Jedoch wurde sie nur als lächerlich dargestellt, da es nicht genügend Beweise gab. Bald faszinierte das Geheimnis einen Jungen namens Jacob, der einer der talentiertesten und rebellischsten Schüler zu seiner Zeit an der Schule war. Deshalb interessierte sich Dumbledore schnell für den jungen Jacob und beschloss ihn sogar zu unterrichten, als er entdeckte, dass Jacob ein geborener Legilimentiker war.
\vspace{10pt}
\newline
Dumbledore merkte jedoch, dass Jacob sich während der Suche nach den Verliesen immer mehr veränderte. Er wurde immer rücksichtsloser, verletzte viele Schulregeln und gefährdete sowohl sich, als auch seine Freunde. Schließlich wurde er größenwahnsinnig und die Suche trieb ihn in den Wahnsinn. Darum bestrafte Dumbledore Duncan Ashe und Jacob. Schließlich starb Duncan, während einem von Jacobs Experimenten. Dumbledore konnte nichts mehr an diesem Vorfall ändern, weshalb er das Britische Zaubereiministerium um Hilfe bat. Das Ministerium schickte sofort zwei Auroren, die Jacob zu dem Vorfall befragten. Nachdem die Auroren ihn im Die drei Besen fanden, nahmen sie ihn fest. Während der Vernehmung gab Jacob zu, seinen Freund unter Druck gesetzt zu haben, der einen Trank gebraut hatte, bei dem er starb. Angesichts von Jacobs Geständnis hatte Professor Dumbledore keine andere Wahl, als ihn von der Schule zu verweisen.
\vspace{10pt}
\newline
Duncan Ashe war nicht das einzige Opfer. Immer, wenn ein Verlies geöffnet wurde, wurde ein Fluch ausgelöst, der die Schule heimsuchte. Verfluchtes Eis begann sich im ganzen Schloss auszubreiten, verwandelte ganze Räume in eiserne Gefängnisse und drohte, die ganze Schule einzuhüllen und verletzte mehrere Schüler schwer. Ein Schlafwandler-Fluch wurde ebenfalls ausgelöst, der die Schüler in einen tiefen Schlummer versetzte und sie dazu brachte, achtlos in den Verbotenen Wald in Richtung eines der Verliese zu wandern, wo diejenigen, die das Personal nicht aufhielt, aufgrund ihres Zustandes und ihrer Unfähigkeit, sich zu verteidigen, von den Tieren angegriffen und sogar getötet wurden. Andere, bemerkte Madam Pomfrey, wachten einfach nie wieder auf. Obwohl Dumbledore und der Rest des Personals es letztlich schafften, die Flüche einzudämmen und die Situation in den Griff zu bekommen, verneinte die Schule dennoch jegliches Wissen, dass die Verliese tatsächlich existierten, in der Hoffnung, andere Studenten davon abzuhalten, sie zu öffnen. Jedoch wurde schnell von den Ereignissen im Tagespropheten berichtet.
\vspace{10pt}
\newline
Irgendwann nach all dem wurde Dumbledore jedoch auf die Existenz einer geheimnisvollen Gesellschaft aufmerksam, die einfach durch das anfängliche "R" bekannt war, das einige Zeit zuvor in der Zauberwelt aktiv geworden war. Eine abweichende, heimliche Gruppe gefährlicher Dunkler Zauberer, über die sehr wenig bekannt war, deren rücksichtsloses Streben nach "Erleuchtung und Unsterblichkeit" zu verschiedenen geheimen Aktivitäten auf der ganzen Welt geführt hatte und schließlich richten sie ihre Aufmerksamkeit auf Hogwarts. Ihre Beteiligung an der Störung der verfluchten Verliese war jedoch verdeckt gewesen, und nicht bekannt für Dumbledore und den Rest des Stabes zu der Zeit. Zwang war eine treibende Kraft hinter Jacobs Besessenheit mit den Verliesen gewesen. Nachdem Jacob sich zunächst mit „R“ zusammengetan hatte, in der Hoffnung, dass sie ihm helfen könnten, die Gewölbe zu finden, hatte die bösartige Organisation ihm stattdessen den Spieß umgedreht und ihn gezwungen, ihm bei ihrer eigenen Suche zu helfen, indem sie drohte, während gleichzeitig ein Versprechen der zukünftigen Mitgliedschaft vorhanden war. Es war auch 'R', die Duncan Ashe dazu drängten, den Trank zu machen, der ihn das Leben kostete. Jacob nahm die Schuld aus Angst auf sich, weil er eine solche Angst vor der Drohung hatte, wenn er sie verraten hätte.
\vspace{10pt}
\newline
Kurz nachdem er aus der Schule geworfen worden war, rannte Jacob von zu Hause weg und verschwand auf mysteriöse Weise. Schließlich gab es Gerüchte über ihn an der Hogwarts-Schule, dass er in der Unterwelt von Großbritannien sehr bekannt wurde und sich sogar Lord Voldemort als Todesser angeschlossen hat. Dies war jedoch nicht der Fall. In Wahrheit hatte Jacob vom
%SEITE7
Moment seiner Vertreibung bis zu seinem Verschwinden weiter für "R" gearbeitet, um die Geheimnisse der verwünschenden Verliese zu enthüllen, um seine Liebsten zu schützen.

\subsubsection*{\large Zwischen den Kriegen (1981 - 1995)}
\subsubsection*{Niedergang des Dunklen Lords}
%ABSATZ
Dennoch tobte der Zauberkrieg außerhalb der Mauern von Hogwarts: Kurz vor dem Tod der Potters entdeckte Dumbledore, dass James Unsichtbarkeitsumhang tatsächlich eines der Heiligtümer des Todes war. Obwohl Dumbledore seinen Wunsch, die Heiligtümer zu vereinen und Meister des Todes zu werden, lange aufgegeben hatte, konnte er dem Gedanken, den Mantel zu analysieren, nicht widerstehen. Zu diesem Zweck lieh er sich diesen Mantel von James, während James und Lily untergetaucht waren. Später wandte sich Severus Snape an Dumbledore, nachdem er Voldemort bereits von der Prophezeiung erzählt hatte. Snape war schockiert über die Tatsache, dass Voldemort zu dem Schluss gekommen war, dass es bei der Prophezeiung, über die Person, die ihn besiegen konnte, um Harry Potter, den Sohn von James Potter und Snapes geliebte Jugendfreundin Lily Evans ging. Darum bat er Dumbledore darum sie zu schützen. Als Gegenleistung wollte Dumbledore, dass Snape als Doppelagent für dem Orden arbeitet.
\vspace{10pt}
\newline
Dumbledore traf Vorkehrungen, um die Sicherheit der Familie zu gewährleisten; er riet ihnen, in ihrem Haus in Godric's Hollow versteckt zu bleiben, welches von einem Fidelius-Zauber geschützt wurde. Doch ihr Geheimniswahrer (von Dumbledore und allen anderen wurde gedacht, dass dieser Sirius Black sei; in Wahrheit war es Peter Pettigrew) verriet sie, was zu James und Lilys Tod durch Voldemort führte. Doch als Voldemort versuchte Harry zu ermorden, schlug der Fluch aufgrund von Lily Potters Liebe zu ihrem Sohn zurück, da dies ein mächtiger Schutzzauber war, und traf ihn selbst, worauf seine Seele, welche durch die Horkruxe geschützt war, seinen toten Körper verlies und floh.
\vspace{10pt}
\newline
Vermutlich wohl wissend, dass Lord Voldemort nicht wirklich besiegt war, sorgte Dumbledore dafür, dass Harry mit seiner Familie, den Dursleys, in Sicherheit gebracht wurde. Er überzeugte Snape, ihm zu helfen, Harry zu schützen, obwohl Dumbledore Jahre später heimlich plante, Voldemort zu erlauben, Harry am Ende zu töten, was das Stück Voldemorts Seele in Harry zerstören würde. Er hoffte jedoch auch, Harry in eine solche Person zu beeinflussen, die bereitwillig sein Leben für das größere Wohl opfern würde, was ihm die Möglichkeit gab, am Ende zwischen Leben und Tod zu wählen, und die Bedeutung einer solchen Tat richtig erraten würde. Kurz darauf stellte Dumbledore Severus Snape als Tränkemeister in Hogwarts an und schütze Snape später vor dem Britischen Zaubereiministerium, die die verbliebenen Todesser aufsuchten und nach Askaban schickten. Dumbledore nahm an vielen Todesser-Prozessen teil, darunter die der Familie Lestrange und von Barty Crouch Jr.. Er war auch Zeuge von Igor Karkaroffs Aussage, die seine Freilassung aus Azkaban als Gegenleistung für die Namen von anderen Todessern sicherte. Der Orden blieb bestehen und informierte ihn Regelmäßig über den Zustand von Harry. Auch leitete er weiterhin Hogwarts.

\subsubsection*{1984 - 1985}
\subsubsection*{Beginn des Schuljahres}
%ABSATZ
Beim Fest zum Beginn des Schuljahres während des Schuljahres 1984–1985 saß Professor Dumbledore auf dem prächtigen goldenen Stuhl im Zentrum des Lehrertisches in der Großen Halle. Unter den Neuankömmlingen war der zweitälteste Sohn von Arthur und Molly Weasley und die Tochter von Andromeda Black und ihrem muggelgeborenen Ehemann Edward Tonks. Charles wurde wie erwartet in das Haus Gryffindor untergebracht, während letztere unerwartet nach Hufflepuff einsortiert wurde. Der neue Jahrgang brachte auch Tulip Karasu, die junge Tochter von zwei Mitgliedern der Abteilung für magische Strafverfolgung im Zaubereiministerium, die Dumbledore vermutlich kannte, da er viele Verbindungen zum Zaubereiministerium hatte. Merula Snyde und Barnaby Lee, die beide Eltern hatten, die während des Zauberkrieges als Todesser bekannt geworden waren und wegen ihrer vielen Verbrechen nach Askaban geschickt wurden, fanden ihren Weg ins Haus Slytherin. Rowan Khanna, der Junge einer Zaubererfamilie, die eine ziemlich bekannte Baumfarm besaß, die berühmt dafür ist, verschiedene hochwertige Hölzer für die Herstellung von Zauberstäben und Besenstielen zu liefern. Als hoffnungsvollster Schüler aber galt Jacobs Geschwister, von den Dumbledore sowie viele andere Lehrer einen Eindruck bekomme sollten.
\vspace{10pt}
\newline
Nachdem diese altehrwürdige Tradition ihren Lauf genommen hatte, erhob sich Dumbledore von seinem Platz und hielt die übliche Begrüßungsrede, erklärte, wie das Hauspunktsystem funktionierte und ermutigte sie alle, alles zu tun, um dem neuen Haus Anerkennung zu sein. Jedoch hielten sich nicht alle Schüler an diese Regeln. Professor Snape sprach Dumbledore kurz nach der Rede an, dass er den Geschwister von Jacob und Merula Snyde beim Duellieren erwischt hatte. Severus Snape war der Meinung, dass man ihn deshalb von der Schule verweisen sollte, jedoch lehnte Dumbledore dies ab und erklärte, dass jeder Schüler seine Zeit bräuchte, um sich einzuleben. Dumbledore nahm Snape das Recht, schlimme Strafen zu vergeben, jedoch eilte Snape bald zurück zu Dumbledore und informierte ihn darüber, dass der Hausmeister verfluchtes Eis im fünften Stock gefunden hatte. Dies war ein Zeichen dafür, dass wieder jemand versuchte, die verfluchten Gewölbe zu öffnen.
\vspace{10pt}
\newline
Nach dieser Entdeckung gelang es den Lehrkräften, das Eis davon abzuhalten, sich weiter auszubreiten und den Raum anschließend sicher zu versiegeln. Auch wurde der Raum unter Bewachung von Herrn Filch und Mrs Norris gestellt. Da Dumbledore
%SEITE8
sich jedoch nicht ganz sicher fühlte, holte er Patricia Rakepick zurück an die Schule, um die Gewölbe zu studieren und neue Informationen zu sammeln.
\subsubsection*{Das heulende Halloween}
%ABSATZ
Auch in diesem Jahr wurde das Halloween Fest wieder durchgeführt und die Schüler sollten freiwillig helfen das Fest vorzubereiten. Die Feier verlief jedoch nicht wie die letzten, weil gegen Ende des Festes Dumbledore nicht mehr anwesend war, sondern außerhalb des Gebäudes unterwegs war.
\vspace{10pt}
\newline
Dort stellte er fest, dass die Schule von einem Rudel Werwölfe angegriffen worden war. Sofort begann er den Rest des Personals zu alarmieren. Erfolgreich entfernte sie fünf von ihnen, die Hagrids Hütte fast überrannt hatten, bevor er in die Burg zurückkehrte. Dort kam er gerade noch rechtzeitig, um Jacobs Geschwister vor drei weiteren zu retten, die versucht hatten, seine Abwesenheit zu nutzen, um ein paar Studenten anzugreifen. Schließlich flüchtete Fenrir Greyback und seine Anhänger zurück in den Verbotenen Wald. Nachdem er sichergestellt hatte, dass der oben genannte Schüler unverletzt blieb, bemerkte er jemanden, der in der Nähe bewusstlos auf dem Boden lag. Sofort erkannte er, dass es eine Schülerin war - eine junge Hexe aus dem Hufflepuff-Haus namens Chiara Lobosca. Er untersuchte kurz ihre Verletzungen und befahl den restlichen Schülern zurück in die Häuser zu gehen. Nach dem Angriff informierte Dumbledore das Zaubereiministerium über den Vorfall und bat darum, dass sie jemanden in die Schule schickten, der die Angelegenheit untersuchen konnte und hoffentlich die Werwölfe gefangen nehmen konnte, bevor jemand anderes verletzt wurde. Als Reaktion darauf schickten sie Cecil Lee, um die Werwölfe aufzuspüren. Auch gab er die Erlaubnis die Schüler zu befragen. Nach einem Monat konnte Mr. Lee mithilfe des Geschwister von Jacob Greyback und seinen Rudel gefangen nehmen.
\subsubsection*{Hagrids Geburtstag}
%ABSATZ
Greyback und der restliche Rudel wurde auf dem Trainingsgelände der Quidditchmannschaften gefesselt. Ein weiblicher Werwolf nannte Jacobs Geschwister "einen Freund". Deshalb wurde beschlossen den Werwolf ins Zaubereiministerium zu bringen. Tatsächlich handelte sich dabei um Chiara Labsca, die, wie sich herausstellte, schon als Kind gebissen wurde und zu Greybacks Opfern zählte. Da es in der Zauberwelt immer noch viele Vorurteile gegen Werwölfe gab, nahm Dumbledore, wie damals bei Lupin, die Schülerin wieder an Hogwarts auf. Severus Snape stimmte dem zu. Auch entschloss sich Dumbledore, die Halloweenparty zweimal zu veranstalten, weil es immer ein Highlight für die Schüler war und die erste nicht beendet worden war.
\vspace{10pt}
\newline
Rubeus Hagrid verlor irgendwann später eines seiner Haustiere und war bereits die gesamte letzte Zeit sehr ruhig und unauffällig. Deshalb beschloss Dumbledore ihn aufzumuntern. Anfang Dezember schickte er deshalb Jacob Geschwister und seinen Freunden eine Eule, dass sie seinen Geburtstag planen sollen. Dumbledore, Jacobs Geschwister und Rowan Khanna planten eine Überraschungsfeier für Hagrid.
\vspace{10pt}
\newline
Der Schulleiter erzählte Minerva McGonagall, dass er mit Amos Diggory gesprochen hatte, der sich bereit erklärt hatte für Hagrid zum Geburtstag magische Kreaturen zu suchen, die legal seien. Damit die Schüler die Überraschung vorbereiten konnten, traf sich Dumbledore mit Hagrid auf dem Trainingsgelände, um ihm mitzuteilen, dass es in Hogsmeade "dringende Geschäfte" gebe. Jacobs Geschwister begleitete ihn, um aufzupassen, dass Hagrid nicht zu schnell zurückkehrte. Die Überraschungsparty wurde ein großer Erfolg.
\vspace{10pt}
\newline
Auch wollte Dumbledore an diesem Weihnachten die Schüler, die in Hogwarts blieben, beim Dekorieren einbinden.
\subsubsection*{Ende des Schuljahres}
%ABSATZ
Das verfluchte Eis einzudämmen scheiterte, da eine Gruppe von Studenten in den verschlossenen Raum eingeboren war. Dumbledore versteht schnell, dass Jacobs Geschwister daran beteiligt war. Jedoch wird ihm nach langem Überlegen klar, dass er dies nicht aus Egoist und nach Gier zum Reichtum gemacht hatte, sondern weil er seinen Bruder finden wollte. Deshalb lud er den Schüler zu sich ein, um mit ihm zu reden. Die Schüler erwarteten einen Vortrag über ihr Verhalten, jedoch sparte sich Dumbledore das zum größten Teil. Er erzählte ihnen die gesamte Geschichte, wie Jacob verschwunden war und bat sie ihre Erkundung zu beschränken, und nichts Lebensbedrohliches zu tun. Auch hatte Dumbledore nicht vor, sie zu bestrafen, sondern lobte sie wegen ihres Mutes. Er gab ihnen wegen ihrer Kreativität 100 Hauspunkte und versicherte ihnen, dass er nach den Ferien mit ihnen reden würde. Am Ende des Jahres während dem Fest zum Schulende verkündete Dumbledore den Gewinner des Haus Cups und wünschte ihnen schöne Ferien.

\subsubsection*{1985 - 1986}
\subsubsection*{Auf der Suche nach Antworten}
%ABSATZ
Obwohl das verfluchte Eis wieder unter Kontrolle war, war dem Personal immer noch nicht ganz im Klaren, was das Öffnen für die Zukunft der Schule bedeuten würde. Zum Beginn des Schuljahres mahnte Dumbledore zur Vorsicht. auch sagte er, dass die Schüler sofort einen Lehrer informierten sollten, falls etwas Außergewöhnliches passierte. Nach der Rede von Dumbledore
%SEITE9
stellte sich heraus, dass ein Gryffindor Zweitklässler namens Ben Cooper nicht anwesend war und niemand wusste, wo dieser sich aufhielt. Nach dem Fest wurde beschlossen, dass die Lehrer nach ihm suchen sollten. Minerva McGonagall traf sich mit dem Geschwister von Jacob, da vermutet wurde, dass dieser etwas mit dem Verschwinden zu tun hatte. Jacobs Geschwister erzählte, dass es eine geheime Nachricht geben würde, die eine Drohung enthielt.
\vspace{10pt}
\newline
Nachdem man die Warnung vor einem "Eisritter" und einer "Verschwundenen Treppe" gelesen hatte, wurde im Schloss wieder verfluchtes Eis gefunden. Dumbledore hate den Einfall, dass es außerhalb des Grundstückes Antworten geben würde und verließ Hogwarts deshalb kurz danach. Davor bat er Rubeus Hagrid, Forschungen über Geflügelte Pferde in Schottland zu betreiben. Hagrid sollte vor allem Abraxaner beobachten, da bald die französische Partnerschule zu Besuch käme. Jacobs Geschwister sollte Hagrid dabei helfen. Kurz nachdem Dumbledore nun die Schule verlassen erhielt er einen Brief von Severus Snape per Eulenpost mit der Nachricht, dass "R" zurückgekehrt war und die Suche nach den verfluchten Gewölben fortsetzen würde.
\vspace{10pt}
\newline
Die Lehrer suchten überall nach Hinweisen, um Cooper wieder zu finden. Dabei suchten sie Zettel, die ein Mitglied von "R" im Schloss Hogwarts versteckt hatte. Wenn man den Forderungen nicht nachkäme, drohten "schwere Bestrafungen". Schließlich hatte Jacobs Geschwister und Rowan Khanna die Suche aufgenommen und folgten den Anweisung. Der Mitschüler wurde schließlich in der Schule gefunden, gefangen im verfluchten Eis. Das Eis hatte ihn so viel Schaden zugefügt, dass viele Dinge aus seinem Gedächtnis verloren gingen und er nichts mehr von seinem Entführer wusste.
\subsubsection*{Rückkehr nach Hogwarts}
%ABSATZ
Dumbledore plagten immer mehr die Sorgen, dass sich das verfluchte Eis weiter ausbreiten würden und Schüler verletzt werden. Deshalb reiste er den Rest des Schuljahres um die gesamte Welt, um einen Fluchbrecher zu finden, der das Problem lösen konnte. Nachdem er aber zurückgekehrt war, verschwand das verfluchte Eis plötzlich spurlos. Dies war ein Zeichen dafür, dass jemand die Tür zum Gewölbe geöffnet hatte. Dumbledore erkannte dies, und ging in den Korridor im fünften Stock, um den Ort zu untersuchen, an dem zum ersten Mal das verfluchte Eis aufgetaucht war. Dort fand er den Eingang, der zum Gewölbe des Eises führte. Als Dumbledore die Eingangshalle untersuchte, konnte der Schulleiter feststellen, dass einen "schreckliche Menge Arbeit" dafür nötig gewesen sein musste. Auch bewunderte er die "komplexe und gefährliche Magie". Er erkannte auch, dass die Tür vor kurzem geöffnet wurde, aber er stellte auch fest, dass der Fluch gebrochen worden war.
\vspace{10pt}
\newline
Nach seiner Erkundungstour teilte er seine neuen Erkenntnisse mit den restlichen Lehrern, bevor er in sein Büro zurückkehrte. Kurz drauf bat er Professor Minerva McGonagall, Jacobs Geschwister in sein Büro zu holen. Er lag damit richtig, dass er es in den Eiskeller gewagt hatte und den Fluch gebrochen hatte. So bewahrte er die Schule vor einem schrecklichen Schicksal. Jacobs Geschwister erzählt von seinen Erfahrungen, die Dumbledore sich dabei aufschreibt. Jacobs Geschwister erzählt, dass er die Stimme Jacobs gehört hatte und durch eine Art Visions in das Eisgewölbe gelockt wurde. Danach lobte Dumbledore den Schüler für seinen Heldenmut, den Fluch zu brechen, jedoch mahnte er ihn auch, da er sich und seine Freunde unnötig in Gefahr gebracht hatte. Auch bat er den Schüle die Suche nach den verbleibenden Gewölbe und dem Bruder erfahrenen Leuten zu überlassen.
\subsubsection*{Besuch der Beauxbatons Akademie für Zauberei}
%ABSATZ
Hagrid erstattete anfangs des Jahres Bericht über die Ankunft von Professor Olympe Maxime, der Schulleiterin der Beauxbatons Akademie für Zauberei, jedoch konnte trotz guter Planung keine gute Unterkunft angeboten werden. Als ihr Wagen mit Abraxaner ankam, war Rubeus Hagrid auf Reisen, weshalb sich niemand um die vielen Tiere kümmern konnte. Dumbledore fand jedoch noch Platz auf dem Trainingsgelände, wo er sie mit mehreren großen Fässern Single Malt Whiskey versorgte. Eigentlich war sie hauptsächlich gekommen, um mit Dumbledore über Schulsachen zu sprechen, aber auch um eine ihre Schülerinnen, Aurélie Dumont, nach Hogwarts zu bringen, damit sie ihre Brieffreundin Penny Haywood treffen kann. Dumbledore wurde aber darauf aufmerksam, dass Aurelie nicht in Hogwarts war, um sich umzusehen und ihre Brieffreundin kennenzulernen, sondern er merkte, dass sie etwas suchte. Als er damit ziemlich sicher war bat er Professor Maxime Dumont und Jacobs Geschwister zu finden und in sein Büro zu bringen.
\vspace{10pt}
\newline
Dumbledore erfuhr, dass sich die französische Schülerin für Nicolas Flamel und Alchemie interessierte. Sie erzählte Dumbledore, das es in Frankreich ein Gerücht über einen Schatz gab, der einst in einem geschützten Gewölbe in Hogwarts versteckt war. Dieser sollte einst Flamel gehört haben. Es sei ein altes alchemistisches Artefakt, dass jeden, der es besaß, zu einem großen Alchemisten machen würde. Dumbledore versicherte ihr, dass er sich zwar noch sicher sei, ob es so etwas gebe, aber ein solches Objekt wäre nirgends in Hogwarts. Professor Maxime merkte, dass sie dieses Wissen traurig machte und sagte ihr deshalb, dass sie keinen Schatz bräuchte, um eine große Alchemisten zu werden. Sie hätte genug Talent, um das auch so zu werden.
\vspace{10pt}
\newline
Beim Fest zum Ende des Schuljahres lobte er die Studentenschaft für ihren Mut gegenüber dem verfluchten Eis. Danach verkündete er den Gewinner des Hauspokals.
%SEITE10

\subsubsection*{1986 - 1987}
\subsubsection*{Schuljahresbeginn}
%ABSATZ
Wie jedes Schuljahr hielt auch Dumbledore in diesem Jahr zu Beginn seine Rede. In dieser Rede ging es vor allem um die Anweisungen der Vertrauensschüler in Bezug auf die verfluchten Gewölbe.Da es im letzten Schuljahr so viele schlimme Ereignisse gab, entschloss sich Dumbledore, den Schülern zu sagen, dass es die verfluchten Gewölbe wirklich gäbe. Auch bestätigte er die Gerüchte, dass eine Gruppe von Schülern in ein solches Gewölbe gegangen waren und den Fluch gebrochen haben. Aber auch sagte er, dass dies sehr gefährlich war und nicht wiederholt werden sollte. auch verhängte er ein Verbot, die Gewölbe wieder zu betreten und Flüche zu brechen. Zum Schluss stellte er klar, dass das gesamte Personal alles tut, um die Flüche zu Brechen und so das Schuljahr in Ruhe verbringen zu können.
\vspace{10pt}
\newline
Etwas später traf sich Dumbledore mit dem Hauslehrer von Jacobs Geschwister. Diesem erzählte er, dass er Jacob bei Nichtbeachtung seiner neuen Regel die Erlaubnis entziehen sollte nach Hogsmeade zu gehen. Auch gab er dem Hauslehrer die Erlaubnis, dass Verbot aufzuheben, wenn Jacobs Geschwister sich gut benehmen würde. Die Bemühungen von Dumbledore brachten jedoch nichts, da die Schüler sofort beim ersten Hogsmeade Ausflug die Vermieterin der drei Besen Rosmerta über die verfluchten Gewölbe befragten und diese Informationen für die weitere Suche verwendeten.
\vspace{10pt}
\newline
Wenige Tage nach dem Willkommensfest wurde Dumbledore von Professor Pomona Sprout darüber informiert, dass bei ihr im Unterricht im Gewächshaus ein Irrwicht in Form eines Werwolfs erschienen war und die Hufflepuff-Schülerin Penny Haywood erschreckt hatte. Dies war ein ziemlich komisches Ereignis, da Irrwichte vor allem in dunklen, geschlossenen Räumen erscheinen und nicht in großen, luftigen, sonnigen Gewächshäusern. Dieses Ereignis brachte das Personal zu dem Schluss, dass dieses seltsames Verhalten durch ein Fluch gekommen war. Dies bedeutete wohl auch, dass jemand das Gewölbe gestört hatte. Später erhielt er eine Eule von Rosmerta, die ihm mitteilen wollte, das Jacobs Geschwister ihm geholfen hatte ein neues, extrasüßes Butterbier zu machen. Sie schlug Dumbledore vor, dem Haus des Schülers einige Hauspunkte zu geben.
\subsubsection*{Lügen und Täuschungen}
%ABSATZ
Der Schulleiter war später der Meinung, dass Merula Snyde einen sehr guten Gripsschärfungstrank gemacht hatte. Deshalb wollte er ihr eine Auszeichnung geben. Später stellte sich jedoch raus, dass Dumbledore getäuscht wurde.
\vspace{10pt}
\newline
Bei einem späteren Gespräch mit Minerva McGonagall diskutierten Dumbledore und sie über Harry Potter. Später stieg Severus Snape ins Gespräch ein und redete mit Dumbledore über den Zaubertränke-Lehrplan. Auch redeten sie über die fünf weiteren Kinder der Familie Weasley. Ebenfalls wurde Rowan Khanna als besonders schlau dargestellt.
\vspace{10pt}
\newline
Später im Gespräch rief Snape Merula Snyde und verlangte unerwartet, dass sie ihr Fehlverhalten gestand. Anfangs täuschte die Schülerin Unwissenheit vor. Später gab die Schülerin unter dem Druck nach und gab zu, dass sie den Gripsschärfungstrank von Penny Haywood geklaut hatte. Dumbledore holte später Penny auf das Podium, um seinen Fehler zu korrigieren. Als Penny wieder ging, sprach Dumbledore mit Snape, da er verstanden hat, dass dies Jacobs Geschwister war und er den Vielsafttrank verwendet hatte. Das Haus bekam am Ende zehn Hauspunkte, weil sie sich so viel Mühe gegeben haben, um ein Unrecht zur korrigieren.
\subsubsection*{Dunkle Pferde}
%ABSATZ
Im gleichen Zeitraum gab es in Großbritannien eine Konferenz über Thestrale, die der Wärter von Hogwarts hielt, um dafür zu sorgen die Bevölkerung für das Thema zu sensibilisieren. Dumbledore rat ihm sich Hilfe von Jacobs Geschwister zu suchen, da dieser sich gut mit magischen Kreaturen auskannte. Aber auch sagte er, er solle Merula Snyde fragen. Merula war nämlich im Stande Thestrale zu sehen. Ihre Mutter tötete eine Aurorin auf brutalste Art und weise. Dumbledore hatte Hagrid gebeten, sie mitzunehmen, da sie dadurch ein Tierwesen kennenlernen würde, das nur sie sehen konnte und auch ihre Vergangenheit und den Mord verarbeiten zu können.
\vspace{10pt}
\newline
Nach dem Treffen trafen sich Hagrid, Jacobs Geschwister und Merula mit Dumbledore. Es wurde klar, dass alle viel Erfahrung gesammelt hatten. Der Vortrag war auch ein voller Erfolg, da viele Zauberer sich bereit erklärten die Pferde zu analysieren und zu trainieren, obwohl sie die Tiere teilweise gar nicht sehen konnten. Auch gab man den "dunklen Pferden" Freundlichkeit und Respekt und nicht Angst und Ekel, so wie vorher.
\subsubsection*{Suche nach Rakepick}
%ABSATZ
Kurz vor seinem zweiten Abgang von der Schule nach vielen Jahren, erteilte Dumbledore Rita Kimmkorn die "regelmäßige Erlaubnis", dass Schloß Hogwarts zu besuchen, um die Schüler einzuladen und sie einzuladen an einem "Testwettbewewerb" teilzunehmen, bei dem jeder Gewinner einen Artikel auf der ersten Seite im Tagespropheten bekäme. Als sie zum ersten Mal in die Schule kam, um den Schülern den Grund für ihren Besuch zu erzählen, war Dumbledore bereits gegangen, um die Suche nach Patricia Rakepick fortzusetzen. Es dauerte fast das gesamte verbleibende Schuljahr, bis er sie schließlich fand. Der Schulleiter
%SEITE11
sprach sie jedoch nicht sofort an, weil er neugierig war, warum seine ehemalige Schülerin sich so versuchte vor ihm zu verstecken. Er folgte ihren Bewegungen durch die halbe Welt und sah sie an Orten, wo er sie nicht erwartet hätte. Dumbledore beschloss ihr zu vertrauern, obwohl er viele Fragen an sie hatte. Er hatte sie zu diesem Zeitpunkt bis nach Südamerika verfolgt, wo er sich ihr schließlich näherte. Sie begann dort einige antike Ruinen zu untersuchen, die kürzlich unter der brasilianischen Zauberschule Castelobruxo im Herzen des Amazonas-Regenwaldes entdeckt wurden. Rakepick konnte Brasilien jedoch nicht verlassen, bevor sie den Auftrag erledigt hatte. Jedoch einigten sich beide, dass sie zu Beginn des folgendes Schuljahres ihre Vereinbarung präsentieren würde.
\vspace{10pt}
\newline
Als bekannt wurde, dass Dumbledore für eine Zeitlang nicht mehr in der Schule sein würde, hatten viele Schüler Bedenken, dass die Vorfälle wieder öfters sein würden. Jedoch gab es seit dem Abgang von Dumbledore sogar weniger Irrwichte, als zuvor. Professor Snape hatte von den Schülerinnen Merula Snyde und Ismelda Murk folgende Gerüchte zu hören bekommen: Jacobs Geschwister und vier andere Schüler waren in die Schulbibliothek eingedrungen und in die Verbotene Abteilung gegangen, wo sie ein zweites Gewölbe finden konnte. Dumbledore war über die verbesserten Bedingungen erfreut, gab jedoch auch die Anweisung, niemand von dem Regelbruch der Schüler zu erzählen. Danach machte sich Dumbledore auf die Suche nach Jacobs Geschwister, den er bei einer Siegesfeier mit seinen Freunden bei "Die drei Besen" fand. Dort bat er sie ihn später in seinem Arbeitszimmer zu besuchen, um die Heldentat des Schülers zu besprechen, durch die erneut ein Fluch gebrochen wurde, der die gesamte Schule bedroht hatte. Auch wollte er Informationen über das Gewölbe erhalten. Auch nutze er aber die Gelegenheit, um ihr Fehlverhalten anzusprechen und ihnen zu sagen, dass er beim nächsten Mal wirklich eine Strafe aussprechen müsste. Zum Schluss verteilte er noch hundert Hauspunkte, für die Rettung der Schule.
\vspace{10pt}
\newline
Zum Ende des Schuljahres lobte der Schulleiter die Schüler für ihre Tapferkeit und gab den Gewinner des Hauspokals bekannt.

\subsubsection*{1987 - 1988}
%ABSATZ
Dumbledore redete in seiner Anfangsrede wieder davon, dass es Strafen geben würde, wenn man die Gewölbe betritt. Er war nach eigenen Angaben auch sehr besorgt über die Besessenheit von Jacobs Geschwister die verfluchten Gewölbe zu suchen und zu zerstören. Schon schnell lud Dumbledore Rakepick ein, zur Schule zu reisen. Sie kam am 1. September 1987 zusammen mit den Schülerin der Großen Halle an. Der Schulleiter war neugierig über ihre Meinung zu den verfluchten Gewölben. In seiner Rede gab er ihr dann die Aufgabe, alles rund um die verfluchten Gewölbe zu verwalten und sich darum zu kümmern. Auch bezeichnete er sie offiziell als Personal der Schule. Dies gab ihr auch die Erlaubnis Schüler zu bestrafen und ihnen Befehle zu geben. Auch durfte sie Hauspunkte geben und abgeben. Er stellte sie in seiner Rede vor und erzählte, warum er sie eingeladen hatte und auch, dass Schüler gene mit ihr reden dürften.
\subsubsection*{1988 - 1989}
%ABSATZ
Dumbledores Entscheidung, Rakepick so viel Verantwortung zu geben, erwies sich als falsch, da sie eine von Rs Mitgliedern war und vorhatte Jacobs Geschwister und seine Freunde zu töten. Dies gelang ihr jedoch nicht, jedoch griff sie Merula mit dem Cruciatus-Fluch an. Dumbledore war zu diesem Zeitpunkt nicht anwesend und konnte es deshalb nicht verhindern, weshalb er dafür verantwortlich war. Er gab zu, dass er es bereut hätte sie einzustellen und sie fangen würde, nachdem sie zuvor geflohen war.
\subsubsection*{1989 - 1990}
%ABSATZ
Das nächste Jahr war ungefähr genauso problematisch wie das Vorjahr, da Jacobs Geschwister und seine Freunde entschlossener den je waren, R zu stoppen und das das letzte verfluchte Gewölbe zu finden, obwohl Dumbledore versucht hatte, sie dazu zu bringen sich auf ihr Studium zu konzentrieren. Er war noch beunruhigter darüber, dass sich Ben Copper und Merula Snyde aufgrund der Ereignisse des letzten Jahres verändert hatten. Auch beunruhigte es ihn sehr, dass ein Askaban-Flüchtling, der sich als verwiesener Schüler der Mahoutokoro-Schule für Zauberei herausstellte, ab sofort mit R zusammenarbeitete. Als dann Rowan Khanna von Rakepick getötet wurde, gab es eine riesige Gedenkfeier für ihn.
\subsubsection*{1991 - 1992}
%ABSATZ
Als Harry Potter Hogwarts das erste Mal betrat, war Dumbledore sehr darüber enttäuscht, dass dieser nicht sehr glücklich war. Dies lag daran, dass er durch seine Tante und seinen Onkel sehr vernachlässigt wurde. In diesem Jahr entwickelte Lord Voldemort einen Plan, um wieder seine menschliche Form zurück zu gewinnen. Dumbledore und sein Freund Nicolas Flamel begannen zu vermuten, dass der körperlose Dunkle Lord plante, den Stein der Weisen zu stehlen, der ihn wieder seine volle Macht bringen würde. Dumbledore hatte den Stein vorher in ein Sicherheitsgewölbe in die Gringotts Zaubererbank in der Winkelgasse gebracht, aber später zurück nach Hogwarts geholt. Genau an diesem Tag wurde zufällig ein Einbruchsversuch am Tresor unternommen. Der Stein wurde von Hagrid zurückgeholt, der auch Harry mitnahm, um seine Schulsachen zu kaufen.
\vspace{10pt}
\newline
Dumbledore setzte mehrere Lehrer (McGonagall, Flitwich, Sprout, Snape und fehlerhafter Weise Quirrell) in Hogwarts ein, um den Stein zu schützen. Die Lehrer legten verschiedene magische Schutzmethoden an und Dumbledore legte den stärksten
%SEITE12
Schutz auf den Stein. Er verzauberte den Spiegel Nerhegeb so, dass er den Stein der weisen in sich hatte. Jedoch konnte ihn eine Person nur erhalten, wenn sie ihn nicht wollte, sondern den Stein von dem Bösen schützen wollte. Dumbledore vermutete, dass der Professor für Verteidigung gegen die dunklen Künste Quirinius Quirrell den Stein aus dem Verließ bei Gringotts stehlen wollte, während er für den geschwächten Lord Voldemort arbeitete. Deshalb bat er Snape dem Professor im Auge zu behalten. Später im Schuljahr schenkte er Harry zu Weihnachten anonym den Unsichtbarkeitsumhang, der zuvor James Potter gehört hatte.
\vspace{10pt}
\newline
Harry Potter war häufig nachts im Schulgebäude unterwegs, weshalb er nach kurzer Zeit den Spiegel Nerhegeb fand. Diesen besuchte er immer öfters, da er in diesem die Gesichter seiner Eltern sah. Eines Nachts fand Dumbledore Harry beim Spiegel und erzählte in, was dieser Spiegel zeigte. Da Harry aber eine Art Sucht bekam seine Eltern zu sehen, fand Dumbledore es sinnvoll den Spiegel in einen anderen Raum zu bringen und bat Harry nicht mehr nach dem Spiegel zu suchen. Harry fragte Dumbledore bevor er ging, was er im Spiegel sah. Dumbledore antwortete, dass er ein Paar dicke Wollsocken in der Hand hätte und sagte, dass "man nie genug Socken haben kann". Dumbledore scherzte jedoch, da die Sache, die er sah, sehr persönlich war. Dumbledore sehnte sich immer noch nach Grindelwalds Liebe und sah sich mit seinem ehemaligen Liebhaber im Spiegel. Auch sah er seine gesamte Familie, die sich alle mochten und ohne Schmerzen und Leiden waren. Jedoch ist nicht bekannt, ob er auf diesem Bild mit Grindelwald in einer Beziehung war.
\vspace{10pt}
\newline
Am Ende war es Harry, der den Stein der Weisen vor Quirrell und Voldemort beschützte. Voldemort scheiterte aus den gleichen Gründen, aus denen er Harry zuvor nicht ermordet hatte: wegen Lily J. Potters Liebe zu Harry. Als Voldemort, der den Körper von Quirrell befallen hatte, Quirrell befahl, den Jungen anzugreifen, tötete ihn der Opferschutz, der ihm zum Zeitpunkt des Todes seiner Mutter eingeflößt wurde, und zwang den Geist des Dunklen Lords zur Flucht. Als Harry schließlich wieder im Krankenflügel aufwachte, fragte er Dumbledore, warum Voldemort versucht hatte, ihn als Baby zu ermorden. Dumbledore glaubte nicht, dass Harry bereit war, die Wahrheit über die Prophezeiung zu erfahren, die Sybill Trelawney gemacht hatte. Er antwortete ihm, dass er dies wissen würde, wenn er älter wäre (obwohl sich dies später als Fehler herausstellen würde).
\vspace{10pt}
\newline
Am Ende des Jahres verlieh Dumbledore den Gryffindor-Schülern Harry Potter (60), Ron Weasley (50), Hermine Granger (50) und Neville Longbottom (10) zusätzliche 170 Hauspunkte für Tapferkeit und logische Fähigkeiten. Dank diesen Punkten gewann Gryffindor den Hauspokal und zerstörte somit Slytherin den Traum den Pokal zum siebten Mal zu gewinnen. Dies wirkte möglicherwiese etwas unprofessional, da er Gryffindor sichtbar bevorzugte.
\vspace{10pt}
\newline
Kurz vor Beginn des nächsten Jahres fand Dumbledore, der sich nie von Gilderoy Lockharts Werken täuschen ließ, bald die Wahrheit über ihn heraus. Er fand dies heraus, weil Dumbledore zwei Zauberer kannten, von denen Lockhart die Erinnerungen gelöscht hatte. Er erkannte, dass Lockhart in seinem Wunsch nach Ruhm zu weit gegangen war und beschloss ihn aufzuspüren und für seine Verbrechen bezahlen zu lassen. Deshalb bot er ihm die Stelle als Professor für Verteidigung gegen die dunklen Künste an, da diese Position wieder besetzt wurde. Zuerst war der Autor skeptisch, da er Angst hatte, seine Lügen würden wegen seines wenigen Könnens bekannt werden, aber als Dumbledore ihm sagte, dass Harry Potter sein Schüler war, nahm Lockhart das Angebot an.
\vspace{10pt}
\newline
Dumbledores Entscheidung kam bei den meisten anderen Lehrern nicht sehr gut an. McGonagall beschwerte sich, da die Schüler sicherlich nicht viel lernen würden. Darauf antwortete Dumbledore, dass man von einem schlechten Lehrer viel lernen kann: was man nicht tun sollte und wie man nicht sein sollte
\subsubsection*{1992 - 1993}
%ABSATZ
Im Schuljahr 1992 wurden mehrere Schüler von einem Tier angegriffen, das angeblich aus der Kammer des Schreckens entlassen worden war. Anfangs war noch Harry Potter im Verdacht, der Erbe Slytherins zu sein, der die Kammer nun geöffnet hat, da sich herausstellte, dass er Parsel sprechen kann. Doch nach einem Gespräch mit Dumbledore in seinem Büro machte der Schulleiter Harry klar, dass er Harry nicht für schuldig befindet. Unter der Führung von Lucius Malfoy stimmt der Hogwarts-Schulbeirat dafür, Dumbledore als Schulleiter zu "beurlauben", da er nicht im Stande war, den Täter zu finden, der die Kammer des Schreckens geöffnet hatte. Kurz überlegte man sogar die Schule zu schließen. Der Zaubereiminister Cornelius Fudge ließ den Wildhüter Rubeus Hagrid nach Askaban bringen, da er der Meinung war, dass Hagrid die Kammer bereits 1943 geöffnet hatte.
\vspace{10pt}
\newline
Später wurde bekannt, dass die Erstklässlerin Ginevra Weasley in den Besitz von Tom Riddles Tagebuch gekommen war, einem verzauberten Objekt, dass in der Lage war, mit demjenigen, der darin schrieb, zu interagieren. Ginny verbrachte langsam so viel Zeit mit dem Buch und schrieb so viel, dass das Buch ihre physische Form annehmen konnte und sie so in die Kammer des Schreckens führte. Harry Potter war schlussendlich in der Lage, die Kammer selbst zu finden und das Tier, das dort wohnte, nämlich Salazar Slytherins Basilisk, zu töten. Harry zerstörte auch das Tagebuch mit einem Zahn des Basilisken, rettete Ginny und besiegte die fast vollständig geformte Erinnerung von Riddle. Harry war jedoch nur in der Lage diese Leistung zu erbringen, weil er Dumbledore eine extreme Loyalität entgegenbrachte: denn nur jemand, der dem Schulleiter wirklich treu ist, ist in der Lage sein Haustier Phönix Fawkes zu beschwören, der Potter die nötigen Waffen zu Verfügung zu stellen, um Riddle un den Basilisken zu besiegen.
%SEITE13
\vspace{10pt}
\newline
Dumbledore kehrte auf unbekannte Weise zu seiner Position als Schulleiter zurück, nachdem die Sache geklärt wurde. Er lobte sowohl Harry als auch Ron für ihre Lösung des Rätsels mit jeweils zweihundert Hauspunkten. Dies sicherte Gryffindor zum zweiten Mal in Folge den Hauspokal. Dumbledore sorgte auch dafür, dass Hagrid aus Askaban freigelassen wurde, indem er Rons Eule die Entlassungspapiere ausliefern ließ. Auch ließ Dumbledore alle Prüfungen in diesem Jahr ausfallen. Über diese Nachricht waren die meisten Schüler außer Hermine erfreut. Lucius Malfoy wurde ebenfalls am Ende des Jahres als Elternrat Sprecher entlassen, was Draco nicht sehr erfreute.
\vspace{10pt}
\newline
Dumbledore war jedoch sehr über die Phänomene, die Harry beschrieb, sehr besorgt. Es war bisher unbekannt, dass eine bloße Erinnerung eine physische Form annehmen könnte. Dumbledore begann dann zu vermuten, dass Riddles Tagebuch ein Horkrux war. Später verstand er, dass dies nicht das einzige sein konnte, weil Lord Voldemort das Buch viel zu schlecht beschützt hatte.
\vspace{10pt}
\newline
Während seines Gesprächs mit Harry in seinem Büro, erzählte Dumbledore Harry fast von seiner Verbindung mit Voldemort, verzichtete jedoch darauf, da er das Gefühl hatte, dass der Junge noch zu jung war, um die gesamte Wahrheit zu erfahren und er wollte Harry seinen Triumph geniesen lassen.
\subsubsection*{1993 - 1994}
%ABSATZ
Dumbledore hatte als Nachfolger für Hagrid Silvanus Kesselbrand als Lehrer für Pflege Magischer Geschöpfe eingestellt. Dumbledore hatte nun auch als Ersatz für Lockhart Remus Lupin als Lehrer für Verteidigung gegen die Dunklen Künste eingestellt. Auch beschloss das Britische Zaubereiministerium die Sicherheit an der Hogwarts-Schule für Hexerei und Zauberei zu verbessern. Anlass dafür war, dass Sirius Black, der wegen Mordes an Peter Pettigrew und zwölf Muggel zu lebenslänglicher Haft in Askaban verurteilt worden war, aus Askaban geflohen war. Das Ministerium befürchtete, dass er Harry verfolgen würde, um ihn zu töten. Deshalb schickten sie Dementoren nach Hogwarts, um ihn wieder festzunehmen. Dumbledore war damit mehr oder weniger einverstanden, dass die Dementoren auf dem Schulgelände waren, weigerte sich jedoch, sie in die Schule zu lassen.
\vspace{10pt}
\newline
Die Dementoren verstießen jedoch gegen Dumbledores Vorschriften, als sie das Quidditch-Spiel zwischen Gryffindor und Hufflepuff störten, was dazu führte, dass Harry bewusstlos wurde und vom Himmel stürzte. Dumbledore konnte Harry retten, indem er ihn nicht auf den Bonde fallen ließ und erzeugte einen Patronus, der die Dementoren verscheuchte.
\vspace{10pt}
\newline
Zum Ende des Schuljahrs wurde Harry und seinen Freunden, klar dass Sirius Black unschuldig war, weil diese Verbrechen von Peter Pettigrew begannen wurden, der sich seit zwölf Jahren als Krätze tarnte. Dies konnte jedoch nicht bestätigt wurden. Später wurde Sirius gefangen genommen, im höchsten Turm von Hogwarts eingesperrt und zum Dementorenkuss verurteilt. Dumbledore glaube Harry und seinen Freunden sofort, dass Sirius unschuldig war. Der Minister war jedoch anderer Meinung. Dumbledore war auch nicht in der Lage den Minister umzustimmen.
\vspace{10pt}
\newline
Deshalb forderte Dumbledore Hermine Granger auf, ihren Zeitumkehrer zu nutzen, um Sirius zu retten. Harry und Hermines Rettungsmission war ein Erfolg, weshalb Sirius entkommen konnte. Am Ende des Schuljahres sorgte Dumbledore dafür, dass die Dementoren nicht mehr auf das Gelände durften, weil sie fast Harrys Seele entfernt hatten.
\vspace{10pt}
\newline
Nachdem Severus Snape bekannt gemacht hatte, dass Remus Lupin ein Werwolf war, entschloss sich dieser wieder die Schule zu verlassen. Dumbledore erkannte kurz danach, dass Harry beunruhigt schien. Harry erzählte Dumbledore, dass er sich schuldig fühlte, Pettigrew die Flucht erlaubt zu haben. Nach Sybill Trelawneys zweiten Prophezeiung würde Pettigrew wieder ein Diener von Voldemort werden und in zurück zu seiner alten Macht bringen. Dumbledore versicherte Harry daraufhin, dass er etwas Wunderbares getan hatte, indem er Pettigrew verschont hatte. Auch sagte er, dass Harry eines Tages froh darüber sein würde.
\vspace{10pt}
\newline
Am Ende des Jahres begann Dumbledore sich Sorgen zu machen, dass er Harry immer noch nichts über die erste Prophezeiung gesagt hatte.
\subsubsection*{1994 - 1995}
%ABSATZ
Vor Beginn des neuen Schuljahres wurde das Finale der Quidditch-Weltmeisterschaft 1994 von Todessern unterbrochen und das Dunkle Mal beschworen. Dumbledore vermutete deshalb das der zweite Aufstieg von Lord Voldemort nahe war. Später fand in Hogwarts das erste Trimagische Turnier seit 1792 statt. Es wurde beschlossen, dass nur volljährige Schüler ihren Namen in den Feuerkelch werfen dürfen, da es zu viele Gefahren gab und es regelmäßig Tote gab. Dumbledore selbst zog eine Alterslinie um sicherzustellen, dass die Regel nicht gebrochen werden konnte. In diesem Jahr stellte Dumbledore seinen alten Freund Alastor Moody, einen ehemaligen Auror, als neuen Professor für Verteidigung gegen die dunklen Künste an. Moody nahm die Stelle an, aber nur weil er Albus eine Freude machen wollte.
\vspace{10pt}
\newline
Ein Tag bevor Moody nach Hogwarts reisen sollte, wurde er jedoch von Barty Crouch Jr. und Peter Pettigrew überwältigt. Auf Befehl von Voldemort braute sich Barty Crouch Jr. einen Vielsafttrank und sollte sich damit als Moody ausgeben und seinen Platz in Hogwarts einnehmen. Crouch, in der Gestalt von Moody, täuschte erfolgreich den Feuerkelch und warf Harrys Namen in den
%SEITE14
Kelch. Crouch half Harry heimlich während des Turniers, sodass er als erster den Trimagischen Pokal berühren sollte. Crouch hatte diesen so verhext, dass er zu einem Portschlüssel wurde.
\vspace{10pt}
\newline
Als Harry und sein Gegner Cedric Diggory die Trophäe gleichzeitig berührten, wurden sie auf den Friedhof von Little Hangleton gebracht. Dort wurde Diggory auf Befehl von Lord Voldemort getötet und Harrys Blut wurde als letzte Zutat in einen Trank gegeben, mit dem der Dunkle Lord zur vollen Kraft zurückkehrte. Als Voldemort Harry töten wollte, konnte Harry mit dem Portschlüssel entkommen und nach Hogwarts zurückkehren.
\vspace{10pt}
\newline
Als Crouch, immer noch in der Gestalt von Moody, Harry von Dumbledore entfernte (etwas, was der echte Moody in einer solchen dramatischen Situation niemals getan hätte), erkannte der Schulleiter, dass dies nicht Moody sein konnte, der das gesamte Jahr in der Schule gewesen war. Dumbledore, Snape und McGonagall überwältigten den Todesser schnell. In diesem Moment erkannt Harry, warum Dumbledore der einzige Zauberer war, den Voldemort flüchtete. Barty erzählte dank Veritaserum die gesamten Handlungen von Voldemort.
\vspace{10pt}
\newline
Trotz aller beteiligten Zeugen, weigerte sich der Zaubereiminister Cornelius Fudge zu glaube, dass Lord Voldemort an die Macht zurückgekehrt war. Auch setzte er Crouch einem Dementorenkuss aus, bevor er überhaupt vor einem Gericht stand. Dumbledore war sich trotzdem sehr sicher über die Rückkehr von Voldemort und sorgte deshalb dafür, dass der Orden des Phönix wieder aufgerüstet wurde. Dumbledore entschied sich für die Reform des Ordens, da er sich ziemlich sicher über den Beginn des Zweiter Zaubererkriegs war.
\vspace{10pt}
\newline
Wieder war sich Dumbledore zwar bewusst, dass Harry bald von der Prophezeiung erfahren musste, aber entschied sich erneut dagegen, weil er Harry nicht eine erneute Last neben dem Tod von Cedric Diggory geben wollte.



\subsubsection*{\large Zweiter Zaubererkrieg (1995–1997)}
\subsubsection*{Cornelius Fudge Verleugnung}
%ABSATZ
Das Ministerium begann ziemlich schnell die Behauptungen von Dumbledore und Harry, dass der dunkle Lord zurückgekehrt war, zu bestreiten und starteten eine Kampagne, die Dumbledore und Harry als Lügner darstellten. Deshalb verlor Dumbledore auch seine Funktion als Hexenmeister des Zaubergamots und wurde als Ganz hohes Tier entfernt. Dies war vor allem eine Strafe dafür, dass Dumbledore die Propaganda nicht stoppte. Dumbledore gab bekannt, dass ihm dies egal sei, solang man ihn nicht von den Schokofroschkarten streichen würde. Dumbledore war der Meinung, dass Fudge nicht aufgrund eines Imperius-Fluch diese Meinung hatte, sondern das dies sein eigener Wille sei.
\vspace{10pt}
\newline
Vor Beginn des Schuljahres wurde Dumbledore zum Geheimhalter des Hauptquartiers des Orden des Phönix ernannt und stellte sicher, dass Harrys Freunde keine Infos über den Orden bekannt gaben.
\vspace{10pt}
\newline
Dumbledore nahm an Harry Potters Anhörung im Britischen Zaubereiministerium teil und kam zum Prozess drei Stunden früher um fünf Uhr morgens an. Er präsentierte Arabella Figg zu Harrys Verteidigung, bot an, , Dobby anzurufen, um zu bezeugen, dass Harry seit 1992 keinen Schwebezauber benutz hatte und erinnerte Fudge daran, dass er selbst Harrys versehentliches Aufblasen von Magdalene Dursley begnadigt hatte. Er kommentierte auch das ungewöhnliche Verhalten des Gerichts, das eine vollständige strafrechtliche Anhörung für eine Fall minderjähriger Magie abhielt. Nachdem das Zaubergamot zu Harrys Gunsten entschieden hatte, ging Dumbledore wieder, ohne mit Harry nur ein Wort gesprochen zu haben, oder ihn nur angesehen zu haben.
\vspace{10pt}
\newline
Als es wieder Zeit wurde die Vertrauensschüler auszusuchen, entschied sich Dumbledore gegen Harry, da dieser dadurch zu sehr abgelenkt werden würde. Stattdessen bekam Ron den Job. Diese Entscheidung verwirrte mehrere Leute.
\vspace{10pt}
\newline
Dumbledore bekam Angst vor der Verbindung zwischen dem Dunklen Lord und Harry, weshalb er ihn das gesamte Schuljahr absichtlich mied. Als die Verbindung immer öfters aufgebaut wurde, ließ Dumbledore Severus Snape versuchen, Harry die Kunst der Okklumentik beizubringen. Dumbledore wollte dies nicht selbst machen, da er befürchtete, dass Voldemort über Harry dann auch in seine Gedanken schauen könnte. Das Ministeriumbegann dann sich in Hogwarts einzumischen, indem sie Fudges Untersekretärin Dolores Umbridge zur Professorin für Verteidigung gegen die dunklen Künste ernannten.
\vspace{10pt}
\newline
Umbridge begann ihre Amtszeit in Hogwarts, indem sie Dumbledores Rede zu Beginn des Schuljahres unhöflich unterbrach. Nach diesem Vorfall wurde es immer schlimmer. Umbridge und Fudge begannen neue Regeln und Vorschriften zu verabschieden, die Dumbledore viele Rechte als Schulleiter nahmen, die dafür aber Umbridge bekam. Irgendwann während des Schuljahrs wurde ihr dann die Position der Großinquisitorin übergeben. Als Reaktion auf Umbridge und die Verweigerung des Ministeriums, praktische Verteidigungsmagie zu lehren, gründeten Harry, Hermine und Ron eine Gruppe für Studenten, die sich heimlich trafen und gemeinsam praktische Verteidigungsmagie lehrten. Mit Harry als Anführer und Lehrmeister wurde die Organisation Dumbledores Armee genannt, um für den Kampf gegen Voldemort und seine Todesser vorbereitet zu sein, die das Ministerium leugnete.
%SEITE15
\vspace{10pt}
\newline
Als diese illegale Organisation von Umbridge entdeckt wurde, übernahm Dumbledore die Verantwortung, obwohl er nichts von der Organisation wusste. Harry versuchte dies abzustreiten und die Schuld auf sich zu nehmen, worauf Dumbledore darauf verwiese, dass es "Dumbledores Armee, nicht Potters Armee" genannt wurde. Fudge befahl deshalb seine Festnahme, während Dumbledore erklärt, dass er sich nicht mit einem Zauber überwältigen ließe und disapperierte. Danach arbeitete er vor allem nur noch für den Orden. Nachdem Dumbledore "entlassen" wurde, wurde Umbridge zur neuen Schulleiterin von Hogwarts ernannt und begann ihre Macht gegenüber Studenten und dem Personal auszunutzen.
\subsubsection*{Duell mit Lord Voldemort}
%ABSATZ
Leider, wie Dumbledore bereits befürchtet hatte, handelte Voldemort schließlich so, um die Verbindung zwischen Harry zu nutzen. Mit Legilimentik schuf Voldemort eine Version, die darauf hinwies, dass sein Pate Sirius Black von Voldemort in der Mysteriumsabteilung gefangen und gefoltert wurde. Harry, Ron, Hermine, Ginevra Weasley, Luna Lovegood und Neville Longbottom eilten zum Ministerium in der Hoffnung, dass sie Sirius retten konnten. Die sechs Schüler wurden jedoch von einer Gruppe Todesser angegriffen, die hofften die Aufzeichnungen der Prophezeiungen über Harry und den Dunklen Lord zu bekommen. Voldemort selbst war anfangs nicht anwesend, da er es für zu gefährlich hielt, selbst ins Ministerium einzudringen. Glücklicherweise geling es Harry vor seiner Abreise Snape darauf aufmerksam zu machen, dass Sirius angeblich in Gefahr war. Diese Information gab Snape an den Orden weiter und so kamen mehrere Mitglieder ins Ministerium, um mit den Todessern zu kämpfen.
\vspace{10pt}
\newline
Gegen Ende der Schlacht traf Dumbledore selbst ein und nahm schnell alle Todesser gefangen, außer Bellatrix Lestrange, die kurz darauf Sirius ermordet hatte und mehreren Zaubern auswich. Dumbledore wickelte die Todesser in eine Schnur ein, die mit einem Anti-Disapparier-Fluch ausgestattet war. Danach ging er ins Atrium des Ministeriums, wo Bellatrix Voldemort begleitet hatte, der versuchte Harry zu töten.
\vspace{10pt}
\newline
Voldemort Todesfluch wurde jedoch gestoppt, indem Dumbledore schnell eine goldene Figur zauberte, die Harry vor den Fluch schützte. Er nutzte die Statuen des Brunnen der magischen Geschwister, um Bellatrix zu stoppen und die Beamten des Ministeriums zu alarmieren. Nachdem Voldemort schließlich im Ministerium war, nutzte Dumbledore die Möglichkeit, Fudge die Möglichkeit zu geben, Voldemorts Rückkehr real zu sehen. Auch kämpfte Dumbledore gegen den Dunklen Lord, um Harry zu schützen und dafür zu sorgen, dass er lange genug im Ministerium blieb. Dumbledore präsentierte in diesem Duell seine außergewöhnlichen magischen Fähigkeiten, indem er Voldemorts Angriffen ruhig entgegentrat oder ihnen auswich, aber sofort mit einem eigenen Angriff antwortete, der Voldemort immer dazu zwang, in die Defensive zu gehen. Dumbledore kämpften in diesem Duell jedoch nicht mit Zaubern, die Voldemort töten konnten, was diesen verwunderte. Schon bald gelang es Dumbledore, den Dunklen Lord in einer Kugel aus Wasser zu fangen. Vor dieser Tätigkeit wurde Dumbledore jedoch von Fawkes gerettet, der einen Todesfluch abfing. Voldemort beschwor eine feurige Schlange, bemerkte jedoch, dass es ihm nicht gelingen konnte, seinen ehemaligen Lehrer zu überwältigen. Deshalb gab er auf und verschwand.
\vspace{10pt}
\newline
Bevor Voldemort jedoch aus dem Ministerium flog, kehrte er kurz in Harrys Gedanken, in der Hoffnung, dass Dumbledore den Teenager opfern würde, um Voldemort zu töten. Dumbledore fiel jedoch nicht auf diesen Trick rein. Glücklicherweise musste Dumbledore nicht auf gefährliche Mittel zurückgreifen, die er benötigt hätte, um Voldemort von Harry zu entfernen, da die enorme Kraft von Harrys Trauer um seinen verstorbenen Paten Voldemort Schmerzen zugefügt hatte. Deshalb blockierte Voldemort die Verbindung und verschwand. Nachdem Voldemort die Prophezeiung nicht bekommen hatte und auch Harry nicht töten konnte, floh er mit Bellatrix aus dem Ministerium. Vorher wurde er jedoch von Fudge und seinen Mitarbeitern gesehen. Nachdem Dumbledores Glaubwürdigkeit wieder hergestellt war, war Fudges Glaubwürdigkeit verschwunden. Deshalb verlor er seinen Posten und Rufus Scrimgeour wurde neuer Zaubereiminister. Zuerst stellte Dumbledore erfolgreich sicher, dass Fudge Umbridge als Professorin entließ und Hagrid aus Askaban entlassen wurde. Nachdem Dumbledore wieder offiziell Schulleiter wurde, stellte er Hagrid sofort wieder an. Schließlich redete Dumbledore mit dem verstörten harry und klärte ihn über die Verbindung mit Voldemort auf und erzählte ihm vin der Prophezeiung. Vor Jahresende ließ er auch das Inquisitionskommando auflösen und sorgte dafür, dass diese all ihre Schäden rückgängig machten.

\subsubsection*{Privatunterricht für Harry Potter}
%ABSATZ
Im Sommer 1996 fand Dumbledore ein Horkrux von Voldemort im Haus der Gaunts. Dumbledore reiste nach Little Hangleton, wo er die Überreste der Hütte fand, die hinter Unkraut und einem Busch versteckt waren. Voldemort hatte viele mächtige Schutzzauber um das ehemalige Vorfahren gelegt, jedoch erwies sich Dumbledore als geschickt genug, diese lebendig zu passieren. Der Horkrux war ein Ring, der früher Vorlost Gaunt gehörte. Auch bemerkte er, dass der Stein im Ring der Stein der Auferstehung war, eines der Heiligtümer des Todes.
\vspace{10pt}
\newline
Der Legende nach hatte dieser Stein die Kraft, die Toten wiederzubeleben, jedoch war dies in Wirklichkeit nicht mehr ein Abbild der Person. Seinen Wunsch, seine Familie wieder zu sehen, setzte seinen gesunden Menschenverstand für einen Moment außer Kraft, weshalb er den Ring anlegte und vergaß, dass der Ring von Voldemort verflucht war. Der Fluch breitete sich schnell aus, und wäre er kein so talentierter Magier gewesen, wäre er innerhalb weniger Augenblicke gestorben. Trotz dieser Verletzung zerstörte Dumbledore den Ring und zerstörte somit ein Teil von Voldemorts Selle. So kam er wieder ein Schritt näher an die Niederlage von Voldemort.
%SEITE16
\vspace{10pt}
\newline
Dumbledore kehrte schnell nach Hogwarts zurück, wo es Snape gelang, den Fluch nur in Dumbledores Hand zu halten. Sowohl Snape als auch Dumbledore wussten jedoch, dass es nicht lange brauchen würde, bis es sich auf den Rest des Körpers ausbreitete. Dumbledore wusste, dass der Fluch ihn töten wurde und Snape bestätigte, dass er nicht länger als ein Jahr überleben könnte. Deshalb bat er Snape, dass er ihn früher töten würde und nicht Draco Malfoy. Malfoy hatte nämlich die Aufgabe erhalten, den Schulleiter zu ermorden und Dumbledore wollte dies verhindern, da er befürchtete, dass Draco dies nicht verkraften würde. Deshalb musste Snape diese Aufgabe übernehmen. Auch würde es ihn helfen, nicht aus seiner Tarnung als Spion des Ordens aufzufliegen und ein besseres Bündnis zu Voldemort aufzubauen. Snape wollte diese Aufgabe anfangs nicht machen, jedoch erinnerte Dumbledore ihn daran, dass er ihn so vom Leiden erlösen würde. Auch erzählte er Snape, dass Harry sterben muss, damit Voldemort sterblich wird. Während dieser Zeit teilte Dumbledore Harry auch seine Erkenntnisse über Voldemort mit, indem er ihm die verschiedenen Erinnerungen zeigte, die er gesammelt hatte. So gab er Harry ein besseres Wissen über Voldemorts Psychologie, das ihm in seinen zukünftigen Duelle helfen würde. Auch sollte er mit diesen Informationen das Werk von Dumbledore fortsetzen.
\vspace{10pt}
\newline
Dumbledore gab Harry später Privatunterricht für Harry über Voldemorts Vergangenheit, sein Leben und Horkruxe. Besonders hilfreich war dafür das Denkarium im Schulleiterbüro, wo man sich alte Erinnerungen anschauen konnte. Dumbledore berichtete Harry, dass er es sich zur Aufgabe gemacht hatte, etwas über Voldemorts Vergangenheit herauszufinden, um mögliche Horkrux Verstecke zu finden. Er befragte auch Menschen, die Voldemort als Kind und Teenager kannten und erhielt von ihnen wichtige Informationen.
\vspace{10pt}
\newline
Er sammelte Informationen von Bob Ogden, Morfin Gaunt, Horace Slughorn und der Hauselfe Hokey. Anhand dieser Erinnerungen konnte Dumbledore erschließen, dass Voldemort sechs Horkruxe erschaffen hatte. Dies schockierte Dumbledore. Er fand heraus, dass viele seiner Horkruxe mit Hogwarts zu tun haben. Er vermutete, dass es in jedem Haus ein Horkrux geben würde, außer in Gryffindor, weil es nur den Sprechenden Hut und das Gryffindor Schwert als Artefakt gab und Voldemort keinen Zugang zu diesen hatte. Slughorn wurde von Dumbledore als neuer Zaubertränke Professor eingesetzt, während Snape Lehrer für Verteidigung gegen die Dunklen Künste wurde.

\subsubsection*{Suche nach den Horkruxen}
%ABSATZ
Nach langem suchen fand Dumbledore schließlich ein weiteres Horkrux, was sich in einer Höhle am Meer befand, die Voldemort als junges Waisenkind einmal besucht hatte. Dumbledore hatte Harry versprochen ihn in die Höhle zu begleiten und ihm dabei zu helfen, den Horkrux zu zerstören. Als er und Harry Anfang Juni 1997 an einem Abend von Hogwarts aus aufbrachen, apparierte er vor die Höhle und konnte aufgrund ihrer schützenden Verzauberungen nicht direkt eintreten. Als sie in der Höhle waren, fand Dumbledore eine versteckte Tür, die sich erst öffnen ließ, wenn sie ein Blutopfer erhielt.
\vspace{10pt}
\newline
Als nächstes befanden sie sich in einer riesigen Kammer mit einem unterirdischen See, in dessen Zentrum sich eine einsame Insel befand, von der ein schwaches grünes Licht sichtbar war. Nach einem gescheiterten Versuch von Harry, den Horkrux mit einem Accio-Zauber zu beschwören, fand Dumbledore ein gut verstecktes Boot, das ihnen eine sichere Reise über den See ermöglichte. Während der Überfahrt bemerkte Harry die dritte Verteidigungslinie. Der See war mit Inferi gefüllt, die jeden angreifen würden, der das Wasser des Sees berührte. Dumbledore wies Harry an, Feuer als Waffe der Wahl gegen Inferi einzusetzen, falls es notwendig sein würde.
\vspace{10pt}
\newline
Als sie auf der kleinen Insel ankamen, fanden sie ein Becken mit einem leuchtendem grünen Trank. Diese letzte Verteidigung konnte nicht durch Verschwinden oder das Herausschöpfen beseitigt werden, sondern jemand würde es trinken müssen, um zu dem Horkrux zu gelangen, der darin lag. Dumbledore befahl Harry, ihm zu helfen, den Trank zu trinken, egal welche Auswirkungen er hatte. Er sollte, egal wie sehr Dumbledore ihn bitten würde es aufhören zu lassen, ihm die gesamte Flüssigkeit in den Mund bringen.
\vspace{10pt}
\newline
Während Dumbledore die Flüssigkeit trank, sah er eine Vision des Duells zwischen ihm, Gellert Grindelwald und Aberforth Dumbledore. Auch sah er den Tod seiner Schwester Ariana Dumbledore. Das Getränk schwächte ihn, machte ihn extrem durstig und verursachte ihm schlimme Schmerzen. Er konnte es nicht ertragen, aber Harry gelang es trotzdem, es ihm komplett in en Mund zu bekommen, indem er ihm sagte, dass es bald vorbei sei. Bei der letzten portion wurde Dumbledore bewusstlos. Harry versuchte Dumbledore wiederzubeleben, jedoch gelang ihm dies nicht. Dumbledore wachte aber nach einer kurzen Zeit Bewusstlosigkeit auf und wollte Wasser.
\vspace{10pt}
\newline
Harry versuchte dann mit dem Wasser-Aufrufzauber Wasser für Dumbledore zu beschwören, aber dies funktionierte nicht, da das Wasser aus unbekannten Gründen nicht aus dem Krug kam. In seiner Verzweiflung füllte Harry eine Tasse mit Wasser aus dem See. Nachdem Dumbledore das Wasser getrunken hate, stiegen die Inferi aus dem Wasser und versuchten sie zu töten. Harry geriet in Panik und versuchte sich mit einer Vielzahl von Zaubersprüchen und Flüchen zu wehren. Unter anderem verwendete er den Lähmzauber und Sectumsempra, jedoch funktionierte keiner dieser Zauber, da es zu viele waren und keinen Schmerz fühlten. Nachdem Dumbledore wieder bei vollem Bewusstsein war, schuf er einen Feuerring, der die Inferi abwehrte und sie zurück in den See trieb, während er und Harry mit dem Horkrux im Boot flüchteten. Harry benutzte dieses Mal sein Blut, um die verborgene Tür wieder zu öffne. Als sie außerhalb der Höhle waren apparierten sie zurück nach Hogsmeade.
%SEITE17
\subsubsection*{Schlacht auf dem Astronomieturm}
%ABSATZ
Jedoch würden beide in Hogwarts nicht zur Ruhe kommen können, da die Schlacht auf dem Astronomieturm bereits begonnen hatte. Harry und Dumbledore nahmen Besen, die hinter der Bar von Rosmerta standen und flogen zum Astronomieturm. Am Astronomieturm angekommen, wurden Dumbledore und ein in den Unsichtbarkeitsumhang gehüllter Harry von Draco Malfoy erwartet. Bevor Harry sich zeigen konnte, machte Dumbledore ihn mit einem nonverbalen Petrificus Totalis bewegungsunfähig, sodass er bei dem bevorstehenden Ereignis nur zusehen und nicht einschreiten konnte.
\vspace{10pt}
\newline
Malfoy entwaffnete seinen Schulleiter und gab zu, dass er hinter den Angriffen auf Hogwarts-Studenten steckte. Auch erzählte er, dass Voldemort ihm befohlen hatte, Dumbledore zu töten. Dumbledore enthüllte, dass er von Dracos Mission und den Versuch sein Leben mit der Opal-Halskette und dem vergifteten Met zu beenden, wusste. Er erlaubte es ihm jedoch trotzdem in Hogwarts zu bleiben, weil er wusste, dass Voldemort ihn töten würde, wenn er den Auftrag nicht erfüllen würde. Dumbledore bot Draco Schutz an und versprach, dass er ihn und seine Mutter an einer Stelle verstecken könnte, wo Voldemort sie nie finden würde. Draco war nach der Rede von Dumbledore jedoch nicht im Stande, ihn zu töten, selbst nach der Ankunft der Todesser, die Draco in die Burg geschmuggelt hatte.
\vspace{10pt}
\newline
Snape, der kurz zuvor von Filius Flitwick gerufen wurde, um den Orden des Phönix zu unterstützen, kam jedoch in den Turm und tötete Dumbledore, nachdem dieser ihn darum gebeten hatte, ohne dies zu enthüllen. Mit den Worten "Severus, bitte." wird Dumbledore vom Todesfluch getroffen. Der Körper von Dumbledore fiel nach dem Todesfluch vom Turm und fiel auf den Boden.
\vspace{10pt}
\newline
Eine Weile später, nachdem die Schlacht vorbei war, fand Harry Dumbledores Körper tot und zerfetzt vom Sturm auf dem Boden vor Hunderten von schockierten Menschen. Harry fand das Medaillon und las einen Brief von R. A. B. und erkannte, dass es kein Horkrux war. Er verstand, dass es sich dabei um eine Fälschung handelte. Harry wusste weder, noch kümmerte es ihn, was diese Nachricht bedeutete und er verstand, dass er und Dumbledore in dieser Nacht nichts erreicht hatten. Auch dachte er, dass Dumbledore nur gestorben war, weil er von dem Trank geschwächt war und diesen umsonst getrunken hatte. Harry fing an über Dumbledores Leiche zu weinen und wischte mit seinem Ärmel einen Blutstropfen aus dem Mund des toten Schulleiters.
\vspace{10pt}
\newline
Anfang Juli fand die Beerdigung von Dumbledore statt, an der Schüler und Mitarbeiter von Hogwarts, Beamte des Britischen Zaubereiministerium (unter anderem Cornelius Fudge, Rufus Scrimgeour, Dolores Umbridge und Percy Weasley), Zentauren, die Mitglieder des Orden des Phönix (außer Snape, um nicht seine Tarnung zu verlieren und Mundungus Fletcher, der in Askaban inhaftiert war), Einwohner von Hogsmeade (darunter auch Dumbledores Bruder Aberforth Dumbledore), Fawkes, Hagrids Halbbruder Grawp, Wassermenschen, Tom (Vermieter des Zum Tropfenden Kessel), Olympe Maxime(Schulleiterin der Beauxbatons Akademie für Zauberei), Donaghan Tremlett (Bassist der Schwestern des Schicksals), Ernie Prang (Busfahrer des Fahrenden Ritters), Madam Malkin (Inhaberin des Madam Malkins Anzüge für alle Gelegenheiten Laden in der Winkelgasse), die Dame, die den Wagen im Hogwarts-Express schob und viele andere Hexen und Zauberer teilnahmen. Dumbledore Leiche wurde in einem weißen Grabmal bestattet, in welchem auch sein Zauberstab beigelegt wurde.

\subsubsection*{Post-mortem und Vermächtnis}
\subsubsection*{1997}
%ABSATZ
Nach seinem Tod beschmutzte Rita Kimmkorn den Ruf von Dumbledore, indem sie eine beleidigende Biographie mit dem Titel "Das Leben und die Lügen von Albus Dumbledore" veröffentlichte. Diese stellte Dumbledore als Bösewicht da und enthielt den Hinweis, dass er seinen Angriff auf Grindelwald verzögerte, weil er zu ihm immer noch eine gewisse Zuneigung hatte. Auch wurde Dumbledore häufig lächerlich dargestellt.
\vspace{10pt}
\newline
Dumbledore vererbt an Harry Potter einen Goldenen Schnatz, den Harry bei seinem ersten Quidditchspiel gefangen hat, in dem sich der Stein der Auferstehung befindet. An Ron Weasley vererbte er seinen Deluminator, den er benutzte, als er Harry Potter zu den Dursleys brachte. Und an Hermine vererbte er eine Ausgabe des Buches "Die Märchen von Beedle dem Barden" aus dem das Trio später von den Heiligtümern des Todes erfährt.
\subsubsection*{1998}
%ABSATZ
Im Jahr 1998, kurz vor Beginn der Schlacht von Hogwarts, versuchte Lord Voldemort Gellert Grindelwald, dem früheren Besitzer des Elderstab, den Aufenthaltsort des Elderstabs zu entlocken. Grindelwald gab Voldemort keine Auskunft, jedoch vermutete Voldemort, dass Dumbledore ihn bekommen hatte, nachdem er Grindelwald in den 1940er Jahren besiegt hatte. Er brach in das Weißes Grab des ehemaligen Schulleiter ein und stahl den Zauberstab, in der Hoffnung die volle Kraft dieses nutzen zu können.
\vspace{10pt}
\newline
Wie sich herausstelle, war die Nacht von Dumbledore Tod nicht das letzte Mal, dass Harry ihn lebendig sah. Nachdem er herausgefunden hatte, dass er einer von Voldemorts Horkruxen war (Voldemort hatte Harry bei der versuchten Ermordung als Baby versehentlich zu einem Horkrux gemacht), ging Harry tapfer in Voldemorts Stützpunkt, um sich töten zu lassen, genauso wie
%SEITE18
Dumbledore es mehr oder weniger geplant hatte. Aber Voldemorts Todesfluch konnte Harry erneut nicht töten. Stattdessen traf Harry in einen Zustand ein, in dem er mit Dumbledores Geist sprach. Dumbledore erklärte Harry alles, was er ihm zu Lebzeiten nicht erzählt hatte: über die Heiligtümer des Todes, seine Freundschaft mit Grindelwald und viele andere Dinge.
\vspace{10pt}
\newline
Vor allem enthüllte er, dass Harry nicht tot war, denn als Voldemort Harrys Blut verwendet hatte, um seinen Körper wiederherzustellen, hatte er unbeabsichtigt Harrys Leben mit seinem eigenen verbunden, mit dem Ergebnis, dass Voldemorts Fluch nur das Fragment seiner Seele "getötet" hatte, das all die Jahre in Harry geblieben war. Harry versuchte auch Dumbledore zu trösten, als sein alter Lehrer fragte, ob er auf seiner Suche nach Macht besser als Voldemort gewesen sei. Harry wies ihn darauf hin, dass Dumbledore niemals getötet hatte, wenn er es vermeiden konnte und argumentierte, dass Dumbledores Suche nach den Heiligtümern etwas völlig anderes gewesen sei als Horkruxe zu erschaffen, so wie es Voldemort getan hatte.
\vspace{10pt}
\newline
Nachdem er und Harry sein Gespräch beendet hatten, trennten sich die Wege der Beiden wieder: Harry kehrte in die Welt der Sterblichen zurück und Dumbledore ging, wie er es sagen würde, "weiter". Er war der einzige Schulleiter, der auf dem Gelände von Hogwarts beigesetzt wurde, nachdem Rubeus Hagrid argumentiert hatte, dass kein anderer Schulleiter jemals so viel für die Schule getan hatte. Sein Porträt blieb auch in Hogwarts. Nach Voldemorts Tod enthüllte Harry, dass er der wahre Meister des Elderstabs war und sagte zu Dumbledore Porträt, dass er den Zauberstab zu seinem Grab zurückbringen würde, wo er bleiben würde, bis Harry starb. Er sollte unbesiegt sein und würde der letzte Meister des Stabs sein. Er benutzte den Zauberstab nur für eine Sache: um seinen eigenen Zauberstab zu reparieren, der zuvor gebrochen worden war.
\vspace{10pt}
\newline
Harry und Ginny Potters zweiter Sohn Albus Severus Potter wurde nach Albus Dumbledore und Severus Snape benannt.
\subsubsection*{Schuljahr 2020-2021}
%ABSATZ
Nachdem der Sohn von Harry Albus Potter verletzt wurde, weil er einen Prototyp eines Zeitumkehrers zur Rettung von Cedric Diggory verwendet hatte, besuchte Harry das Porträt von Dumbledore. Er gab Harry den Rat, seinen Sohn so zu sehen, wie er ist, anstatt ihn zu etwas zu machen, das er nicht sein kann. Nach diesen Worten verließ er das Porträt.
\vspace{10pt}
\newline
Harry besuchte ihn erneut, nachdem er gehört hatte, dass Albus und Scorpius Malfoy, der Sohn von Draco Malfoy, von Delphini, der Tochter von Lord Voldemort, entführt wurden. Dumbledore fragte Harry, was er tun würde und beantwortete Harry ein paar Fragen.

\subsection*{\Large Äußerliche Erscheinung}
%ABSATZ
Albus Dumbledore war groß und dünn, hatte silbernes Haar und Bart (in seiner Jugend rotbraun). Sein Bart war so lang, dass er diesen in seinen Gürtel stecken konnte. Er hatte eine sehr lange und krumme Nase, die aussah, als wäre sie mindestens zweimal gebrochen worden. Unter anderem wird vermutet, dass der Schlag seines Bruders Aberforth Dumbledore bei der Beerdigung von Ariana Dumbledore an der Verformung der Nase von Dumbledore schuld war. Dumbledore hatte auch sehr lange Finger. Es wurde beschrieben, dass seine Augen einen strahlenden Blauton hatten und normalerweise vor Freundlichkeit funkelten.
\vspace{10pt}
\newline
Wenn Dumbledore jedoch wütend wurde, verwandelte er sich Dumbledore von einem gutartig aussehenden, alten Mann, in einen Zauberer, der noch schrecklicher war als Lord Voldemort, mit einem Gesicht, dass Wut ausstrahlte. In der Tat hatte Harry - nachdem er einen seltenen Moment von Dumbledore Wut erlebt hatte - gestand, völlig verstanden zu haben, warum die Leute immer sagen, dass Dumbledore der einzige Zauberer war, den Voldemort jemals gefürchtet hatte.
\vspace{10pt}
\newline
Dumbledore trug eine halbmondförmige Brille und eine Reihe von bunten Roben, die sowohl lila als auch purpurrot waren. Er behauptete einmal eine Narbe über seinem linken Knie in der genauen Form der Karte der Londoner U-Bahn zu haben, aber die wurde nie bestätigt. Sein Verhalten war oft - wenn nicht immer - ruhig und fachlich. Er sprach normalerweise mit einer ruhigen, angenehmen Stimme, selbst wenn Harry dachte, dass er tatsächlich wütend war.
\vspace{10pt}
\newline
Während seinem letzten Lebensjahr hatte Dumbledore seine rechte Hand vernarbt, da er den verfluchten Ring von Vorlost Gaunt anzog. Hätte Severus Snape keinen Gegenfluch gewusst, wäre Dumbledore in wenigen Minuten gestorben. Trotzdem sah dir Hand aber weiterhin geschwärzt und tot aus. Keine Heilung konnte dieses Aussehen ändern. Snape versicherte Dumbledore aber, dass der Fluch sich trotzdem weiter ausbreiten würde und er innerhalb von einem Jahr sterben würde. Da jedoch Dumbledore vor dieser Frist getötet wurde, ist nicht bekannt, ob sich schließlich das tote Aussehen auf den gesamten Körper ausgebreitet hätte.
\vspace{10pt}
\newline
Der Tod diente vor allem dazu, ihn wieder zu voller Gesundheut und Kraft zu bringen: Sein Geist wurde als lebhaft beschrieben. Auch waren seine Hände waren wieder weiß und unbeschädigt.
%SEITE19
\subsection*{\Large Persönlichkeit und Verhalten}
\subsubsection*{\large Allgemeines}
%ABSATZ
Dumbledore wird von vielen als der mächtigster Zauberer, abgesehen von Merlin angesehen. Dumbledore hatte fast immer ein ruhiges und entspanntes Auftreten und strahle auch unter Druck stets eine Aura der Gelassenheit aus. Gefühle wie Wut oder Angst zeigte er selten. Es wird behauptet, dass Dumbledore der einzige Zauberer war, den Lord Voldemort jemals wirklich gefürchtet hatte.
\vspace{10pt}
\newline
Dumbledore liebte Stickmuster und trug häufig extravagante Kleidung. Auch hatte er einen etwas seltsamen Geschmack bei Süßigkeiten: Er liebte Muggelsüßigkeiten (insbesondere Sorbet-Zitronen/ Zitronenbrausebonbons), die in der Zaubererwelt fast gar nicht bekannt waren. Die Passwörter seines Büros waren fast immer seine aktuelle Lieblingssüßigkeit. Außerdem war er ein Fan von Kammermusik und Bowling und hatte einen guten Sinn für Humor. Letzteren behielt auch in brenzligen oder sehr ernsten Situationen, beispielsweise als er mit Severus Snape seinen eigene Ermordung plante. Dabei sagte er sein Tod sei genau so sicher wie die Tatsache, dass die Chudley Cannons, wieder letzer in der Quidditch Liga werden würden.
\subsubsection*{\large Intelligenz}
%ABSATZ
Da Dumbledore überaus talentiert war, hatte er zahlreiche außergewöhnliche Fähigkeiten. Er war sehr belesen und sehr gut darin Personen einzuschätzen und deren Beweggründe zu verstehen. So erkannte er beispielsweise, dass ein Mangel an Vertrauen, Liebe und Freundschaft als Voldemorts größte Schwächen waren. Außerdem fand er schnell heraus, dass Quirinus Quirrell mit einem gefälschten Brief des Britischen Zaubereiministeriums ablenken wollte, ließ sich nie von den Werken von Gilderoy Lockhart täuschen, weil er wusste, wie er wirklich war und erkannt auch, dass Alastor Moody in Wahrheit Bartemius Crouch Junior war, als er Harry Potter bat in sein Büro zu kommen.
\vspace{10pt}
\newline
Während Dumbledore (zumindest in seinen späteren Jahren) niemals Arroganz oder Einbildung zeigte, hatte er auch keinen Platz für falsche Bescheidenheit und war, wenn die Situation dies rechtfertigte, vollkommen bereit, seine Intelligenz und Macht anzuerkennen. Er sagte Harry einmal, dass er intelligenter sei als die meisten Menschen, seine Fehler deshalb, aber auch entsprechend größer wären.
\vspace{10pt}
\newline
Im Allgemeinen war Dumbledore seinen Mitmenschen meist einen Schritt voraus, wusste von deren Beweggründen und Plänen und hatte meist auch Informationen, über die andere nicht verfügten.
\subsubsection*{\large Zwischenmenschliche Beziehungen}
%ABSATZ
Dumbledore verlor nie seinen Glauben an das Gute in Personen und gab Leuten immer eine Chance sich zu bessern (u.a., weil er einst selbst eine erhalten hatte). Er sprach sich unter anderem sehr gegen die Suspendierung von Newton Scamander aus und später auch gegen die von Rubeus Hagridaus. Dumbledore erlaubte auch Remus Lupin Schüler an seiner Schule zu sein und später sogar Professor zu werden, obwohl Remus ein Werwolf war. Er hatte auch eine gute Freundschaft mit Hagrid, obwohl viele dem Halbriesen nicht vertrauten. Das beste Beispiel dafür ist jedoch Dumbledores Beziehung zu Severus Snape, dem Dumbledore eine Chance gab zur guten Seite zu wechseln, nachdem Snape Reue gezeigt hatte. Im Allgemeinen setzte sich Dumbledore immer für die ein, die nicht für sich selbst sprechen konnte und behandelte Außenseiter und Schwächere immer auf Augenhöhe und mit gebührendem Respekt.
\vspace{10pt}
\newline
Trotz Dumbledores vieler außergewöhnlichen Eigenschaften war er auch eine etwas machiavellistische Figur und war oft gezwungen schwierige Entscheidungen zu treffen. Zum Beispiel brachte er Harry Potter zur Familie Dursley, da dies der einzige Ort war, wo Voldemort ihn nicht erreichen konnte, obwohl er von McGonagall wusste, dass die Dursleys Harry wahrscheinlich nie gut behandeln würden.
\vspace{10pt}
\newline
Außerdem war Dumbledore äußerst geheimnisvoll und ließ auch in späteren Jahren nie jemanden vollständig an sich heran (die Ausnahme hier bildetete Gellert Grindelwald). Vollständige Einsicht in seine Pläne und Denkweisen gab er niemandem. Auch seine Vergangenheit war weitestgehend unbekannt und wurde erst nach seinem Tod in Rita Kimkorns Buch "Leben und Lügen des Albus Dumbledore" enthüllt, weshalb Harry z.B. den Eindruck kriegte, er hätte Dumbledore, trotz deren enger Beziehung, nie wirklich gekannt.
\subsubsection*{\large Schwächen und Einfluss seiner Vergangenheit}
%ABSATZ
Dumbledore größte Schwäche, von der er Harry 1998 erzählte, war sein Wunsch nach Macht. Er gab zu, dass er in seinen jüngeren Jahren ziemlich egoistisch gewesen war und aufgrund seiner außergewöhnlichen Fähigkeiten geglaubt hätte er sei zu Größerem bestimmt. Als er siebzehn war planten er und Gellert Grindelwald eine Zaubererrevolution mit dem Ziel die Muggel zu versklaven und Zauberer und Hexen wieder zu den Herrschern über die Welt zu machen. Außerdem wollten er und Grindelwald die Heiligtümer des Todes suchen und finden. Sie planten Dumbledores gebrechliche Schwester Ariana mit auf diese
%SEITE20
Suche zu nehmen, doch Dumbledores Bruder Aberforth war damit nicht einverstanden, da er meinte, dass Arianas Zustand dafür zu instabil wäre. Daraufhin kam es zu einem Streit und später zu einem Duell zwischen Aberforth, Albus und Gellert. Während des Duells geriet Ariana zwischen die Fronten, wurde von einem Fluch getroffen und starb. Dieses Ereignis veränderte Dumbledore grundlegend. Er machte sich Zeit seines Lebens verantwortlich für ihren Tod, kam zu dem Schluss, dass Grindelwalds Ziele falsch waren und er selbst in Machtpositionen nur Schaden anrichtete. Unter anderem deshalb weigerte er sich mindestens dreimal das Amt des Zaubereiminister zu übernehmen.
\vspace{10pt}
\newline
Arianas Tod hatte für Dumbledore Nachwirkungen bis ins hohe Alter. Er entwickelte einen starken Selbsthass und kam von den Schuldgefühlen nie los. Dies sieht man als er 1996, kurz vor seinem Tod, mit Harry, auf der Suche nach Horkruxen, in einer Unterwassergrotte einen Zaubertrank zu sich nimmt, der ihn zwingt das Ganze nochmals zu durchleben. Während der Einnahme des Tranks erleidet Dumbledore einen Zusammenbruch, sagt er wisse doch, dass dies alles seine Schuld sei und fleht Harry an ihn zu töten.
\subsubsection*{\large Dumbledore und sein Verhältnis zum Tod}
%ABSATZ
Dumbledore betrachtete den Tod nie als schlimm. So sagte er zu Harry, dass dieser für den gut vorbereiteten Geist nur das nächste größere Abenteuer sei. Außerdem zeigte sich dies bei seinem Duell mit Voldemort, als dieser Dumbledore fragte, ob er sich nicht trauen würde ihn zu töten. Dumbledore erwiderte daraufhin nur, dass Voldemort zu töten ihn nicht zufriedenstellen würde, da es weitaus schlimmere Dinge gäbe. Als Dumbledore 1996 erfuhr, dass er aufgrund des Fluches auf seiner Hand nur noch ein Jahr zu leben hatte, schien ihn dies nicht weiter zu beunruhigen. Stattdessen plante er seine Ermordung durch die Hand von Severus Snape, um dessen Rolle als Doppelagenten sicherzustellen.
\subsubsection*{\large Dumbledore und sein Verhältnis zur Liebe}
%ABSATZ
Dumbledore hatte immer einen großen Glauben an die Liebe und erinnerte Harry häufig daran, dass Liebe die größte Magie von allen war. Aufgrund seines Selbsthasses hatte er jedoch einige Schwierigkeiten Liebe anzunehmen oder sich zu erlauben andere zu lieben, u.a., weil er glaubte, dass er, wo er liebte nur Schaden anrichtete, wie aus einem Gespräch mit Harry aus Harry Potter und das verwunschene Kind hervorgeht:
\vspace{10pt}
\newline
Harry: „Years. Years I spent there alone without knowing what I was, or why I was there, without knowing that anybody cared!“
\vspace{10pt}
\newline
Dumbledore: „I – I did not wish to become attached to you–“
\vspace{10pt}
\newline
Harry: „Protecting yourself, even then!“
\vspace{10pt}
\newline
Dumbledore: „No, I was protecting you. I did not want to hurt you …"
\vspace{10pt}
\newline
DUMBLEDORE attempts to reach out of the portrait but he can’t. He begins to cry but tries to hide it.
But I had to meet you in the end… eleven years old and you were so brave. So good. You walked uncomplainingly along the path that had been laid at your feet. Of course I loved you… and I knew that it would happen all over again… that where I loved I caused irreparable damage… I am no fit person to love… I have never loved without causing harm…“
\vspace{10pt}
\newline
Harry: „You would have hurt me less if you had told me this, then.“
\vspace{10pt}
\newline
Dumbledore: „(openly weeping now): I was blind. That is what love does. I didn’t know that you needed to hear that this closed-up, tricky, dangerous, old man… loved you…“
\vspace{10pt}
\newline
— Harry spricht mit Dumbledores Porträt über die zwischenmenschliche Beziehung zwischen beiden
\subsubsection*{\large Politische Standpunkte}
%ABSATZ
In seiner Jugend glaubte Dumbledore an die Überlegenheit der Zauberer gegenüber Muggeln, dies änderte sich jedoch nach Arianas Tod. Dumbledores Standpunkte sind (auf die Zauberergesellschaft bezogen) progressiv, z.B. setzt er sich für mehr Rechte für Werwölfe, Zentauren und Riesen ein und hält die Zauberer nicht für überlegen oder für besser als andere magische oder nichtmagische Gruppierungen. In Teilen könnten Dumbledores Standpunkte, als politisch liberal gewertet werden, da er normalerweise nicht davon hält, wenn das Zauberministerium sich in Hogwarts einmischt und Regierungsinstitutionen eher zu misstrauen scheint. Dies zeigt sich 1996 z.B. in seiner Weigerung Harry dem Zauberminister Scrimgeour vorzustellen um den Eindruck zu erwecken, Harry Potter wäre auf der Seite des Ministeriums.
%SEITE21
\subsubsection*{\large Psychologie}
%ABSATZ
Dumbledore zeigt Züge suizidalen Verhaltens. Beispielsweise scheint ihm die Tatsache, dass er nicht mehr lange zu Leben hat, 1996 egal zu sein. Als Snape ihm dies erläutert bleibt er vollkommen ruhig und beauftragt diesen später mit seiner Ermordung. In der Nacht von seinem Tod nimmt er einen Zaubertrank zu sich, der ihn seine schlimmste Erinnerung nochmals durchleben lässt und bettelt Harry an ihn zu töten. Auch später kurz vor seiner Ermordung auf dem Astronomieturm scheint ihm die Tatsache, dass er sterben wird nichts auszumachen.

\subsection*{\Large Besitztümer}
%ABSATZ
In Dumbledores Büro befanden sich eine Reihe eigenartiger Gegenstände. Unter ihnen war beispielsweise ein Denkarium, ein Steinbehälter, in dem Erinnerungen gespeichert und betrachtet werden können, die Magier als eine Art Flüssigkeit aus ihren Köpfen bekommen. Harry sah dieses Gerät zum ersten Mal, um Dumbledores Erinnerungen an Igor Karkaroffs Prozess zu betrachten. Auch benutzte er es, um Snapes Kindheitserinnerungen anzusehen. Während des Privatunterrichts zwischen Harry Potter und Albus Dumbledore wurde das Denkarium verwendet, um eine Vielzahl von Erinnerungen an Tom Riddles Vergangenheit zu betrachten.
\vspace{10pt}
\newline
Dumbledore wurde auch häufig mit dem Deluminator gesehen, einem Gerät zum Entfernen und späteren Zurückgeben von Licht. Als das Objekt an Ronald Weasley vererbt wurde, wurde betätigt, dass der Deluminator von Dumbledore entworfen wurde. Dumbledore war auch der Besitzer und Meister des Elderstabs, der auch als "Zauberstab des Schicksals" oder "Todesstab" bekannt ist, und einer der Heiligtümer des Todes ist. Den Zauberstab bekam er, nachdem er Grindelwald besiegt hatte. Davor hatte er einen anderen eigenen Zauberstab. Meister über den Elderstab war er bis zum Zeitpunkt, als er von Draco Malfoy am Astronomieturm entwaffnet wurde. Die Treue des Zauberstabs bekam dann fast ein Jahr später Harry Potter. Im Gegensatz zu vielen früheren Besitzern des Zauberstabs hielt Dumbledore seine Identität streng geheim.
\vspace{10pt}
\newline
Er besaß als Haustier einen Phönix namens Fawkes, der das Schloss Hogwarts nach Dumbledores Tod verließ. Der Phönix war Dumbledore zu Lebzeiten jedoch so verbunden, dass er Leuten half, die Dumbledore mochte und die Dumbledore mit viel Würde ansahen. Es zeigte sich auch, dass zwei von den Phönixfeder von Fawkes die magischen Kerne von Harrys und Voldemorts Zauberstab waren.
\vspace{10pt}
\newline
Professor Dumbledore führte auch ein Tagebuch, in das er regelmäßig schrieb. Er hatte auch eine Taschenuhr mit zwölf Zeigern, aber anstelle von Zahlen bewegten sich kleine Planeten am Rand.

\subsection*{\Large Magische Fähigkeiten und Fertigkeiten}
%ABSATZ
Dumbledore war ein außerordentlich talentierter Zauberer, auch als er noch Schüler in Hogwarts war. Er war in vielen Bereichen der Magie so herausragend, dass er als der brillanteste Schüler anerkannt wurde, der jemals die Schule besucht hat, dicht gefolgt von Tom Riddle. Dumbledore erhielt viele Auszeichnungen und Preise, für die unglaubliche magische Fähigkeiten erforderlich waren. Als Erwachsener war Dumbledore ein außerordentlich mächtiger Zauberer mit Fähigkeiten und Kräften in allen Gebieten. Wegen seinem herausragendem magischem Talent wurde er bewundert, aber auch gefürchtet. Schließlich wurde er sogar von vielen als größter Zauberer aller Zeiten angesehen. In der Tat war Dumbledore der einzige Zauberer, der Grindelwald besiegen und später Voldemort in die Defensives zwingen konnte.
\paragraph{Magisches Können:}
%ABSATZ
Nach seinem ersten Jahr in Hogwarts war Dumbledore bereits als außerordentlich talentiert bekannt und wurde von vielen Schülern als der begabteste Schüler angesehen, den die Schule jemals hatte. Er gewann bereits in seinem erstem Schuljahr viele Auszeichnungen und Preise für magische Fähigkeiten, die eigentlich für höhere Altersklasse bestimmt waren. Als Erwachsener wurde Dumbledore von vielen als der begabteste Zauberer der modernen Geschichte angesehen, wenn nicht sogar der gesamten Zauberergeschichte. Dies wurde meistens mit seiner beispielhaften Beherrschung der Magie begründet. Er kehrte später zu seiner ehemaligen Schule zurück, um Verteidigung gegen die dunklen Künste zu unterrichten, und schlug sich dabei so gut, dass er von vielen seiner Schüler als bester Lehrer von Hogwarts angesehen wurde. In der früherer Phase von Gellert Grindelwalds Machtübernahme gab sogar der früher Leiter der Abteilung für magische Strafverfolgung, Torquil Travers, widerwillig zu, dass nur Dumbledores Grindelwald besiegen konnte, obwohl er Dumbledore nicht besonders mochte. In der Tat gab sogar Grindelwald zu, dass Dumbledore eine große Bedrohung für seine Ziele darstellte, als jeder andere lebende Zauberer und glaubte, dass nur der äußerst mächtige Obscurial Credence Barebone ihn möglicherweise töten könnte. Letztendlich war es Dumbledores Engagement, das den Globalen Zauberkrieg beendete. Er konnte sich gegen Grindelwald durchsetzen, obwohl dieser den Elderstab besaß. Laut Augenzeugen war das legendäre Duell eine so erstaunliche Demonstration von Können und Macht, dass es wohl der spannendste Duell zwischen zwei Zauberern jeweils war. Albus gab zu, dass er nur etwas geschickter als Grindelwald war. später in seiner Lehrerkarriere unterrichtete er Verwandlung, ein komplexer und schwieriger Zweig der Magie und stieg bald in die Position des Leiter der Verwandlung von Hogwarts auf. Schließlich wurde er zum Schulleiter der hoch angesehenen Zaubererschule und wurde bald als fähigster Schulleiter aller Zeiten beschrieben, auch von Tom Riddle.
%SEITE22
Danach wurde ihm auch mehrmals der Posten als Zaubereiminister angeboten, obwohl er diesen jedes al ablehnte. dies begründete er, weil er angeblich nicht in der Lage war, diese herausragende Leistung zu übernehmen. Seine Fähigkeit, die Verteidigung der Schule zu fördern, war so gut, dass nicht mal Voldemort es jemals hinbekam in die Schule einzudringen oder dies überhaupt wagte, während Dumbledore Schulleiter war, da Dumbledore der einzige Zauberer war, den Voldemort jemals fürchtete. Hogwarts wurde deshalb häufig als sicherster Ort in der Zaubererwelt bezeichnet, da er mit der einzige Ort war, der von Voldemort sicher war. Als sich schließlich Dumbledore und Voldemort im Kampf gegenüber standen, erwies sich Dumbledore als fähig, Voldemorts überwältigende magische Fähigkeiten und Kräfte mit seinen eigenen unter Druck zu setzen. Laut Voldemort kämpfte Dumbledore jedoch nicht darum ihn zu töten, sondern nur um dafür zu sorgen, dass Voldemort in die Defensives ging und verschwand. Dumbledore behauptete auch, dass er leicht aus dem sicheren Askaban ausbrechen könnte, selbst wenn er selbst inhaftiert wäre. Dumbledore hatte schließlich auch ansehen von berühmten magischen Theoretikern wie Adalbert Schwahfel, der magischen Historikerin Bathilda Bagshot und dem Alchemisten Nicolas Flamel. Während seines gesamten Lebens trug er viel zur Weiterentwicklung des magischen Wissens in der Neuzeit bei und wurde zu einem berühmten Mitwirkenden an vielen magischen Zeitschriften. Dumbledore konnte auch die mächtigen dunklen Schutzzauber von Voldemort, die er auf das Versteck seiner Horkruxe gelegt hatte, überwinden. Beim Überwinden dieser Zauberer zeigte er seine Fähigkeit Magie nur zu erkennen und zu analysieren, indem er Sachen lediglich in seine Hände legte oder berührte.
\paragraph{Intellektuelles Genie:}
%ABSATZ
Dumbledore war nicht nur ein großartiger Zauberer, sondern besaß auch einen außergewöhnlichen Intellekt. Er galt als einer der brillantesten Schüler, der jemals Hogwarts besucht hatte (dicht gefolgt von Tom Riddle) und wurde als Erwachsener von vielen als klügster Zauberer angesehen. Seine Intelligenz zeigte sich besonders in seinem taktischen und strategischen Genie, das sich als fähig erwies, sehr gute Masterpläne zu entwickeln, um die dunkelsten Zauberer zu besiegen. Dumbledore war besonders gut darin, zu raten, Vorhersagen zu treffen und Informationen zu finden, die schwer zu finden waren. Beispielsweise wusste er, wo Newton Scamander einen illegalen Donnervogel finden konnte. Er wusste, dass Newt es eilig hatte, den Donnervogle zu retten und nach Amerika zu bringen, um ihn freizulassen, sobald er wieder gepflegt war. Auch konnte Dumbledore durch dies Newt dazu bringen, Grindelwald zu beobachte und das Obscurial in New York City zu suchen. Dies gelang auch und Grindelwald konnte festgenommen werden. Auch konnte man durch Newt den Angriff des Obscurials auf New York im Jahr 1926 die Muggel vergessen lassen. Danach konnte Dumbledore auch Grindelwald bei der Rekrutierung seiner Anhänger beobachten und seine Pläne nachvollziehen. Trotz seines Blutpakts mit Grindelwald konnte Dumbledore ihn schließlich besiegen. Dumbledore spielte auch während des gesamten Globalen Zaubererkriegs eine wichtige Rolle, auch wenn es eine sehr indirekte Rolle war. Als dann schließlich Lord Voldemort eine Generation später an die Macht kam, führte Dumbledore den Orden des Phönix während des Ersten Zaubererkriegs, wo er viel taktisches verwendete. Auch im Zweiten Zaubererkrieg konnte Dumbledore viele taktischen Pläne schmieden und Pläne von Voldemort durch Manipulation von Schlüsselpersonen (wie Severus Snape und Harry Potter) zerstören. So gab er nach seinem Tod Harry die Möglichkeit die Horkruxe von Voldemort zu zerstören und den Dunklen Lord zu besiegen. Dumbledores Intellekt, insbesondere sein strategisches Denken, ist eines seiner Schlüsseltalente. Auch hatte er die besondere Fähigkeit, sehr gute und strategische Pläne auszuarbeiten. So konnte er beispielsweise Sirius Black retten, indem er mit Hermine Granger und ihrem Zeitumkehrer die Vergangenheit so ändern konnte, dass Sirius fliehen konnte und für den Orden kämpfen konnte. Sein Genie machte sich auch nach seinem Tod durch seinen Wille bemerkbar. Dumbledore war auch im Stande alle Möglichen Dinge vorherzusehen. So wusste er beispielsweise, dass Ronald Weasley zwar anfangs mit auf Horkrux-Jagd gehen würde, aber später wieder gehen würde. Auch wusste er, dass er zurückkommen würde und zu welchem Zeitpunkt. Deshalb vererbte er Ron den Deluminator, der dem Trio bei der Suche helfen würde. Auch nutzte er den Wissensdurst von Harry und Hermine, um sie neugierig auf das Symbol der Heiligtümer des Todes zu machen, indem er sie in die Kopie seines Buches von den Die Märchen von Beedle dem Barden zeichnete. Dies führte dazu, dass Harry von den Heiligtümern des Todes erfuhr und schließlich den Stein der Auferstehung in dem Goldenen Schnatz fand, den Dumbledore ihn vererbt hatte. So gab er Harry die emotionale Unterstützung, die er brauchte, um sich vom Dunklen Lord töten zu lassen. Dumbledore besaß auch eine große Menge an emotionaler Intelligenz, sodass er die Maske von Tom Riddle während seiner Schulzeit durchschauen konnte. Dies gab ihm später auch die Möglichkeit ein Verständnis dafür zu haben, wo Voldemort die Horkruxe versteckt hatte. Auch war er intelligent genug, um Riddle seit seiner ersten Öffnung der Kammer des Schreckens so aufmerksam im Auge zu behalten, dass er den Erben von Slytherin effektiv davon abhielt, die Kammer zum zweiten Mal zu öffnen. Er war auch einer der einzigen Zauberer, der sich nicht von Gilderoy Lockhart Geschichten täuschen ließ und fand heraus, dass dieser ein Betrüger war, weil Dumbledore zwei Zauberer kannte, deren Erinnerungen von Lockhart verändert worden waren. Dumbledore ließ sich auch nicht von Quirinus Quirrells schüchternen Art täuschen. Anfangs ließ er sich zwar von Quirrells falschen Brief täuschen, der besagte, dass Dumbledore zum Ministerium musste, fand aber heraus, dass es eine Fälschung war und kehrte rechtzeitig nach Hogwarts zurück, um Harry zu retten. Er schaffte es schließlich auch Bartemius Crouch Junior Verkleidung als Alastor Moody zu enttarnen, als dieser Harry von Dumbledore nahm, weil Moody dies nie gemacht hatte. Snape und McGonagall verstanden die Verkleidung zuerst nicht, aber als Dumbledore diesen befahl Okklumentik anzuwenden und ihm Veritaserum zu geben, verstanden auch diese den gut durchdachten Plan. Es wurde klar, dass Crouch sich als Moody verkleidet hatte und diesen mit dem Imperius-Fluch schwächte, damit Moody ihn nicht überwältigen konnte, aber trotzdem am Leben hielt, um den Vielsafttrank herzustellen. Den Vielsafttrank hatte er in seiner Flachen, um diesen ohne Verdacht zu erregen zu trinken. Auch konnte er Jacobs Geschwisterkind leicht erkennen, als dieser sich als Snape ausgab. Selbst Cornelius Fudge bewunderte das Genie von Dumbledore, weshalb Fudge zu Beginn seiner Karriere als Zaubereiminister Dumbledore mit Briefen bombardierte, in denen er Dumbledore um Rat fragte.
%SEITE23
\paragraph{Charisma:}
%ABSATZ
Obwohl er aufgrund seiner außergewöhnlichen emotionalen Intelligenz nicht für seine Fähigkeiten als Manipulator berühmt war, war er dennoch außerordentlich geschickt darin andere zu manipulieren, was er durchaus bereit war, für das Wohl der Allgemeinheit zu tun. Dadurch zeigte er eine immense Menge an Charisma. Er war in der Lage die Eigenschaften einer Person effektiv zu ihrem Vorteil zu nutzen, um sie dazu zu bringen, was getan werden musste. Im Gegensatz zu Voldemort nutzte Dumbledore immer das Gute im Menschen, um sie dazu zu bringen, was getan werden musste. er nutzte seinen Charisma, um viele auf die Seite des Guten anstatt des Bösen zu bringen. Er war in der Lage, Newt dazu zu bringen, dass er ihm mit seinem Donnervogel unterstützen wollte, obwohl er ihn zum Bekämpfen von Grindelwald einsetzen wollte. Er konnte auch den Menschen von Voldemorts Vergangenheit erzählen, obwohl sich die Leute vor ihm fürchteten. Die bemerkenswerteste Demonstration seiner Manipulationsfähigkeiten war, als er Snape effektiv manipulieren konnte, um zuzustimmen, Harry zu beschützen und als Spion in Voldemorts Reihen für Dumbledore zu agieren, trotz Snapes Hass gegenüber Harrys Vater und vollem Wissen über das Risikos seiner Spionage. Dies gelang Dumbledore, weil er von Snapes Liebe zu Lily Evans wusste. Eine weitere Demonstration seiner Manipulationsfähigkeiten war, als er Lockhart mühelos dazu manipulierte Lehrer in Hogwarts zu werden, indem er nur schlau andeutete, dass Harry Schüler der Schule war. So konnte er Lockhart in die Schule locken und dort entlarven, dass er nur ein Betrüger war. Dumbledores Charisma brachte ihm so viel Respekt bei seinen Verbündeten ein, dass die meisten von ihnen Dumbledore auch nach seinem Tod immer noch die Treue hielten und in der Schlacht von Hogwarts tapfer für ihn kämpften. In der Tat war Dumbledore in der Lage sich schnell mit Harry anzufreunden und ihn erfolgreich dazu zu bringen die Jagd nach den Horkruxen fortzusetzen und sich schließlich sogar selbst zu opfern, so wie Dumbledore es geplant hatte.
\paragraph{Lehrfähigkeiten:}
%ABSATZ
Dumbledore hatte schon in jungen Jahren großes Interesse am Unterrichten. Dank seiner magischen Fähigkeiten und seinem Lehrtalent war er in der Lage, die Position des Professors für Verteidigung gegen die dunklen Künste und Verwandlung in Hogwarts zu übernehmen. Von vielen seiner Schüler wurde er als bester Lehrer in Hogwarts bezeichnet. Tatsächlich respektierten sogar die beiden eher strengeren Schulleiter, unter denen Dumbledore arbeitet, seine Kompetenz als Lehrer und ließen sich auch von Dumbledores Meinung im Bezug auf Bestrafung eines Schülers überzeugen und verringerten die Strafen. So konnte er beispielsweise Phineas Nigellus Black davon überzeugen nicht den Zauberstab von Newt zu zerstören, als dieser zum Ausschluss verurteilt wurde. Auch konnte er Armando Dippet davon überzeugen, dass Rubeus Hagrid zumindest als Wildhüter in Hogwarts bleiben durfte. So zeigte er seinen ehemaligen Schülern eine große Menge an Respekt und Treue. Dumbledore gelang es auch Minerva McGonagall erfolgreich im Alter von 17 Jahren zu einem Animagus zu machen, was bekanntlich sehr schwer war. Auch war Dumbledore zuversichtlich Harry Okklumentik zu lehren, obwohl er dies nie tat. Auch gab Dumbledore Harry einen sehr guten Privatunterricht und konnte ihm in kurzer Zeit während der normalen Schullaufbahn viel beibringen. Aufgrund seiner emotionalen Fähigkeiten, allen Mitgefühl zu zeigen, war Dumbledore auch sehr gut darin selbst einige zurückgezogene Schüler dazu zu bringen, sich zu öffnen. Dumbledore war dadurch im Stande Leta Lestrange dazu zu bringen über den Tod ihres Bruders zu kommen, indem Dumbledore seinen Fehler erzählte, als er Ariana verloren hatte. Dies ermöglichte Leta letztendlich Credence Barebone vor Yusuf Kamas Rache zu retten und gab ihr die geistige Kraft Widerstand gegen die überzeugende Rede von Grindelwald zu leisten. Auch konnte er Harry während der gemeinsamen Privatstunden viel über Voldemort erzählen, was hilfreich für Harrys Sieg über den Dunklen Lord war, da er so viele Hintergrund Informationen bekam. Als Schulleiter stimmte Dumbledore sogar zu, den Werwolf Remus Lupin an die Schule zu lassen, was sehr riskant war. So zeigte er sich gegenüber Remus sehr respektvoll und fürsorglich. Dumbledore und einige andere Lehrer von Hogwarts waren auch teilweise dafür verantwortlich, Voldemort zu unterrichten und auszubilden, sodass er schließlich als einer der besten Schüler jemals an der Schule galt. Dieser Unterricht machte Voldemort erst zu einem so mächtigen Zauberer, obwohl sich dies später als Nachteil herausstellte.
\paragraph{Führungsqualitäten:}
%ABSATZ
Dumbledore war bereits in seiner Jugend ein ausgezeichneter Anführer, da er zum Britischen Jugendvertreter im Zaubergamot wurde und er mit Grindelwald die Revolution für das allgemeine Wohl begann und die gesamte Zaubererwelt unterwerfen wollte. Dumbledore war auch der Meinung, dass Newt und Harry eine sehr gute Führungsqualität hatten und weniger machtgieriger waren als er. So waren sie im Stande viel Verantwortung zu übernehmen, egal wie groß sie war. Dumbledore hatte aus seinen jugendlichen Fehlern gelernt und wurde so von vielen respektiert und gefürchtet, da er unheimliche Fähigkeiten hatte, seine Anhänger schnell dazu zu bringen, ihm zu vertrauen. Er wurde als Anführer so sehr geschätzt, dass ihm sogar oft der Posten als Zaubereiminister angeboten war. Harry war zwar der Meinung, dass er ein besserer Minister wäre als Cornelius oder Rufus, aber Dumbledore lehnte den Posten immer aus Angst vor zu viel Macht ab. Fudge selbst schätzte sogar die Führungsqualitäten von Dumbledore, dass er ihn häufig um Rat bat. Während des globalen Zaubererkriegs war Dumbledore in der Lage, sowohl Chef Auror Theseus Scamander als auch den Leiter der Abteilung für magische Strafverfolgung Torquil Travers nützliche Ratschläge zur Führung zu geben, wie Grindelwald am besten bekämpft werden kann, obwohl Travers diese ignorierte. Diese Ratschläge waren jedoch sehr wichtig für die Kriegsführung und halten vielen Zauberern. Ungefähr zu dieser Zeit unterrichtete Dumbledore auch persönlich den jungen Tom Riddle, der später zum mächtigsten dunklen Zauberer aller Zeiten wurde und auch zum größten Rivalen von Dumbledore. Nach Dumbledores Beförderung zum Schulleiter von Hogwarts leitete er die Schule so effektiv, dass er von vielen als bester Schulleiter aller Zeiten bezeichnet wurde. Dumbledores Führungskompetenzen machten ihn auch zum Hexenmeister des Zaubergamots und Ganz hohen Tier. Später gründete Dumbledore den Orden des Phönix der gegen Voldemort und dessen Todesser sowohl im Ersten als auch im Zweiten Zaubererkrieg im Geheimen kämpfte. Seine Führung ermöglichte es Dumbledore auch gegen die Dunklen Mächte zu kämpfen. Viele hochrangige Beamte des Ministeriums
%SEITE24
wie Travers, Dolores Umbridge und Fudge fürchteten die Führungskompetenz von Dumbledore und hatten Angst, dass er das Ministerium stürzen würde. Jedoch hatte Dumbledore nie die Absicht dies zu tun.
\paragraph{Duellkünste:}
%ABSATZ
Dumbledore war ein Duellant von außergewöhnlichem, fast konkurrenzlosem Können. Dumbledore war bereits in seinen jungen Jahren ein hochqualifizierter Duellant, da er sich im Drei-Wege-Duell zwischen Grindelwald und Aberforth behaupten konnte. Als Erwachsener hatte er während seiner Zeit als Professor für Verteidigung gegen die dunklen Künste große Erfolge darin, den Hogwarts-Schülern die Kunst der Kampfmagie beizubringen. Dies war an den Fähigkeiten seiner Schüler Newt, Theseus und Leta zu erkennen, die zu sehr guten Zauberern während des Globalen Zaubererkriegs wurden. Selbst Grindelwald gab zu, dass Dumbledore leicht in der Lage sein würde, jeden seiner Anhänger zu überwältigen, die eine Bedrohung in der Revolution darstellten. Dumbledore und Grindelwald konnten aufgrund ihres Blutpakts nicht gegeneinander kämpfen und so fand Grindelwald heraus, dass Dumbledore nur durch das Obscurial Credence Barebone getötet werden konnte. Selbst auf dem Höhepunkt von Grindelwalds Macht, wo der Dunkle Zauberer so viele Menschen getötet hatte, gaben viele Zauberer immer noch zu, dass sie glaubten, Dumbledore könne Grindelwald besiegen. Und tatsächlich konnte Dumbledore Grindelwald, trotz des Elderstabs besiegen, besiegen. Von vielen wurde es als größtes Duell der Zaubergeschichte bezeichnet. Dumbledore gab später zu, dass er nur etwas geschickter als Grindelwald war. Vor seiner Verletzung und dem Trinken des Medaillon-Sicherungstrank im Jahr 1996 blieb es eine Tatsache, dass jede Hexe oder jeder Zauberer, unabhängig von seinen oder ihren Fähigkeiten, Dumbledore nicht hätte besiegen können. Selbst als Dumbledore den Vorschlag von Voldemort ablehnte Professor für Verteidigung gegen die dunklen Künste zu werden, und Voldemort sehr wütend war, verzichtete Voldemort darauf seinen Zauberstab zu ziehen und Dumbledore anzugreifen. Während des ersten Zaubererkriegs entwaffnete Dumbledore Snape sofort, als dieser ihn über die Prophezeiung konfrontierte, bevor sein Gegner reagieren konnte. Während der Zeit als Patricia Rakepick in Hogwarts anwesend war, bemerkte Snape, dass Dumbledore der einzige war, der die extrem gefährliche Fluchbrecherin mit Sicherheit besiegen konnte und sogar Rakepick erschien nicht bereits zu sei, Dumbledore direkt zu konfrontieren. Auch Jacobs Geschwisterkind, vielleicht der beste aktuelle studentische Duellant, bemerkte, dass es nicht sehr schlau sein würde ins Büro von Dumbledore einzudringen. Dumbledore konnte auch erfolgreich fünf Werwölfe, die Hagrids Hütte während Fenrir Greybacks Angriff auf Hogwarts angriffen, abwehren. Nach der Rückkehr von Voldemort, als Dumbledore entdeckte, dass einer der Lehrkräfte tatsächlich ein Todesser war, der sich als Moody ausgab, unterwarf er Crouch Junior mit einem einzigen atemberaubenden Zauber von solcher Kraft, dass er die versiegelte Tür zerstörte und den Betrüger bewusstlos zu Boden schlug. Während seiner versuchten Verhaftung durch das Zaubereiministeriums im Schuljahr 1995-1996 besiegte Dumbledore mit einem einzigen Zauber sofort Fudge, Umbridge, Kingsley Shacklebolt (der heimlich auf Dumbledores Seite aktiv war) und John Dawlish, als sie versuchten ihn zu verhaften und entkam. Während der Schlacht in der Mysteriumsabteilung nahm Dumbledore alle Todesser zusammen gefangen, die nicht einmal alle Ordensmitglieder zusammen besiegen konnte. Kurz danach fand das Duell im Atrium statt. Vor diesem nahm Dumbledore auch Bellatrix Lestrange gefangen, die eine der gefährlichsten Anhänger von Voldemort war. Anschließend duellierte er sich ruhig mit Voldemort, konterte seine Angriffe und reagierte mit einem so mächtigen Angriff, dass er den Dunklen Lord ständig in die Defensive zwang. Schließlich hielt er Voldemort in einer Kugel von Wasser gefangen und zwang ihn so zu verschwinden. auch war Dumbledore sehr gut darin non-verbale Zauber durchzuführen. Dumbledores Duellstiel war äußerst vielseitig und geschickt und konnte sich mühelos an die Situation anpassen. Währen der Schlacht in der Mysteriumsabteilung konnte Dumbledore so beispielsweise die Todesser mit leichter Magie überwältigen. Aber als Dumbledore gegen den unglaublich mächtigen Voldemort kämpfte, benutzte er nicht nur einige sehr mächtige Zaubersprüche, sondern manipulierte auch die Umgebung, um sicherzustellen, dass Bellatrix nicht in den Kampf einschreiten konnte und Harry in Sicherheit war. Er nutzte auch seinen Phönix Fawkes während des Kampfes, indem er dessen Unsterblichkeit ausnutzte, um Voldemorts Todesfluch zu blockieren und manchmal zu apparieren, um Voldemorts Zaubersprüchen auszuweichen, bevor er schnell konterte. Dies war eine Methode die sich als äußerst effizient gegen den Dunklen Lord erwies.
\paragraph{Verteidigung gegen die dunklen Künste:}
%ABSATZ
Albus Dumbledore war mehr als zwanzig Jahre lang als Professor für Verteidigung gegen die dunklen Künste tätig, bevor Travers ihm verbot das Fach zu unterrichten. Dumbledore wurde als einer der besten Lehrer der gesamten Schule angesehen und zeigte sein Fachwissen bei der Abwehr von Dunklen Flüchen. Dumbledore hatte den Vorteil, dass er schon früh wegen Grindelwald in Kontakt mit den dunklen Künsten kam. Er konnte sein Wissen als Lehrer so gut weitergeben, dass Schüler wie Leta und Newt zu sehr guten Zauberern wurde und sich sogar gegen Grindelwald wehren konnten. Seine gesamten Fähigkeiten auf diesem Gebiet zeigten sich bei den Duellen gegen Grindelwald und Voldemort, die beide als sehr gefährliche dunkle Zauberer galten. Während beider Rebellionen waren viele Zauberer der Meinung, dass nur Dumbledore die Schreckenherrschaften von beiden dunklen Zauberern beenden konnte. Dies gelang Dumbledore schließlich und er konnte Beide besiegen. Sein Wissen und Können in Bezug auf Schutzverzauberungen zeigte sich am besten während der Zeit als Schulleiter von Hogwarts , als er persönlich für die Aufrechterhaltung des magischen Schutzes in der Schule verantwortlich war. Viele meinten es gäbe keinen sichereren Ort auf der Welt als das Schloss Hogwarts, selbst die Gringotts Zaubererbank sei nicht so sicher wie das Schloss. Dieses Level der Sicherheit hatte zuvor kein anderer Schulleiter erreicht. In der Tat war Dumbledores Verteidigungszauber so mächtig, dass nicht mal Voldemort diesen durchbrechen konnte. Dumbledore geling es auch sehr leicht dunkle Kreaturen wie Irrwichte abzuwehren. Auch konnte er einen äußerst mächtigen Patronuszauber produzieren, der in der Lage war alle, von Fudge in Hogwarts stationiert Dementoren zu vertreiben, obwohl es sich bei dem Patronus um einen nicht-körperlichen handelte. Dumbledore war auch mit dem Konzept vertraut, dass man Inferi mit
%SEITE25
Feuer abwehren konnte. Er erklärte Harry, wie die Inferi von Voldemort in der Horkrux-Höhle erschaffen wurden, was zeigte, dass er sich gut mit der Materie auseinandergesetzt hatte. Anschließend wehrte Dumbledore die Inferi mit einem einzigen Feuersturm ab. Dumbledore war sich auch sicher, dass ein Obscurial möglicherweise geheilt werden kann, indem die Gefühle, die eine Person zu dem Obscurial hat, das Obscurial durch Liebe und Zugehörigkeit heilt. Dumbledore hatte auch viel Wissen über Nebenwirkungen verschiedener dunkler Zaubersprüche. So erkannte er beispielsweise, dass Mrs Norris nicht getötet wurde, sondern nur versteinert wurde. Auch wusste er wie man die Versteinerung durch den Alraunen-Wiederbelebungstrank rückgängig machen konnte, die sonst nur ein erfahrener Auror wusste. Dumbledore wusste auch, dass der Todesfluch einen Phönix nicht töten konnte und so benutzte er seinen Phönix Fawkes häufig in Duellen, um sich vor den Todesfluch zu schützen. Er war auch in der Lage den außergewöhnlich starken Fluch auf Vorlost Gaunts Ring schnell genug zu identifizieren, sodass er seine Auswirkung so lange unterdrücken konnte, dass Snap später die Behandlung fortsetzen konnte. Dumbledore konnte auch beide Verteidigungsanlagen bei den Horkruxen erfolgreich überwinden. Auch geling es Dumbledore den Ring von Gaunt zu zerstören, sodass auch Voldemorts Seelenteil zerstört wurde, aber nicht der Stein der Auferstehung zerstört wurde. Auch konnte Dumbledore Flüche von dunklen Objekten entfernen. So war er beispielsweise in der Lage die Flüche in Rakepicks dunklen Artefakten zu entfernen, um sie zu untersuchen, was nicht einmal Snape gelang. Dumbledore kannte sich auch gut in der alten Magie aus und konnte diese effektiv nutzen. Auch hatte Dumbledore außergewöhnliche Fähigkeiten, dunkle Kreaturen zu verstehen und abzuwehren. Er konnte mühelos fünf Werwölfe von Hagrids Hütte fernhalten und schließlich sogar Jacobs Geschwister vor Greyback retten, der ein sehr gefährlicher Werwolf war. Dumbledore hatte eine starke Abneigung gegen dunkle Kreaturen wie Dementoren, da diese nicht sehr vertrauenswürdig waren. Trotzdem hatte er ein gutes Sachwissen über diese Kreaturen und konnte auch Jacobs Geschwister Irrwichte fehlerfrei erklären. Auch lernte er diesem den Riddikulus-Zauber.
\paragraph{Dunkle Künste:}
%ABSATZ
Dumbledore verwendete sehr selten die Dunklen Künste, war jedoch trotzdem sehr talentiert auf dem Gebiet dieser schwarzen Magie. Laut McGonagall wäre aber Dumbledore genauso im Stande wie Voldemort die zerstörerischen dunklen Kräfte einzusetzen. Jedoch war Dumbledore "zu edel, um sie zu benutzen". Dumbledore konnte sich dank seines umfangreiches Wissen zum Thema Dunkle Magie gut in dunkle Zauberer wie Voldemort versetzen und so ihre nächsten Angriffe in Duellen vorhersehen. Auch konnte er verstehen wie extrem fortgeschrittene alte Magie wie Horkruxe hergestellt wurden. Ebenfalls erkannte er früh das Mrs. Norris nicht getötet wurde, sondern nur versteinert wurde. Obwohl Dumbledore wegen seiner vornehmen Art selten solche Zauber verwendete, konnte er eine ganze Reihe von sehr begabten Ministeriumsbeamten mit nur einem Zauberer zurückschleudern. Auch schien Dumbledore im Stande zu sein den Todesfluch auszuführen. Als Snape während seiner Zeit als Todesser zu nah an die Schule kam entwaffnete ihn Dumbledore und Snape befürchtete deshalb, dass Dumbledore ihn töten würde. auch konnte Dumbledore Zauber aus dem Bereich der Nekromantik verwenden. Für komödiantische Zwecke konnte er mehrere Skelettstücke zu ganzen Skeletten zusammenfügen und kontrollierte sie so, dass sie an Halloween tanzten.
\paragraph{Okklumentik und Legilimentik:}
%ABSATZ
Dumbledore war sowohl in Okklumentik und Legilimentik sehr geschickt. Legilimentik konnte er ohne Zauberstab und non-verbal nutzen. Harry vermutete häufig, dass Dumbledore in einen Gedanken war, jedoch gab es Personen, die sehr gute Okklumentik beherrschten wie Anhänger von Voldemort und seine Mitarbeiter wie Slughorn. Bei diesen konnte Dumbledore keine Legilimentik anwenden. Seine Fähigkeiten in Legilimentik waren so gut, dass er sogar verfälschte Versionen von Erinnerungen wieder zu richtigen machen konnte. Als Dumbledore beispielsweise nach Informationen über Voldemort bei Zeugen suchten, fand er Leute, bei denen Voldemort die Erinnerungen verändert hatte wie Morfin Gaunt und Hokey. Bei diesen konnte er jedoch trotzdem die richtigen Erinnerungen finden. Dumbledore gab zu, dass er dafür "viel geschickte Legilimentik" benötigte, um dies zu erreichen und es schwierig für ihn war. Auch lehrte er Jacob die Kunst der Legilimentik. Dumbledores Geschicklichkeit in Okklumentik war so groß, dass er zu den wenigen neben Grindelwald und Snape gehörte, die in der Lage waren, Gedanken vor Voldemort zu verbergen, der angeblich der mächtigster Legilimentiker aller Zeiten war. Irgendwann bemerkte er, dass Harry vielleicht auch Okklumentik lernen könnte, um zu verhindern, dass Voldemort die Gedanken von Harry manipulieren konnte. Jedoch wollte Dumbledore Harry selbst nicht trainieren, da er Angst hatte, dass so Voldemort in die Gedanken von ihm schauen konnte. Deshalb gab er diese Aufgabe unglücklicherweise an Snape weiter. Er konnte auch seine Gedankenverteidigung so weit senken, dass ein schlechter Legilimentiker in seinen Geist eindringen konnte. Jedoch konnte er so trotzdem einschränken, was sie sehen konnten. Dies tat er beispielsweise für Jacobs Geschwister.
\paragraph{Zauberstablose und nonverbale Magie:}
%ABSATZ
Dumbledore beherrschte sowohl nonverbale als auch zauberstablose Magie sehr gut. Beide Gebiete der Magie waren sehr komplex und schwer auszuführen. Die Fähigkeiten von Dumbledore waren so gut, dass er sogar beides gleichzeitig nutzen konnte. Einige Leute wie Antonin Dolohow, wie während der Schlacht
%SEITE26
um die Mysteriumsabteilung zu sehen waren, beherrschten zwar nonverbale Magie, konnten aber nur sehr schwache Zauber ausführen. Dumbledore dagegen konnte zum Beispiel Hunderte von lila Schlafsäcken für die Schüler von Hogwarts zaubern, ohne ein Wort zu sagen. Während des Abschlussfestes im Jahre 1992 musste er nur inn die Hände klatschen, um die Banner der Großen Halle von Slytherin zu Gryffindor zu ändern. Während des Begrüßungsfestes 1993 löschte er außerdem eine Kerze und zündete sie nur mit einer bloßen Handbewegung an. Auch während er in der Höhle war, besiegte Dumbledore die von Voldemort errichteten Verteidigungsmaßnahmen nur mit zauberstabloser und nonverbaler Magie. Später sprach er auch den Gefrierzauber nonverbal aus, um Harry davon abzuhalten, in seinen Tod einzugreifen. Auch konnte er den Zauberspruch "Homenum revelio" zauberstablos und nonverbal ausführen. Er duellierte sich auch mit Voldemort und besiegte die Todesser während der Schlacht in der Mysteriumsabteilung, ohne jemals einen einzigen Zauber auszusprechen. Auch warf Dumbledore Harry während des Duells wiederholt weg und dies mit einer bloßen Geste, um den Jungen aus dem Duell mit Voldemort herauszuhalten. Er war auch in der Lage, die Lichtquellen der Großen Halle zu dimmen und die Namen der Champions des Trimagischen Turniers auf seine Hand zu rufen. Er setzte sogar die Garderobe des jungen Riddles im Weißenhaus in Brand, ohne überhaupt die Garderobe anzusehen.
\paragraph{Zaubererstellung:}
%ABSATZ
Dumbledore war bekannt für seine Fähigkeit, einzigartige magische Zauber und Gegenstände zu erschaffen und bereits vorhandene Zauber zu verbessern oder zu modernisieren. So erfand er beispielsweise den Deluminator. Dieses Gerät konnte nicht nur Lichtquellen aus der unmittelbaren Umgebung entfernen, sondern auch Lichtquellen erhellen. Später wurde dieser an Ron vererbt. Ron stellte fest, dass er durch den Deluminator auch manchmal die Stimmen von Harry und Hermine hören konnte. Auch konnte er mit dem Deluminator zu dem Standort der Beiden apparieren, was anscheinend eine Funktion war, die Dumbledore vor seinem Tod zu dem Deluminator hinzufügte. Er entwickelte auch eine Methode zur Kommunikation mit dem Patronuszauber, die er den Mitgliedern des Orden des Phönix beibrachte.
\paragraph{Beherrschung der Elementarmagie:}
%ABSATZ
Dumbledore war auch fähig die Elemente, insbesondere Feuer, Wasser und Wind, mit Magie zu manipulieren, um sie in Duellen oder als Abwehr zu verwenden. Um sich beispielsweise heimlich mit Newt zu unterhalten, kombinierte Dumbledore Wasser und Wind, um einen dicken Nebel zu bilden, und zerstörte diesen wieder, nachdem er mit dem Gespräch fertig war. Am besten konnte er jedoch das Feuer manipulieren. So setzte er beispielsweise Riddles Kleiderschrank leicht in Brand, ohne die Inhalte zu verbrenn, um Tom seine Zauberkräfte zu zeigen. Auch konnte er einen Feuerring erzeugen, um eine Menge Inferi in der Horkrux-Höhle abzuwehren, der den riesigen Ozean unauslöschlich einnahm. Auch konnte er Feuerbälle ins Wasser werfen, um Harry von den Inferi zu befreien, die ihn festhielten. Auch erzeugte er einmal Win, um eine Kerze zu löschen und zündete sie später leicht mit einer bloßen Geste wieder an. Auch war Dumbledore in der Lage das Gubraith-Feuer zu beschwören, was eine unglaubliche fortschrittliche Magie voraussetzte. Dies konnten nur wenige und es beeindruckte Hermine sehr. In seinem Duell gegen Voldemort benutzte er ein Feuriges Seil, das es schaffte die Abwehr von Voldemort zu überwinden. Auch manipulierte er as Wasser im Brunnen der magischen Brüder, um Voldemort anzugreifen.
\paragraph{Magisch Mehrsprachig:}
%ABSATZ
Dumbledore konnte viele fremde magische Sprachen sprechen. Unter anderem konnte er Meerisch, Koboldogack und Parsel sprechen. Dumbledore konnte bei Morfin Gaunt in seinen Gedanken die Parsel nachvollziehen und zitieren, was darauf hinwies, dass er zumindest etwas davon sprechen konnte.
\paragraph{Verwandlung:}
%ABSATZ
Dumbledore war auch sehr bekannt dafür, dass er die komplexe Kunst der Verwandlung bereits vor seinem Abschluss beherrschte. Da sein Wissen und seine Kenntnisse in diesem Bereich so gut waren, schrieb er bereits einige Artikel für Verwandlung heute. Die Artikel hatten einen so hohen Standard, dass Bathilda Bagshot die UTZ-Professorin Griselda Marchbanks konnte und er seine UTZ-Prüfung in Verwandlung machte. Als Erwachsener war Verwandlung eines der wichtigsten Fachgebiete von Dumbledore geworden, da Dumbledore, nachdem ihm Travers das Unterrichten von Verteidigung gegen die Dunklen Künste verboten hatte, als nächstes Unterrichtsfach Verwandlung wählte. Dies Fach unterrichtete er so gut, dass er viele Jahre lang Leiter der Verwandlung an der Hogwarts-Schule war, bis er zum Schulleiter aufstieg. Dumbledore wurde auch als einer der besten Lehrer des Faches aller Zeiten angesehen. Dumbledores Nachfolgerin auf dieser Position war McGonagall, die Dumbledore selbst ausbildete. Dumbledore arbeitete als Kolumnist für Verwandlung heute. Einer seiner Artikel war so gut, dass er es sogar in den Tagespropheten schaffte. Dumbledore war auch ziemlich gut im Gebiet Animagus. So gelang es ihm beispielsweise McGonagall im Alter von siebzehn Jahren zu einem Animagus zu machen, was bekanntlich sehr schwierig war. Dumbledore beherrschte auch die Beschwörung und das Verschwinden sehr gut. So beschwor er unteranderem eine Trage für den bewusstlosen Harry, mehrere Hüte für die Halloween-Skelete, die er danach auch wieder verschwinden ließ, hunderte lila Schlafsäcke für die Gryffindor-Studenten, um diese in der Großen Halle unterzubringen, als Sirius in das Schloss einbrach und zwei Sessel, auf denen Arabella Figg, während Harrys Anhörung sitzen konnte. Dumbledore verwendete seine Künste der Verwandlung auch im Kampf. So verwendete er beispielsweise im Duell mit Voldemort im Zaubereiministerium den Zauberspruch Piertotum Locomotor, damit die Statuen des Brunnes im Atrium Bellatrix festnahmen, Harry schützen und Voldemort Flüche abwehrten. Auch beschwor er ein Feuerseil, dass Voldemort in eine Schlange verwandeln musste. Auch verwandelte er die Banner in der Großen Halle von Slytherin zu Gryffindor und dies mit einem bloßen Händeklatschen. Dumbledore hatte auch ein unheimliches Talent, um Verwandlungszauber, die von anderen ausgeführt wurden vorherzusehen und ihnen so entgegenzuwirken. Dies war daran zu erkennen, dass er schnell Slughorn erkannte, der sich in einen Stuhl verwandelt hatte. Nach einem leichten Stoß seines Zauberstabs verwandelte sich Slughorn zurück in seine eigene Gestalt. Als Dumbledore mit Jacobs Geschwister sprach, bemerkte er, dass sie ein Käfer belauschte. Dieser Käfer war die Animagus-Form von Rita Kimmkorn. Er forderte sie auf, damit aufzuhören und in ihre wahre Form zurückzukehren. Vermutlich hat er auch den Verwandlungszauber aufgehoben, den McGonagall in der Schachbrettkammer verwendet hatte, um die Kammer schnell zu durchqueren und so zu Harry zu kommen.
\paragraph{Zauberkunst:}
%ABSATZ
Albus war bereits in jungen Jahren im Zaubern äußerst erfolgreich, da er Artikel, die er über diesen Zweig der Magie schrieb, in der Fachzeitschrift Zentralfragen der Zauberkunstveröffentlicht wurden. Auch seine UTZ-Prüferin und Professorin Marchbanks war von ihm sehr beeindruckt und sagte, dass sie so etwas noch nie zuvor gesehen hatte. Er erhielt
%SEITE27
auch den Barnabus-Finkley-Preis für Außergewöhnliche Zauberei und konnte bereits einen starken Desillusionierungszauber hervorbringen, der es ihm ermöglichte sich ohne einen Unsichtbarkeitsumhang unsichtbar zu machen, was eine enorme Leistung war. Als Erwachsener war Dumbledores Beherrschung der Zauber so gut, dass ihm der Titel des Großzauberers verliehen wurde. Seine Kenntnisse des unglaublich komplexen Patronuszaubers waren so weit fortgeschritten, dass er mehrere neue Verwendungsmöglichkeiten für den Zauber entwickeln konnte, sodass der Zauber als Kommunikationsform verwendet werden konnte (wobei der Zaubernde mit seiner eigenen Stimme durch den Patronus sprechen konnte). Dumbledore war auch in der Lage, den äußerst komplexen Fidelius-Zauber effektiv auszuführen und schütze so den Grimmauldplatz 12, den er zum Hauptquartier des Orden des Phönix ausgewählt hatte. Dumbledore war sogar in der Lage, den äußerst mächtigen und uralten Blutsverwandtschaftsschutz auszuführen, um Harrys Schutz vor Voldemort für 17 Jahre weiter zu gewährleiten, während er bei seiner Tante Petunia Dursley lebte. Auch konnte Dumbledore den Portus-Zauber erfolgreich ausführen, um Portschlüssel zu erstellen. So machte er den schwarzen Kessel in seinem Büro zu einem Portschlüssel, der Harry und die Weasley-Kinder zum Grimmauldplatz schickte. Später verwandelte er den Kopf der goldenen Zaubererstatue in einen Portschlüssel, der Harry nach Hogwarts brachte. Selbst ohne den Deluminator konnte Dumbledore leicht Lichtquellen mit einem unbekannten Zauber löschen, was er zweimal tat, während er in seinem Büro war und bei der Ankündigung des Trimagischen Turniers. Er verzauberte auch den Deluminator und verlieh ihm so die Fähigkeiten. Er war auch für den Zauber des Deluminators verantwortlich, der es ihm ermöglichte, Ron nach seiner Abreise zurück zu Harry und Hermine zu führen. Er konnte eine Vielzahl mächtiger und einzigartige Zauber. Diese verwendete er beispielsweise um den Spiegel Nerhegeb zu verzaubern, um den Stein der Weisen nur einer Person zu geben, die ihn finden, aber nicht verwenden wollte, was selbst für Quirrell galt. Das Wissen über Voldemort und seine eigene Erfahrung ermöglichten ihm es die Sicherheitsmaßnahmen leicht zu umgehen. Beispielsweise besaß Newt eine Hand, die es ihm ermöglichte den Standort von Dumbledore zu finden. Auch beschwor Dumbledore einen dichten Nebel, als er mit Newt redete, sodass sie nicht vom Ministerium lokalisiert werden konnten. Dumbledores Beherrschung und Kenntnis über Vereidigungszauber waren sehr gut, da Dumbledore als Schulleiter von Hogwarts für die Kontrolle der Schutzmaßnahmen des Schlosses verantwortlich war und die Abwehr so stark war, dass sie nicht einmal Voldemort durchdringen konnte. Hogwarts galt unter Dumbledore als einer der sichersten Orte der Welt. Er konnte die Zauber auch leicht abschalten und neu erschaffen, was sich zeigte als er den Anti-Apparier Zauber aufhob, damit die Schüler lernen konnten, wie man appariert. Dumbledore war auch in der Lage Türen so zu verschließen, dass sie auf irgendeine Weise geöffnet werden konnte. Nur Dumbledore selbst konnte die Verzauberung rückgängig machen. Auch war er in der Lage eine mächtige Alterslinie um den Feuerkelch zu entwerfen, um sicherzustellen, dass niemand unter 17 Jahren diese überqueren konnte. Dies war nicht nur eine mächtige Verzauberung, sondern sie war auch humorvoll. Sobald jemand versuchte sie mit alternden Tränken zu überqueren, wuchs dieser Person ein Bart. Zwar wurde die Alterslinie schließlich doch überwunden, jedoch auf eine nicht vorhersehbare Weise, nachdem Bartemius Crouch Junior, verkleidet als Moody, den Namen von Harry in den Kelch warf. Er verstand auch den Feuerkelch nd deshalb war ihm bewusst, dass Harry trotzdem am Wettbewerb teilnehmen müsste. Auch als Dumbledore noch nicht Schulleiter war, war er anscheinend für die Schutzzauber der Schule verantwortlich, die Grindelwald erfolgreich abhielten. Auch erlaubte er Newt, Tina Goldstein und ihren Verbündeten nach Grindelwalds Kundgebung in Paris nach Hogwarts zu kommen, um dort Rat zu finden. Auch war Dumbledore sogar in der Lage, dass unortbare Hogwarts-Schloss vorübergehend ortbar zu machen, um es dem No-Maj Jacob Kowalski zu ermögliche, mit Newt das Schloss zu besuchen, um Rat zu erhalten. Dumbledore konnte auch das verwüstete Haus von Slughorn wieder reparieren und alles wieder zu seinem fest Ort bringen. Dumbledore verwendete den Zauber Partis Temporus, um durch den Feuerring zu kommen, den er zur Abwehr der Inferi erzeugt hatte, sodass er und Harry ihr Boot sicher erreichen konnten. Dumbledore extrahierte oft seine Erinnerungen und tat sie später in sein Denkarium, um sie beispielsweise Harry zu zeigen oder sie sich selbst nochmal anzusehen. Dumbledore war auch im Stande den Verwirrungszauber non-verbal auszusprechen, was nicht einmal Snape gelang. Er war auch in der Lage, den Auferstehungsstein in den Schnatz einzuführen, den er später Harry vererbte und ihn so verzauberte, dass er sich nur öffnete, wenn Harry den Schnatz in den Mund nahm. Sein Zauber Homenum revelio war sehr mächtig. Dumbledore war der einzige Zauberer, der bisher in der Lage war, die mächtige Magie des Umhangs der Unsichtbarkeit zu durchdringen und fortan die Anwesenheit von Harry erfolgreich zu spüren, während Harry den Umhang auf sich hatte. Er beherrschte auch den Fall-Verlangsamungszauber nonverbal und stoppte damit Harrys schnellen Fall, nachdem er von Dementoren während eines Quidditch-Spiels angegriffen wurde. So verletzte sich Harry nicht und auch seine Brille ging nicht kaputt. Er rief auch mit einer einfachen Geste die Champions für das Trimagische Turnier aus dem Feuerkelch, obwohl dies auch mit einem einfachen Zauber ging. Dumbledore bevorzugte aber diese Variante. Dies war auch Jahre zuvor so, als Dumbledore die Phiole mit dem Blutpakt zwischen Grindelwald und ihm schweben ließ. Dumbledore benutzte einen so mächtigen Entwaffnungszauber, der die Tür zu Crouchs Büro sprengte und den als Moody verkleideten Todesser zurückschleudern ließ. Er war auch in der Lage einen Zauber zu wirken, der einen flüchtenden Todesser in der Mysteriumsabteilung mühelos zurückzog. Dumbledore konnte auch eine neue Version des Silencio-Zaubers nutzen, die anstatt das Ziel zum Schweigen zu bringen, die natürlichen Geräusche abschaltete und andere Geräusche wiedergab, die von allen Außerstehenden nicht gehört wurde. Dies tat er, um sicherzustellen, dass er und Snape bei einem Gespräch nicht gestört wurden, als sie sich darüber unterhielten, dass Snape die Prophezeiung an Voldemort weitergab. Dumbledore nutzte den Lumos-Zauber nicht nur in dunklen Bereichen der Horkrux-Höhle und im Verbotenen Wald, sondern auch, um das von dem Zauber erzeugte Licht in die Mitte des Ozeans zu werfen, um die gesamte Höhle zu beleuchten. Dumbledore benutzte auch den Piertotum Locomotor Zauber erfolgreich, da er diesen im Vorraum des Zaubereiministeriums benutzte, um die Statuen des Brunnens im Duell mit Voldemort zu manipulieren. Dumbledore war auch im Stande den Rennervate-Zauber zu nutzen, um Viktor Krum nach der Attacke von Bartemius Crouch Junior wieder zu beleben. Er wärmte Harry auf und trocknete seine Klediung mit nur einem einzigen unbekannten Zauber.
%SEITE28
\paragraph{Zaubertränke:}
%ABSATZ
Dumbledore war auch im Bereich Zaubertränke sehr gut informiert und auch erfolgreich. Als er noch jung war schrieb er laut Elphias Doge gute Artikel für die Angewandte Zaubertrankkunde, einer akademischen Zeitschrift über Zaubertränke. Dumbledore hatte auch viel Wissen und Verständnis über die Eigenschaften vom Liebestrank, da er verstand das Liebestrank nur das Gefühl von vorgetäuschter Liebe gab und keine wahre Liebe vorherrief. Auch wusste er von dem Alraune-Wiederbelebungstrank, den er, Sprout und Snape herstellten, um die Versteinung des Basilisken der Kammer des Schreckens rückgängig zu machen. Auch erkannte er, dass das Getränk in Moodys Flasche Vielsafttrank war. Als sich herausstellte, dass dies Bartemius Crouch Jr in Gestalt von Moody war, wusste Dumbledore, dass Veritaserum nicht in der Lage war, die Abwehrkräfte des talentierten Okklumentoren zu überwinden. Deshalb betäubte er den Todesser, bevor er den Wahrheitstrank einsetzte und sorgte dafür, dass eine stärke Menge verwendet wurde. Dumbledore wollte anfangs selbst einen Wahrheitstrank brauen, um die Erinnerung über Voldemort von Slughorn zu bekommen, jedoch entschied er sich dagegen, da Slughorn auch ein guter Okklumentor war und vielleicht einen Gegentrank brauen könnte. Auch erkannte Dumbledore, das seine Alterslinie um den Feuerkelch möglicherweise mit einem Alterungstrank überwunden werden konnte und behob diesen Fehler, sodass Leute, die noch nicht alt genug waren und einen Alterungstrank nahmen, einen Bart und weiße Haare bekamen. Außerdem war es vermutlich Dumbledore, der McGonagall lernte den komplexen Animagus-Trank auszuführen. Dies wurde benötigt, um ein Animagus zu werden. Manche Leute behaupten sogar, Dumbledore hätte den Trank für sie gebraut. Dumbledore erkannte auch schnell die Qualität des Gripsschärfungstrank, den Merula ihm und Snape vorstellte. Dumbledore behauptete, dass dieser Trank der beste wäre, den er jemals gesehen hatte. Später stellte sich heraus, dass der Trank eigentlich von Penny war. Als Dumbledore in die Unterirdische Kammer ging, hatte er anscheinend auch kein Problem den Feuerschutztrank zu finden, um die Türblockierende Flammen von Snape zu durchqueren und so Harry zu retten.
\paragraph{Alchemie:}
%ABSATZ
Dumbledore war offenbar seit seiner Jugend in der alten magischen Wissenschaftskunst der Alchemie sehr begabt. Als Schüler bekam er sogar Aufmerksamkeit von Nicolas Flamel, der als einer der berühmtesten Alchemisten aller Zeiten galt. Bekannt auf dem Gebiet der Alchemie wurde er unter anderem durch die Freundschaft mit Flamel. Seinen Durchbruch schaffte er, nachdem er die Goldmedaille für einen bahnbrechenden Beitrag zur Internationalen Alchemie Konferenz in Kairo für seine Vorstellung bekam. Als Erwachsener trug er neben seinem alten Freund Flamel erheblich zur Weiterentwicklung des Gebiets bei. Dumbledore bewahrte auch später den Stein der Weisen in Hogwarts auf, aber entschied sich später mit Flamel ihn zu zerstören. Auch hat Dumbledore mit Flamel die 12 Anwendungen von Drachenblut entdeckt.
\paragraph{Magische Chemie:}
%ABSATZ
Dumbledore hatte auch viel Wissen über magische Substanzen, insbesondere die Substanz von Drachenblut, da er auf diesem Geiet als führender Experte angesehen wurde und es ihm gelang, zwölf Möglichkeiten zu finden, es praktisch zu verwenden. Sein Wissen uns seine Erfahrung in der Erforschung von Drachenblut ermöglichten es ihm auch es leicht von menschlichem Blut zu unterscheiden, indem er eine kleine Menge davon probierte.
\paragraph{Magische Rechtskompetenz:}
%ABSATZ
Dumbledore wusste viel über die Gesetze der Zaubererwelt und in seinem Büro waren auch mehrere Bücher über dieses Thema zu finden. Seine juristische Expertise ermöglichte es ihm Harry während seines Prozesses wegen Magie eines Minderjährigen erfolgreich von allen Anklagen zu befreien. Es gelang ihm auch die Unschuld von Morfin Gaunt über die Ermordung von Tom Riddle Senior und dem Rest der Familie Riddle darzustellen, nachdem er die Erinnerungen an das Ereignis gesehen hatte. So konnte er das Ministerium überzeugen Morfin aus Askaban freizulassen, bevor er starb. Er zeigte auch sein Wissen, als er Harry zu Slughorn mitnahm und das seine Erlaubnis ausreichen würde, um Harry zu erlauben als Minderjähriger Magie anzuwenden, falls es zu einem Angriff kommen sollte.
\paragraph{Pflege Magischer Geschöpfe:}
%ABSATZ
Dumbledores Wissen und Verständnis für magische Kreaturen war außergewöhnlich und fast vergleichbar mit herausragenden Meistern des Gebietes wie Scamander oder Hagrid. Dumbledore wusste, dass die alte Legende, dass ein Phönix zu einem Mitglieder der Familie Dumbledore kam, wenn das Mitglied Hilfe brauchte, wahr war. Dumbledore verstand ach schnell, wie Newts Haustier Niffler das Fläschchen mit dem Blutpakt von ihm und Grindelwald gestohlen hatte. Dumbledore war auch der einzige neben Hagrid, der wusste wie man an Fluffy vorbei kam. Dumbledore war auch äußerst geschickt darin mit magischen Kreaturen wie Phönixen, Thestralen, Wassermenschen, Zentauren und Kobolden umzugehen. Teilweise konnte er auch die Sprachen der Kreaturen sprechen. Auch blieb ihm sein Haustier Fawkes immer treu. Er blieb sogar bis zu Dumbledores Tod in der Schule und nachdem Dumbledore tot war flog er weg und kehrte nie wieder zurück. Auch benutzte Dumbledore den Phönix im Duell mit Voldemort, um den Todesfluch abzuwehren, der dem Tier nichts tut. Wenn Dumbledore nicht apparierte nutzte er auch häufig Thestrale als Transportmittel. Dumbledore kommunizierte während der zweiten Aufgabe des Trimagischen Turniers mit den Wassermenschen des Sees, um den Fortschritt der Champions zu erfahren. Dumbledore war auch einer der wenigen Zauberer, die den Hogwarts Zentauren vertrauten und sie respektierten. Deshalb konnte er diese auch überzeugen ihm Umbridge freizulassen ohne sie zu verletzen. Es ging schließlich so weit, dass die Zentauren und Wassermenschen an der Beerdigung von Dumbledore teilnahmen. Dumbledore war auch ein berühmter Mitwirkender bei der Verteidigung der Riesen vor Zauberergewalt. Deshalb gelang es ihm Karkus dazu zu bringen, den Orden des Phönix zu unterstützen. Als Golgomath später Karkus tötete, gelange es Dumbledore die Riesen, die sich Golgomath nicht angeschlossen hatten, davon zu überzeugen, sich nicht den Todessern anzuschließen. Diese Beispiele zeigen, dass er auch sehr gut mit Risen verhandeln und umgehen konnte. Er war auch einer der wenigen Magier, die gut mit Hauselfen umgingen. So gab er beispielweise
%SEITE29
Dobby und Winky Jobs in Hogwarts und ermutigte Harry Kreacher freundlicher zu behandeln. Diese ganzen Beispiele zeigen, dass Dumbledore insgesamt sehr gut mit magischen Kreaturen umgehen konnte.
\paragraph{Geschichte der Zauberei:}
%ABSATZ
Dumbledore verfügte über ausgezeichnete Kenntnisse der magischen Geschichte. Er war befreundet mit der berühmten Historikerin Bathilda Bagshot und beschäftigte sich während seiner Freundschaft mit Grindelwald sehr über die Heiligtümer des Todes. Auch wusste Dumbledore viel über mächtige magische Artefakte neben den Heiligtümern des Todes, die schon sehr alt waren. Dank dieses Wissens fand Dumbledore heraus das es sich bei dem Unsichtbarkeitsumhang von James Potter um eines der Heiligtümer handelte. Auch fand er heraus, dass der Stein in Gaunts Ring der Auferstehungsstein war. Außerdem war Dumbledore in der Lage den Inhalt des historischen poetischen Buches über die Wahrsagerei Die Voraussagungen des Tycho Dodonusleicht zu rezitieren, als er von Torquil Travers davon befragt wurde. Er wusste auch viel über das Trimagische Turnier und konnte auch erklären, warum ursprünglich entschieden wurde, dass das Turnier nicht mehr stattfinden würde. Dumbledore war auch einer der Wenigen, der die Erstaufgabe von Die Märchen von Beedle dem Barden erworben hatte. Nachdem er den Runeninhalt übersetzt und in rot in das Buch geschrieben hatte, war er sich über die original Geschichten bewusst. Deshalb veröffentlichte er später Notizen über die Geschichten.
\paragraph{Astronomie:}
%ABSATZ
Dumbledore schien ziemlich gut und geschickt im Bereich Astronomie zu sein, da er eine spezielle Taschenuhr besaß, die sich bewegende Planeten am Rand hatte. Er konnte die Bedeutung übersetzen und so die aktuelle Zeit erkennen.
\paragraph{Wahrsagerei:}
%ABSATZ
Obwohl Dumbledore das Thema selten aufgegriffen hatte und zugegeben hatte das Gebiet nicht zu mögen, zeigte er ein anständiges Wissen über Wahrsagerei. Zum Beispiel konnte er die Fragen von Travers über das Wahrsagungsbuch Die Voraussagungen des Tycho Dodonus richtig beantworten. Seine Kenntnisse ermöglichten es ihm auch den Unterschied zwischen echten und falschen Sehern zu sehen. So ließ er sich nicht von Sybills falschen Vorhersagen beeindrucken, merkte aber schließlich doch, dass sie eine echte Seherin war, als sie eine wirkliche Vorhersage machte. Dumbledore konnte die Prophezeiung auch erfolgreich entschlüsseln und fand heraus, dass der Auserwählte die Person war, die Lord Voldemort besiegen konnte. Auch fand er heraus, dass es in der Prophezeiung über Harry und nicht Neville Longbottom ging. Dumbledore befand mindestens ein Gerät, dass ihm von Ereignissen erzählte, die an anderen Orten stattfanden. So konnte er erkennen, dass Arthur Weasley wirklich von Nagini angegriffen wurde und seine Wunde sehr stark waren.
\paragraph{Muggelkunde:}
%ABSATZ
Dumbledores Wissen über Muggel und die Fähigkeiten mit ihnen zu interagieren waren bemerkenswert. Er verwendete häufig den Namen von Muggelsüßigkeiten als Passwort für das Schulleiterbüro. Auch besuchte er das Wool's Waisenhaus, um Tom Riddle kennenzulernen. Er brauchte nur wenig Hilfe von Magie, um Mrs Cole vorher zu dem Bezug zu überzeugen.
\paragraph{Heilmagie:}
%ABSATZ
Dumbledore war in der Heilmagie erfolgreich, als er es schaffte, den mächtigen Fluch, den er durch Gaunts Ring abbekomme hatte, vorübergehenden zu kontrollieren bis er nach Hogwarts zurückkehrte, wo Snape die Behandlung fortsetzte. Auch benutzte Dumbledore sein Blut, um in die Höhle mit dem Medaillon zu kommen. Danach verwendete er einen Heilzauber. Er schien auch in der Behandlung magischer Kreaturen erfolgreich zu sein. Als Cecil Lee Chiara Lobosca in ihrer Werwolfsform zur Behandlung in sein Büro brachte, nachdem Greyback Chiara verletzte hatte, gelang ihm anscheinend die Behandlung.
\paragraph{Apparieren und Disapparieren:}
%ABSATZ
Dumbledore war sehr gut im Apparieren, da er neben Voldemort der einzige bekannte Zauberer war, der in der Lage war, lautlos zu apparieren. Er konnte sich sehr genau zu einem anderen Ort apparieren, auch wenn er im Duell war. Dies tat er beispielsweise im Duell mit Voldemort.
\paragraph{Besenkunst:}
%ABSATZ
Dumbledore beherrschte die Besenkunst, da er, obwohl er durch den Medallion-Sicherungstrank geschwächt war, von Hogsmeade zum Astronomieturm auf einem Besen flog. Es gelang ihm auch schnell genug auf halbem Weg zum britischen Zaubereiministerium nach Hogwarts zurückzukehren, um Harry vor Quirrell zu retten. Außerdem gelang es ihm anscheinend mühelos den richtigen Schlüssel aus den Fliegende Schlüssel zu finden, um die Tür zu dem Schach-Hindernis zu öffnen.
\paragraph{Liebe:}
%ABSATZ
Dumbledore kannte sich mit der Natur der Liebe und der damit verbundenen Magie aus. Auch war er im Stande wahre Liebe zu fühlen. Dumbledore gab Newt den Tipp, dass er das Obscurial Credence heilen könnte, indem er ihn wahre Liebe gibt. Später riet er Leta Lestrange auch sich wegen des Todes ihres Bruders nicht allzu sehr zu quälen, da er selbst seine eigene Schwester Ariana verloren hatte. Auch war er im Stande Harry zu erklären, warum er den Todesfluch überlebt hatte. Dies hing mit der Liebe seiner Mutter zusammen. Dumbledores Verständnis von Liebe und Opferschutz brachte ihn auch dazu, dass er Harry 17 Jahre lang weiterhin vor Voldemort schützte. Dumbledore war auch im Stande zwischen echter Liebe und Liebe durch einen Liebestrank zu unterscheiden. In jungen Jahren war er in seinen Freund Grindelwald verliebt, wurde jedoch enttäuscht. Danach wurde er laut eigenen Angaben ziemlich asexuell und kümmerte sich mehr um seine Familie, enge Freunde, Schüler, Verbündete und die weitere Zauberergemeinschaft. Seine Fähigkeit zu lieben gab ihm die Kraft seine Selbstsucht und Machtbesessenheit zu überwinden. So konnte er sich auch Grindelwald stellen, als dieser seinen Bruder Aberforth angriff. Seine Fähigkeiten zu lieben sorgten auch dazu, dass er die Angst überwand, Ariana getötet zu haben. auch brachte es
%SEITE30
ihm dazu Grindelwald zu duellieren, als er zu viele Menschen getötet hatte. Dumbledore Fähigkeit zu lieben brachte ihn auch dazu ein besserer Mensch zu werden, indem er viele Menschen beschützte, die ihm wichtig waren, besonders als Voldemort an die Macht kam. Seine Fähigkeit zu lieben sorgte schließlich dazu, dass er sich Sorgen um Harrys Sicherheit machte, sodass er die Prophezeiung vor Harry Geheim hielt. Im fünften Jahr distanzierte er sich sogar von Harry, in der Hoffnung, dass dies Harry helfen würde.
\paragraph{Alte Runen:}
%ABSATZ
Dumbledore beherrschte auch die alten Runen, da er den Inhalt seiner Erstauflage von Die Märchen von Beedle dem Barden, die vollständig in Runen geschrieben wurde, perfekt verstehen konnte. Dies konnte man daran erkennen, dass er eine Reihe ausführliche Notizen zu verschiedenen Geschichten des Buches verfasst hatte, die die Ereignisse der Geschichten interpretierten und zeigten, wie die Geschichten sich vermutlich im Laufe der Zeit verändert hatten.
\paragraph{Kräuterkunde:}
%ABSATZ
Dumbledore hatte auch Wissen über magische Pflanzen und ihre Eigenschaften, da er die heilenden Eigenschaften von Alraunen kannte. Deshalb begann er mit der Herstellung von Alraunen-Wiederbelebungstrank, um die Versteinung zu heilen, die auf die Öffnung der Kammer des Schreckens gefolgt waren. Obwohl er nie gesehen wurde, wie er mit magischen Pflanzen umging, konnte er anscheinend auch die Teufelsschlinge durchqueren, um Harry zu retten.
\paragraph{Zauberstabkunde:}
%ABSATZ
Durch seine Freundschaft mir Grindelwald, in der es häufig um die Heiligtümer des Todes gin, lernte er viel über den Elderstab. So konnte er auch den Fluchumkehr-Effekt erklären, der im Duell zwischen Voldemort und Harry auftrat. Als Dumbledore den Elderstab hatte, war sein großer Wunsch eines natürlichen Todes zu sterben, damit die Kraft des Zauberstabzerstört werden konnte. Dieser Wunsch scheiterte jedoch.
\paragraph{Unbezwingbare Willenskraft:}
%ABSATZ
Dumbledore hatte wie ein wahrer Gryffindor eine enorme Willenskraft. So kam er beispielsweise über die schrecklichen Verluste in junge Jahre hinweg. So konnte er auch Grindelwald in einem Duell angreifen, obwohl er in ihr verliebt war. Dumbledore war auch sicher, dass er den Dementoren ihn Askaban lang genug wieder verstehen konnte und fliehen konnte. Auch war er dafür bekannt, dass er einmal in Askaban eingedrungen zu sein, um Morfin Gaunt über Voldemort zu befragen. Außerdem ließ sich Dumbledore nie von hochrangigen Beamten des Zaubereiministeriums einschüchtern, wenn Fudge ihn beispielsweise nach Askaban schicken wollte. Auch war er einer der Wenigen, der sich trauten den Namen von Voldemort auszusprechen. Auch beherrschte er Okklumentik und nonverbale Magie, die ein hohes Maß an mentaler Stärke erforderten. Zusätzlich half Dumbledore seinen Verbündeten eine ähnliche Willenskraft zu entwickeln. Dies tat er beispielsweise bei Leta Lestrange und Harry.
\paragraph{Einschüchterung:}
%ABSATZ
Trotz seiner wohlwollender Persönlichkeit war Dumbledore, sofern es nötig war, auch ein guter Einschüchterer. Dies gelang Dumbledore vor allem wegen seiner magischen Kraft. Seine magische Kraft sorgte sogar dafür, dass Voldemort und Grindelwald sich vor ihm fürchteten. Auch führte dies dazu, dass Hogwarts nie angegriffen wurde, wenn Dumbledore vor Ort war. Dumbledores bloße Anwesenheit war so einschüchternd, dass eine Gruppe von Todessern während der Schlacht in der Mysteriumsabteilung bei seinem Eintreffen versuchte zu fliehen. Selbst Bellatrix, die man nur sehr schwer einschüchtern konnte, versuchte vor ihm zu fliehen, sobald sie in der Lage war. Snape flehte auch Dumbledore bei einer Begegnung an ihn nicht zu töten, obwohl auch Snape sehr schwer war einzuschüchtern. Auch Harry war der Meinung, dass Dumbledore einschüchternder war als McGonagall, die sehr streng war. Dumbledore konnte auch Leute einschüchtern, wenn er sehr ruhig blieb. Dies tat er häufig bei Harry, wenn dieser beispielsweise über Snape und Draco herzog. Harry wünschte sich in manchen dieser Fälle, dass Dumbledore ihn lieber angeschrien hätte. Der Höhepunkt Dumbledores Zorn, der bekannt war, war es, als er erkannte, dass der Professor Moody in Wahrheit Crouch Jr. war. In diesem Moment wurde er so aggressiv, dass sich selbst McGonagall und Snape fürchteten.
\paragraph{Dirigieren:}
%ABSATZ
Dumbledore war auch in der Lage die Schulhymne zu dirigieren, was zeigte, dass er auch musikalische Fähigkeiten hatte.
\paragraph{Soziale Kontakte:}
%ABSATZ
Aufgrund seines hervorragenden Rufs hatte Dumbledore viele Anhänger und internationale Kontakte in der gesamten Zaubererwelt, um immer informiert zu sein. Selbst der Leiter der Abteilung für magische Strafverfolgung Travers, bewunderte widerwillig Dumbledores internationales Kontaktnetzwerk. Nach dem Bewerbungsgespräch mit Dumbledore, war sogar Voldemort selbst beeindruckt, wie gut der neue Schulleiter von Hogwarts informiert war. Durch seine Kontakte fand er heraus, dass Grindelwald sich als Percival Graves verkleidet hatte. Auch fand er von Flamel über Letas Tod bei Grindelwalds Kundgebung heraus. Aus Muggelzeitungen fand er von Frank Bryce Verschwinden in Albanien heraus. Von Snape fand er viel über Voldemorts Pläne heraus. Durch seine Verbündeten bekam er auch Häuser zur Verfügung gestellt. So bekam er von Nicolas Flamel ein Haus für Newt und durch Sirius das Hauptquartier für den Orden des Phönix. Auch fand er die Animagus-Fähigkeit von Rita Kimmkorn heraus. Obwohl Dumbledore normalerweise sehr gut informiert war, wusste er 1927 nicht, dass sein Verwandter Aurelius angeblich ein Schiffsunglück überlebt hatte und nun "Credence Barebone" hieß. 1981 war er sich auch nicht von der Unschuld von Sirius bewusst. Auch wusste er 1995 nicht, dass Moody in Wahrheit Bartemius Junior war.
%SEITE31

\paragraph{Körperliche Eignung:}
%ABSATZ
Trotz seines Alters war Dumbledore in sehr guter körperlichen Verfassung, da er in der Lage war, die Handlungen zu tun, die in seinem Alter eher unüblich waren. Auch konnte er mit viel jüngeren konkurrieren, was zeigte, dass er noch sehr fit war. Seine Kraft war groß genug, dass er Harry von Cedric Diggory wegdrücken konnte. Dumbledore war auch für jemanden in seinem Alter erstaunlich schnell und beweglich, wie man während der Schlacht in der Mysteriumsabteilung sehen konnte. Auch war Dumbledore ein perfekte Brustschwimmer und sein Körper konnte sich auch schnell an Kälte gewöhnen. Dumbledore hatte auch viel Ausdauer, da er, nachdem er den Sicherungstrank in der Höhle getrunken hatte, schnell einen Feuerring wirken konnte. Danach war er sogar im Stande mit einem Besen nach Hogwarts zu fliegen. Außerdem hatte er eine starke Schmerztoleranz, wie sich zeigte, als er seine Hand mit einem Messer schnitt, um mit seinem Blut in die Höhle mit dem Horkrux zu kommen.
\paragraph{Weisheit:}
%ABSATZ
Dumbledore war auch für seine Weisheit bekannt. Er fand zum Beispiel heraus, dass Tom Riddle die Kammer des Schreckens geöffnet hatte und nicht Hagrid. Auch merkte er, dass der echte Moody niemals Harry während eines Gesprächs mit Dumbledore weggeschafft hätte. Viele bekannte Magier wussten auch von seiner Weisheit, weswegen ihn auch Minister Fudge oft um Hilfe bat. Nur zu einem Punkt Dumbledores Biographie wurde die Weisheit von Dumbledore weltweit angezweifelt. Dies war als 1995 die Rückkehr von Voldemort von ihm und Harry verkündet wurde. Beide wurden als Lügner und Dumbledore als verrückter alter Mann dargestellt.


\subsection*{\Large Karriere}
%ABSATZ
In seiner langen Karriere als Lehrer an der Hogwarts-Schule für Hexerei und Zauberei war er zuerst Professor für Verteidigung gegen die dunklen Künste und später Professor für Verwandlung (Schließlich sogar Leiter der Verwandlung). Später schrieb er wissenschaftliche Beiträge für die Verwandlung heute, Zentralfragen der Zauberkunst und die Angewandte Zaubertrankkunde. Von vielen wurde er auch als bester Schulleiter von Hogwarts bezeichnet. Er war auch ein berühmter Alchemist, der mit Nicolas Flamel zusammenarbeitete und zwölf Verwendungsmöglichkeiten von Drachenblut entdeckte.
\vspace{10pt}
\newline
Dumbledore war er Hexenmeister des Zaubergamots und ein Ganz hohes Tier. Er wurde zwischendurch als Schulleiter gefeuert, nachdem niemand ihm die Rückkehr von Voldemort glaubte. Auch hatte er den Orden des Merlin, erster Klasse.

\subsection*{\Large Beziehungen und Freundschaften}
\subsubsection*{\large Familie}
\subsubsection*{Eltern}
%ABSATZ
Albus Beziehung zu seinen Eltern war nicht immer besonders gut. Es wird vermutet, dass er seinen Vater nicht sehr geschätzt hat, da er in Askaban inhaftiert wurde, weil er drei Muggeljungen, die Ariana Schäden hinzugefügt hatten, magisch angegriffen hatte. Laut Aberforth lehrte die Mutter Albus die Geheimhaltung dieses Verbrechens. Seine Mutter las Dumbledore angeblich oft Geschichten vor, was darauf schließen lässt, dass sie eine gute Beziehung zu ihm hatte. Der Tod seiner Mutter machte ihn sehr wütend und traurig, da er sich danach um seine verstörte Schwester und einen Bruder kümmern musste. Dadurch fühlte Dumbledore sich gefangen.
\vspace{10pt}
\newline
Später erklärte Dumbledore Harry Potter, dass er wirklich jeden aus seiner Familie geliebt hatte, aber von seinem Ehrgeiz und seinen Wahnsinn geblendet wurde. Ursprünglich hatte er vorgehabt, den Stein der Auferstehung zu verwenden, um seine Eltern zurückzubringen, damit er keine allzu große Verantwortung mehr gehabt hätte. Nachdem schließlich auch seine Schwester Ariana auf tragische Weise starb, wollte er den Stein (falls er ihn je finden würde), dazu verwenden, um seine Eltern und seine zurückzubringen und sich dann bei ihnen zu entschuldigen. Dieser Wunsch führte jedoch letztendlich zu seinem Tod: Dieser Wunsch führte jedoch schließlich zu seinem Tod: Der Auferstehungsstein war Teil eines Rings mit einem Fluch geworden. Diesen legte Dumbledore schließlich an und durch den Fluch wurde er langsam getötet.
\subsubsection*{Ariana Dumbledore}
%ABSATZ
Dumbledore war Ariana Dumbledores ältester Bruder und verbrachte nur wenig Zeit mit ihr. Obwohl er seine jüngere Schwester liebte, war Albus ärgerlich, als er nach dem Tod seiner Mutter die Verantwortung für seine kranke Schwester übernehmen musste. Durch seine Schwester musste er viel zu Hause bleiben und dadurch seine ehrgeizigen Pläne teilweise aufgeben. Als er und Gellert Grindelwald planten die Muggel zu unterwerfen und die Macht der Zaubererwelt zu übernehmen, plante Albus Ariana mit auf seine Reise zu nehmen. Daraufhin wies ihn Aberforth Dumbledore darauf hin, dass Ariana wegen ihrer Krankheit dazu nicht im Stande war. Dies führte später zu dem Drei-Wege-Duell zwischen den beiden Brüdern und Grindelwald. In diesem Duell wurde Ariana versehentlich getötet. Albus war durch den Tod seiner Schwester am Boden zerstört und sein Leben lang beschäftigte ihn, dass er mehr oder weniger am Tod seiner Schwester schuld war. Er sprach selten über dieses Ereignis und den Tod seiner Schwester. Nur Leta Lestrange gestand er, dass er seine Schwester während ihres Lebens nicht genug geliebt hatte. Dumbledore hatte auch sein Leben lang Angst, dass er den Fluch gesendet hatte, der Ariana getötet hatte. Auch fürchtete er,
%SEITE32
dass Grindelwald den Fluch ausgeführt hatte, weshalb sich das Duell zwischen Albus und Grindelwald sehr verzögerte. Es ist bekannt, dass der Irrwicht von Albus die Leiche seiner Schwester ist.
\subsubsection*{Aberforth Dumbledore}
%ABSATZ
Aberforth Dumbledore war Albus jüngerer Bruder. Obwohl Albus seinen Bruder mochte, standen sich die beiden Brüder nicht ehr nahe, da sie sehr unterschiedliche Persönlichkeiten hatten. Auch hielt Aberforth von Albus Abstand, da er immer im Schatten vom Albus stand. Trotzdem kümmerte sich Albus später um seinen Bruder und half ihm bei seiner Ausbildung. Als Aberforth die Schule abbrechen wollte, sorgte Albus dafür, dass er seine Ausbildung fortsetzte, weil er sich um die Zukunft seines Bruders sorgen machte. Als Grindelwald später Aberforth angriff eilte Albus ihm sofort zu Hilfe, was zeigte, dass er sich doch um seinen Bruder sorgte. Nach Arianas Tod wurde das Verhältnis zwischen Albus und Aberforth sehr schlecht, da Aberforth Albus für den Tod seiner Schwester verantwortlich machte, für die Aberforth und seine Mutter sehr viel Zeit investiert haben. Bei der Beerdigung von Ariana brach Aberforth seinem Bruder sogar die Nase. Obwohl sie sich später wieder etwas verstanden und Aberforth dem Orden des Phönix beitragt und an Albus Beerdigung teilnahm, hatten Beide nicht allzu viel miteinander zu tun, da Aberforth Albus nicht vergeben konnte. Albus selbst versuchte es nicht die Meinung von seinem Bruder zu ändern, da er es nicht anders verdient hatte. Trotzdem war Albus traurig, dass er und Aberforth sich nie versöhnen konnten und hatte Verlangen mit ihm Frieden zu schließen. Im Spiegel Nerhegeb sah er sich mit Aberforth als Freund und seiner gesamten verlorenen Familie. Albus war auch der Meinung, dass Aberforth ein besserer Mensch war, da er sich sehr um Ariana gekümmert hatte. Aberforth änderte nach dem Tod seines Bruder die Meinung über ihn, nachdem Harry ihm über Albus Reue erzählte.
\subsubsection*{\large Gellert Grindelwald}
%ABSATZ
Dumbledore traf Gellert Grindelwald zum ersten Mal im Alter von siebzehn Jahren. Grindelwalds Großtante Bathilda Bagshot stellte die beiden Jugendlichen gegenseitig vor. Laut Bathilda kamen Beiden schnell miteinander aus. Beide hatten den Wunsch die Heiligtümer des Todes zu finden. Beide schlossen später ein Blutpakt und schworen sich nie gegeneinander zu kämpfen. Es ist auch bekannt, dass beide romantisch miteinander Zeit verbrachten. Es wird gesagt, dass die Beziehung intensiv und leidenschaftlich war. Diese Liebe war so stark, dass er eine Zeitlang Grindelwald im Spiegel Nerhegeb als seinen größten Wunsch sah.
\vspace{10pt}
\newline
Beide verstanden sich zwei Monate lang sehr gut, bis Aberforth Dumbledore Albus damit konfrontierte, dass er sich nicht um Ariana Dumbledore kümmern würde. Grindelwald stieg in den Konflikt mit ein und griff Aberforth an. Albus versuchte seinem Bruder zu helfen, jedoch wurde während des Duells seine Schwester Ariana getötet. Grindelwald floh daraufhin und Albus beendete deshalb die Freundschaft, obwohl er weiterhin in weiterhin liebte.
\vspace{10pt}
\newline
Grindelwald verbrachte seine späteren Jahre damit Macht in Europa zu erlangen, während Dumbledore seine Karriere als Lehrer in Hogwarts begann. Während Grindelwald und seine Armee Europa terrorisierten, blieb Dumbledore in Großbritannien und weigerte sich Grindelwald zu stellen. Dies lag teilweise am Blutpakt, war jedoch auch ein Vorteil für Großbritannien, da Grindelwald Großbritannien nicht angreifen konnte, da dies ein Duell mit Albus bedeuten würde und so das Pakt verletzt werden würde. Ein weiterer Grund war die Liebe zu Grindelwald, die weiterhin bestand. Trotzdem versuchte Dumbledore aber indirekt an Grindelwalds Niederlage teilzunehmen, wozu er Newton Scamander benutzte. Diesen schickte er nach Amerika, um angeblich seinen Donnervogel Frank zu helfen. Dumbledore wusste nämlich, dass Grindelwald in Amerika nach dem Obscurial suchen wollte, mit dem er am Ende Dumbledore töten wollte. Nach dem Tod von Leta Lestrange entschloss sich Dumbledore jedoch das Blutpakt zu brechen.
\vspace{10pt}
\newline
Obwohl er das Fläschchen mit dem Blutpakt nicht zerstören ließ, stellte er sicher, dass die Zaubererwelt Grindelwalds Terrorherrschaft überleben würde, indem er etwas magische Hilfe leistete. Nachdem Dumbledore endlich Erfolg hatte, spürte er Grindelwald auf. Als Dumbledore und Grindelwald sich schließlich fanden, starteten sie das Duell zwischen Albus Dumbledore und Gellert Grindelwald im Jahre 1945, was als eines der legendärsten Duell der Zauberergeschichte zählt. Dumbledore gewann schließlich das Duell, obwohl Grindelwald den Elderstab besaß. Deshalb wurde Dumbledore schließlich neuer Meister des mächtigen Zauberstabs. Anschließend brachte er Grindelwald zu den magischen Behörden, die Grindelwald in sein eigenes Gefängnis Nurmengard einsperrten.
\vspace{10pt}
\newline
Jahre später, 1998, weigerte sich Grindelwald Lord Voldemort Informationen über den Besitzer des Elderstabs zu geben, obwohl Grindelwald keinen Zauberstab besaß, um sich gegen den brutalen Zauberer zu wehren. Dies deutet drauf hin, dass Grindelwald nicht mehr so brutal wie damals war und seine Sünden bereute. Von Harry Potter wurde deswegen spekuliert, dass Grindelwald sein Leben gelassen hat, um Voldemort daran zu hindern, Dumbledores Grab zu entweihen, was daraufhin deutete, dass er Dumbledore weiterhin respektierte und vielleicht sogar wieder liebte.
\subsubsection*{\large Newton Scamander}
%ABSATZ
Dumbledores Beziehung zum berühmten Magizoologen war sehr gut. Während Newts Zeit in Hogwarts war Dumbledore sein Professor für Verteidigung gegen die dunklen Künste. Eines von Leta Lestrange Experimenten mit einem magischen Tier ging zu
%SEITE33
weit und so wurde das Leben von einem anderen Schüler gefährdet. Newt wollte dich, dass seine gute Freundin aus der Schule geworfen wird, also übernahm er due Schuld für die Handlung von Leta. Deshalb wurde er später von der Schule verwiesen. Albus versuchte dien Verweis der Schule zu verhindern, jedoch misslang ihm dies. Dafür gelang es Dumbledore dafür zu sorgen, dass Newt wenigstens seinen Zauberstab behalten durfte. Nach seiner Schulzeit stand Newt weiterhin mit seinem ehemaligen Lehrer in Kontakt. Newt respektierte seinen ehemaligen Lehrer sogar soweit, dass er mehrere Aufgaben für ihn erledigte. Dumbledore sagte ihm so zum Beispiel, wo Newt einen schlecht gepflegten Donnervogel finden konnte und schickte seinen ehemaligen Schüler nach New York, um das Tier freizulassen, dessen natürliche Umgebung Arizona war. Zu dieser Zeit half Newt auch Gellert Grindelwald zu fangen, der versuchte das Obscurial Credence Barebone davon zu überzeugen, sich anzuschließen. Als Grindelwald entkommen konnte, hatte Albus das Gefühl, dass er nicht in der Lage seinen würde, gegen den dunklen Zauber zu kämpfen und bat deshalb Newt um Hilfe. Dumbledore bewunderte Newt, weil er sich nicht um Macht oder Popularität sehnte, sondern nur das beste für seine Tierwesen wollte. Später schickte er Newt nach Paris, um Credence zu retten. Newt zögerte anfangs, erkundete sich jedoch trotzdem später nach einer sicheren Wohnung. Dies Gerüchte über Nets Reise nach Frankreich erreichten das Britische Zaubereiministerium und eine Delegation darunter Newts Bruder Theseus Scamander, Leta Lestrange und Torquil Travers wurden nach Hogwarts geschickt, um Dumbledore zu befragen, da Newt ein Reiseverbot hatte. Albus tat so, als wüsste er nichts davon und erklärte mit der Zustimmung von Newts Bruder, dass Newt sich nicht an Befehle hält. Während Gellert Grindelwalds Kundgebung in Paris konnte der Niffler von Newt, ohne Grindelwalds Wissen, die Phiole mit dem Blutpakt stehlen. Kurz darauf brachte Newt die Phiole nach Hogwarts zu Dumbledore. Dumbledore war sehr traurig über Letas Tod. Newt hinterfragte auch nicht, warum es überhaupt einen Blutpakt zischen Grindelwald und Dumbledore gab, sondern bat ihn einfach darum die Phiole zu zerstören, damit Newt wieder ein normales eben führen konnte. So schaffte es schließlich Dumbledore Grindelwald zu besiegen, obwohl dies sehr lange dauerte.
\subsubsection*{\large Tom Riddle (Lord Voldemort)}
%ABSATZ
Dumbledore wandte sich persönlich an Mrs Cole, die Leiterin des Wool's Waisenhaus, in dem Tom Riddle wohnte, um diesen für Hogwarts zu gewinnen. Dumbledore traf dort dann auch zum ersten Mal Tom von dem er noch nicht wusste, dass er zum gefährlichsten und mächtigsten Dunklen Zauberer aller Zeiten werden würde. Er war jedoch fasziniert von Tom Riddle, da er sehr starke Kräfte hatte. Dumbledore war jedoch trotzdem etwas beunruhigt über die Tatsache, dass Tom ein Parselmund war, eine seltene Fähigkeit, die angeblich mit den Dunklen Künsten verbunden war. Noch mehr beunruhigter war er über die Grausamkeit, da Tom Kinder im Waisenhaus mit Magie verletzte. Dumbledore kehrte dann nach Hogwarts zurück, um Riddle im Auge zu behalten, da er noch niemanden kannte, was eine gefährliche Situation war, die Dumbledore bemerkte. Trotzdem glaubte er weiterhin, dass Riddle eine gute Seite hatte, was er jedoch später bereute.
\vspace{10pt}
\newline
Als der damalige Schulleiter Armando Dippet Dumbledore anvertraute, dass Tom sich ihm annäherte, um nach seine Abschluss weiterzumachen, um Verteidigung gegen die dunklen Künste zu unterrichte - was Dippet nur aufgrund von Tom Alter ablehnte und ihm vorschlug später erneut zu fragen - riet Dumbledore ihm davon an. Dumbledore nannte Dippet zwar keinen Grund, weil der Schulleiter von Tom genauso entzückt war wie die anderen Professoren, fühlte sich jedoch sehr unwohl, dass Tom eine solche Anfrage machte und wollte auch nicht, dass Riddle Macht in Hogwarts bekäme.
\vspace{10pt}
\newline
Währen Dumbledore sich Toms Veränderung und Aufstieg zur Macht bewusst war, hielt er dennoch zu ihm Kontakt und lud ihn sogar zu einem angeblichen Vorstellungsgespräch ein. Als Tom (Zu diesem Zeitpunkt nannte er sich schon Lord Voldemort) die Schule besucht, war Dumbledore sehr höflich. Dumbledore vermutete jedoch, dass Voldemort mehr als einen Job wollte und vielleicht sogar ein Armee aufstellen wollte. Diese Vorurteile bekam er vor allem durch Gerüchte. Dumbledore lehnte schließlich die Bitte von Voldemort ab, was dazu führte, dass Voldemort den Rosten des Lehrers für "Verteidigung gegen die dunklen Künste" verfluchte.
\vspace{10pt}
\newline
Dumbledore hatte jedoch weder Wut noch Angst vor Voldemort, da er von seiner bemitleidenswerter Kindheit wusste und völlig unfähig war, Liebe zu verstehen, da er sie noch nie richtig empfunden hat. Dumbledore fühlte sie sogar Schuld daran, dass er Riddle nicht fröhlich machen konnte. Dumbledore verstand das Böse von Riddle, nachdem er viele schlimme Taten begannen hatte und studierte seine Geschichte, bis er schließlich Horkruxe von Voldemort entdeckte. Nach Riddles Verwandlung in Voldemort ließ er sich nicht sehr leicht von ihm einschüchtern. Deshalb war er einer der Wenigen, die weiterhin den Namen "Voldemort" laut aussprachen. Im Gespräch mit Voldemort selbst nannte er ihn sogar "Tom", damit diesem die Verbindung zu Dumbledore klar wurde. Dies tat später auch Harry Potter.
\vspace{10pt}
\newline
1996 duellierten sich im Britischen Zaubereiministerium Dumbledore und Voldemort. Voldemorts Ziel war es sowohl Harry als auch seinen ehemaligen Lehrer zu töten. Da Voldemort Dumbledore fürchtete, war Dumbledore leicht in der Lage in die Offensive zu gehen und zwang Voldemort schließlich zur Flucht. Kurze Zeit später wurde Dumbledore auf Voldemorts Pläne aufmerksam, Draco Malfoy zu benutzen, um ihn zu töten. Voldemort hatte erkannt, dass er keine realistische Chance hatte, Dumbledore zu töten und ihm wurde bewusst, dass Dumbledore nicht seinen aktuellen Schüler angreifen würde. Dumbledore bat daher Severus Snape ihm anstelle von Malfoy zu töten. Dumbledore wurde 1997 von Snape getötet, jedoch konnte Harry 1998 Voldemort für immer besiegen.
%SEITE34
\subsubsection*{\large Patricia Rakepick}
%ABSATZ
Dumbledore war während der Schulzeit von Patricia Rakepick in Hogwarts Leiter der Verwandlung unter dem damaligen Schulleiter Armando Dippet, bis er während der Ausbildung von Patricia selbst Schulleiter wurde. Dumbledore sorgte regelmäßig dafür, dass Patricia nicht rebellisch wurde, was seinen Kollegen nicht immer gelang. Durch ihre rebellische Art vertrauten ihr viele nicht und gingen von viel Unheil aus. Dabei ging unter, dass sie eine sehr gute Schülerin war. Später kehrte Rakepick auf den Wunsch von Dumbledore zurück, um die Verfluchten Gewölbe sicherzustellen.
\vspace{10pt}
\newline
Nachdem sich Rakepick sich als Verräterin herausstellte, die für R arbeite, schien Dumbledore nicht sehr überrascht zu sein und sagte, dass er sie verdächtigt hatte, aber sie nicht ls so große Gefahr sah.
\subsubsection*{\large Harry Potter}
%ABSATZ
Dumbledore war bekannt für die Fürsorge und den Einsatz für Harry Potter. Dumbledore sorgte dafür, dass Harry von Godric's Hollow weggebracht wurde. Später brachte er Harry zu seiner Tante Petunia Dursley und zu seinem Onkel Vernon Dursley nach Surrey, Little Whinging. Dumbledore legte Harry vor die Haustür der Familie. Dumbledore bekam von Arabella Figg Informationen, dass Harry nicht sehr gut behandelt wurde, schritt jedoch nicht ein. Während Harrys Schulzeit in Hogwarts bot Dumbledore ihm Schutz und gab ihm Sonderunterricht über die Geschichte und Persönlichkeit von Lord Voldemort.
\vspace{10pt}
\newline
Durch eine Prophezeiung hatte er von dem Schicksal des Jungen erfahren, der später als einziger dazu in der Lage sein wird Voldemort zu töten. Dieser tötete bereits Harrys Eltern und Dumbledore übernahm für ihn teilweise eine Vaterrolle. Er bereitete ihn auf sein Schicksal vor, auch wenn er lange Zeit nicht dazu kam, da er immer plante ihm am Ende des Schuljahres von der Prophezeiung zu erzählen. Doch er entschied sich mehrmals dagegen, da Harry immer ziemlich aufgebracht war. Erst in Harrys fünften Schulajhr wurde Harry darüber informiert.
\vspace{10pt}
\newline
Nach der Wiedergeburt von Lord Voldemort weigerte sich Dumbledore all seine Pläne mit Harry zu teilen, da Voldemort die Gedanken von Harry sehen konnte. Dadurch gingen sich Beide immer mehr aus dem Weg.
\vspace{10pt}
\newline
Während Harrys sechstem Schuljahrs lernte Dumbledore ihm viel über Voldemorts Vergangenheit zu verstehen, da Voldemort immer stärker wurde und Beide zusammen einen Schwachpunkt erkennen wollten. Dies geschah vor allem durch den Einsatz des Denkariums. Auch beauftragte Dumbledore Harry Horace Slughorn über Horkruxe zu befragen, da dies wichtig für das weitere Vorgehen gegen Voldemort war. Über Dumbledores Tod war Harry sehr traurig.
\vspace{10pt}
\newline
Harry und Dumbledore schmiedeten trotz des sehr großen Altersunterschieds eine starke Bindung zwischen Schüler und Schulleiter, vergleichbar, als wäre Harry sein Enkel. Dies war daran zu erkennen, dass Dumbledore mit Harry schimpfte, Harry vor Dumbledore weinte, seine Ängste zugab und ihm auch sein Leben anvertrauen würde. Dumbledore fand, dass Harry ein sehr talentierter und mächtiger Zauberer war, der die besten Eigenschaften seiner Eltern geerbt hatte und sehr gute Führungsstärke hatte. Auch war Dumbledore davon beeindruckt, dass Harry nie Macht oder Ansehen wollte. Dumbledore hatte auch viel Vertrauen zu Harry und ließ sich so in einem geschwächten Zustand apparieren, obwohl Harry noch nie richtig über ein paar Meter appariert war.
\vspace{10pt}
\newline
Nachdem Harry sich während der Schlacht von Hogwarts geopfert hatte und von Voldemort mit Todesfluch getroffen und somit fast im Verbotenen Wald gestorben wäre, ging seine Seele an einen Ort, der ähnlich wie die King's Cross Station aussah, wo Dumbledore auf ihn wartete, um mit Harry zu sprechen. Das Dumbledore dort auf ihnen wartete, zeigte die Verbundenheit zwischen den Beiden. Harry beendete danach das Werk von Dumbledore und vernichtete Voldemort. Nach dem Krieg benannte Harry auch seinen zweiten Sohn nach Dumbledore.
\subsubsection*{\large Rubeus Hagrid}
%ABSATZ
Dumbledore vertraute Rubeus Hagrid schon während seiner Schulzeit in Hogwarts. Dumbledore glaubte, dass Hagrid die Kammer des Schreckens nicht geöffnet hatte und bat deshalb den damaligen Schulleiter Armando Dippet Hagrid als Wildhüter auszubilden, damit er in Hogwarts bleiben könne. Dies wurde später dann auch umgesetzt. Hagrid war auch die Person, die von Dumbledore die Aufgabe bekam Harry Potter zu beschützen, nachdem seine Eltern getötet wurden, und dann zur Familie Dursley zu bringen. Dumbledore schickte Hagrid auch zu anderen wichtigen Missionen, zum Beispiel zu dem Versuch, die Riesen zu überzeugen, sich dem Orden des Phönix anzuschließen. Dumbledore erlaubte auch Hagrid seinen Halbbruder Grawp mit nach Hogwarts zu nehmen. Grawp blieb vorerst im Verbotenen Wald und kam späte in die Berge in der Nähe von Hogsmeade. Hagrid war auch sehr traurig über Dumbledores Tod. Hagrid bewunderte und vertraute Dumbledore auch sehr. Wenn Leute Dumbledore als Schulleiter kritisierten, reagierte Hagrid darauf sehr aggressiv.
\subsubsection*{\large Minerva McGonagall}
%ABSATZ
Minerva McGonagall kannte Dumbledore den größten Teil ihres Lebens. Sie besuchte Hogwarts als Schülerin, als Dumbledore Leiter der Verwandlung war und ersetzte ihn später in dieser Position, als er Schulleiter wurde. Sie war viele Jahre lang
%SEITE35
stellvertretende Schulleiterin, trat für ihn ein und wurde sogar mehrfach Interims-Schulleiterin.
\vspace{10pt}
\newline
Minerva hatte großes Vertrauen in Dumbledore. Sie war häufiger gleicher Meinung mit Dumbledore, jedoch war sie sehr dagegen, dass Harry Potter zur Familie Dursley gebracht wurde, da sie fand, dass die Muggel sehr schlimm waren. Häufig half sie Dumbledore, wenn er mit anderen Sachen beschäftigt war.
\vspace{10pt}
\newline
Dumbledore entdeckte Minerva eines späten Abends in ihrem Klassenzimmer weinend, nachdem sie erfahren hatte, dass ihre junge Liebe Dougal McGregor eine andere Frau geheiratet hatte. Minerva erzählte ihm die ganze Geschichte, woraufhin Dumbledore sie tröstete und ihr seine Familiengeschichte erzählte, die ihr bisher unbekannt war. Das private Gespräch zwischen den Beiden brachte viel Vertrauen und war die Grundlage für eine dauerhafte gegenseitige Wertschätzung und Freundschaft.
\vspace{10pt}
\newline
Minerva hatte keine Angst, Dumbledore ihre Loyalität zu zeigen. Als Cornelius Fudge 1996 Auroren zur Verhaftung von Dumbledore schickte, trat sie diesen mutig vor Dumbledore und kündigte hm an, die Auroren in seinem amen zu kämpfen. Dumbledore verbot ihr dies und sagte ihr, dass die Schüler sie in der Abwesenheit von ihm brauchen würde.
\vspace{10pt}
\newline
Minerva war sehr traurig über den Tod von Dumbledore und nahm deshalb auch an seiner Beerdigung teil. Als Harry später nach Hogwarts zurückkehrte und ihr von Dumbledores Auftrag erzählte, setzte ich McGonagall für ihn ein und versuchte ihm zu helfen.
\subsubsection*{\large Severus Snape}
%ABSATZ
Severus Snapes Beziehung zu Dumbledore war eine sehr enge. Nachdem Snape Lord Voldemort indirekt versehentlich zu Lily Evans geschickt hatte, um sie zu töten, wandte er sich an Dumbledore, damit dieser sie beschützen könne. Dumbledore entschied sich, Snape für seine Taten als Todesser zu verzeihen und machte ihn zum Doppelagenten im Orden des Phönix.
\vspace{10pt}
\newline
Als Spion des Ordens behielt Snape weiterhin seinen Platz in den Reihen der Todesser und informierte Dumbledore über ihre Pläne. Ebenfalls warn er auch für Voldemort in der Rolle des Geheimagenten und versorgte ihn mit Informationen über den Orden. Die Inhalte dieser Berichte wurden ihm meistens von Dumbledore aufgetragen. Dumbledore gab Snape fast alle Informationen, die für den Erfolg des Ordens relevant waren, während Snape Voldemort meistens falsche Informationen weitergab.
\vspace{10pt}
\newline
Dumbledore vertraute Snape jedoch nicht alles an, da die Gefahr bestand, dass Voldemort doch so an Informationen kommen würde. Snape ärgerte sich sehr über die Geheimhaltung von Dumbledore und war häufig beleidigt über den Mangel an Vertrauen in seine Fähigkeiten. Als Snape jedoch entdeckte, dass er dafür verwendet wurde, Harry Potters eigene Zerstörung vorzubereiten, war er entsetzt, da er dadurch einen Verrat an Lily sah.
\vspace{10pt}
\newline
Dumbledore unterschätzte Snape Fähigkeiten im Bereich Liebe und war sehr darüber überrascht, dass er auch noch Jahre nach Lilys Tod n sie verliebt war. Außerdem lobte er Snapes Tapferkeit und erklärte, dass er weitaus mutiger als Igor Karkaroff sei. Dumbledore beauftragte auch Snape Harry in Okklumentik zu unterrichten, was jedoch nach hinten losging.
\vspace{10pt}
\newline
Im Jahr 1996 wurde Dumbledore verflucht, da er Vorlost Gaunts Ring anlegte, der ein Horkrux war. Er zog den Ring an, weil in diesem der Stein der Auferstehung war. Sein Verlangen seine toten Familienangehörigen zu sehen, war so groß, dass er den Verstand verlor und deshalb den Ring anzog. Es gelang ihm Snape zu kontaktieren, der den Fluch später in seiner rechten Hand halten konnte. Hätte Snape dies nicht gemacht, wäre Dumbledore in kurzer Zeit gestorben und so hat Snape ihm das Leben gerettet. Snape wusste jedoch auch, dass der Fluch ihn trotzdem mit der Zeit schmerzhaft töten würde und sagte Dumbledore, dass er maximal ein Jahr überleben würde. Zu diesem Zeitpunkt war sich Dumbledore Voldemorts Plan bewusst ihn zu töten, indem er Draco Malfoy diese Aufgabe übergab, sonst würde er seine Eltern töten. Er wusste, dass er in einem Jahr sterben würde und wollte Draco versuchen nicht seine Kindheit durch einen Mord zu zerstören. Deshalb befahl er Snape ihn anstatt von Draco zu töten, wenn die Zeit gekommen war. Snape wollte diese Aufgabe anfangs nicht übernehmen, führte es aber in der Schlacht auf dem Astronomieturm durch, als Draco es geschafft hatte, Todesser nach Hogwarts zu bringen. Dumbledore sorgte auch etwas dafür, das Snape nach seinem Tod und während der Herrschaft von Voldemort Schulleiter von Hogwarts wurde.
\vspace{10pt}
\newline
Obwohl Dumbledore viel Vertrauen zu Snape hatte, erlaubte er es ihm nie Professor für Verteidigung gegen die dunklen Künste zu werden, da er befürchtete, dass Snape so wieder zu seinen "alten Gewohnheiten" zurückkehren würde. Dies war einer der Möglichkeiten, jedoch war es wahrscheinlicher, dass Dumbledore Snape diesen Wunsch verwahrte, da er ihn vor dem Fluch schützen wollte, den Voldemort auf das Fach gelegt hatte. Dieser Fluch verhinderte, dass ein Professor länger als ein Schuljahr auf dem Posten blieb. Dies geschah manchmal sogar durch tödliche Folgen. Nachdem Dumbledore Snape befohlen hatte ihn zu töten, erlaubte er es ihm doch schließlich den Posten für Verteidigung gegen die dunklen Künste zu übernehmen, da er wusste, dass sobald Snape ihn töten würde, er von dem Posten zurücktrete und fliehen müsste.
%SEITE36
\subsubsection*{\large Elphias Doge}
%ABSATZ
Elphias Doge wurde schon am ersten Schultag in Hogwarts ein enger Freund von Dumbledore. Elphias litt unter Drachenpocken, weshalb ihn viele Mitschüler mieden. Dumbledore aber erkannte aber den Charakter von Elphias. Auch wollten Beide nach der Schule eine Weltreise machen, diese wurde aber wegen dem Tod von Dumbledores Mutter abgesagt. Elphias hielt Dumbledore für einen sehr guten Zauberer und eine große Persönlichkeit und glaubte, dass seine vielen frühen Verluste ihn mit großer Menschlichkeit und starkem Mitgefühl ausstatteten, wie er nach Dumbledores Tod beschrieb.
\vspace{10pt}
\newline
Elphias bemitleidete Dumbledore wegen den Tod seines Vaters, seiner Mutter und seiner Schwester. Elphias wurde auch Mitglied des Orden des Phönix, einer Organisation, die Dumbledore geleitet hatte. Elphias nahm auch an der Beerdigung von Albus Dumbledore teil.
\vspace{10pt}
\newline
Elphias sagte Harry Potter auch, dass er der Verleumdung von Rita Kimmkorn und Muriel über Dumbledore Familienleben niemals glauben sollte, einschließlich der Tatsache, dass Ariana ein Squib war. Es ist möglich, dass Elphias in der Schlacht von Hogwarts gekämpft hat.
\subsubsection*{\large Ronald Weasley und Hermine Granger}
%ABSATZ
Ronald Weasley und Hermine Granger waren Harry Potters beste Freunde und wie Harry bewunderten sie Dumbledore. Dumbledore mochte sie auch, obwohl er mit ihnen nicht so viel zu tun hatte, wie Harry.
\vspace{10pt}
\newline
Nach seinem Tod hinterließ er Ron seinen Deluminator (ein einzigartiges Artefakt, das Dumbledore selbst erschaffen hatte und Hermine seine Kopie von "Die Märchen von Beedle dem Barden". Diese Erbstücke erwiesen sich als sehr nüzlich, als Ron Harry und Hermine früher verieß, um zu seiner Familie zurückzugehen. Danach fand er mit dem Deluminator die Beiden wieder. Das Trio las auch die Geschichten von Beedle dem Barde und erfuhr so von den Heiligtümern des Todes. Harry, Ron und Hermine besuchten nach der Schlacht von Hogwarts auch Dumbledores Porträt.
\subsubsection*{\large Familie Weasley}
%ABSATZ
Dumbledore respektierte nicht nur Ronald Weasley, sondern auch den Rest der Familie Weasley. Sie betrachteten ihn als den größten Zauberer aller Zeiten und unterstützen ihn bei allem (außer Percy Weasley, der nicht an Lord Voldemorts Rückkehr glaubte). Auch hatten alle Mitglieder der Familie Weasley Albus Dumbledore als Schulleiter, während sie Hogwarts besuchten.
\vspace{10pt}
\newline
Arthur Weasley und Molly Weasley respektierten Dumbledore immer und waren ihm sehr treu. Arthur glaubte genau wie Dumbledore an die Gleichheit aller Hexen, Zauberer und Muggel. Molly hielt Dumbledore für einen großartigen Zauberer, der keine Fehler machte. Als das Britische Zaubereiministerium anfing Dumbledore und Harry Potter nicht zu glauben, unterstützen sie beide Zauberer und wurden Mitglied des zweiten Orden des Phönix.
\vspace{10pt}
\newline
Dumbledores Beziehung zu Bill Weasley war zufällig eine freundschaftliche, da Beide Gryffindors waren und sich mochten. Es ist wahrscheinlich, dass er auch eine gute Beziehung zu Charlie Weasley hatte.
\vspace{10pt}
\newline
Percy Weasley hielt Dumbledore auch für einen großartigen Zauberer, aber fand ihn manchmal seltsam. 1995 wurde Percys Sympathie für Dumbledore unterbrochen, als Dumbledore und Harry behaupteten, Voldemort sei zurückgekehrt. Dies geschah dadurch, dass Percy dem Ministerium sehr treu war und so gegen Dumbledore und Harry arbeite. Percy erkannte jedoch an, dass er einen Fehler gemacht hatte. Später nahm er sogar an der Beerdigung von Dumbledore teil.
\vspace{10pt}
\newline
Fred und George Weasley schienen Dumbledore sehr zu respektieren. Dumbledore schien auch die Zwillinge zu mögen, trotz dem Unheil, was sie manchmal anrichten, weil er wusste dass sie gute und lustige Menschen waren. Die Brüder schlossen sich dem zweiten Orden des Phönix an und wurden sogar Mitglieder der Dumbledores Armee.
\vspace{10pt}
\newline
Ginevra Weasley bewunderte Dumbledore sehr und wurde 1992 Schülerin an seiner Schule. Dumbledore tröstete Ginny, nachdem sie von einem Fragment von Voldemorts Seele besessen war und während des Schuljahres 1992-1993, die Kammer des Schreckens geöffnet hatte. Ginny war eine der ersten, die von Dumbledores Tod erfuhr und die einzige, die Harry danach trösten konnte. Ginny benannte auch einen ihrer Söhne, Albus Potter, nach Dumbledore.
\subsubsection*{\large Lily und James Potter}
%ABSATZ
Dumbledore traf James Potter und Lily Evans zum ersten Mal in ihrem ersten Jahr in Hogwarts, wo er der Schulleiter war. Dumbledore hatte eine gute Beziehung zu Beiden und wusste, dass Beide sehr loyal waren. James hatte Dumbledore seinen Unsichtbarkeitsumhang, der eines der drei Heiligtümer des Todes war, geliehen. Nach James Tod gab Dumbledore den Umhang an James Sohn Harry Potter
%SEITE37
zurück.
\vspace{10pt}
\newline
Als die Potters aufgrund einer Prophezeiung über ihren Sohn und den Dunklen Lord zu Voldemorts Zielen wurden, versuchte Dumbledore sie zu beschützen, indem er den Fidelius-Zauber ausübte und anbot, ihr geheimer Bewahrer zu sein, aber die Rolle des geheimen Bewahrer ging stattdessen an Peter Pettigrew. Dies stellte sich jedoch als fatalen Fehler heraus, da Peter sich als Spion herausstellte und Lilly und James an Voldemort verriet, der sie tötete. Dumbledore war sehr traurig über ihren Tod und widmete den Rest seines Leben dem Schutz ihres Sohnes Harry. Er erzählte Harry of von seinen Eltern und erklärte Harry, dass ihn seine Eltern niemalds wegen ihrer Liebe wirklich verlassen würde. Dumbledore glaubte auch, dass Harrys Liebe zu seinen Eltern eine große Kraft war.
\subsubsection*{\large Familie Dursley}
%ABSATZ
Es war bekannt, dass Dumbledore vor der Geburt ihres Neffen sich kurz mit der Familie Dursley beschäftigt hatte. 1971 schrieb Petunia Dursley einen Brief an Dumbledore und fragte, ob sie zusammen mit ihrer Schwester Hogwarts besuchen könne. Dumbledore lehnte ihre Bitte freundlicherweise ab, da sie ein Muggel war, was wahrscheinlich zu Petunias Hass gegenüber der Magie beitrug. Zehn Jahre später ließ Dumbledore Petunias Neffen Harry, kurz nach den Mord an ihrer Schwester Lily und ihrem Schwager James, zu ihr vor die Haustür bringen. Die Entscheidung war jedoch sehr umstritten, da Minerva McGonagall fand, dass die Familie sehr schlimm waren. Dumbledore hoffte, dass sich dies ändern würde, jedoch ärgerten sich Petunia und ihr eigener Ehemann sehr über die Anwesenheit von Harry.
\vspace{10pt}
\newline
Nachdem Harry seinen Hogwarts-Aufnahmebrief von Rubeus Hagrid erhalten hatte, hatte Vernon Dumbledore als "verrückten alten Narren" bezeichnet und weigerte sich die Schuldgelder von Harry zu bezahlen. Als Reaktion auf diesen respektlosen Ausbruch verpasste Hagrid Vernons Sohn Dudley Dursley einen Schweineschwanz. Im Sommer 1995 wollte Vernon Harry aus seinem Haus werfen, worauf Dumbledore Petunia einen Heuler schickte. Weniger als ein Jahr später sprachen Harry und Dumbledore darüber, warum er überhaupt bei der Familie Dursley bleiben müsste.
\vspace{10pt}
\newline
Ein paar Monate später traf sich Dumbledore zum letzten Mal in seinem Leben mit den Dursleys. Während seines Besuchs in ihrem Haus ging Dumbledore auf den Schutz von ihnen ein. Auch ging es um den Schutz, den sie Harry gewährten, Harrys Erbe von Sirius Black und Änderungen in der Zaubererwelt. Außerdem beschwerte sich Dumbledore über die Fürsorge von Harry und die Bevorzugung von Dudley. Vernon und Petunia waren wahrscheinlich empört über Dumbledores Bemerkungen, wehrten sich aber nicht.
\subsubsection*{\large Sirius Black}
%ABSATZ
Dumbledore hatte auch eine gute Beziehung zu Sirius Black, James Potter bestem Freund und Harry Potters Paten. Auch war Sirius Mitglied des Orden des Phönix. Wie der Rest der Zaubererwelt glaubte Dumbledore anfangs, dass Sirius derjenige war, der Lily und James an Lord Voldemort verriet. Deshalb half Dumbledore auch ihn zu verurteilen und ihn nach Askaban zu schicken, obwohl es nicht genügend Beweise gab. 1994 erfuhr er aber, dass der Täter Peter Pettigrew war. Da er aber keine Möglichkeit hatte Sirius Unschuld zu beweisen, obwohl er Hexenmeister des Zaubergamots war und die Macht hatte ihn vor Gericht zu stellen, entschädigte er Sirius, indem er Hermine Granger und Harry half, Sirius vor dem Kuss des Dementors zu retten und dem Britischen Zaubereiministerium zu entkommen, indem er ihnen Tipps für den Zeitumkehrer gab.
\vspace{10pt}
\newline
Die Beiden nahmen dann wieder ihre Freundschaft auf und Sirius erinnerte Dumbledore daran, den Orden nach Voldemorts Rückkehr wieder ins Leben zu rufen. Ihre Freundschaft war jedoch schnell angespannt, als Dumbledore Sirius zwang am Grimmauldplatz 12 zu bleiben, weil er weiterhin vom Zaubereiministerium gesucht wurde und weil er häufig vorschnell handelte, was ihn dazu brachte, dass er leicht in große Gefahr geriet. Sirius konnte diesen Ansichten zwar nachvollziehen, aber das Haus bracht viele unglückliche Erinnerungen für ihn auf und er war häufig alleine in dem Haus, wo er sich einsam fühlte. Trotzdem vertraute Sirius Dumbledore weiterhin. Dumbledore war traurig über den Tod von Sirius und glaubte, dass es seine Schuld war. Auch tröstete er Harry, indem er sagte, dass Sirius sehr lebhaft war und auch lustig und mutig, der sich immer um seine Liebsten kümmerte. Sirius Tod brachte Dumbledore auch dazu Harry von der Prophezeiung über ihn und Voldemort und den Opferschutz seiner Mutter zu erzählen.
\subsubsection*{\large Remus Lupin}
%ABSATZ
Ein weiterer guter Freund von Dumbledore war Remus Lupin, ein guter Freund von James Potter und Sirius Black und Werwolf. Dumbledore sorgte mit einigen besonderen Vorsichtsmaßnahmen dafür, dass Remus trotz seines Werwolfsstatus nach Hogwarts kommen konnte, um eine angemessene magische Ausbildung zu erhalten. Nach seinem Abschluss wurde er auch Mitglied des Orden des Phönix. Später, 1993, gab Dumbledore ihm den Posten des Professors für Verteidigung gegen die dunklen Künste, eine fragwürdige Entscheidung. Er half Remus sogar dabei seine Werwolf-Seite zu kontrollieren, indem er Severus Snape regelmäßig den Wolfsbanntrank brauen ließ. Remus schätze dies sehr, da er sonst von der Gesellschaft eher gemieden wurde, weil es viele Vorurteile gegenüber Werwölfen gab. Remus trat auch 1995 wieder dem Orden des Phönix bei und beide kämpften in der Schlacht in der Mysteriumsabteilung, wo Remus einen für ihn sehr schweren Verlust erlitt. Sirius starb, nachdem er von Bellatrix Lestrange mit dem Todesfluch getroffen wurde. Auch Remus war sehr traurig über den Tod von Dumbledore
%SEITE38
und zeigte seine Loyalität gegenüber Dumbledore, indem er in der Schlacht von Hogwarts kämpfte.
\subsubsection*{\large Alastor Moody}
%ABSATZ
Alastor Moody war ein bekannter Auror und Dumbledores Freund. Sie kämpften in vielen Schlachten zusammen und waren befreundet. 1994 nahm Moody Dumbledore Angebot an Verteidigung gegen die dunklen Künste zu unterrichten, wurde jedoch von Bartemius Crouch Junior gefangen genommen und eingesperrt. Bartemius verkörperte ihn das gesamte Schuljahr 1994-1995, um Harry Potter zu Lord Voldemort zu schicken.
\vspace{10pt}
\newline
Dumbledore und Moody kämpften zusammen in der Schlacht in der Mysteriumsabteilung. Nachdem Dumbledore starb, war Moody die nächste Person des Ordens, die starb.
\subsubsection*{\large Fawkes}
%ABSATZ
Albus Dumbledore hatte eine sehr starke Beziehung zu Fawkes, seinem Phönix Haustier, der ihm als sehr enger Begleiter diente. Dumbledore kümmerte sich immer um Fawkes, wenn er nach einem seiner unzähligen Todesfällen wiedergeboren wurde. Ihm Gegenzug dazu war Fawkes ihm äußerst treu und lebte mit Dumbledore in seinem Schulleiterbüro. Fawkes fing zum Beispiel einen Todesfluch ein, um Dumbledore zu schützen und ihn vor Lord Voldemort zu verteidigen. Daraufhin verspottete ihn Voldemort. Er half Albus auch bei der Flucht vor mehreren Beamten des Britischen Zaubereiministerium, als sie versuchten Dumbledore festzunehmen. Als Dumbledore starb, trauerte Fawkes um ihn, indem er ein Trauerlied sang. Danach verließ er Hogwarts und wurde anscheinend nie wieder gesehen.
\subsubsection*{\large Orden des Phönix}
%ABSATZ
Der Orden des Phönix war eine Organisation, die Dumbledore gegründet hatte, um Lord Voldemort und die Todesser zu bekämpfen. Er hatte ganz gute Beziehungen zu den meisten Mitgliedern wie seinem Bruder Aberforth Dumbledore, Elphias Doge, Minerva McGonagall, Rubeus Hagrid, Severus Snape, James Potter, Lily Evans, Sirius Black, Remus Lupin und Alastor Moody.
\vspace{10pt}
\newline
Er war auch mit Kingsley Shacklebolt befreundet, der an Gleichheit und Fairness für alle Zauberer, Hexen und Muggel glaubte. Arabella Figg gab er die Aufgabe Harry Potter im Haus der Familie Dursley zu beobachten. Auch im Orden Befreundete waren Nymphadora Tonks, Dädalus Diggel, Hestia Jones, Emmeline Vance, Sturgis Podmore, Mundungus Fletcher und die Familie Weasley. Emmeline wurde von Todessern während des Zweiten Zaubererkriegs getötet.
\vspace{10pt}
\newline
Andere mögliche Freunde von Dumbledore, die während des Ersten Zaubererkriegs getötet oder schwer verletzt wurden, waren Edgar Bones, Caradoc Dearborn, Benjy Fenwick, Marlene McKinnon, Dorcas Meadowes, Fabian Prewett, Alice Longbottom, Frank Longbottom. Die Beiden Longbottoms wurden von Bellatrix Lestrange und Bartemius Crouch zum Wahnsinn gefoltert.
\vspace{10pt}
\newline
Dumbledore hatte auch viele Freunde, die Verbündete des Ordens waren, darunter Olympe Maxime, die Schulleiterin der Beauxbatons Akademie für Zauberei, die er während des Trimagischen Turniers traf. Auch waren seine Mitarbeiter Filius Flitwick, Pomona Sprout und Horace Slughorn Verbündete des Ordens. Weitere Verbündete waren Augusta Longbottom, Ted Tonks und Andromeda Tonks. Auch waren seine Schüler teilweise Verbündete des Ordens wie Ginevra Weasley, Neville Longbottom, Luna Lovegood, Lee Jordan und Percy Weasley. Er mochte auch die Hauselfen Dobby und Kreacher und bat sie in der Küche von Hogwarts auszuhelfen.
\vspace{10pt}
\newline
Es ist wahrscheinlich, dass alle Mitglieder und Verbündete des Ordens, die weder getötet noch inhaftiert wurden, in der Schlacht von Hogwarts kämpften. Es ist nur ziemlich sicher, dass Arabella Fig nicht mitkämpfte, da sie ein Squib war. Möglich ist jedoch, das sie auf eine andere Weise geholfen hat.
\subsubsection*{\large Hogwarts-Personal}
%ABSATZ
Neben den bereits erwähnten Minerva McGonagall, Rubeus Hagrid, Severus Snape, Remus Lupin und Alastor Moody hatte Dumbledore während seiner Karriere in Hogwarts viele weitere Mitarbeiter. Armando Dippet war einer der Wenigen, der Dippet offenbar aufgrund seiner Unterrichtseffizienz vertraute. Herbert Beery, ehemaliger Professor für Kräuterkunde, war anscheinend auch mit Dumbledore befreundet, jedoch legte er sein Lehramt in Hogwarts nieder, um weiter an der Magischen Akademie für Schauspielkunst zu unterrichten. Auch arbeitete Dumbledore zusammen mit Galatea Merrythought, Professor für Verteidigung gegen die dunklen Künste bis zu Dippets Tod, Silvanus Kesselbrand, Professor für Pflege Magischer Geschöpfe und Horace Slughorn, Professor für Zaubertränke, der zwei Amtszeiten an Hogwarts hatte.
\vspace{10pt}
\newline
Nach Dippets Tod wurde er der neue Schulleiter von Hogwarts. Danach war er auch mit Filius Flitwick, dem Professor für Zauberkunst und Leiter des Hauses Ravenclaw, Pomona Sprout, Professorin für Kräuterkunde und Leiterin des Hauses Hufflepuff und Poppy Pomfrey, der Krankenschwester im Krankenflügel befreundet.
%SEITE39
\vspace{10pt}
\newline
Mit Ausnahme von Dolores Umbridge respektierte jeder Mitarbeiter von Hogwarts Dumbledore und war von ihm sehr überzeugt. Andere Mitarbeiter der Schule waren Argus Filch, der Hausmeister von Hogwarts; Wilhelmina Raue-Pritsche, Ersatzprofessorin für Pflege magischer Geschöpfe; Firenze, ein Zentaur und Professor für Wahrsagen; Rolanda Hooch, Professorin für Besenkunst; Aurora Sinistra, Professorin für Astronomie; Bathsheda Babbling, Professorin für das Studium der Alten Runen; Septima Vektor, Professorin für Arithmantik; Irma Pince, die Bibliothekarin; Cuthbert Binns, ein Geist und Professor für Geschichte der Zauberei; Charity Burbage, Professorin für Muggelkunde.
\vspace{10pt}
\newline
Dumbledore kannte Gilderoy Lockhart, den Professor für Verteidigung gegen die dunkle Künste, als er noch Schüler war. Es ist nicht völlig bekannt, wie Dumbledore Lockhart betrachtete, jedoch hielt er nicht sonderlich viel von seinen Ideen, seiner Eitelkeit und Eingebildetheit. Dumbledore glaubte jedoch an das Gute in jedem Menschen und wollte Lockhart auch eine weitere Chance geben, da er sicher war, dass er auch seine Eingebildetheit überwinden konnte und aufgrund seiner Talente in großer Zauberer werden könne. Als Lockhart jedoch die Erinnerungen von zwei Zauberern, die Dumbledore kannte, änderte, fand Dumbledore, der sich nie von Lockharts großen Geschichten täuschen ließ, leicht die Wahrheit heraus und beschloss sofort Lockhart einen Posten als Professor in Hogwarts anzubieten, um seine Betrug aufzudecken. Ein weiterer Grund war, dass sich keine andere Person finden ließ, die den Posten übernehmen wollte. Trotz Dumbledores Abneigung von Lockharts Taten, schien er ihn dennoch nicht wirklich zu hassen, da er weiterhin höflich gegenüber ihn war, wenn sie miteinander redeten. Als Lockhart später seine Erinnerungen selbst vollständig löschte, war Dumbledore mitfühlend genug, dass er sich entschied, Lockharts Lügen und Verbrechen nicht zu enthüllen. Dies geschah wahrscheinlich, um Lockhart vor dem Hass und den Strafen der Zauberergemeinschaft zu schützen.
\vspace{10pt}
\newline
Eine weitere Professorin war Sybill Trelawney, die Wahrsagen unterrichtete. Dumbledore hatte anfangs vor Wahrsagen nicht weiter an Hogwarts zu unterrichten, ging aber aus Höflichkeit trotzdem zum Vorstellungsgespräch von Trelawney. Kurz vor Ende des Gesprächs, merkte er jedoch, dass sie keine wirkliche Seherin war. Kurz vor Ende machte Trelawney jedoch in Trance eine echte Vorhersage über Lord Voldemort und Harry Potter. Deshalb stellte er Trelawney schließlich doch ein, um sie vor den Todessern zu schützen. Als Umbridge 1995 entließ und versuchte sie aus Hogwarts zu vertreiben, trat Dumbledore für Trelawney ein und sorgte dafür, dass sie in Hogwarts bleiben durfte, obwohl sie nicht mehr unterrichtete.
\vspace{10pt}
\newline
Quirinus Quirrell war Professor für Muggelkunde, bevor er anfing Verteidigung gegen die dunklen Künste zu unterrichten. Es ist nicht bekannt, was Dumbledore von Quirrell hielt, während er Muggelkunde unterrichtete. Jedoch wird er von seinen akademischen Leistungen als Schüler beeindruckt gewesen. Auch stellte Dumbledore Quirrell persönlich ein. Es ist nicht bekannt, ob Dumbledore von Quirrells Zartheit wusste, und dass er gemobbt wurde. Dumbledore wird jedoch gehofft haben, dass Quirrell mehr Selbstvertrauen bekommt. Dies geschah jedoch nicht, da Quirrell auf Voldemort bei einer Weltreise traf und mit ihm seinen Körper teilte. Seitdem diente er Voldemort treu und versuchte den Stein der Weisen zu stehlen und Harry zu töten. Dumbledore durchschaute Quirrell schnell und merkte, dass er verdient durch Harrys Berührungen verbrannt wurde und daran starb.
\vspace{10pt}
\newline
1995 wurde Umbridge vom Britischen Zaubereiministerium zur Professorin für Verteidigung gegen die dunklen Künste ernannt. Umbridge war eine sadistische Frau, die es genoss, Schüler zu bestrafen. Umbridge wurde anfangs Hogwarts-Großinquisitorin und später sogar Schulleiterin. Ende des Jahres wurde sie jedoch von Zentauren verfolgt und verließ Hogwarts.
\vspace{10pt}
\newline
Jeder Mitarbeiter von Hogwarts war über Dumbledores Tod sehr traurig und es ist wahrscheinlich, dass fast alle Mitarbeiter in der Schlacht von Hogwarts gekämpft haben.
\subsubsection*{\large Dumbledores Armee}
%ABSATZ
Dumbledores Armee war eine Organisation, die von Harry Potter gegründet und geführt wurde, um die Schüler für die kommenden Schlachten gegen Lord Voldemort und seine Todesser vorzubereiten.
\vspace{10pt}
\newline
Mitglieder der Organisation waren Harry, Ronald Weasley, Hermine Granger, Ginevra Weasley, Neville Longbottom, Luna Lovegood, Fred Weasley, George Weasley, Cho Chang, Dean Thomas, Seamus Finnigan, Parvati Patil, Padma Patil, Lavender Brown, Hannah Abbott, Ernest Macmillan, Susan Bones, Justin Finch-Fletchley, Michael Corner, Terry Boot, Anthony Goldstein, Katie Bell, Alicia Spinnet, Angelina Johnson, Colin Creevey, Dennis Creevey, Zacharias Smith und Marietta Edgecombe.
\vspace{10pt}
\newline
Alle der Mitglieder waren über Dumbledores Tod sehr traurig und kämpften wahrscheinlich alle in der Schlacht von Hogwarts. Die Mitglieder Fred Weasley, Lavender Brown und Colin Creevey starben während der Schlacht. Die beiden Mitläufer Marietta Edgecombe und Zacharias Smith glaubten von Angang an nicht an Dumbledore und haben auch nicht an der letzten Schlacht teilgenommen.
\subsubsection*{\large Weitere Freundschaften}
%ABSATZ
Während seines Lebens freundete sich Dumbledore mit vielen berühmten Zauberern und Hexen an. Darunter waren zum Beispiel Nicolas Flamel, Nicolas Frau Perenelle Flamel, Bathilda Bagshot und Griselda Marchbanks.
%SEITE40
\vspace{10pt}
\newline
Dumbledore lernte Nicolas durch seine Interesse an Alchemie kennen. Dumbledore erarbeitete 12 Anwendungen von Drachenblut und verfasste zusammen mit Flamel ein Werk über Alchemie. Als Dumbledore 1927 seinen ehemaligen Schüler Newt Scamander nach Paris schickte, damit dieser Credence Barebone aufspüren konnte, gewährte ihm Flamel Unterschlupf in seinem Haus. Nicolas und seine Frau Perenelle waren über 650 Jahre alt, als sie sich dazu entschlossen, den Stein der Weisen zu zerstören, um Lord Voldemorts Rückkehr nicht zu verschlimmern.
\vspace{10pt}
\newline
Bathilda Bagshot war die Nachbarin der Familie Dumbledore in Godric's Hollow. Ihr Großneffe war Gellert Grindelwald, der bei ihr wohnte, nachdem er von Durmstrang verwiesen wurde. Die beiden Jugendlichen freundeten sich an und interessierten sich für ähnliche Themen in der Magie. Durch Bagshot fanden sie viel über die Geschichte der Magie heraus und konnten so ihre Forschungen vertiefen. Als die Freundschaft zwischen Dumbledore und Grindelwald beendet wurde, hatte Dumbledore weiterhin Kontakt zu Bathilda. Bathilda starb einige Monate nach Dumbledore, da sie auch auf den Auftrag von Voldemort getötet wurde.
\subsubsection*{\large Todesser}
\subsubsection*{Bellatrix Lestrange}
%ABSATZ
Bellatrix Lestrange war eine Todesserin, die in Beiden Zaubererkriegen kämpfte. Lord Voldemort war ihre einzige wahre Liebe, weshalb sie ihm fanatisch treu war und ihn als mächtigsten Zauberer betrachtete, sogar mächtiger als Dumbledore, obwohl sie auch einsah, dass Dumbledore sehr mächtig war. Bellatrix gehört zu den wenigen Todessern, die von dem Plan wussten Dumbledore zu ermorden und das Draco Malfoy diese Aufgabe übernehmen solle. Bellatrix erlebte nicht, dass ihr anfänglicher Verdacht über Severus Snapes Loyalität richtig war, da sie während eines Duells mit Molly Weasley starb. Dumbledore fand, dass Bellatrix unglaublich gefährlich und sadistisch war. Dumbledore erzählte Snape, dass er lieber einen schnellen und sauberen Tod sterben würde, anstatt Bellatrix zum Opfer zu fallen, die häufig mit "ihrem Essen spielte". Bellatrix tötete auch viele von Dumbledores Freunden wie Sirius Black, Dobby und Nymphadora Tonks.
\subsubsection*{Lucius Malfoy}
%ABSATZ
Lucius Malfoy war Leiter des Hogwarts-Schulbeirat in Dumbledores führen Jahren als Schulleiter von Hogwarts. Lucius gehört zu den Todessern, die nach dem Ersten Zaubererkrieg nicht nach Askaban mussten. 1991 besuchte Lucius Sohn Draco Malfoy zum ersten Mal Hogwarts. Die Familie Malfoy hatten eine starke Abneigung zu Dumbledore, da er Muggelgeborene und Muggelrechte schätzte. 1992 manipulierte Lucius Ginevra Weasley, damit sie die Kammer des Schreckens wieder öffnete, indem er ihr Tom Riddles Tagebuch gab, um so Dumbledore aus Hogwarts zu werfen. Dies schlug fehl, weil Dumbledore die Wahrheit erkannte, als Harry Potter Ginny vor dem Basilisken rettete. Deshalb ließ Dumbledore Harry über Lucius Bestrafung entscheiden. Harry schenkte dem Hauselfen der Familie Malfoy Dobby die Freiheit und Lucius verlor seine Position im Schulrat. Im Jahr 1996 kämpften Dumbledore und Lucius in der Schlacht in der Mysteriumsabteilung gegeneinander. Lucius wurde deshalb nach Askaban gesperrt. Da Lucius Lord Voldemort die Prophezeiung über den Auserwählten nicht gab wurde Voldemort wütend un machte Draco zum Todesser. Seine erste Aufgabe als Todesser war es Dumbledore zu töten, sonst würde Voldemort seine Familie töten. Dumbledore sagte Harry, er solle ein gewisses Mitgefühl für Lucius ausdrücken, da er glaubte, dass Lucius froh war in Askaban eingesperrt wurde, sodass er keine weiteren grauenvollen Taten für Voldemort mehr machen musste. Draco versuchte Dumbledore auf Voldemorts Befehl zu töten und sagte ihm, dass Voldemort ihn sonst töten würde, wenn er es nicht tat. Dumbledore zeigte Sympathie für Dracos Notlage, aber spürte, dass der junge Zauberer nicht fähig war, ihn zu ermorden. Deshalb schlug er vor, dass er seinen Posten als Todesser verlassen könnte und Dumbledore ihn und seine Mutter beim Orden des Phönix verstecken könne. Dieses Gespräch wurde jedoch von weiteren Todessern unterbrochen. Ein Jahr später wechselte die Familie Malfoy die Seite.
\subsubsection*{Peter Pettigrew}
%ABSATZ
Peter Pettigrew war Mitglied des Orden des Phönix und Freund von James Potter. Es ist möglich, dass er auch ein Freund von Dumbledore war. 1981 verriet Pettigrew jedoch James und Lily Evans, als er Voldemort als Spion den Standort der Familie verriet. Er täuschte seinen eigenen Tod vor, indem er sich in einer überfüllten Londoner Straße den Finger abschnitt und eine Menge von Muggel mit dem Sprengfluch tötete. Dadurch wurde Sirius Black als Täter vorgetäuscht. Wie der Rest der Zaubererwelt glaubte Dumbledore zunächst auch, dass es wirklich Sirius war und er Pettigrew getötet hatte. Er erfuhr 1994 die Wahrheit, als er davon erfuhr, dass Harry Potter Pettigrews Leben vor der Rachen von Sirius und Remus Lupin gerettet hatte. Dumbledore erklärt Harry zwar, dass seine Tat gut war, jedoch Pettigre2 weiterhin für die Todesser arbeitete. Dumbledores Vermutungen erwiesen sich als wahr, als Pettigrew versuchte Harry zu erwürgen. Pettigrew zögerte jedoch, als Harry ihn an seine Verbrechen erinnerte. Daraufhin erwürgte die silberne Hand, die Voldemort Pettigrew gab, den Zauberer, wodurch er starb.
\subsubsection*{Igor Karkaroff}
%ABSATZ
Igor Karkaroff war ein Todesser im Ersten Zaubererkrieg. Dumbledore nahm an seinem Prozess teil, wo Karkaroff viele Todesser verriet, um eine Inhaftierung in Askaban zu verhindern. Nach seiner Freilassung kam er nicht mehr mit Lord Voldemort in
%SEITE41
Kontakt und wurde Direktor des Durmstrang Institut. Er traf Dumbledore 1994 während des Trimagischen Turniers in Hogwarts. Zuerst begrüßten sich beide Schulleiter wie alte Freunde und zeigten scheinbar keine Feindseligkeit. Karkaroff warf Dumbledore jedoch später Betrug vor, als er Harry Potters Namen aus dem Feuerkelch zog. Noch wütender wurde er, als er von dem Angriff auf den Durmstrang-Champion Viktor Krum erfuhr. Er verließ Hogwarts nach Voldemorts Wiedergeburt und wurde 1995 von Todessern getötet. Dumbledore bezeichnete Karkaroff später als Feigling.
\subsubsection*{Bartemius Crouch Junior}
%ABSATZ
Ein weiterer sehr loyaler Todesser war Bartemius Crouch Junior, der von seinem eigenen Vater nach Askaban gebracht wurde, aber mit Hilfe seiner Mutter floh. Crouch spielte eine Schlüsselrolle bei Voldemorts Wiedergeburt, da er Alastor Moody gefangen nahm und eingesperrt hatte, um seine Gestalt einzunehmen und so in Hogwarts zu unterrichten. Währenddessen war Dumbledore der Meinung, Barty wäre tot. Crouch war auch derjenige, der Harry Potters Namen in den Feuerkelch warf und ihm half, die Aufgaben des Trimagischen Turniers zu schaffen, damit Harry zu Lord Voldemort gelangen konnte. Als Harry jedoch entkommen konnte, versuchte Crouch ihn zu töten, wurde jedoch von Dumbledore aufgehalten. Crouch starb 1995 durch den Dementorenkuss. Dumbledore zeigte Wut und Ekel über Crouch Juniors Taten und Handlung. Dennoch war Dumbledore enttäuscht und empört darüber, dass Cornelius Fudge Crouch zum Tode beurteilte. So gab es keine Zeugen mehr zu Voldemorts Rückkehr und die gesamte Familie Crouch wurde ausgelöscht. Dies zeigte auch, dass Dumbledore trotz seines Ekels auch Mitleid mit Crouch hatte.
\subsubsection*{Fenrir Greyback}
%ABSATZ
Fenrir Greyback war ein wilder Werwolf, der in der Vergangenheit öfter mit Dumbledore zu tun hatte, da Dumbledore irritiert war, dass der Todesser zur Schlacht auf dem Astronomieturm nach Hogwarts gekommen war. Dumbledore war auch davon angewidert, dass Greyback außerhalb des Vollmonds angriff und den Geschmack von menschlichem Fleisch genoss. Er griff auch Kinder in Hogwarts an. Deshalb war Dumbledore sehr enttäuscht, als er fälschlicherweise glaubte, dass Draco Malfoy Greyback nach Hogwarts eingeladen hatte.
\subsubsection*{Weitere Todesser}
%ABSATZ
Es ist wahrscheinlich, dass Dumbledore weitere Todesser, bekämpfte und an ihren Prozessen teilnahm, wen sie gefangen genommen wurden. Dumbledore verabscheute aber fast alle Todesser.

\subsection*{\Large Etymologie}
%ABSATZ
Dumbledores Vornamen können so interpretiert werden, dass die vier Hauptteile der britischen Insel dargestellt werden:
\vspace{10pt}
\newline
Albus steht im Lateinischen für "weiß" oder "hell". Das "weiß" kann für das Gute im Menschen stehen oder bezieht sich auf seine Haarfarbe. Auch kann es die männliche Form von "Alba" sein, der gälische Name für Schottland oder das italienische Wort für "Sonnenaufgang". Dies soll möglicherweise eine Anspielung auf die Wiedergeburtssymbole des Phönix sein. Albus kann auch mit der griechischen Göttin Aphrodite (der Göttin der Liebe) in Verbindung gebracht werden. Laut Rowling selbst wurden die Vornamen von Albus Dumbledore und Rubeus Hagrid anhand von "Weiß" und "Rot" ausgewählt, die als wesentliche mystische Bestandteile der Alchemie angesehen wurden. Dumbledore erhielt das Weiß für Aspekte "ein spiritueller Theoretiker, brilliant, idealisiert und etwas distanziert" zu sein.
\vspace{10pt}
\newline
Percival ist ein legendärer Anhänger von König Arthur aus Wales, der an der Gralssuche beteiligt war. In Le morte d'Arthur von Thomas Mallory ist Percival sowohl Held der Suche, als auch Erzähler der Geschichte, wie dies bei Dumbledore in Teilen der Geschichte der Fall ist. Auch hieß der Vater von Albus Dumbledore Percival.
\vspace{10pt}
\newline
Clodius Albinus war ein römischer Usurpator, der von den Legionen in Großbritannien und Hispania nach der Ermordung von Pertinax im Jahr 193 zum Kaiser ernannt wurde.
\vspace{10pt}
\newline
Wulfric ist ein angelsächsischer Name, der in der angelsächsischen Chronik vorkommt und für England repräsentativ sein könnte. Wulfric bedeutet wörtlich übersetzt "Wolfskraft" und erinnert an einen anderen ähnlichen Namen, Beowulf, was "mächtiger Wolfsbär" bedeutet. Der legendäre Held Beowulf tötete das Monster Grendel als Jugendlicher, ein Name, der sich ähnlich anhört wie Gellert Grindelwald, dunkler Zauberer und Verlierer im Duell mit Dumbledore. Beowulf wurde von einem Drachen in einer Meereshöhle tödlich verwundet, und der einzige, der ihm helfen würde, war sein Knappe, ein verwaister Sohn einiger treuer Anhänger. Dumbledore wurde auch durch einen Trank in einer Meereshöhle schwer verwundet und der verwaiste Anhänger würde Harry Potter darstellen.
\vspace{10pt}
\newline
Brian ist der Name des legendären irischen Hochkönigs und Helden Brian Boru, der die Wikinger in der Schlacht von Clontarf bei Dublin besiegte. Dumbledore selbst war ein Kämpfer, der viele dunkle Zauberer besiegte. Auch ist Brian ein gewöhnlicher englischer Name im Vergleich zu den anderen Namen, mit der Bedeutung "stark" und "edel".
%SEITE42
\vspace{10pt}
\newline
Dumbledore ist ein altes englisches Wort aus dem 18. Jahrhundert für "Hummel".Das Wort wird noch in Neufundland und Kanada verwendet. Joanne Rowling erklärte, dass sie sich selbst vorgestellt hatte, wie Dumbledore vor sich hinsummend im Schloss herumlief.
\vspace{10pt}
\newline
"Albus Dumbledore" könnte auch von Rowlings alten Schulleiter Alfred Dunn inspiriert gewesen sein.
Von Madam Maxime wurde er einmal Dumbly-Dorr genannt, welches sich mit "dumm" oder "Dummkopf" übersetzen lässt.

\subsection*{\Large Hinter den Kulissen}
%ABSATZ
Dumbledore wird während der Verfilmungen von Harry Potter und der Stein der Weisenund Harry Potter und die Kammer des Schreckens vom verstorbenen irischen Schauspieler Richard Harris gespielt. Als Harris 2002 an der Hodgkin-Krankgeit starb, wurde Michael Gambon neuer Schauspieler von Dumbledore in den restlichen Filmen. Dadurch ist Dumbledore eine der 14 Figuren, die in allen 8 Verfilmungen vorkommt.
\vspace{10pt}
\newline
Ein jüngerer Dumbledore (ungefähr 62 Jahre alt) wird auch von Harris in Harry Potter und die Kammer des Schreckens gespielt und Gambon spielte Dumbledore in Harry Potter und der Feuerkelch(ungefähr 100 Jahre alt) und in Harry Potter und der Halbblutprinz(ungefähr 57 Jahre alt). Toby Regbo spielt einen noch jüngeren Dumbledore (ungefähr 18 Jahre alt) in Harry Potter und die Heiligtümer des Todes 1 und Phantastische Tierwesen: Grindelwalds Verbrechen.
\vspace{10pt}
\newline
Jude Law spielt den etwa 47 Jahre alten Dumbledore in Phantastische Tierwesen: Grindelwalds Verbrechen.
Auch die Synchronsprecher der deutschen Filme haben bereits mehrmals gewechselt. So sprach Klaus Höhne Dumbledore in den Filmen 1-4 und Wolfgang Hess sprach ihn in den restlichen Filmen der Harry Potter-Filmreihe. Alexander Brem sprach Dumbledore in Phantastische Tierwesen: Grindelwalds Verbrechen (Film).
Die Darstellung von Gambon als Dumbledore unterscheidet sich deutlich von der Darstellung von Harris und den Beschreibungen in den Büchern. In den Büchern wird beschrieben, dass Dumbledore immer sehr ruhig und miit einer sanften Stimme sprach. Es ist auch bekannt, dass Dumbledore eine spitze Zauberermütze, lange Gewänder und einen langen, nicht gebunden Bart trug. Gambons Dumbledore ist jedoch in dieser Hinsicht etwas anderes. Obwohl er die meiste Zeit ruhig und friedlich ist, tendiert er manchmal dazu, doch etwas lauter zu werden. So wird Dumbledore in Harry Potter und der Feuerkelch und Harry Potter und der Orden des Phönix gegenüber Harry Potter manchmal etwas lauter. Dieses Verhalten ist für Dumbledore in den Büchern und vorherigen Filmen ein eher untypisches Verhalten.
\vspace{10pt}
\newline
Gambons physische Darstellung als Dumbledore ist ebenfalls etwas anderes. Es war ein eher böhmischer Look mit einem Quastenhutes, zusammen mit langen Gewändern, die arabische oder indische Einflüsse hatten. Sein Bart ist auch mit einer Kette zusammengebunden. Dumbledore trägt seine Brille jedoch nicht immer in den Filmen (außer im dritten Teil). Im vierten Teil trug er seine Brillen immer Mal wieder. Im sechsten Film trug er dagegen seine Brille nur zu Beginn des Films und als er nach dem Tagebuch von Riddle suchte. Zu dieser Szene ist jedoch auch ein Filmfehler gekommen, da in der nächsten Einstellung die Brille verschwunden war.
\vspace{10pt}
\newline
Joanne K. Rowling hat angegeben, dass Albus Dumbledore ihr Lieblingscharakter ist.
Dumbledore sagte in der Verfilmung von Harry Potter und der Gefangene von Askaban kein einziges Mal den Namen von Harry Potter. In den Filmen davor war dies immer der Fall.
In den früheren Entwürfen zu Harry Potter und der Orden des Phönix wurde Dumbledore nach der Entdeckung von Dumbledores Armee wirklich verhaftet und fragte humorvoll, ob er seine Zahnbürste nach Askaban mitnehmen dürfte. Im späteren Original floh er.
Albus Dumbledore ist in den Filmen der erste Mensch, der zu sehen ist. In den Büchern ist es hingegen die Familie Dursley.
\vspace{10pt}
\newline
Beide Schauspieler, die den Dumbledore zu Harrys Schulzeit gespielt haben, waren Iren. In einem Interview auf der DVD von Harry Potter und der Gefangene von Askabanerwähnte Gambon, dass er häufig bei Dumbledore mit irischem Akzent sprach, aber ihn niemand am Set bat, dies zu ändern. Bei Harris war der Akzent eher weniger ausgeprägt.
\vspace{10pt}
\newline
Im Jahr 2007 enthüllte Rowling, dass Dumbledore schwul war und in Gellert Grindelwald war. Rowling kommentierte auch, dass Dumbledore deshalb über die Veränderung von Grindelwald entsetzt war. Die Liebe zu Grindelwald wurde hauptsächlich dadurch erzeugt, dass Dumbledore jemand fand, der so brilliant wie er selbst war. In Heiligtümern des Todes findet man auch kleine Hinweise dazu. Berichten zufolge kam jedoch der früheste Hinweis auf Dumbledores Sexuelle-Orientierung, als Rowling ein Veto gegen eine Dialoglinie einlegte, in der Dumbledore über eine romantische Beteiligung einer Frau redete. Rowling erklärte auch, dass Dumbledore seinen moralischen Kompass während der Zeit mit Grindelwald verlor. Rowling sagte auch, dass Dumbledore nach dem Verrat von Grindelwald ziemlich asexuell wurde. So führte Dumbledore ein Leben ohne zu heiraten. Später in der Bonusausgabe von Phantastische Tierwesen: Grindelwalds Verbrechenbestätigte Rowling, dass Dumbledore und Grindelwald in ihrer Jugend eine romantische Beziehung führte, die ziemlich intensiv und leidenschaftlich war.
\vspace{10pt}
\newline
Bei Twitter wurde gefragt, warum Dumbledore häufig andere Leute schickte, um gegen Böse vorzugehen (in diesem Fall war es Grindelwald). Darauf antwortete sie, dass er seine Gründe hatte und sich selbst Zeit ließ.
\vspace{10pt}
\newline
Die veröffentlichte Version von Die Märchen von Beedle dem Bardenenthüllt, dass Lucius Malfoy versuchte Dumbledore vom 
%SEITE43
Posten des Schulleiters zu entfernen, nachdem er Pro-Muggel-Geschichten verbreitete.
Auf Rowlings Website in der Rubrik Zauberer des Monats war Dumbledore, vor Harry Potter, der letzte Zauberer, der vorgestellt wurde.
\vspace{10pt}
\newline
Ciarán Hinds, der Dumbledores Bruder Aberforth Dumbledore spielte, behauptete, dass Dumbledore 123 Jahre alt war, als er starb. Dies würde bedeuten, dass Dumbledore 1874 geboren wurde.
In einem Interview sagte Rowling, dass Dumbledore 150 Jahre alt war. Dies ist jedoch eine mögliche Anspielung auf die Szene im Film von Harry Potter und der Halbblutprinz, wo das Tiro humorvoll spekulierte, dass Dumbledore 150 Jahre alt sei.
\vspace{10pt}
\newline
Am Ende des ersten Teil der Heiligtümer des Todes wird gezeigt, dass Dumbledore auf der Insel im Schwarzen See begraben liegt, wo Harry und Lupin im Film Gefangener von Askaban liefen.
Dumbledore ähnelt im Aussehen und Rolle Gandalf von J.R.R Tolkiens Herr der Ringe-Trilogie. Seine Ähnlichkeit mit Gandalf ist am besten in der Verfilmung von Harry Potter und der Halbblutprinz. Gandalf dem Weißen ähnelt er auch sehr in Harry Potter und die Heiligtümer des Todes 1. Der Feuerangriff auf die Inferi ist auch vergleichbar mit einem ähnlichen Angriff von Gandalf. Viele Fans dachten deshalb, dass Rowling auch den angeblichen Tod von Gandalf, der eigentlich überlebte, einfügen würde. So wurden auch die Theorien unterstützt, dass Dumbledore überlebt hätte, aber Rowling gab bekannt, dass diese Theorie falsch sei. Ironischerweise wurde Ian McKellen, dem Schauspieler von Gandalf, nach dem Tod von Harris die Rolle des Dumbledore angeboten, jedoch lehnte dieser die Rolle ab.
\vspace{10pt}
\newline
Neben Gandalf hat Dumbledore auch Parallelen zu Obi Wan Kenobi aus George Lucas Star Wars-Original-Trilogie. Alle drei Charaktere besitzen übermenschliche bzw. magische Fähigkeiten und dienen als Hauptmentoren für die Hauptakteure (Harry Potter, Frodo Beutlin und Luke Skywalker) der jeweiligen Geschichten, die sie während ihrer Ausbildung begleiten. Alle Protagonisten wurden auch später Zeuge des Todes ihres Mentors, der sie schwer beschäftigte. Die Hauptakteure konnten jedoch später wieder mit ihrem Mentoren kommunizieren, entweder durch Visionen oder durch Wiedergeburt wie in Gandalfs Fall. Auch wird in jedem der Filme, wenn er gegen einen bösen Gegner antritt, jeder Mentor verspottet, dass er schwach gewesen sei und als "alter Mann" bezeichnet wurde.
\vspace{10pt}
\newline
Dumbledore und Harry Potter sind sich auch in gewisser Hinsicht ziemlich ähnlich: Sie waren Gryffindor-Schüler, Beide tragen eine Brille (Harry trägt Vollmond und Dumbledore trägt Halbmond), Beide sind Halbblüter, beide Mütter waren Muggelgeborene und Beide waren brillante Schüler. Beide besiegten mächtige dunkle Zauberer, waren talentierte Duellanten, kannten sich in der Magie auf dem Gebiet Liebe aus und fungierten als Anführer von verdeckten Organisationen.
\vspace{10pt}
\newline
Evanna Lynch, die Darstellerin von Luna Lovegood, benannte ihre Katzen nach Harry Potter - Charakteren. Eine ihrer Katzen heißt Luna und die anderen Dumbledore.
In der Verfilmung von Harry Potter und der Halbblutprinz sagte Dumbledore zu Draco Malfoy: "Vor Jahren kannte ich einen Jungen, der die falschen Entscheidungen getroffen hat." Auf diesen Satz folgte eine Debatte darüber, auf wen er sich bezog. Möglichkeiten wären Tom Riddle, Gellert Grindelwald, Severus Snape und sogar Dumbledore selbst.
\vspace{10pt}
\newline
Zusätzlich zu dem Porträt von Dumbledore, das nach seinem Tod automatisch in seinem Büro aufgehängt wurde, wurden weitere Porträts von ihm an mehreren Stellen im Schloss Hogwarts aufgehängt.
J. K. Rowling antwortete auf die Frage, mit welcher Figur der Harry-Potter-Romane sie sich zu einem Abendessen treffen würde, dass es Dumbledore sei, weil sie das Gefühl hat, dass sie viel mit ihm besprechen könnte und er gute Ratschläge für sie hätte.
\vspace{10pt}
\newline
Die LEGO-Version von Dumbledore trat 2014 auch im Film The LEGO Movie zusammen mit anderen Charakteren aus beliebten LEGO-Sets auf. In dem Film ist er ein Baumeister und sitzt während eines Treffens mit anderen Baumeistern mit Gandalf. Dumbledore wird von Vitruvius angesprochen, jedoch ist dieser nicht fähig seinen Namen richtig auszusprechen, worüber er sich aufregt. Er wurde zusammen mit anderen Baumeistern von den Bösewichten gefangen genommen und wird später gerettet. Dumbledore hat in dem Film eine hohe weibliche Stimme. Es ist unbekannt, ob Dumbledore während eines Angriffs auf ein Gebiet mit anschließender Zerstörung getötet wurde.
\vspace{10pt}
\newline
Die Magie, die Dumbledore benutzt, um Spuren von Magie in der Horkrux-Höhle zu entdecken, könnte auch die Magie gewesen sein, die die Aurologen benutzten, wie im New Yorker Geisterbote aus Phantastische Tierwesen und wo sie zu finden sinderwähnt.
Dumbledore ist neben Nagini, Minerva McGonagall und Gellert Grindelwald, eine von vier Figuren, die bisher sowohl in der Filmreihe von Harry Potter als auch in Phantastische Tierwesen zu sehen waren.
%SEITE44
\end{document}


\end{document}
