\documentclass[a4paper, 10pt]{article}
\usepackage[T1]{fontenc}
\usepackage[sfdefault]{AlegreyaSans} %% Option 'black' gives heavier bold face
\usepackage[a4paper, left=2cm, right=2cm, bottom=2cm, top=2cm]{geometry} % Adjust left and right margins
\usepackage{amsmath} % for \boxed

% Set the width of the lines around the box
\setlength{\fboxrule}{2pt}

\begin{document}

\begin{minipage}[t]{\textwidth}
    \vspace*{-1.5cm} % Move the content up by 0.5cm
    \begin{flushright}
        \hspace*{\fill} % Move the content to the right edge
        $\boxed{\textbf{\Huge\phantom{00}6\phantom{00}}}$ % Your content here with increased padding
    \end{flushright}
\end{minipage}

\section*{\huge Tom Riddle}
\marginpar{1} 
Tom Vorlost Riddle (  Engl.  Tom Marvolo Riddle ) alias Lord Voldemort (* 31. Dezember 1926, † 2. Mai 1998) war der schrecklichste schwarze Magier seit langer Zeit und nach Gellert Grindelwald auch der stärkste. Seine größten Feinde waren Harry Potter und Albus Dumbledore. Voldemorts Schreckensherrschaft dauerte von 1970 bis 1981 und 1995 bis 1998, bis er in der Schlacht von Hogwarts von Harry Potter besiegt wurde. Er wurde nach seinem Vater Tom Riddle sr. und seinem Großvater Vorlost Gaunt (  Engl.  Marvolo Gaunt ) benannt. Er hat zusammen mit Bellatrix Lestrange eine Tochter namens Delphini.
\vspace{10pt}
\newline
\marginpar{2}  
Voldemorts größtes Ziel war es "den Tod zu besiegen". Um an Unsterblichkeit zu gelangen, erschuf er sieben Horkruxe, davon einen unwissentlich. Demnach bestand seine Seele aus acht Bruchstücken.

\subsection*{\Large Biografie}
\subsubsection*{\large Kindheit}
\marginpar{3} 
Die Hexe Merope aus dem reinblütigen Haus Gaunt sollte einen anderen Zauberer heiraten, verliebte sich allerdings in den Muggel Tom Riddle sr., der ihre Liebe jedoch nicht erwiderte. Also nutzte sie einen Liebestrank, um ihn für sich zu gewinnen, und so zeugten sie einen Sohn. Merope dachte, durch den Liebestrank und die mit ihr verbrachte Zeit habe sich auch bei Tom Liebe für sie entwickelt, doch sie lag falsch. Daraufhin entschied sie, ihn nicht weiter mit Magie an sie zu binden. Tom Riddle wurde am 31. Dezember 1926 in einem Waisenhaus in London geboren, wo er auch aufwuchs, da seine Mutter kurz nach der Geburt verstarb; sein Vater, Tom Riddle sr., war ein wohlhabender Muggel, den Tom hasste.
\vspace{10pt}
\newline
\marginpar{4}  
Bereits als Kind quälte er gerne andere Kinder und Tiere.

\subsubsection*{\large Schulzeit in Hogwarts}
\marginpar{5} 
Im Alter von elf Jahren, im Jahre 1938, wurde er von Albus Dumbledore aus einem Waisenhaus nach Hogwarts geholt. Damals erwähnte Tom, dass er mit Schlangen reden könne, aber Albus Dumbledore machte sich darüber nicht sehr viele Gedanken. Er bewies sich schnell als begabter Schüler. Sein damaliger Hauslehrer, Horace Slughorn (Slytherin), war stolz darauf, ihn in seinem Slug-Club zu haben. Tom wurde allgemein von der Lehrerschaft gemocht, da er immer fleißig war und einen unstillbaren Hunger nach Wissen besaß. Bloß sein Lehrer in Verwandlung, Albus Dumbledore, misstraute ihm. Nichtsdestotrotz wurde er Vertrauensschüler.
\vspace{10pt}
\newline
\marginpar{6}  
Tom Riddle begann, sich mit seinen Ahnen zu beschäftigen und fand heraus, dass er ein Nachkomme von Salazar Slytherin war und kam zu dem Schluss, dass er der wahre Erbe Slytherins sei. In dieser Zeit legte er sich auch sein Pseudonym zu.
\vspace{10pt}
\newline
\marginpar{7}  
TOM VORLOST RIDDLE
\vspace{10pt}
\newline
\marginpar{8}  
Er versetzte die Buchstaben und heraus kam:
\vspace{10pt}
\newline
\marginpar{9}  
IST LORD VOLDEMORT
\vspace{10pt}
\newline
\marginpar{10}  
(Im Original wird TOM MARVOLO RIDDLE zu I AM LORD VOLDEMORT)
\vspace{10pt}
\newline
\marginpar{11}  
Er öffnete die Kammer des Schreckens in seinem fünften Schuljahr. Das Monster darin, der Basilisk, tötete ein muggelstämmiges Mädchen. Ihr Geist blieb fortan am Ort ihres Todes, der Mädchentoilette und sie wurde nun aufgrund ihres unablässigen Klagens die "Maulende Myrte" genannt. Als Vertrauensschüler genoss Tom allerdings ein hohes Ansehen und führte den damaligen Schulleiter, Armando Dippet, auf die Spur, dass angeblich Rubeus Hagrid die Kammer geöffnet habe. Alle verdächtigten Hagrid wegen seiner Acromantula Aragog und seiner Vorliebe für Monster. Deshalb wurde er von der Schule verwiesen.
\vspace{10pt}
\newline
\marginpar{12}  
Nachdem er in einem Buch von Horkruxen und deren Versprechen der Unsterblichkeit gelesen hatte, wollte er von Professor Slughorn wissen, ob und wie es möglich wäre seine Seele mehr als einmal zu teilen. In diesem Gespräch erfuhr Riddle alles Wichtige über die mehrmalige Teilung der Seele. Diese Informationen hätte er durch kein Buch bekommen können. Nachdem Slughorn bemerkte, dass Riddle nicht nur theoretisch an diesem Wissen interessiert ist, erkannte er seinen Fehler und wollte diesen verschleiern. Die Erinnerung an dieses Gespräch wurde von dem Zaubertränkelehrer ebenfalls nachträglich verändert, und Slughorn lebte von nun an mit der Schmach, dem späteren Lord Voldemort wichtige Informationen zur Unsterblichkeit geliefert zu haben.
\vspace{10pt}
\newline
\marginpar{13}  
Da Dumbledore Riddle aber nach wie vor misstraute, wagte letzterer es nicht, die Kammer des Schreckens noch einmal zu öffnen 
%SEITE1
und "Salazar Slytherins edles Werk" zu vollenden. Er schrieb ein Tagebuch, welches einen Teil seiner Seele enthielt, so dass die Erinnerung an den wahren Erben Slytherins erhalten blieb — er tat dies auch, da er erreichen wollte, dass nach seiner Schulzeit die Kammer des Schreckens wieder geöffnet würde. Riddle machte das Tagebuch später zu einem seiner Horkruxe.

\subsubsection*{\large Mord an Riddles Familie}
\marginpar{14} 
In seinem letzten Jahr in Hogwarts war Tom Riddle der Inbegriff eines Musterschülers: Vertrauensschüler, Schulsprecher und einer der brillantesten Schüler, die Hogwarts jemals besucht hatten.
\vspace{10pt}
\newline
\marginpar{15}  
Tom hatte wegen seines Muggel-Vaters, der seine Mutter verließ, einen Hass auf Muggel entwickelt. Er reiste im Sommer des Jahres 1943 nach Little Hangleton, dem Wohnort der Riddles. Er stieß im Anwesen der Gaunts auf seinen Onkel Morfin. Er benutzte einen Schockzauber und setzte ihm eine falsche Erinnerung ein. Die Ereignisse, die daraufhin folgten, konnten auch von Dumbledore nur vermutet werden. Riddle nahm Morfins Zauberstab, ging zum Haus der Riddles, tötete seinen Vater, seinen Großvater und seine Großmutter, brachte den Zauberstab zurück zu Morfin, raubte den Ring von Vorlost Gaunt, verwandelte ihn in einen Horkrux, versteckte ihn lange Zeit später im Haus der Gaunts in einer goldenen Schachtel unter Dielenbrettern und verließ das Dorf.
\vspace{10pt}
\newline
\marginpar{16}  
Aufgrund seiner eingepflanzten Erinnerung gestand Morfin, der als Muggelhasser bekannt war, den Mord an den Riddles. Er wurde nach Askaban gebracht, wo er kurz vor seinem Tod von Dumbledore besucht wurde und ihm die Erinnerung an Tom Riddle überließ.

\subsubsection*{\large Beschäftigung nach Hogwarts}
\marginpar{17} 
Tom Riddle hatte sich in seinen Jahren in Hogwarts einen Ruf als genialer Zauberer gemacht. Alle rechneten damit, dass er eine Arbeit beim Zaubereiministerium annehmen und irgendwann Zaubereiminister werden würde. Doch entgegen aller Erwartung begann er bei Borgin und Burkes in der Nokturngasse zu arbeiten. Diese Arbeit war allerdings nur zweite Priorität für ihn, denn nach seinem Schulabschluss fragte er Armando Dippet, ob er in Hogwarts bleiben und Verteidigung gegen die Dunklen Künste unterrichten könne. Dippet antwortete ihm aber, er sei noch zu jung und solle in ein paar Jahren wiederkommen.
\vspace{10pt}
\newline
\marginpar{18}  
Bei seiner Arbeit für Borgin und Burkes konnte er eines seiner größten Talente einsetzen: Leuten zu schmeicheln. Er übernahm die Aufträge seines Chefs, die besondere Überredungskunst beanspruchten. Auf diesem Wege geriet Tom auch an Hepzibah Smith, die ihn sehr schätzte und ihm sogar so stark vertraute, dass sie ihm ihre größten Schätze zeigte: einen Kelch von Helga Hufflepuff und das Medaillon von Salazar Slytherin. Riddle erkannte den ungeheuren materiellen, aber vor allem auch den hohen persönlichen Wert dieser Gegenstände. Als Harry Potter diese Erinnerung in Dumbledores Denkarium sah, erkannte er deutlich “ein rotes Blitzen in Voldemorts Augen“. Zwei Tage, nachdem Riddle Hepzibah Smith besucht hatte, war sie tot, angeblich von ihrer Hauselfe Hokey vergiftet. Dies dachten die Ministeriumsbeamten, da Hokey sich daran erinnerte, etwas in den Kakao ihrer Herrin gemischt zu haben. Es wurde allgemein vermutet, dass die alte und verwirrte Hauselfe Gift mit Zucker verwechselt hatte. Doch laut Dumbledore gleicht dieser Mord zu sehr demjenigen an den Riddles. Offensichtlich hatte Riddle auch hier jemandem eine Erinnerung eingepflanzt, um seine Schuld zu vertuschen, diesmal traf es Hokey. Nach dem Mord fehlten jedenfalls die beiden wertvollen Gegenstände in Hepzibahs Haus. Die Verwandten Hepzibahs dachten, die senile Frau habe das Medaillon und den Becher irgendwo versteckt und mit vielen Schutzzaubern umgeben.
\vspace{10pt}
\newline
\marginpar{19}  
Nach dem Mord und dem Diebstahl verwandelte Riddle diese Gegenstände in Horkruxe.

\subsubsection*{\large Verwandlung zum Dunklen Lord}
\marginpar{20} 
Nach diesen Ereignissen verschwand Tom Riddle für lange Zeit von der Bildfläche. Er benutzte auch den Namen seines Vaters nicht mehr, er nannte sich nur noch Lord Voldemort. Es kann vermutet werden, dass Voldemort in diesen Jahren auf der Suche nach Rowena Ravenclaws Diadem war, denn die Graue Dame, Rowena Ravenclaws Tochter, hatte ihm in Hogwarts verraten, dass sie das Diadem in Albanien versteckt hatte.
\vspace{10pt}
\newline
\marginpar{21}  
Nachdem er das Diadem gefunden hatte, verwandelte er es ebenfalls in einen Horkrux und kehrte nach Hogwarts zurück, um sich noch einmal für die Stelle als Lehrer für Verteidigung gegen die Dunklen Künste zu bewerben. Albus Dumbledore, der in der Zwischenzeit Schulleiter geworden war, verweigerte ihm die Stelle und wies darauf hin, dass Riddle inzwischen dreimal der Posten als Zaubereiminister angeboten wurde. Voldemort wollte auch gar nicht Lehrer werden, sondern brauchte nur einen Vorwand, um nach Hogwarts zu kommen: Er wollte das Diadem im Raum der Wünsche verstecken, was er schließlich auch tat, ein Versteck, von dem Lord Voldemort glaubte, dass nur er von diesem Ort Kenntnis hatte. Doch von diesem Tag an war die Stelle des Lehrers für Verteidigung gegen die Dunklen Künste verhext und kein Lehrer schaffte es, länger als ein Jahr im Amt zu bleiben. Dieser Zauber löste sich erst mit Voldemorts Tod auf.
\vspace{10pt}
\newline
\marginpar{22}  
Um diese Zeit begann Voldemort auch damit, seine "Freunde" aus Hogwarts um sich zu scharen, viele von diesen Leuten wurden die ersten Todesser.
%SEITE2
\vspace{10pt}
\newline
\marginpar{23}  
In dieser Zeit verlor Voldemort auch sein gutes Aussehen. Das Auseinanderreißen seiner Seele verschaffte ihm zwar einige Sicherheit gegenüber dem Tod, es setzte seinem Körper und seiner Seele aber auch stark zu. Er wurde immer weniger menschlich.

\subsubsection*{\large Aufstieg des Dunklen Lords}
\marginpar{24} 
Voldemort scharte seine treuesten Anhänger in den 70er Jahren um sich und begann mit ihnen die Jagd auf Schlammblüter.
\vspace{10pt}
\newline
\marginpar{25}  
In dieser Zeit spielte er auch einmal mehr seine große Stärke, die Überredungskunst, aus und gewann die Wesen für sich, die von der Gesellschaft zu einem Leben im Verborgenen gezwungen wurden. Riesen und Vampire schlossen sich ihm an. Auch sehr wichtig für Voldemort wurden die Werwölfe, angeführt von Fenrir Greyback, einem der fanatischsten Anhänger Voldemorts. Es wurde auch befürchtet, die Kobolde würden sich Voldemort anschließen, doch er schaffte es nie, die Hüter Gringotts für sich zu gewinnen, da diese stolz darauf waren, unabhängig zu denken und sich niemandem unterzuordnen. Wohl am verheerendsten für die Zauberergesellschaft war, dass Voldemort es schaffte, die Dementoren dazu zu bringen, sich ihm anzuschließen.
\vspace{10pt}
\newline
\marginpar{26}  
Während dieser Zeit herrschte Verwirrung und Panik in der Zaubererwelt. Niemand traute mehr seinem Gegenüber und das Ministerium sperrte viele Unschuldige nach Askaban, auch, um den Zauberern zu beweisen, dass sie die Lage im Griff haben.
\vspace{10pt}
\newline
\marginpar{27}  
Der einzige wirklich sichere Platz war Hogwarts, wo Albus Dumbledore, der einzige Zauberer, den Voldemort wirklich fürchtete, die Schüler beschützte. Zu dieser Zeit rief Dumbledore auch den Orden des Phönix ins Leben, um Voldemort die Stirn zu bieten. Dem Orden schlossen sich unter anderem viele Auroren aus dem Ministerium an.
\vspace{10pt}
\newline
\marginpar{28}  
1979 war Voldemort auf der Höhe seiner Macht. Er genoss die Unterstützung vieler Hexen und Zauberer. Die einen wollten Schutz vor dem Bösen, die anderen wollten etwas von Voldemorts Ruhm abhaben, und nur die wenigsten glaubten voll und ganz an Voldemorts Ideale (letztere waren der engste Kreis der Todesser des Dunklen Lords). Diese verschiedenen Beweggründe, sich den Todessern anzuschließen, waren Voldemort egal, es lief alles darauf hinaus, dass er fast uneingeschränkte Macht hatte.

\subsubsection*{\large Fall}
\marginpar{29} 
Einer seiner treuesten Todesser, Severus Snape, belauschte in Hogsmeade im Eberkopf Sybill Trelawney, die sich bei Albus Dumbledore als Lehrerin für das Fach Wahrsagen bewarb, hörte allerdings nur den ersten Teil der Prophezeiung, bevor er vom Barmann des Eberkopfs, Aberforth Dumbledore, hinausgeworfen wurde. Der zukünftige Zaubertränkelehrer überbrachte daraufhin die Nachricht an seinen Herren, dass es eine Prophezeiung über Voldemort und einen Jungen, der seine Macht schließlich brechen sollte, gab.
\vspace{10pt}
\newline
\marginpar{30}  
Die Prophezeiung lautete folgendermassen:
\vspace{10pt}
\newline
\marginpar{31}  
„Der Eine mit der Macht, den Dunklen Lord zu besiegen, naht heran...jenen geboren, die ihm drei Mal die Stirn geboten haben, geboren, wenn der siebte Monat stirbt...und der Dunkle Lord wird Ihn als sich Ebenbürtigen kennzeichnen, aber Er wird eine Macht besitzen, die der Dunkle Lord nicht kennt...und der Eine muss von der Hand des Anderen sterben, denn keiner kann leben, während der Andere überlebt...der Eine mit der Macht, den Dunklen Lord zu besiegen, wird geboren werden, wenn der siebte Monat stirbt...“
\vspace{10pt}
\newline
\marginpar{32}  
— Sybill Trelawney
\vspace{10pt}
\newline
\marginpar{33}  
Die Prophezeiung traf auf zwei Jungen zu, Harry Potter, ein Halbblut und Neville Longbottom, ein Reinblut. Da Voldemort diese "Gefahr" ausschalten wollte, entschied er sich für den Jungen, dessen Mutter muggelstämmig war, Harry Potter. Dies, laut Dumbledore, vielleicht auch, da Voldemort Parallelen zwischen sich und Harry Potter erkannte.
\vspace{10pt}
\newline
\marginpar{34}  
Lord Voldemort machte sich auf nach Godric's Hollow, um die Potters zu töten. Der Wohnort der Potters wurde ihm von Peter Pettigrew verraten, dem Geheimniswahrer der Potters. Voldemort tötete James Potter mit dem Todesfluch und anschließend traf er auf Lily Potter und ihren Sohn, Harry. Severus Snape liebte Lily schon seit seiner Kindheit, deshalb hatte er Voldemort darum gebeten, sie zu verschonen. Voldemort hätte dies auch angeblich getan, wenn sie ihm aus dem Weg gegangen wäre, doch sie schützte ihren Sohn, indem sie sich für ihn opfern wollte, und wurde deshalb von Voldemort ebenfalls umgebracht.
\vspace{10pt}
\newline
\marginpar{35}  
Da Lily Potter für ihren Sohn starb, erschuf sie einen Schutz für Harry, ein Kraftfeld aus Liebe. Als Voldemort Harry mit dem Todesfluch töten wollte, prallte dieser ab und traf Voldemort selbst, dessen Kraft sofort schwand. Der Körper von Voldemort war zerstört, doch da er bis dahin fünf Horkruxe erschaffen hatte, konnte er nicht sterben. Während er versuchte, Harry umzubringen, erschuf er, ohne es zu merken, einen sechsten Horkrux: Aufgrund der Kraft des abgeprallten Todesfluchs spaltete sich seine Seele noch einmal und dieses Seelenteil ruhte von nun an in Harry.
\vspace{10pt}
\newline
\marginpar{36}  
Bereits vor den Morden an den Potters verlor Voldemort allerdings einen treuen Todesser. Severus Snape stellte sich von Beginn der Verfolgung der Potters durch Voldemort gegen den Dunklen Lord und bot sich Dumbledore als Doppelagent an.
%SEITE3
\vspace{10pt}
\newline
\marginpar{37}  
Nach dem Fall Voldemorts kamen viele Todesser zur Zauberergesellschaft zurück, doch die meisten von ihnen wurden verhaftet und nach Askaban gebracht. Einige bereuten ihre Taten tatsächlich, einige täuschten die Reue nur vor, einige behaupteten, sie hätten Voldemort nur wegen des Imperius-Fluchs unterstützt; einige Todesser blieben ihrem Herrn aber dennoch treu, so etwa Bellatrix Lestrange.
\subsubsection*{\large Leben im Exil}
\marginpar{38} 
Nach dem gescheiterten Mord an Harry Potter verschwand Voldemort. Da der Großteil der Zaubererwelt, Todesser mit eingeschlossen, dachte, er wäre tot, war der körperlose Voldemort gezwungen, sich in Albanien zu verstecken, wo er durchgehend von Kleintieren u.ä. Besitz ergriff.
\subsubsection*{\large Rückkehrversuch}
\marginpar{39} 
Der körperlose Voldemort fand schließlich im jungen Quirinus Quirrell einen willigen Diener. Quirrell scheiterte nur knapp daran, den Stein der Weisen aus Gringotts Zaubererbank zu stehlen, der seinen Herrn unsterblich gemacht hätte. Nach diesem Versagen seines Dieners schlüpfte Voldemort in Quirrells Körper. Er lebte fortan an Quirrells Hinterkopf, verdeckt von einem Turban. Um Voldemort zu stärken, trank Quirrell das Blut von Einhörnern. Als Quirrell schließlich erneut versuchte, den Stein der Weisen zu stehlen, wurde er von dem 11-jährigen Harry Potter daran gehindert. Voldemort verließ Quirrells Körper und überließ ihn dem Tod, da Quirrell den Schutz von Harrys Mutter nicht durchdringen konnte.
\vspace{10pt}
\newline
\marginpar{40}  
Nach diesem gescheiterten Versuch, unsterblich zu werden, verschwand Voldemort wieder. Er floh, so schwach, wie er es noch nie war, erneut nach Albanien, um auf einen Diener zu warten.
\subsubsection*{\large Wiederöffnung der Kammer des Schreckens}
\marginpar{41} 
Das Tagebuch des jungen Lord Voldemort, ein Horkrux, befand sich für lange Zeit im Besitz von Lucius Malfoy. Dieser war sich der Bedeutung des Tagebuchs aber nicht bewusst und steckte es am Anfang von Harry Potter und die Kammer des Schreckens Ginny Weasley zu.
\vspace{10pt}
\newline
\marginpar{42}  
Ginny nahm das Tagebuch mit nach Hogwarts und begann, darin zu schreiben. Darin eingesperrt war das Seelenteil des 16-jährigen Tom Riddle. Er antwortete ihr und gewann immer mehr die Kontrolle über Ginny. Besessen von Riddle öffnete sie die Kammer des Schreckens und ließ das Monster auf Muggelstämmige los.
\vspace{10pt}
\newline
\marginpar{43}  
Als Harry Potter schließlich in die Kammer des Schreckens vorstieß, traf er auf Tom Riddles Seele, nicht wissend, dass Riddle und Voldemort ein und dieselbe Person sind. Als ihm dieses Geheimnis schließlich offenbart wurde, hetzte Tom Riddles Seelenteil den Basilisken auf Harry, der ihn in den Arm biss, doch Harry hatte das Schwert von Gryffindor aus dem Sprechenden Hut gezogen, der ihm vom Phönix Fawkes gebracht wurde, womit er den Basilisken tötete. Mit dem Zahn des Basilisken, der in Harrys Arm steckte, stach er in Riddles Tagebuch und zerstörte unbewusst einen Horkrux. Die Tränen von Fawkes heilten Harrys Verletzung schließlich und bewahrten Harry vor dem Tod.
\vspace{10pt}
\newline
\marginpar{44}  
Das Seelenteil von Voldemort hatte bei diesen Ereignissen selbstständig gehandelt, Voldemorts körperloser Geist hatte keinerlei Einfluss auf die Geschehnisse in Hogwarts.
\subsubsection*{\large Wiedergeburt}
\marginpar{45} 
Einer der treuesten Diener von Voldemort, Peter Pettigrew, wurde lange für tot gehalten. Die gesamte Zaubererwelt verdächtigte Sirius Black, den schusseligen Zauberer umgebracht zu haben, die Potters verraten zu haben und ein Todesser zu sein. Doch alles war genau umgekehrt. Pettigrew hatte die Potters verraten und er war ein Todesser. Als er von Sirius Black in die Enge getrieben wurde, schnitt er sich einen Finger ab, beschwor eine Explosion, die einige Muggel tötete, herauf und verwandelte sich in eine Ratte, denn Pettigrew war ein nicht registrierter Animagus. Er lebte seither als Hausratte Krätze bei den Weasleys.
\vspace{10pt}
\newline
\marginpar{46}  
Am Ende von des Schuljahres 1993/1994 verwandelte Peter Pettigrew sich zurück in eine Ratte und floh, um Lord Voldemort zu finden und zu dienen.
\vspace{10pt}
\newline
\marginpar{47}  
Pettigrew sorgte für Voldemort und gab ihm Milch von Voldemorts Schlange, um ihn zu stärken und ihm eine Art Körper zu geben. Voldemort wohnte mit Pettigrew im Haus seines Vaters in Little Hangleton. Als am Anfang von Harry Potter und der Feuerkelch ein Muggel namens Frank Bryce Voldemort und Pettigrew entdeckte, wurde er von Voldemort mit den Todesfluch umgebracht und Nagini wurde der siebte Horkrux.
\vspace{10pt}
\newline
\marginpar{48}  
Während des Trimagischen Turniers befand sich ein Todesser in Hogwarts. Barty Crouch jr. hatte sich mittels Vielsafttrank in Alastor "Mad-Eye" Moody verwandelt und war als Lehrer für Verteidigung gegen die Dunklen Künste tätig. Er sorgte dafür, dass Harry Potter das Turnier gewann und mit Hilfe des Pokals, der in einen Portschlüssel verwandelt worden war, zum Friedhof in 
%SEITE4
Little Hangleton gebracht wurde. Harry und Cedric Diggory hatten den Pokal beide gleichzeitig angefasst und kamen beide auf den Friedhof. Peter Pettigrew tötete auf Voldemorts Befehl den "Überflüssigen", nämlich Cedric.
\vspace{10pt}
\newline
\marginpar{49}  
Pettigrew tauchte die schleimige, hässliche babyhafte Gestalt Voldemorts in das Gebräu eines mannsgroßen Kessels, danach ergänzte Pettigrew das Gebräu, welches Voldemort einen Körper verschaffen sollte, mit den folgenden Zutaten:
\vspace{10pt}
\newline
\marginpar{50}  
"Knochen des Vaters, unwissentlich gegeben, du wirst deinen Sohn erneuern!"
Die erste Zutat waren die Knochen in Form eines schmalen Staubwirbels von Tom Riddle sr..
\vspace{10pt}
\newline
\marginpar{51}  
"Fleisch des Dieners, willentlich gegeben, du wirst deinen Meister wieder beleben!"
Die zweite Zutat war die rechte Hand mit dem fehlenden Finger von Peter Pettigrew. Nach dem Ritual gab Voldemort Pettigrew eine neue, silberne Hand.
\vspace{10pt}
\newline
\marginpar{52}  
"Blut des Feindes, mit Gewalt genommen, du wirst deinen Gegner wieder erstarken lassen!"
Die letzte Zutat war Blut von Harry Potter.
\vspace{10pt}
\newline
\marginpar{53}  
Nach einer Weile quoll weißer Dampf in dicken Schwaden, die Sicht behindernd, aus dem Kessel und heraus trat der wieder erstarkte Lord Voldemort.
\vspace{10pt}
\newline
\marginpar{54}  
Anschließend bewunderte er seinen Körper und rief die Todesser zu sich. Die treuesten von ihnen kamen und küssten seinen Umhang .Einige blieben verschwunden, Voldemort ließ die Fehlenden jagen und umbringen.
\vspace{10pt}
\newline
\marginpar{55}  
Auf dem Friedhof wollte Voldemort Harry Potter umbringen, doch er gab ihm die Chance, sich zu duellieren. Voldemort benutzte den Todesfluch, Harry den Entwaffnungszauber, Expelliarmus. Doch da Harry Potter und Lord Voldemorts Zauberstäbe den gleichen Kern hatten (jeder mit der Feder des gleichen Phönix, nämlich Fawkes), prallten die Flüche aufeinander und es entstand Priori Incantatem, der Fluchumkehreffekt. Die letzten Morde des Zauberstabs von Voldemort erschienen, Schattenbilder der Personen — darunter auch Harrys Eltern — die von Voldemort getötet wurden und lenkten ihn ab. Das Schattenbild von Cedric Diggory bat Harry darum, seine Leiche mitzunehmen. Harry schnappte sich Cedrics Leiche und gelangte mit dem Portschlüssel zurück nach Hogwarts. Dort versuchte er, allen zu sagen, dass Voldemort zurück war, doch niemand außer Dumbledore glaubte ihm. Dies war ein Vorteil für Voldemort, der nun im Geheimen arbeiten konnte. Der falsche Mad-Eye Moody nahm Harry gleich, als er wieder in Hogworts angekommen war, gefangen und befragte ihn über Voldemorts Auferstehung. Dann wollte Mad-Eye Harry töten, weil es Voldemort ja nicht geschaft hatte, doch zum Glück retteten Dumbledore und noch ein paar andere Lehrkräfte Harry und der falsche Moody verlor seine Seele durch einen Dementorenkuss.
\subsubsection*{\large Suche nach der Prophezeiung}
\marginpar{56} 
Fast niemand glaubte Harry, dass Voldemort zurück sei. Besonders das Ministerium distanzierte sich von dem Gedanken, dass die größte Gefahr für alle Zauberer wieder anwesend sein sollte. Voldemort suchte nach seiner Wiedergeburt nach der Prophezeiung, die Severus Snape nur unvollständig mitbekommen hatte. Voldemort war sich bewusst, dass sich die Kristallkugel mit der Prophezeiung in der Mysteriumsabteilung im Zaubereiministerium befand.
\vspace{10pt}
\newline
\marginpar{57}  
Voldemort organisierte nach seiner Wiederauferstehung unter anderem einen Massenausbruch aus Askaban, bei welchem auch Bellatrix Lestrange aus dem Gefängnis entkam.
\vspace{10pt}
\newline
\marginpar{58}  
in Harry Potters fünften Schuljahr begann Harry zu merken, dass zwischen ihm und Voldemort eine geistige Verbindung bestand und dass er Voldemorts Gedanken mitbekommen konnte. So rettete Harry Arthur Weasley, der für den Orden des Phönix Nachtwache hielt, als dieser von der Schlange Nagini angegriffen und lebensgefährlich verletzt wurde. Nagini war vom Geist Voldemorts besessen, so konnte Harry den Angriff aus der Sicht der Schlange mitbekommen . Als Voldemort die Verbindung entdeckte, ließ er Harry einen gefälschten Gedanken sehen, nämlich, dass er Harrys Patenonkel Sirius folterte. Harry reiste daraufhin mit Ronald Weasley, Hermine Granger, Luna Lovegood, Ginny Weasley und Neville Longbottom ins Ministerium, um Sirius zu retten. In der Mysteriumsabteilung fanden sie die Prophezeiung. Die Freunde wurden von Todessern überrascht, die sie zur Herausgabe der Prophezeiung zwangen. Doch Neville Longbottom ließ die Prophezeiung versehentlich fallen, so das keiner sie zu hören bekam.
\vspace{10pt}
\newline
\marginpar{59}  
Einige Ordensmitglieder kamen, um zu kämpfen, dabei wurde Sirius Black von Bellatrix Lestrange getötet. Als Harry die Mörderin seines Paten, Bellatrix Lestrange, jagte, erschien Voldemort, um Harry zu töten, doch zu Harrys Glück erschien Albus Dumbledore. Dieser lieferte sich mit Lord Voldemort einen sehr heftigen Kampf. Schließlich ergriff Voldemort von Harry Besitz. Aber er musste aus Harrys Körper flüchten, weil er erkannte, dass es ihm höllische Schmerzen bereitete, in einem Körper zu sein, der erfüllt war von dem Gefühl, welches Voldemort am meisten verabscheute, nämlich Liebe. Er floh aus dem Ministerium, doch der Zaubereiminister, Cornelius Fudge, sah Voldemort verschwinden und wusste, dass Harry Potter und Dumbledore Recht damit hatten, dass der Dunkle Lord zurückgekehrt sei.
\vspace{10pt}
\newline
\marginpar{60}  
Viele der Todesser, die im Ministerium gegen die Auroren gekämpft hatten, wurden festgenommen und nach Askaban gebracht, darunter auch Lucius Malfoy.
%SEITE5
\subsubsection*{\large Zweiter Zaubererkrieg}
\marginpar{61} 
Nach den Ereignissen im Zaubereiministerium war allen klar, dass Voldemort endgültig zurückgekehrt war. Es herrschte Panik und vieles war so, wie beim ersten Krieg gegen Voldemort: Leute verschwanden oder wurden tot aufgefunden, es gab Dementorenangriffe, kleine Kinder wurden von Werwölfen, die von Fenrir Greyback angestachelt wurden, angegriffen und niemand wusste, wem man trauen konnte, da jeder ein Todesser hätte sein können oder unter dem Imperius-Fluch hätte stehen können.
\vspace{10pt}
\newline
\marginpar{62}  
1996 erteilte Voldemort Draco Malfoy die Aufgabe, einen Weg zu finden, wie man unbemerkt Todesser nach Hogwarts bringen könnte und Dumbledore zu töten. Malfoy schaffte es zwar dank des Verschwindekabinetts im Raum der Wünsche, Todesser via Verschwindekabinett bei Borgin und Burkes nach Hogwarts zu bringen, doch seine zwei Versuche, Dumbledore zu töten, scheiterten. Schließlich tötete Severus Snape Dumbledore - Snape hatte nach Aufforderung durch Bellatrix Lestrange Dracos Mutter Narzissa den Unbrechbaren Schwur geschworen, im Falle von Dracos Versagen die Aufgabe zu übernehmen, die der Dunkle Lord eigentlich Draco zugedacht hatte.
\vspace{10pt}
\newline
\marginpar{63}  
Nach diesem Kampf begann auch Harry Potters Jagd nach den verbliebenen Horkruxen. Zwei wurden bereits zerstört: Der Horkrux-Ring und Tom Riddles Tagebuch in Harrys zweitem Schuljahr, und zwar wurde das Tagebuch von Harry selbst zerstört, in dem er einen giftigen Basiliskenzahn in das Tagebuch hineinstieß.
\vspace{10pt}
\newline
\marginpar{64}  
Im Sommer des Jahres 1997 nahm Voldemort Charity Burbage, die Lehrerin für Muggelkunde in Hogwarts, gefangen, weil sie einen Artikel im Tagespropheten veröffentlichte, in dem sie forderte, dass Muggel und Zauberer gleichgestellt werden sollten. Sie wurde von Voldemort mit dem Todesfluch getötet und an Nagini verfüttert. Außerdem versuchte Voldemort, aus dem gefangenen Zauberstabmacher Garrick Ollivander herauszubekommen, was er tun müsste, um die Verbindung zwischen seinem Zauberstab und dem von Harry Potter zu überwinden. Auf Ollivanders Rat hin nahm Voldemort den Zauberstab von Lucius Malfoy an sich.
\vspace{10pt}
\newline
\marginpar{65}  
Das Hauptquartier der Todesser war von nun an im Haus der Malfoys. Von seinem Informanten Snape erfuhr Voldemort, dass Harry Potter einige Tage vor seinem siebzehnten Geburtstag aus dem Ligusterweg fliehen würde (der Schutz im Ligusterweg hob sich selber am Tag von Harrys siebzehnten Geburtstag auf). Voldemort und einige Todesser passten die Auroren und die Mitglieder des Ordens des Phönix ab und griffen sie an. Bei diesem Kampf starben Harrys Eule Hedwig und Alastor "Mad-Eye" Moody und George Weasley verlor ein Ohr. Als Voldemort Harry umbringen wollte, stießen Harry und Hagrid gerade durch den Schutzzauber des Hauses von Ted und Andromeda Tonks, den Voldemort nicht durchdringen konnte.
\vspace{10pt}
\newline
\marginpar{66}  
Während des Kampfes reagierte Harrys Zauberstab (wie beim Priori Incantatem) erneut auf Voldemort. Dieser folterte Garrick Ollivander, um zu erfahren, warum der Zauberstab von Harry Potter immer noch auf ihn reagierte. Ollivander sagte Voldemort, wahrheitsgemäß, dass er es nicht wüsste und erzählte ihm unter Folter alles, was er über den Elderstab wusste. Voldemort wusste - ohne die leiseste Ahnung, dass es sich beim legendären Zauberstab um ein Heiligtum des Todes handelte -, dass er den Zauberstab an sich bringen müsste, um unbesiegbar zu werden.
\vspace{10pt}
\newline
\marginpar{67}  
Bei der Hochzeit von Bill Weasley und Fleur Delacour berichtete der Patronus von Kingsley Shacklebolt, dass das Zaubereiministerium gefallen und der Zaubereiminister, Rufus Scrimgeour, tot sei und dass die Todesser auf dem Weg zum Fuchsbau seien.
\vspace{10pt}
\newline
\marginpar{68}  
Neuer Zaubereiminister wurde Pius Thicknesse, der unter dem Imperius-Fluch stand. Doch de facto war nun Voldemort Minister, denn Thicknesse stand unter seinem Kommando.
\vspace{10pt}
\newline
\marginpar{69}  
In der Zwischenzeit machte sich Voldmort auf nach Bulgarien, um den Zauberstabmacher Gregorowitsch aufzusuchen, denn Voldemort versprach sich Informationen über den Elderstab (Elder Wand). Gregorowitsch klärte ihn darüber auf, dass er gestohlen wurde, daraufhin setzte Voldemort Legilimentik ein und sah in den Gedanken des Zauberstabmachers einen jungen, blonden Mann, der den Zauberstab in der Hand hielt und im Dunkeln verschwand. Als Gregorowitsch Voldemort nicht sagen konnte, wer der Jüngling war, brachte Voldemort ihn kurzerhand um.
\vspace{10pt}
\newline
\marginpar{70}  
In Godric's Hollow lauerte im Körper der toten Bathilda Bagshot Voldemorts Schlange Nagini und wartete darauf, dass Harry kam, um sie zu besuchen. Er und Hermine kamen an Weihnachten zu ihr ins Haus und Nagini lotste Harry mittels Gesten auf den Dachboden, wo sie anfing, Parsel zu reden. Nach kurzer Zeit schoss Nagini aus der Leiche heraus und biss Harry, um ihn festzuhalten. Doch Hermine stürmte rechtzeitig nach oben, lenkte die Schlange und den ankommenden Voldemort mit einer Explosion ab und floh mit Harry. Voldemort konnte die beiden nicht töten, doch er erkannte auf einem am Boden liegendem Foto den blonden Jüngling aus Gregorowitchs Gedächtnis, der den Elderstab stahl. Er wusste nun, dass es sich um Gellert Grindelwald handelte. Grindelwald war nach Vodlemort der größte Schwarze Magier aller Zeiten und wurde 1945 von Albus Dumbledore besiegt.
\vspace{10pt}
\newline
\marginpar{71}  
Voldemort reiste zum Gefängnis Nurmengard, wo Gellert Grindelwald einsaß und befragte ihn über den Elderstab. Als Grindelwald sich aber weigerte, Informationen preiszugeben, tötete er ihn. Er erkannte, dass Dumbledore der wahre Meister des Zauberstabs war und dass er ihn aus dessen Grab holen musste. Voldemort brach Dumbledores Grab auf und nahm den Zauberstab
%SEITE6
an sich.
\vspace{10pt}
\newline
\marginpar{72}  
Als Voldemort berichtet wurde, dass jemand in Bellatrix Lestranges Verlies in Gringotts eingebrochen war und  dass der Becher Helga Hufflepuffs (ein weiterer Horkrux) mitgenommen wurde, begann er zu merken, dass seine Horkruxe gejagt wurden. Er reiste zu den Verstecken seiner Horkruxe und stellte fest, dass einige von ihnen weg waren. Daraufhin machte er sich auf den Weg nach Hogwarts, um zu sehen, ob Rowena Ravenclaws Diadem noch sicher im Raum der Wünsche war. Doch während er die anderen Verstecke aufgesucht hatte, fanden Harry Potter und seine Freunde das Diadem und zerstörten es. Nun verblieben noch zwei Horkruxe: Nagini und Harry Potter selbst Zu diesem Zeitpunkt wusste allerdings nur Severus Snape, dass Harry auch ein Horkrux war.
\vspace{10pt}
\newline
\marginpar{73}  
„Und sein Name ist Voldemort, sie können ihn ruhig so nennen, er wird versuchen Sie zu töten, egal wie sie Ihn nennen.“
\vspace{10pt}
\newline
\marginpar{74}  
— Minerva McGonagall vor der Schlacht von Hogwarts
\vspace{10pt}
\newline
\marginpar{75}  
Als Voldemort nach Hogwarts reiste, wurde ihm bewusst, dass Harry Potter anwesend und ganz Hogwarts zum Kämpfen bereit war. Er berief eine Armee von Todessern, Dementoren, Riesen und Acromantulas ein, um gegen die Armee von Hogwarts (Auroren, Lehrer, Schüler) zu kämpfen. Es kam zur großen Schlacht von Hogwarts. Doch Voldemort nahm nicht an der Schlacht teil. Er hielt sich in der Heulenden Hütte auf und unterhielt sich mit Severus Snape über den Elderstab. Da Snape Dumbledore umgebracht hatte, so schloss Voldemort, wäre der Zauberstab Snape treu, deshalb müsse er getötet werden. Voldemort hetzte Nagini auf Snape, der vor seinem Tod den heimlich dieses Gespräch belauschenden Harry Potter noch eine wichtige Erinnerung anvertraute. In dieser Erinnerung war die Information, dass Harry ein Horkrux sei und zerstört werden müsse. Voldemort befahl mit magisch verstärkter Stimme inzwischen einen Waffenstillstand, der eine Stunde andauern sollte. Er sagte an Harry Potter gerichtet, er würde eine Stunde lang im Verbotenen Wald warten, dass Harry Potter zu ihm kommen und sich stellen müsse. Harry, der sich mit seinem Schicksal abgefunden hatte, stellte sich Voldemort, der ihn darufhin mit dem Todesfluch umbrachte. Doch der Elderstab gehorchte nur Harry Potter und tötete ihn nicht. Voldemort ließ Harry als Zeichen seines vermeintlichen Sieges durch den gefangenen Rubeus Hagrid zum Hogwarts-Schloss tragen. Er verkündete, dass jeder, der ihm jetzt noch die Stirn bieten würde, getötet werden würde. Neville Longbottom stürmte hervor und die Schlacht begann erneut. Neville zog das Schwert von Gryffindor aus dem Sprechenden Hut und köpfte Nagini, so wie es ihm von Harry aufgetragen worden war. Der letzte Horkrux war zerstört. Voldemort versuchte nun, Neville zu töten, doch Harry Potter, der unter dem Tarnumhang steckte, stellte sich zwischen die beiden und erzeugte einen Schildzauber. Voldemort begann jetzt auch zu kämpfen. Er kämpfte gegen Horace Slughorn, Kingsley Shacklebolt und Professor McGonagall gleichzeitig. Doch als er sah, wie Bellatrix Lestrange von Molly Weasley getötet wurde, wollte er diese umbringen. Harry stellte sich vor Voldemort und riss sich den Tarnumhang vom Leib, um allen zu zeigen, dass er lebte. Er verkündete, dass er allein mit Voldemort kämpfen müsse und dass ihm niemand helfen solle.
\vspace{10pt}
\newline
\marginpar{76}  
Harry Potter offenbarte Voldemort, dass Severus Snape immer auf der Seite von Albus Dumbledore gewesen war und dass der Elderstab Draco Malfoy gehorchte, aber da Harry Draco seinen Zauberstab entwendet hatte, gehörte der Elderstab nun nur noch Harry. Voldemort wurde provoziert von Harry, da dieser ihn "Tom Riddle" nannte, was Voldemort nicht ausstehen konnte.
\vspace{10pt}
\newline
\marginpar{77}  
Voldemort und Harry Potter begannen also das ungleiche Duell, Voldemort benutzte einen Todesfluch, Harry einen Entwaffnungszauber. Die Zauberlichter rot und grün prallten aufeinander und der Todesfluch aus dem Zauberstab, der eigentlich Harry gehörte, flog zurück zu Voldemort, der daran starb. Da Harry keinen Todesfluch benutzt hatte, könnte man sagen, dass Voldemort sich selbst umgebracht hatte.

\subsection*{\Large Etymologie}
\marginpar{78} 
Der Name Voldemort kann in drei Teile geteilt werden: Vol-de-mort. Auf Deutsch würde das "vom Tod entflohen" heißen.
\vspace{10pt}
\newline
\marginpar{79}  
Sein Irrwicht ist sein eigener Tod. In einem Interview sagte Rowling auch, dass sie Voldemort zwar ursprünglich gern französisch aussprach, es dazu allerdings keinen Bezug gäbe.

\subsection*{\Large Magische Fähigkeiten}
\marginpar{80} 
\paragraph{Duell Künste:}  
\marginpar{81} 
Tom war ein sehr mächtiger und starker Duellant, legte jedoch mehr Wert auf einen starken Angriff und weniger auf eine sichere Deffensive; er nutze schwarze Magie, um seinen Gegner schnell zu besiegen und meistens auch anschließend zu töten. Sein beliebtester Zauber war der Todesfluch: Avada Kedavra. Außerdem konnte er mit Albus Dumbledore mithalten, der als der mächtigster Zauberer galt.
\paragraph{Nonverbale Magie:}  
\marginpar{82} 
Tom konnte auch nonverbal zaubern, wie man im 5. Teil bei dem Duell von Albus und Tom im Ministerium lesen kann, als er den Todesfluch auch ohne ein Wort nutzte.
%SEITE7
\paragraph{Dunkle Künste:} 
\marginpar{83} 
Tom war ein Profi der Schwarzen Magie Er nutzte alle Unverzeilichen Flüche (Todesfluch, Folterfluch und den Imperius-Fluch), das Dämonenfeuer, das er auch kontrollieren konnte, er hatte auch sechs beziehungsweise sieben Horkruxe erstellt: 1. sein Tagebuch 2. Der Erbring seiner Familie Gaunt, 3. Der Trinkpokal von Hufflepuff, 4. Das Medallion von Slytherin 5. Das Ravenclaw Diadem 6. Die Schlange Nagini und 7. Harry Potter.
\paragraph{Sprache:}  
\marginpar{84} 
Tom war in der Lage Parsel (Schlangensprache) zu sprechen, was eine seltene Fähigkeit ist, so konnte er den Basilisken in der Kammer des Schreckens Befehle erteilen.
\paragraph{Besenloses Fliegen:} 
\marginpar{85} 
Tom konnte ohne Besen fliegen, womit er einen Vorteil gegenüber anderen Zauberern hatte, die einen Besen hatten, da dieser zerstört werden kann.

\subsection*{\Large Trivia}
\marginpar{86} 
Voldemort hat einen Auftritt in The LEGO Batman Movie, wo er vom Joker aus dem Weltraumgefängnis Phantom Zone befreit wird, um ihm bei der Zerstörung von Gotham zu helfen.
In der französischen Version von Avengers: Infinity War nennt Tony Stark Ebony Maw bei ihrer ersten Begegnung "Voldemort", vermutlich in Anspielung auf Ebony Maws blasse Haut, Zauberkräfte und das Fehlen einer Nase.

\subsection*{\Large Hinter den Kulissen}
\marginpar{87} 
In den Büchern werden die Augen des erwachsenen Tom Riddles als rot beschrieben. Erstmals sah Harry Potter in Voldemorts Erinnerungen beim Besuch Hezibah Smiths ein rotes Aufblitzen. In den Filmadaptationen ist seine Augenfarbe blaugrau.
\vspace{10pt}
\newline
\marginpar{88}  
Die beliebtesten Filmrequisiten erläuterte der Schauspieler Ralph Fiennes auf einer Pressekonferrenz zur Filmeröffnung Harry Potter und die Heiligtümer des Todes (Film 2)
\begin{itemize}
\marginpar{89} 
    \item Sein Zauberstab
    \item Sein Gebiss
    \item Sein Voldemortkostüm: Fiennes fand es sehr irritierend, es war zu lang, er stolperte oft darüber. Darunter trug er beinlange Strümpfe, die regelmäßig runterrutschten, was ein würdevolles Gehen erschwerte. Deswegen verlangte er von den Kleidermachern, dass sie ihm Strumpfhalter anfertigten. In Szenen, in denen er mit Stunts zusammenarbeiten musste, hatte er diese mit den Strumpfhaltern an der Schenkelinnenseite geärgert und belästigt.
    \item Ralph Fiennes' beliebteste Filmtextpassage war: "Jetzt... kann ich dich berühren!" in der Szene auf dem Friedhof von Little Hangleton bei der Rückkehr von Lord Voldemort.
\end{itemize}
%SEITE8

\end{document}


\end{document}
