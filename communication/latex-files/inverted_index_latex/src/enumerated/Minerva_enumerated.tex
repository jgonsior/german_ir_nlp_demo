\documentclass[a4paper, 10pt]{article}
\usepackage[T1]{fontenc}
\usepackage[sfdefault]{AlegreyaSans} %% Option 'black' gives heavier bold face
\usepackage[a4paper, left=2cm, right=2cm, bottom=2cm, top=2cm, bottom=2cm]{geometry} % Adjust left and right margins
\usepackage{amsmath} % for \boxed

% Set the width of the lines around the box
\setlength{\fboxrule}{2pt}

\begin{document}

\begin{minipage}[t]{\textwidth}
    \vspace*{-1.5cm} % Move the content up by 0.5cm
    \begin{flushright}
        \hspace*{\fill} % Move the content to the right edge
        $\boxed{\textbf{\Huge\phantom{00}4\phantom{00}}}$ % Your content here with increased padding
    \end{flushright}
\end{minipage}

\section*{\huge Minerva McGonagall}
\marginpar{1} 
Minerva McGonagall (geb. 4. Oktober ) ist eine Hexe und eine Animaga, die selbst von 1947 - 1954 die Hogwartsschule für Hexerei und Zauberei besuchte. Von Dezember 1956 bis 1998 unterrichtete sie Verwandlung in Hogwarts. Gleichzeitig war sie Hauslehrerin von Gryffindor und die stellvertretende Schulleiterin unter Albus Dumbledore, wurde im Schuljahr 1997/1998 aber von den Carrows als Stellvertretende Schulleiterin abgelöst. Nach deren Rauswurf zusammen mit Severus Snape, wurde sie zur festen Schulleiterin ernannt. McGonagall war auch ein Mitglied des zweiten Ordens des Phönix. 1995 stellte sie sich gegen die damalige Inquisitorin von Hogwarts, Dolores Umbridge. Sie versuchte auch die Schüler im Schuljahr 1997/1998 bestmöglich vor den Carrows zu schützen. Sie kämpfte in mehreren Schlachten beider Kriege und überlebte.
\vspace{10pt}
\newline
\marginpar{2}  
Minerva McGonagall war eine der mächtigsten Hexen, die es je gab, wenn nicht sogar die mächtigste Hexe.
\subsection*{\Large Biografie}
\subsubsection*{\large Frühes Leben}
\marginpar{3} 
Minerva McGonagall wurde am 4. Oktober 1935 am Stadtrand von Caithness in den schottischen Highlands als Tochter des Muggels und presbyterianischen Pfarrers Robert McGonagall sr. und seiner Frau, der Hexe Isobel Ross, geboren. Sie war das erste Kind und wurde nach ihrer Großmutter mütterlicherseits, einer überaus talentierten Hexe, benannt.
\vspace{10pt}
\newline
\marginpar{4}  
Ihre Geburt erwies sich als ein zweischneidiges Schwert: Isobel hatte die Magie für ihre Liebe zu Robert aufgegeben und ihm aus Angst nie gesagt, dass sie über magische Fähigkeiten verfügte. Minerva jedoch zeigte schon von frühester Kindheit an kleine, aber sichtbare Zeichen von Magie. Als Säugling hatte Minerva keine Kontrolle über ihre Magie und ließ damit Spielzeuge, die auf den höher gelegenden Regalen lagen, zu ihrem Kinderbett schweben, die Katze der Familie nach ihrer Pfeife tanzen und den Dudelsack ihres Vaters selbstständig spielen, was die kleine Minerva glücklich glucksen ließ. Ab diesem Zeitpunkt war es Isobel klar, dass sie ihrem Mann die seltsamen Vorgänge und deren Ursache beibringen musste. Sie erklärte Robert alles und betonte aber, dass sie durch das Internationale Geheimhaltungsabkommen verpflichtet war, ihre magischen Fähigkeiten vor anderen Muggeln zu verbergen.
\vspace{10pt}
\newline
\marginpar{5}  
Dennoch hatte die Ehe Bestand und auf Minerva folgten noch zwei Söhne, Malcolm und Robert jr. Wie ihre Schwester zeigten auch sie bereits früh Anzeichen von magischen Fähigkeiten. Minerva vertuschte mithilfe ihrer Mutter die dadurch entstehenden Vorfälle.
\vspace{10pt}
\newline
\marginpar{6}  
An Minervas elftem Geburtstag, dem 4. Oktober 1946, erhielt sie die Zusage von der Hogwartsschule für Hexerei und Zauberei, damals unter der Leitung von Professor Armando Dippet. Minervas Mutter weinte, als der Brief ankam, da sie gleichsam stolz und auch neidisch auf ihre Tochter war.
\subsubsection*{\large Schulzeit in Hogwarts}
\marginpar{7} 
Von 1947 bis 1954 war Minerva McGonagall selbst Schülerin von Hogwarts. Schon am ersten Abend, als ihr der Sprechende Hut aufgesetzt wurde, zog sie ungewollt Aufmerksamkeit auf sich, denn sie erwies sich als sogenannter "Hutklemmer". Nach fünfeinhalb Minuten entschied der Sprechende Hut, der zwischen den Häusern Ravenclaw und Gryffindor geschwankt hatte, Minerva nach Gryffindor zu schicken.
\vspace{10pt}
\newline
\marginpar{8}  
Minerva entwickelte sich zu einer der besten Schülerinnen ihres Jahrgangs, die ein besonderes Talent für das Fach Verwandlung mitbrachte. Während ihrer Schulzeit erreichte Minerva Spitzennoten in den ZAG- und UTZ-Prüfungen, wurde Vertrauensschülerin, Schulsprecherin und Gewinnerin des Bester-Newcomer-Preises von Verwandlung Heute. Schließlich wurde sie auch mit Hilfe ihres Verwandlungslehrers Albus Dumbledore ein Animagus. Ihre Animagusgestalt stellte sich als getigerte Katze mit quadratischem Brillenmuster um die Augen heraus und diese trug sie rechtens in das Animagus-Register im Zaubereiministerium ein.
\vspace{10pt}
\newline
\marginpar{9}  
Zusatzlich zu ihren schulischen Fähigkeiten war Minerva eine durchaus fähige Quidditchspielerin. In ihrem letzten Schuljahr fiel sie in einem Spiel gegen Slytherin vom Besen und zog sich eine schwere Gehirnerschütterung und mehrere gebrochene Rippen zu. Daraus resultierte ihr lebenslanger Wunsch, Slytherin beim Quidditch besiegt zu sehen.
\subsubsection*{\large Früher Liebeskummer}
\marginpar{10} 
Nachdem Minerva Hogwarts verlassen hatte, kehrte sie für einen letzten Sommer zu ihrer Familie zurück, bevor sie ihre Stelle beim Zaubereiministerium antreten wollte.
%SEITE1
\vspace{10pt}
\newline
\marginpar{11}  
Im jungen Alter von 18 Jahren verliebte sie sich Hals über Kopf in den Muggeljungen Dougal McGregor, einen hübschen, klugen und humorvollen Sohn eines ortsansässigen Bauern. Beide hatten einen gemeinsamen Sinn für Humor, stritten sich ungehalten und vermuteten geheimnisvolle Tiefen im jeweils anderen. Eines Tages sank McGregor mitten auf einem Acker nieder und machte ihr einen Heiratsantrag. Minerva willigte ein, doch in dieser Nacht fand sie keinen Schlaf. Gedanken um ihre Zukunft plagten sie, denn Dougal wusste nicht, dass sie in Wirklichkeit eine Hexe war. Alle ihre ehrgeizigen Ziele wären mit dieser Ehe begraben und ihr Zauberstab wohl für immer weggesperrt.
\vspace{10pt}
\newline
\marginpar{12}  
Früh am nächsten Morgen ging Minerva zu Dougal, um ihm zu sagen, dass sie es sich anders überlegt habe und ihn nicht heiraten könne. Wegen des internationalen Geheimhaltungsabkommens konnte sie ihm keinen Grund für ihren plötzlichen Sinneswandel liefern und so ließ sie ihn mit gebrochenem Herzen zurück. Drei Tage später reiste sie nach London ab.
\subsubsection*{\large Arbeit beim Zaubereiministerium}
\marginpar{13} 
Nach ihrem Schulabschluss arbeitete Minerva zwei Jahre beim Zaubereiministerium in der Abteilung für magische Strafverfolgung.
\subsubsection*{\large Rückkehr nach Hogwarts}
\marginpar{14} 
Nach den zwei Jahren im Ministerium schickte sie schließlich eine Eule nach Hogwarts mit der Frage, ob ein Lehrposten für sie in Frage käme. Innerhalb weniger Stunden kehrte die Eule mit einem Stellenangebot für das Fach Verwandlung, Hauslehrerin von Gryffindor und stellvertretende Schulleiterin zurück.
\vspace{10pt}
\newline
\marginpar{15}  
Minerva McGonagall blieb auch nach ihrer Rückkehr als Lehrerin nach Hogwarts mit ihrem ehemaligen Chef Elphinstone Urquart freundschaftlich in Kontakt. Während seiner Ferien in Schottland besuchte er sie und machte ihr überraschend in Madam Puddifoot's Café einen Antrag. Minerva, die immer noch in Dougal McGregor verliebt war, lehnte jedoch ab. Elphinstone jedoch hörte nie auf, sie zu lieben und machte ihr ab und zu auch neuerliche Anträge, während sie ihn weiterhin abwies.
\subsubsection*{\large Erster Zaubererkrieg}
\marginpar{16} 
Nach dem ersten Aufstieg des dunklen Zauberers Lord Voldemort nahm Minerva den Kampf gegen Voldemort und seine Anhänger auf. Während dieser Zeit unterrichtete sie noch in Hogwarts.
\subsubsection*{\large Orden des Phönix}
\marginpar{17} 
In den Anfängen des Ersten Krieges, schloss sich Minerva einer Gruppierung zur Bekämpfung von Voldemort an, nämlich dem Orden des Phönix unter der Führung von Albus Dumbledore. Minerva war aktiv - obwohl zeitgleich als Professorin in Hogwarts tätig -  im Widerstand beteiligt und engagierte sich weiterhin im Orden, obgleich ein großer Mitgliederschwund zu verzeichnen war. Sie überstand den Krieg ohne nennenswerte Blessuren.
\subsubsection*{\large Schutz von Harry Potte}
\marginpar{18} 
Als Harry Potter 1981 als Kleinkind in Godric's Hollow Lord Voldemort besiegte, starben in diesem Zuge seine Eltern James und Lily Potter (beide aktive Mitglieder des Ordens des Phönix), als sie ihren Sohn vor Voldemort beschützen wollten. Der Sieg über Voldemort war der Auslöser für das Ende des Krieges. Seine Anhänger wurden größtenteils gefasst und verurteilt. Die wenigen, die entkommen konnten, verschwanden von der Bildfläche. Minerva reiste nach Surrey, um dort die Dursleys  zu beobachten und abzuschätzen, ob sie geeignet waren, um für Harry Potter sorgen zu können, da sie seine einzigen noch lebenden Verwandten waren. Obwohl sie Zweifel an der Familie hatte, ließ sie es zu, dass Dumbledore Harry auf deren Türschwelle legte.
\subsubsection*{\large Zwischen den Kriegen}
\marginpar{19} 
Nach dem Ende des Ersten Zaubererkrieges, setzte Minerva ihre Lehrtätigkeit für das Fach Verwandlung an der Hogwartsschule für Hexerei und Zauberei fort. Sie blieb ein loyales Mitglied des Ordens des Phönix und behielt Harry weiterhin im Auge, um ihn zu beschützen.
\vspace{10pt}
\newline
\marginpar{20}  
Während eines sommerlichen Spaziergangs im Jahr 1982, machte ihr Elphinstone am See auf dem Gelände von Hogwarts erneut einen Antrag. Diesmal willigte McGonagall ein, der schockierende Tod von Dougal McGregor schien Minerva befreit zu haben. Elphinstone, inzwischen pensioniert, kaufte für beide ein Haus in Hogsmeade, von dem Minerva jeden Tag problemlos zur Arbeit gelangen konnte. Minerva, die schon immer eine Art Feministin war, behielt ihren Namen auch nach der Hochzeit. Die kurze Ehe verlief bis zum Unfalltod von Elphinstone sehr glücklich. Obwohl das Ehepaar keine eigenen Kinder hatte, waren Minervas Nichten und Neffen häufig zu Besuch. Dies war eine sehr erfüllte Zeit für Minerva.
\vspace{10pt}
\newline
\marginpar{21}  
Durch den Biss einer Giftigen Tentakel, nach drei Jahren Ehe, starb Elphinstone Urquart und brachte somit entsetzliches Leid 
%SEITE2
über Minerva. Sie ertrug es nicht länger, allein in dem Haus in Hogsmeade zu wohnen und kehrte zurück in ihr steingefliestes Schlafzimmer in Hogwarts, das durch eine verborgene Tür in ihrem Büro zugänglich war. Es ist unbekannt, ob sich das Haus noch in ihrem Besitz befindet oder nicht. McGonagall, als schon immer in sich gekehrter und zurückhaltender Mensch, wandte all ihre Aufmerksamkeit nun ihrer Arbeit zu. Nur wenige Menschen erkannten damals, wie sehr sie unter diesem Verlust litt. Doch vor allem war es wohl Albus Dumbledore, der ihr Leid sah.
\subsubsection*{\large Schutz des Steins der Weisen}
\marginpar{22} 
1991 brachte Nicolas Flamel, ein Freund Albus Dumbledores, den Stein der Weisen wegen der besseren Schutzmöglichkeiten aus seinem Verlies in Gringotts nach Hogwarts. Dort beteiligte sich Minerva an den Maßnahmen, die zum Schutz rund um das Artefakt aufgebaut wurden. Sie beschwor ein riesiges Schachspiel herauf und verzauberte es so, dass es sich wie ein Zaubererschachspiel verhalten würde, sobald jemand an ihm vorbeigelangen wollte. Die Person(en), die an diesem Hindernis vorbeiwollte(n), mussten den Platz einer Spielfigur einnehmen, wodurch nicht ungefährliche Verletzungen nicht unwahrscheinlich waren. 
\vspace{10pt}
\newline
\marginpar{23}  
Im selben Jahr kam auch Harry Potter als Erstklässler an die Hogwartsschule für Hexerei und Zauberei, von dessen Existenz und der Tatsache, dass er ein Zauberer war, er nichts gewusst hatte. Als Harry dann schließlich in Hogwarts ankam, wurde er dem Haus Gryffindor zugeteilt, dessen Hauslehrerin Minerva war.
\vspace{10pt}
\newline
\marginpar{24}  
Obwohl Minerva ihre Strenge auch vor dem Jungen nicht versteckte, hielt es sie nicht davon ab, für ihn bei Gelegenheit die Regeln etwas abzuändern, so als sie ihn ohne Aufsicht auf einem Besen fliegend erwischte. Anstatt Madam Hoochs angekündigte Konsquenz für alle, die sich beim Fliegen erwischen lassen, von Hogwarts verwiesen zu werden, durchzuziehen, brachte sie ihn zu Oliver Wood, dem Kapitän der Quidditchmannschaft von Gryffindor, damit er Harrys phänomenale Flugkünste beurteilen könne.
\vspace{10pt}
\newline
\marginpar{25}  
Nachdem Harry als jüngster Sucher des Jahrhunderts in die Mannschaft kam, organisierte McGonagall ihm den damaligen schnellsten Rennbesen, einen Nimbus 2000. Gegen Ende des Jahres, wurden die, von Minerva und den anderen Lehrern errichteten und zum Schutz des Steins der Weisen getroffenen Maßnahmen, auf die Probe gestellt, als Professor Quirrell versuchte, den Stein zu stehlen. Hermine Granger, Ronald Weasley und Harry Potter gelang es, den Diebstahl des Steins zu verhindern und am Ende des Schuljahres durch ihre und Neville Longbottoms Tapferkeit so viele Extrapunkte zu bekommen, sodass Gryffindor den Hauspokal gewann.
\subsubsection*{\large Erneute Öffnung der Kammer des Schreckens}
\marginpar{26} 
Am Anfang des 1992er Schuljahres, musste sich Minerva mit Harry und Ron auseinandersetzen, die - als Reaktion auf die von Dobby verschlossene Barriere zum Gleis 9¾ und das Verpassen des Hogwarts-Expresses - mit Mr. Weasleys fliegendem Ford Anglia nach Hogwarts und dann in die Peitschende Weide geflogen sind. Das offenbarte den Muggeln beinahe die magische Welt, aber bei der Bestrafung war Minerva fairer als Snape. Minerva ließ sie mit einer Strafarbeit und einem Brief an ihre Eltern und Erziehungsberechtigten gehen; sie zog ihnen keine Hauspunkte ab, als Harry anfügte, dass das Schuljahr noch gar nicht begonnen hätte, als sie das Auto genommen hatten.
\vspace{10pt}
\newline
\marginpar{27}  
Später sollte Minerva an der erneuten Öffnung der Kammer des Schreckens teilhaben. Sie durchkämmte, zusammen mit dem Rest des Lehrkörpers, ohne Erfolg das Schloss. Sie musste sich außerdem dem mit weniger hilfreichen Gilderoy Lockhart, der als Professor für Verteidigung gegen die dunklen Künste angestellt worden war, auseinandersetzen. Minerva hatte wenig Sympathie für ihn übrig und am Ende des Jahres war sie eine der vielen Professoren, die ihn sogar verachteten.
\vspace{10pt}
\newline
\marginpar{28}  
In den letzten Wochen des Schuljahres wurde Minerva zeitweilige Schulleiterin von Hogwarts, als die Schulräte unter dem Einfluss Lucius Malfoys Dumbledore suspendierten, da er die Angriffe auf Schüler nicht in den Griff bekommen könne. Sie war deswegen ziemlich aufgewühlt. Als Harry und Ron sich wegzuschleichen versuchten, um dem Verursacher der Angriffe auf den Grund zu gehen und von ihr erwischt wurden, ließ sie sie unter dem Vorwand, die versteinerte Hermine besuchen zu wollen, gehen; normalerweise würde sie sie mit Strafarbeiten bestrafen. Als Ginny Weasley später in die Kammer verschleppt wurde, "Ihr Skelett wird für immer in der Kammer liegen", glaubte Minerva verzweifelt, dass sie dem Ende von Hogwarts entgegensehen würde. Als der Basilisk in der Kammer erledigt wurde, kehrte Dumbledore an die Schule zurück und Minerva widmete sich wieder ihrer eigentlichen Tätigkeit.
\subsubsection*{\large Flucht von Sirius Black}
\marginpar{29} 
In den Sommer 1993 fiel unter anderem die Flucht von Sirius Black aus Askaban. Als das Schuljahr im September begann, musste sich Minerva mit den Auswirkungen, die die Dementoren auf ihre Schüler, wie Harry Potter, hatten, auseinandersetzen. Sie autorisierte Hermine Granger, einen Zeitumkehrer für zusätzliche Schulstunden zu nutzen. Eine Entscheidung, die sich für zukünftige Ereignisse in diesem Jahr als sehr nützlich erweisen würde. Dieses Jahr würde wegen der Dementoren, die das Schloss bewachen würden, mit Sicherheit ein weiteres schwieriges Jahr werden.
%SEITE3
\vspace{10pt}
\newline
\marginpar{30}  
Am Anfang des Jahres, als McGonagall ihre Animagusform ihrer dritten Klasse vorführte, bemerkte sie, wie still alle waren. Als Hermine dann von ihrer erste Stunde Wahrsagen berichtete, verstand McGonagall und fragte ihre Klasse, wer von ihnen denn in diesem Jahr sterben werde. Harry Potter hob seine Hand. McGonagall versicherte ihm, dass Sybill Trelawneys Vorhersagen eine besondere Art waren, eine Klasse zu begrüßen, und dass sich bisher noch kein einziger Todesfall unter den Schülern eingestellt hat.
\vspace{10pt}
\newline
\marginpar{31}  
Während des Schuljahres half Minerva bei der Sicherung des Schlosses, dennoch gelang es Black ins Schloss zu gelangen. Nachdem er die Fette Dame attackiert hatte, versuchte Minerva, Harry zu sagen, dass Black hinter ihm her sei, aber sie wurde von Harry unterbrochen, der ihr erklärte, dass Mr. Weasley es ihm schon gesagt hatte. Später im Jahr gelang es Black mit Hilfe von Hermines Kater Krummbein, mit dem er sich angefreundet hatte und der ihm die Liste der aktuellen Passwörter gab, in den Gryffindorturm zu kommen. Als Neville Longbottom als derjenige indentifiziert wurde, der die Passwörter aufgeschrieben hatte, gab McGonagall ihm eine Strafarbeit und schrieb an seine Großmutter.
\vspace{10pt}
\newline
\marginpar{32}  
Die Folge davon war, dass Neville einen Heuler bekam. Als Harry einen Feuerblitz nach der Zerstörung seines Nimbus 2000 zugeschickt bekam, konfiszierte Minerva ihn, um ihn zu untersuchen, da sie glaubte, er wurde von Black geschickt und müsste demnach verflucht sein. Dieser Vorfall führte dazu, dass sich Harry und Hermine stritten, da Hermine Minerva auf den Besen aufmerksam gemacht hatte. Er wurde ihm später zurückgegeben. Während des Quidditchspiels gegen Ravenclaw erwischte Minerva Draco Malfoy, Gregory Goyle, Vincent Crabbe und Marcus Flint dabei, wie sie Dementoren nachmachten, um Harry vom Spiel abzulenken. Sie zog Slytherin fünfzig Punkte ab und gab den Vieren Strafarbeiten. Im Frühling 1994, nach der unfairen Spielweise der Quidditchmannschaft von Slytherin beim Finale, wo Minerva Zeuge wurde, wie Draco Malfoy das Ende von Harrys Feuerblitz festhielt, um ihn am Fangen des Goldenen Schnatzes zu hindern, erlebte sie auch, wie das Quidditchteam der Gryffindors das erste Mal, seit dem Abgang von Charlie Weasley, wieder den Quidditchpokal gewann. McGonagall war vom Sieg der Gryffindors begeistert, aber war betrübt darüber, dass ihr Kapitän Oliver Wood nach diesem Schuljahr gehen würde. Im späteren Verlauf des Jahres hatten sich drei Studenten ihres Hauses - Harry Potter, Ron Weasley und Hermine Granger - in Schwierigkeiten gebracht, nachdem Sirius Black nach Hogwarts zurückgekehrt war.
\subsubsection*{\large Trimagisches Turnier}
\marginpar{33} 
Im Herbst 1994 empfing Hogwarts die magischen Schulen Beauxbatons Akademie für Zauberei und das Durmstrang Institut im Zuge des Trimagischen Turniers. Harry Potter wurde, obgleich minderjährig, als Trimagischer Champion ausgewählt und McGonagall sorgte sich um seine Sicherheit. Minerva protestierte mit Dumbledore und Crouch, um Harry wieder aus dem Turnier auszuschließen, doch es war gegen die Regeln des Turniers und Potter musste weitermachen, egal wie gefährlich es für ihn sein könnte.
\vspace{10pt}
\newline
\marginpar{34}  
Während des Schuljahrs ermahnte Minerva ihre Schüler mehr als einmal, um die beiden anderen Schulen Beaxbatons und Durmstrang zu beeindrucken, sich und Hogwarts von der besten Seite zu präsentieren. Sie besuchte den Weihnachtsball mit Albus Dumbledore und trug ihr Haar zum ersten Mal seit Harrys Ausbildungsbeginn in Hogwarts offen. Sie stellte außerdem auch ihren Klassenraum zum Erproben von Zaubersprüchen für Harry, Ron und Hermine zur Verfügung, um Harrys Chancen in der Dritten Aufgabe zu verbessern. Am Ende des Turniers begleitete Minerva Dumbledore und Severus Snape, die Harry vor dem Todesser retteten, der sich das ganze Jahr als der diesjährige Verteidigung gegen die dunklen Künste-Professor und Auror Alastor Moody verkleidet hatte.
\vspace{10pt}
\newline
\marginpar{35}  
Sie wurde angewiesen, Crouch jr. zu bewachen, nachdem er seine Mitarbeit bei der Wiedergeburt Voldemorts zugegeben hatte, aber ihr war es nicht möglich, sowohl den Dementor abzuwehren, den Minister Cornelius Fudge mitgebracht hatte, als auch den Kuss, der den Todesser ereilte, zu verhindern. Sie war über alle Maßen zornig auf Fudge wegen dieses Fehlers und schrie ihn lauter an, als Harry es je erlebt hatte. Auch als sich der Minister weigerte, Dumbledore und Harry Potter bezüglich der Rückkehr Voldemorts zu glauben, stand sie auf der Seite ihres Vorgesetzten und ihres Schülers. Bei der Rückkehr des Dunklen Lords wurde der Hufflepuffschüler Cedric Diggory achtlos ermordet und eine Gedenkfeier wurde am Ende des Schuljahrs abgehalten. Der Zweite Zaubererkrieg hatte begonnen, auch wenn das Ministerium versuchte, jegliche Beweise zu vertuschen.
\subsubsection*{\large Zweiter Zaubererkrieg}
\marginpar{36} 
Kaum hatte der Zweite Zaubererkrieg begonnen, trat Minerva wieder in den Orden des Phönix ein und verbrachte viel Zeit im Sommer 1995 mit der Arbeit für den Orden. Obwohl diese Pflichten viel von ihrer Zeit einnahmen, würde sich Minerva in Hogwarts noch größeren Herausforderungen stellen müssen. Da es Dumbledore nicht möglich war, einen neuen Lehrer für Verteidigung gegen die dunklen Künster zu finden, setzte das Zaubereiministerium Dolores Umbridge, eine  hochrangige Ministeriumsangestellte und Unterstützerin Fudges ein. Minerva sah diese Ernennung als einen Affront gegen die Schule selbst und versuchte kaum, ihren Ekel zu verbergen.
%SEITE4
\subsubsection*{\large Großinquisitorin in Hogwarts}
\marginpar{37} 
Als Harry Potter sich mit Umbridge anlegte und sich gleich in seiner ersten Stunde eine Strafarbeit einhandelte, verstand Minerva ihn zwar, warnte ihn aber gleichzeitig, dass mit Umbridge nicht zu spaßen sei, und dass er sich lieber zurückhalten solle. McGonagall tat alles, um Umbridges Autorität zu umgehen, allerdings ohne ihre eigene Stellung als stellvertretende Schulleiterin zu riskieren. In der Tat schenkte sie Umbridge so wenig Glauben und Respekt wie möglich, während Hogwarts unter der Beobachtung des Zaubereiministeriums stand, und gab sogar Peeves Tipps, wie er einen Kronleuchter abschrauben könne, als er versuchte, extra Chaos für Umbridge zu verursachen. McGonagall schaffte es auch, ihre Abneigung gegenüber Professor Trelawney für die Zeit, in der Umbridge die Bewohner Hogwarts' verfolgte, abzulegen. Es wurde gezeigt, dass sie Trelawney, nachdem Umbridge sie entließ, zurück ins Schloss brachte.
\vspace{10pt}
\newline
\marginpar{38}  
Nach Umbridges Ernennung zur Großinquisitorin musste Minerva die Anwesenheit der anderen Frau während der Unterrichtsinspektionen aushalten. Minerva hielt ihr Temperament unter Kontrolle, wahrte eine kalte Maske und ließ dennoch den einen oder anderen trockenen oder spitzen Witz auf Umbridge los. Diese Begegnungen hatten zum Ergebnis, dass Schüler wie Ronald Weasley, die Umbridge von ganzem Herzen hassten, jubeln konnten.
\vspace{10pt}
\newline
\marginpar{39}  
Trotz aller Bemühungen Minervas, den Einfluss des Ministeriums in Hogwarts zu schwächen, begannen die Dinge aus dem Ruder zu laufen. Nachdem Dumbledore aus der Schule gedrängt wurde, wurde Umbridge zur Schulleiterin ernannt und stand damit über Minerva. Als Umbridge sich dann in die Berufsberatung der Fünftklässler einmischte, verlor Minerva ihre Geduld. Während Harry Potters Sitzung, als Umbridge andauernd das Gespräch unterbrach und verkündete, dass Harry niemals ein Auror im Ministerium werden würde, erklärte Minerva Harry, dass sie ihm auf jedem erdenklichen Weg helfen würde, um sein Ziel zu erreichen.
\subsubsection*{\large St. Mungo Hospital}
\marginpar{40} 
Während der ZAG-Prüfungen erreichten die Ereignisse ihren Höhepunkt. Als Dolores Umbridge mit einigen Auroren versuchte, Hagrid aus Hogwarts zu vertreiben, versuchte Minerva ihren Kollegen zu verteidigen. Allerdings, bevor sie überhaupt ihren Zauberstab ziehen konnte, wurde Minerva plötzlich von vier Schockzaubern in die Brust getroffen, was aufgrund ihrer Behandlung eine Verlegung vom Krankenflügel in das St. Mungo zur Folge hatte. Madam Pomfrey sagte, dass es sehr schwierig sein würde, Minerva bei Tageslicht zu treffen und dass sie aus Protest und wegen diesem feigen Angriff ihren Posten aufgeben würde, wenn das Wohlbefinden der Schüler bei ihr nicht an oberster Stelle stehen würde. Viele Schüler waren über diesen Angriff empört und ignorierten deshalb sogar ihre laufende Astronomieprüfung und selbst Professor Tofty tat seinen Zorn darüber lautstark kund.
\subsubsection*{\large Rückkehr nach Hogwarts}
\marginpar{41} 
Minerva kehrte bis kurz nach dem Kampf in der Mysteriumsabteilung nicht nach Hogwarts zurück, doch als sie zurückkam, gab sie sechs Schülern, die am Kampf beteiligt waren, als Snape stichelte, dass Gryffindor Null Punkte habe, jeweils 50 Hauspunkte dafür, dass sie der Welt gezeigt haben, dass Lord Voldemort wieder zurück ist. Mit  etwas Vergnügen beobachtete sie, wie Peeves Umbridge aus der Schule jagte und fügte hinzu, dass sie sich ihm gerne anschließen würde, doch Peeves habe ihren Gehstock genommen. Nach ihrer Genesung im St. Mungo nahm Minerva ihre Aufgaben innerhalb des Ordens des Phönix sowie ihre Lehrtätigkeit, wenn das Schuljahr nach dem Sommer wieder beginnen würde, wieder auf.
\subsubsection*{\large Schlacht auf dem Astronomieturm}
\marginpar{42} 
Später im Schuljahr nahm sie an der Schlacht auf dem Astronomieturm 1997 teil, nachdem Draco Malfoy Todesser durch das Verschwindekabinett im Raum der Wünsche in Hogwarts eingeschleust hatte. Sie blieb auf der Ebene unter Harry und Dumbledore stecken und konnte die Treppen nicht passieren, um zu ihnen zu kommen, doch Snape schaffte es und drängte sich an ihr vorbei.
\vspace{10pt}
\newline
\marginpar{43}  
Nach der Schlacht war sie, als sie von Snapes Mord an Dumbledore erfuhr, dem Mann, den sie fast ihr ganzes Leben als Lehrer, Mentor und Freund gekannt hatte und dessen Mörder sie hatte passieren lassen, am Boden zerstört. Nachdem der Mord begangen war, leuchtete das Dunkle Mal, welches bereits in den Himmel gezaubert worden war, über die Schulgründe. McGonagall streckte ihren Zauberstab in Richtung Himmel, um das Dunkle Mal aus dem Himmel zu entfernen und Dumbledores Leiche von dessen Schein zu befreien. Nach Dumbledores Tod organisierte sie dessen Beerdigung in Hogwarts und für die letzte Zeit des Schuljahres übernahm sie Dumbledores ehemalige Position und wurde Schulleiterin von Hogwarts, obwohl sie zugunsten von Snape nach Voldemorts Übernahme des Ministeriums abgesetzt wurde. Minerva versuchte Harry über den Tod von Dumbledore hinwegzutrösten, doch ihre Bemühungen fielen kurz aus, da sie selbst durch des Tod ihres geliebten Freundes emotional am Ende war.
%SEITE5
\subsubsection*{\large Voldemorts Kontrolle in Hogwarts}
\marginpar{44} 
Als der Krieg weiterging, blieb Minerva in Hogwarts, auch wenn es von Voldemort übernommen und kontrolliert wurde, wahrscheinlich um die Schüler so gut es ging zu beschützen und sie als Insider agieren konnte, um den Orden des Phönix über die Ereignisse in Hogwarts zu informieren. Auch wenn Severus Snape Schulleiter war und Alecto und Amycus Carrow freie Hand bekamen, tat Minerva alles in ihrer Macht stehende, um die Schüler zu schützen. Die von Schülern gegründete Rebellenallianz Dumbledores Armee wurde reformiert und McGonagall half ihnen, ihre Treffen geheimhalten zu können.
\subsubsection*{\large Schlacht von Hogwarts}
\marginpar{45} 
Am 2. Mai 1998 war Minerva zugegen, als Harry Potter, Hermine Granger und Ron Weasley Mitten in der Nacht nach Hogwarts zurückkehrten, um Rowena Ravenclaws Diadem zu finden. Sie war sich erst richtig bewusst, dass sie wirklich da waren, nachdem Harry ihre Ehre durch die erfolgreiche Anwendung des Cruciatus-Fluchs an Amycus Carrow verteidigt hatte.
\vspace{10pt}
\newline
\marginpar{46}  
Viele sind der Meinung, das Snape hätte gewinnen können, doch dies stimmt nicht.
\vspace{10pt}
\newline
\marginpar{47}  
Natürlich nahm er Rücksicht auf Minerva, aber auch wenn er dies nicht getan hätte, hätte sie ihn mit ihrer Intelligenz und Macht geschickt in Schach gehalten. Schließlich gewann sie auch ein Duell zwischen Voldemort und der nahm keine Rücksicht auf Minerva. Mit der Hilfe von Harry und Luna Lovegood setzte sie die Carrows im Ravenclaw-Gemeinschaftsraum außer Gefecht. Im Bewusstsein, dass ihnen eine Schlacht bevorstand, versammelte Minerva Filius Flitwick, Pomona Sprout und Horace Slughorn und zusammen vertrieben sie Snape aus Hogwarts. Minerva nannte ihn sogar einen Feigling, als er aus der Schule floh. Harry informierte McGonagall, dass Dumbledore ihm eine Aufgabe hinterlassen hätte und sie bot an, aus Respekt für Dumbledores Wünsche, Voldemort und seine Armee in der von Harry benötigten Zeit aufzuhalten. Dann fuhr sie fort und erweckte die Rüstungen und Statuen des Schlosses, um ihr bei der Verteidigung der Schule zu helfen. Kurz vor der Schlacht, befahl Minerva Neville Longbottom, die Überdachte Brücke zu zerstören, damit die Greifer nicht in die Schule eindringen können.
\vspace{10pt}
\newline
\marginpar{48}  
Nachdem sie die Evakuierung der jüngeren Schüler durch den Eberkopf organisiert hatte, begann Minerva das Schloss gegen den baldigen Angriff mit ihren Kollegen, den Ordensmitgliedern und älteren Schülern, die zurückgeblieben waren, um zu kämpfen, zu sichern. Als Pansy Parkinson vorschlug, den Vorschlag Voldemorts - ihrer aller Leben gegen das von Harry Potter - anzunehmen, befahl Minerva dem Mädchen, allen Slytherins und allen, die nicht kämpfen wollten, zu gehen.
\vspace{10pt}
\newline
\marginpar{49}  
Minerva kämpfte geschickt im ersten Teil der Schlacht und blieb auch in der Pause stark. Sie stand unter vielen Schülern in einer Menge, als Voldemort selbst nach Hogwarts hineinging. Erst als Minerva sah, dass es Harry Potters augenscheinlich toter Körper war, der von Rubeus Hagrid getragen wurde, begann sie zu verzweifeln und stieß einen Schrei aus, schrecklich für alle, die ihn hörten, vergleichbar mit dem von Ron, Hermine und Ginny zusammen.
\vspace{10pt}
\newline
\marginpar{50}  
Dennoch kämpfte Minerva weiter, als die Schlacht nach dem Angriff der Zentauren und Hauselfen auf Voldemort und die Todesser fortgeführt wurde. In den letzten Minuten der Schlacht duellierte sie sich neben Horace Slughorn und Kingsley Shacklebolt mit Voldemort, allerdings wurden sie von seinem explodierenden Zorn weggesprengt, als Bellatrix Lestrange in einem Duell von Molly Weasley getötet wurde.
\vspace{10pt}
\newline
\marginpar{51}  
Sie hätten das Duell wiederaufnehmen können, aber durch die Tatsache, dass sich Harry offenbarte und darauf bestand, Voldemort selbst zu töten, war ihr Duell beendet. Nachdem Voldemort endlich von Harry getötet wurde, umarmte sie ihn mit vielen anderen in einem uncharakteristischen Ausbruch von Emotionen.
\subsubsection*{\large Ihre Zukunft in Hogwarts}
\marginpar{52} 
Sie wurde nach der Schlacht von Hogwarts am 2. Mai 1998 zur festen Schulleiterin ernannt.
\subsection*{\Large Äußerliche Erscheinung}
\marginpar{53} 
„Sogleich öffnete sich die Tür. Vor ihnen stand eine große Hexe mit schwarzen Haaren und einem smaragdgrünen Umhang. Sie hatte ein strenges Gesicht, und Harrys erster Gedanke war, dass mit ihr wohl nicht gut Kirschen essen wäre.“
\vspace{10pt}
\newline
\marginpar{54}  
— Harry Potters erster Eindruck von McGonagall
\vspace{10pt}
\newline
\marginpar{55}  
Minerva McGonagall wurde zunächst als eine große, ziemlich streng aussehende Frau beschrieben und später als eine "sprühende Mitt-Siebzigerin". Sie trug oft smaragdgrüne Roben oder ihr Lieblings-Tartan-Muster. Oft trug sie auch einen auf eine Seite geneigten, spitzen Hut und hatte immer ein sehr strengen Ausdruck. Sie ließ nur selten ihre schwarzen Haare offen und den größten Teil der Zeit waren sie hinten in einem festen Knoten befestigt.
\vspace{10pt}
\newline
\marginpar{56}  
Minerva war ein Fan von Schottenmuster und trug es oft zu Quidditch-Spielen in Kombination mit einer Wollmütze, die ihre 
%SEITE6
Ohren bedeckte. Minerva hatte eine Brille, die den quadratischen Markierungen um die Augen ihres Animagus' und Patronus' abgestimmt waren.

\subsection*{\Large Persönlichkeit und Verhalten}
\marginpar{57} 
Minerva strahlte fast immer eine Aura der Strenge und gleichsam Großherzigkeit aus und wurde von ihren Schülern und Kollegen respektiert (und teilweise gefürchtet). Ihren eigenen Weg gehend, tolerierte sie bei ihren Schülern weder Frechheit noch Albernheit und war Leuten, die in ihrer Gegenwart etwas Dummes gesagt oder getan hatten oder versuchten, witzig zu sein, gegenüber kalt und abweisend. Trotz ihrer strengen Haltung hatte Minerva einen Sinn für trockenen Humor, der dann zum Vorschein kam, wenn die Lage sehr ernst war.
\vspace{10pt}
\newline
\marginpar{58}  
Entsprechend ihres eigenen Hauses Gryffindor belohnte sie Treue und Tapferkeit hoch und hielt Feigheit für einen schlimmen Fehler. Trotz der Loyalität zu ihrem Haus war sie eine gerechte Person und zog auch Schülern ihres eigenen Hauses Punkte ab, wenn sie dies für nötig hielt.
\vspace{10pt}
\newline
\marginpar{59}  
Minerva hatte keine Angst davor, ihre Meinung zu sagen und zeigte gerade in Streitgesprächen eine spitze Zunge. Meistens blieb sie emotional sehr gelassen, ohne aber gleichgültig zu sein. Obwohl sie nicht immer besonders nett zu ihm war, war sie eine große Unterstützerin von Harry und vertraute Dumbledore, Flitwick, Sprout, Pomfrey und Hagrid. Sie konnte in schwierigen Zeiten sehr warmherzig und nachsichtig sein und kümmerte sich sehr um ihre Schüler, sowohl akademisch als auch persönlich.
\vspace{10pt}
\newline
\marginpar{60}  
Minerva genoss Handarbeiten, Korrektur lesen von Artikeln für Verwandlung heute, Quidditch und sie ist ein Fan der Montrose Magpies.
\vspace{10pt}
\newline
\marginpar{61}  
Als hingebungsvolle Lehrerin für Verwandlung respektierte Minerva diesen spezifischen Zweig der Magie sehr und hielt ihn für komplexer und gefährlicher als jeden anderen. Auf der anderen Seite hatte sie, aufgrund der Ungenauigkeit und Seltenheit von wahren Sehern, wenig Geduld mit Wahrsagen.
\vspace{10pt}
\newline
\marginpar{62}  
Außerdem hatte sie ein beachtliches Temperament, welches sich offenbarte wenn sie wütend oder nervös war.

\subsection*{\Large Magische Fähigkeiten und Fertigkeiten}
\marginpar{63} 
Minerva McGonagall war eine sehr mächtige und talentierte Hexe mit einer Vielzahl von magischen Fähigkeiten. Ihre Fähigkeiten konnte sie zum Abwehren vieler dunkler Zauberer während der Schlacht von Hogwarts nutzen. Minerva konnte sich sogar für eine Weile gegen Voldemort im Duell behaupten. Auf ihr körperliches Durchhaltevermögen sollte auch hingewiesen werden, da sie vier Schockzauber in die Brust bekam und vollständig genas.
\paragraph{Magieausbrüche:} 
\marginpar{64} 
Als Kleinkind war Minerva in der Lage, die Katze der Familie zu manipulieren, ihr Spielzeug in ihr Bett zu zaubern und den Dudelsack ihres Vaters von alleine spielen zu lassen.
\paragraph{Verwandlung:} 
\marginpar{65} 
Während ihrer Lehrtätigkeit in Hogwarts lehrte sie Verwandlung. Schon als junge Hexe zeigte sie ein besonderes Talent in diesem Bereich. Sie hielt es für den komplexesten und gefährlichsten Zweig der Magie, der in Hogwarts unterrichtet wird. Während der Schlacht von Hogwarts zeigte Minerva ihre erstaunliche Beherrschung dieses Fachgebietes, da sie etliche Schreibtische und Rüstungen zum Leben erweckte, um ihr gegen Voldemort zu helfen. Sie verwendete ihr Wissen in diesem Gebiet auch im Duell gegen Severus Snape, einem wirklich guten Zauberer. Wie unten angegeben, wurde Minerva ein Animagus, um ihre Forschungen in diesem Gebiet zu fördern. Sie wurde von Albus Dumbledore gefördert und nahm seinen Platz als Lehrer für Verwandlung ein.
\paragraph{Duellieren:}
\marginpar{66} 
Minerva war eine sehr geschickte Duellantin und war in der Lage, sich selbst gegen viele und weitaus jüngere Todesser, Severus Snape und selbst Voldemort persönlich zu halten. Während der Schlacht auf dem Astronomieturm besiegte sie Alecto Carrow und überstand den Kampf unversehrt. Im Jahr 1998, während der Schlacht von Hogwarts, duellierte sie sich heftigst mit Severus Snape und jagte ihn dann aus der Schule, obwohl es zu beachten ist, dass Snape wegen der Ankunft von Filius Flitwick und Pomona Sprout geflohen ist. Später duellierte sie sich im selben Kampf mit dem Dunklen Lord in einem Kräftegleichgewicht neben Horace Slughorn und Kingsley Shacklebolt.
\paragraph{Animagus:}  
\marginpar{67} 
Ein Animagus zu sein, gab Minerva die Fähigkeit, sich willentlich in die Form einer getigerten Katze mit viereckigen Streifen zu verwandeln. Oft nutzte sie diese Fähigkeit, wenn sie nicht erkannt werden wollte. Ein Beispiel für so eine Situation war, als sie die Familie Dursley beobachtete, bevor Albus Dumbledore Harry Potter zu ihnen gebracht hatte. Sie zeigte ihre Verwandlung auch ihren Schülern in ihrem dritten Jahr und bekam in der Regel Applaus. Als Hermine Granger in ihrem dritten Jahr über Animagi recherchiert hatte, stellte sie fest, dass McGonagall einer der wenigen registrierten Animagi im 20.
%SEITE7
Jahrhundert war. Animagi sind oft Menschen, die sich in Tiere verwandeln um einer Strafe zu entgehen, nachdem sie das Gesetz gebrochen hatten. Minerva bat Albus Dumbledore zu erklären, dass sie jedoch nur für ihre Forschungen auf allen Unterarten der Verwandlung zum Animagi wurde. Albus Dumbledore schrieb etwas dazu in seinen Anmerkungen bezüglich "Die Märchen von Beedle dem Barden". Sie lernte unter der Aufsicht Dumbledores während ihrer Zeit als Schülerin in Hogwarts, ein Animagus zu werden. Sie wurde einmal von der magischen Fachzeitschrift Verwandlung heute mit dem Bester-Newcomer-Preis ausgezeichnet.
\paragraph{Zauberkunst:}
\marginpar{68} 
McGonagall scheint sehr erfahren in Zauberkunst zu sein. 1992, als Harry und Ron in Snapes Büro essen mussten, beschwor sie einen sich immer wieder mit Sandwiches nachfüllenden Teller herauf.
\paragraph{Patronus:} 
\marginpar{69} 
Auch den Patronuszauber hat sie bewältigt, sie ist in der Lage, einen mächtigen, körperlichen Patronus in der Form einer Katze - identisch mit ihrem Animagus - zu schaffen. In der Tat war sie die einzige Person, bei der gesehen wurde, dass sie drei Patroni auf einmal (wenn auch zugegebenermaßen nicht in der Anwesenheit eines Dementors) erschaffen kann. Zusätzlich war sie in der Lage, ihren Patronus, wie von Dumbledore gelehrt, als Kommunikationsmittel zu benutzen.
\paragraph{Dunkle Künste:} 
\marginpar{70} 
Obwohl Professor McGonagall in den dunklen Künsten nicht sehr bewandert ist, kann sie ohne Probleme den Imperiusfluch auf Amycus Carrow anwenden. Das zeigt, dass sie ein wenig erfahren auf diesem Gebiet ist.
\paragraph{Verteidigung gegen die Dunklen Künste:} 
\marginpar{71} 
Auch in der Verteidigung gegen die Dunklen Künste handelte Minerva professionell und kompetent. Das ist daran zu erkennen, dass sie in zwei Kriegen gekämpft und sie beide überlebt hat. Während der Schlacht von Hogwarts focht sie gegen die Todesser, die dafür bekannt sind, die Dunklen Künste zu verwenden. Sie hat auch tapfer gegen Voldemort selbst, an der Seite von Kingsley Shacklebolt und Horace Slughorn, gekämpft.
Studium von Alte Runen: Minerva war mit Alten Runen vertraut und es war ihr möglich, sie zu lesen und zu verstehen. 
\paragraph{Nonverbale Zauber:}
\marginpar{72} 
McGonagall zeigte auch, dass sie den Umgang mit nonverbalen Zaubern sehr gut meisterte. Als Barty Crouch jr. (getarnt als Alastor Moody) Draco Malfoy in ein Frettchen verwandelte, verwandelte sie diesen ohne ein einziges Wort zurück. 1998 fesselte sie die Carrows mit einem silbernen Netz aus ihrem Zauberstab, alles wortlos. Ebenfalls zeigte sie ihre Kompetenz in diesem Gebiet, als Neville versehentlich eines der Beine seines Tisches verschwinden ließ und Minerva es nur durch einen Schlenker ihres Zauberstabes wieder erscheinen ließ. Außerdem kämpfte sie nonverbal mit mächtigen und gefährlichen Zaubersprüchen gegen Snape im Duell.
\paragraph{Führungsqualitäten:} 
\marginpar{73} 
Minerva war eine starke Anführerin und in der Lage, in jeder Situation innerhalb weniger Minuten Verantwortung zu übernehmen. Als Dumbledore die Schule verlassen hatte, trat sie in seine Fußstapfen. In den Anfängen der Schlacht von Hogwarts vertrieb sie den aktuellen Schulleiter Severus Snape und übernahm mit der Führung der Schule auch dessen Schutz und Sicherung gegen Voldemort und seine Todesser. Sie führte die Verteidiger Hogwarts' in die Schlacht und am Ende waren sie siegreich, als Harry Voldemort endgültig besiegte. Nachdem sich alles wieder normalisiert hatte, wurde sie permanente Schulleiterin von Hogwarts.
\paragraph{Logisches Denken und Intelligenz:}
\marginpar{74} 
McGonagall war eine sehr kluge und intelligente Hexe. Ihre Intelligenz ist in vielerlei Hinsicht in der gesamten Serie deutlich geworden. Sie hat immer gezeigt, dass sie fähig war, weise Entscheidungen zu treffen. Im Jahr 1998 löste McGonagall ohne zu zögern ein Rätsel, um Amycus Carrow den Zugang zum Ravenclaw-Turm zu ermöglichen. Ihre Intelligenz gab ihr zu Beginn ihrer Schulzeit die Möglichkeit, eine Ravenclaw zu werden. Allerdings entschied sie sich dann doch für Gryffindor.
\paragraph{Quidditch:} 
\marginpar{75} 
Minerva war auch eine ganz passable Fliegerin und war Mitglied der Quidditchmannschaft von Gryffindor. Allerdings erlitt sie während ihrer Spielzeit etliche Verletzungen, wie z. B. eine Gehirnerschütterung, die sie von der Fortsetzung abhielten. Das ist auch einer der Gründe, weshalb sie so dringend möchte, dass die Gryffindors den Quidditchpokal gewinnen.
\subsection*{\Large Beziehungen und Freundschaften}
\subsubsection*{\large Albus Dumbledore}
\marginpar{76} 
Minerva kannte Albus Dumbledore schon seit ihrer Schulzeit in Hogwarts. Sie besuchte Hogwarts, während Dumbledore als Lehrer für Verwandlung tätig war und ersetzte ihn später, als er die Postion des Schulleiters übernahm. Sie war viele Jahre als seine Stellvertreterin tätig und ersetzte ihn bei mehreren Gelegenheiten.
\vspace{10pt}
\newline
\marginpar{77}  
Obwohl der Liebeskummer bezüglich Dougal McGregor schon einige Zeit zurück lag, nachdem sie ihren Posten in Hogwarts angetreten hatte, erwies es sich als Schock für sie, als ihre ahnungslose Mutter ihr in einem Brief schrieb, dass er die Tochter eines anderen geheiratet habe. Albus Dumbledore entdeckte Minerva daraufhin am Abend darauf in Tränen aufgelöst in einem
%SEITE8
Klassenzimmer. Sie 'beichtete' ihm die Geschichte und Dumbledore bot ihr sowohl Weisheit als auch Trost. Er selbst offenbarte ihr an diesem Abend einiges aus seiner eigenen Familiengeschichte, die bis dahin für Minerva völlig unbekannt gewesen war. Die Geheimnisse, die in jener Nacht zwischen zwei äußerst verschwiegenen und zurückhaltenden Persönlichkeiten ausgetauscht wurden, bildeten die Grundlage von anhaltender gegenseitiger Wertschätzung und Freundschaft.
\vspace{10pt}
\newline
\marginpar{78}  
Minerva hatte keine Scheu, ihre Loyalität Dumbledore gegenüber zu beweisen. Als Cornelius Fudge 1996 Auroren nach Hogwarts brachte, um Dumbledore verhaften zu lassen, trat sie tapfer für ihn ein und kündigte ihre Absicht an, die Auroren in seinem Namen zu bekämpfen. Sie ließ sich erst davon abringen, als Dumbledore sie daran erinnerte, dass sich jemand um Hogwarts und die Schüler kümmern müsse.
\vspace{10pt}
\newline
\marginpar{79}  
Minerva war am Boden zerstört, als Dumbledore im Jahr 1997 während des Kampfes im Astronomieturm getötet wurde. Sie besuchte seine Beerdigung. Ein Jahr später, als Harry Potter zurück nach Hogwarts kam, um Dumbledores Auftrag zu erfüllen, stellte sie die Anweisungen ihres toten Freundes keinen Moment in Frage und versuchte sofort, nach Kräften ihre Unterstützung anzubieten.
\subsubsection*{\large Harry Potter}
\marginpar{80} 
Minerva zeigte ihre Sorge für Harry, seitdem sie, Dumbledore und Hagrid ihn in den Händen der Dursleys zurückgelassen hatten. Besonders, da sie eine recht enge Beziehung mit Harrys Eltern hatte. Sie war eine Art liebende Großmutter für ihn und zeigte sich besorgt mit der Wahl Dumbledores von Hagrid, um Harry sicher nach Little Whinging zu transportieren und von der Auswahl der Familie, wo Harry aufwachsen sollte. Sie verbrachte Stunden damit, die Dursleys auszuspionieren und Informationen über Harrys wahrscheinliches zukünftiges Leben zu bekommen. Es ist jedoch unwahrscheinlich, dass sie nur über ihn wachte, um gegen Dumbledores Plan zu arbeiten, da sie sich niemals gegen ihn wenden würde, egal wie unterschiedlich ihre Meinungen sind.
\vspace{10pt}
\newline
\marginpar{81}  
Als Harry nach Hogwarts kam, war Minerva erfreut, dass er in ihr Haus, Gryffindor, einsortiert worden war. Als sie Harry auf einem Besen fliegen sah erkannte sie sein Naturtalent, ließ von einer Strafe ab und brachte Harry stattdessen zu Oliver Wood, um zu verkünden, dass sie ihren neuen Sucher gefunden hätte. In ihrer Dankbarkeit, dass Harry die Position angenommen hatte, kaufte ihm McGonagall für sein erstes Quidditchspiel einen Nimbus 2000.
\vspace{10pt}
\newline
\marginpar{82}  
Allerdings ließ Minerva ihre Sorge um Harry seine Vergehen in der Schule nicht ignorieren. Obwohl sie Harry oft die Schulregeln ungestraft brechen ließ, geschah dies nur, wenn die Situation lebensbedrohlich war. Minerva verfolgte ihre eigene Methode, um alle Schüler gerecht zu behandeln und zögerte auch nicht, Mitgliedern aus ihrem eigenen Haus harte Strafen aufzuerlegen, wenn sie dachte, sie hätten es verdient. So zog sie Harry und seinen Freunden 150 Punkte ab, trotz dessen Siegerstellung in der Punktefolge zu der Zeit, und sandte sie noch zusätzlich in den Verbotenen Wald. Es wäre möglich gewesen, dass sie sie auch deutlich härter bestraft hätte, wenn Minerva sie direkt beim Herausschmuggeln von Norbert entdeckt hätte.
\vspace{10pt}
\newline
\marginpar{83}  
Während Harrys Schulzeit in Hogwarts hielt Minerva ein wachsames Auge auf ihn und seine Freunde Ron Weasley und Hermine Granger  und griff in ihre Angelegenheiten ein, wann immer sie es für notwendig hielt, um ihre Sicherheit zu gewährleisten. In Harrys fünftem Schuljahr versprach sie auch, dass sie Harry mit allen Möglichkeiten unterstützen würde, damit er seinen Traum, ein Auror zu werden, erreichen könne. Als Minerva später die Details über das Strafmaß herausfand, die Harry und andere Schüler von Dolores Umbridge auferlegt bekommen hatten, äußerte sie offen ihre Abscheu und Ablehnung, trotz der Befugnisse Umbridges in Hogwarts.

\subsubsection*{\large Severus Snape}
\marginpar{84} 
Minerva McGonagall war Severus Snapes Verwandlungsprofessorin während seiner Zeit als Schüler in Hogwarts. Nachdem er die Schule verlassen und dann als Zaubertrankmeister wiedergekehrt war, wurden er und McGonagall Kollegen. Als Hauslehrerin von Gryffindor stritten sich McGonagall und Snape andauernd über Hauspunkte und Quidditchergebnisse. Wie Snape war McGonagall eine sehr strenge, disziplinierte Person, allerdings mit dem Unterschied, dass sie Harry Potter fairer behandelte. McGonagall und Snape waren beide effektive Lehrer und überdurchschnittlich magisch begabt. Sie waren beide loyale Anhänger Dumbledores und nach einer Zeit respektierte Minerva Snape trotz seiner dunklen Vergangenheit. Zwar wusste Minerva, dass Snape in seiner Jugend den Todessern angehörte, vertraute aber Dumbledores Meinung, dass er seine frühere Tätigkeit vollständig hinter sich gelassen habe. 


\subsection*{\Large Hinter den Kulissen}
\marginpar{85} 
Minerva ist der Name der römischen Göttin der Weisheit, welche auf altgriechisch auch als Athene bekannt ist.
Diese wiederum besitzt als Zeichen eine Eule, welche auch das Symbol der Illuminaten ist
In Harry Potter und der Feuerkelch (Film) und Harry Potter und die Heiligtümer des Todes (Film 2) werden Professor McGonagall komplexe Alliterationen in den Mund gelegt, nämlich "[...] blindwütige blamable Bande von Brüllaffen." und "[...] früher eine fatale Vorliebe für Feuerwerk". Letzteres ist nur 
%SEITE9
gesprochen eine vollständig klingende Alliteration.
In der Verflimung von Harry Potter und der Feuerkelch hängt ein Portrait von McGonagall in jüngeren Jahren im Gemeinschaftsraum der Gryffindors.
In LEGO Harry Potter: Jahre 1 - 4 ist es Professor McGonagall nicht möglich, in den Gemeinschaftsraum der Gryffindors zu gelangen, obwohl sie selbst Mitglied dieses Hauses und später sogar Hauslehrerin war.
In LEGO Harry Potter: Jahre 5 - 7 beschwört McGonagall die Rüstungen des Schlosses nicht, um die Schule zu beschützen.

\subsection*{\Large Auftritte}
\vspace{10pt}
\marginpar{86} 
\begin{itemize}
    \item Harry Potter (Buchreihe)
    \item Harry Potter (Filmreihe)
    \item Harry Potter-Videospiele
    \item Harry Potter und das verwunschene Kind
    \item Harry Potter und das verwunschene Kind (Theaterstück)
    \item Phantastische Tierwesen und wo sie zu finden sind (Nur erwähnt)
    \item Phantastische Tierwesen und wo sie zu finden sind: Das Originaldrehbuch (Nur erwähnt)
    \item Phantastische Tierwesen: Grindelwalds Verbrechen (Film)
    \item Phantastische Tierwesen: Grindelwalds Verbrechen (Das Originaldrehbuch)
    \item Phantastische Tierwesen: Dumbledores Geheimnisse (Film)
    \item Phantastische Tierwesen: Dumbledores Geheimnisse - Das Originaldrehbuch
    \item Die Märchen von Beedle dem Barden (Nur erwähnt)
    \item Pottermore
    \item Wizarding World
    \item The Road to Hogwarts Sweepstakes
    \item Harry Potters offizielle Webseite
    \item Harry Potter: Ein Pop-Up Buch
    \item LEGO Harry Potter: Erschaffung einer Magischen Welt
    \item LEGO Harry Potter
    \item LEGO Creator: Harry Potter
    \item Creator: Harry Potter und die Kammer des Schreckens
    \item LEGO Dimensions
    \item LEGO Harry Potter: Jahre 1 - 4
    \item LEGO Harry Potter: Jahre 5 - 7
    \item Harry Potter: Quidditch-Weltmeisterschaft
    \item Harry Potter für Kinect
    \item Wonderbook: Das Buch der Zaubersprüche
    \item Wonderbook: Das Buch der Zaubertränke
    \item Harry Potter Sammelkartenspiel
    \item Harry Potter: Die Welt der magischen Wesen
    \item Harry Potter: Die Welt der magischen Figuren
    \item The Wizarding World of Harry Potter
%SEITE10
    \item Harry Potter: Hogwarts Mystery
    \item Harry Potter: Wizards Unite
    \item Harry Potter: Die Magie erwacht
    \item Harry Potter: Rätsel \& Zauber
    \item Hogwarts Legacy
\end{itemize}
%SEITE11

\end{document}


\end{document}
