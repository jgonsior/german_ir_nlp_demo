\documentclass[a4paper, 10pt]{article}
\usepackage[T1]{fontenc}
\usepackage[sfdefault]{AlegreyaSans} %% Option 'black' gives heavier bold face
\usepackage[a4paper, left=2cm, right=2cm, bottom=2cm, top=2cm]{geometry} % Adjust left and right margins
\usepackage{amsmath} % for \boxed

% Set the width of the lines around the box
\setlength{\fboxrule}{2pt}

\begin{document}

\begin{minipage}[t]{\textwidth}
    \vspace*{-1.5cm} % Move the content up by 0.5cm
    \begin{flushright}
        \hspace*{\fill} % Move the content to the right edge
        $\boxed{\textbf{\Huge\phantom{00}5\phantom{00}}}$ % Your content here with increased padding
    \end{flushright}
\end{minipage}

\section*{\huge Severus Snape}
\marginpar{1} 
Severus Snape (* 9. Januar 1960; † 2. Mai 1998) war ein halbblütiger Zauberer, Zaubertrankmeister (1981-96), Lehrer für Verteidigung gegen die dunklen Künste (1996-97) und Schulleiter (1997-98) der Hogwarts-Schule für Hexerei und Zauberei. Er war Mitglied im Orden des Phönix und spielte als Doppelagent eine entscheidende Rolle in beiden Zaubererkriegen gegen Lord Voldemort.
\vspace{10pt}
\newline
\marginpar{2}  
Snape besuchte von 1971 bis 1978 gemeinsam mit Lily Evans, James Potter, Remus Lupin, Sirius Black und Peter Pettigrew die Hogwarts-Schule und wurde dem Haus Slytherin zugeordnet. Besonders zu James Potter hegte er eine offene Feindschaft.
\vspace{10pt}
\newline
\marginpar{3}  
Severus wuchs in einer Muggelgegend, in der Nähe der Familie Evans, auf. Er lernte Lily und Petunia Evans kennen, als er neun Jahre alt war und verliebte sich bedingungslos in Lily. Sie wusste davon jedoch nichts und die beiden wurden zu sehr engen Freunden. Er kam gemeinsam mit Lily nach Hogwarts, sie wurden aber durch die Rivalität ihrer beiden Häuser voneinander getrennt. Severus wurde der unmittelbare Feind von James Potter und Sirius Black und war ein häufiges Opfer ihrer Mobbingangriffe. Snape entwickelte eine Leidenschaft für die Dunklen Künste, die zunahm, als sein Wunsch nach Rache stärker wurde. Snape kam in die Gesellschaft von späteren Todessern in Slytherin, wodurch seine Freundschaft mit der muggelstämmigen Lily endgültig zerbrach. Nach einem vergeblichen Versuch, Lilys Vertrauen zurückzugewinnen, schloss sich Snape den Todessern an, genauso wie viele seiner Schulkameraden aus Slytherin.
\vspace{10pt}
\newline
\marginpar{4}  
Kurz bevor Lily von Lord Voldemort ermordet wurde, wechselte Snape die Seite und wurde Mitglied des Ordens des Phönix und ein Doppelagent Dumbledores. Unter enormen Schwierigkeiten verhinderte Snape, dass Lord Voldemort die Wahrheit über seine Loyalität herausfand. Trotz des Hasses und des Misstrauens anderer, darunter Harry Potter, vertraute Albus Dumbledore Snape aus Gründen, die bis zu beider Tod nur ihnen beiden bekannt waren. Als Snape starb, zeigte sich, dass seine tiefe Liebe zu Lily Evans ihn dazu veranlasste, sich selbst zu opfern, indem er Dumbledores Sache zu ihrem Schutz (und nach ihrem Tod, dem ihres Sohnes) vor Lord Voldemort unterstütze.
\vspace{10pt}
\newline
\marginpar{5}  
Die Beziehung zwischen Albus Dumbledore und Snape war durch eine ungewöhnlich starke Loyalität geprägt, da Snape bereit war, Dumbledore auf dessen eigenen Wunsch hin zu töten. Vor dem Tod Dumbledores versprach Snape, die Schüler von Hogwarts vor den Todessern zu schützen. Snape nahm später an der Schlacht von Hogwarts teil, wurde aber von Lord Voldemort ermordet, der fälschlicherweise glaubte, dass Snape der Meister des Elderstabs sei. Jedoch war Harry Potter der Herr über den Elderstab, da er den unwissenden Draco Malfoy entwaffnete und dieser wiederum Dumbledore entwaffnet hatte und Snape, durch seine Zusammenarbeit mit Dumbledore, diesen nie besiegt hatte.
\vspace{10pt}
\newline
\marginpar{6}  
Harry benannte seinen zweiten Sohn, Albus Severus, nach Severus Snape.

\subsection*{\Large Geschichte}
\marginpar{7} 
Als Hauslehrer von Slytherin schikanierte er vor allem Harry Potter und seine Mitschüler aus Gryffindor und bevorzugte sein eigenes Haus. Besonders Neville Longbottom lebte in ständiger Angst vor Snape. In ihrem eigentlich 7. Schuljahr (Harry, Hermine und Ron waren auf der Jagd nach Horkruxen) wurde Severus Snape im Auftrag Lord Voldemorts Schulleiter von Hogwarts. Er und Alecto und Amycus Carrow waren dort für die Strafen zuständig, wenn sich Muggelgeborene gegen die Reinblüter auflehnten.

\subsubsection*{\large Kindheit}
\marginpar{8} 
Seit seiner Kindheit war Snape mit Harry Potters Mutter Lily Evans befreundet. Auf einem Spielplatz erzählte er der aus einer Muggelfamilie stammenden Lily erstmals in ihrem Leben von Zauberkräften und der Welt der Zauberer und Hexen. Lilys Schwester Petunia, die keine magischen Fähigkeiten besaß, war eifersüchtig, vor allem, nachdem Snape und Lily, die inzwischen Freunde geworden waren, in ihrem Zimmer einen Brief von Professor Dumbledore gefunden hatten, in dem er erklärte, warum er Petunia nicht in die Hogwarts-Schule aufnehmen konnte.

\subsubsection*{\large Snapes Schulzeit in Hogwarts}
\marginpar{9} 
Auch während der Schulzeit in Hogwarts waren die beiden Freunde, obwohl Snape im Haus Slytherin und Lily in Gryffindor lebte. Aufgrund Snapes Interesse für die Dunklen Künste wandte sich Lily nach und nach von ihm ab. Die Freundschaft war endgültig beendet, als Snape von James Potter attackiert wurde und er Lily, die ihn vor James verteidigte, als Schlammblut beschimpfte. Snape versuchte hartnäckig und verzweifelt, sich bei Lily zu entschuldigen, was die Freundschaft aber nicht mehr retten konnte.
%SEITE1
\vspace{10pt}
\newline
\marginpar{10}  
Als Schüler im Hause Slytherin wurde ihm der Weg zum Anhänger von Lord Voldemort eröffnet, dem er sich später auch anschloss. Er spionierte für Lord Voldemort Teile der Prophezeiung von Sybill Trelawney aus. Er belauschte eher zufällig eine Unterhaltung zwischen dem Direktor Albus Dumbledore und der Bewerberin Trelawney und berichtete dem Dunklen Lord davon. Glücklicherweise hatte er nur einen Teil der Prophezeiung gehört, da er von dem Inhaber der Schenke, Aberforth Dumbledore, ertappt und rausgeschmissen wurde. Schon kurz nachdem er den gehörten Teil der Prophezeiung an den Dunklen Lord verraten hatte, kehrte Snape nach Hogwarts zurück und zeigte Reue. Es stellte sich heraus, dass er seit seiner Kindheit in Lily verliebt war und ihren Tod nicht wollte. Voldemort, der die Informationen so auslegte, dass er den Sohn der Potters töten müsse, suchte die Potters auf. Lily wollte Harry schützen und kam dabei, genauso wie dessen Vater, ums Leben. Snape hatte ausdrücklich um die Verschonung seiner großen Liebe Lily gefleht. Dumbledore vereinbarte mit Snape, dass er versuchen würde, sie zu schützen.

\subsubsection*{\large Lehrer in Hogwarts}
\marginpar{11} 
Severus versprach, im Gedenken an Lily, das Leben ihres Sohnes Harry zu schützen. Dieser geheime Bund zwischen den Männern blieb aber auf Snapes eindringlichen Wunsch hin geheim, was Dumbledore sehr bedauerte. Snape arbeitete an der Seite Dumbledores als Spion und wurde als Lehrer in Hogwarts eingestellt. Von anderen wurde jedoch immer wieder bezweifelt, dass Snape tatsächlich zur guten Seite übergelaufen war.

\subsubsection*{Schuljahr     1995/1996}
\marginpar{12} 
Im Harry Potters 5. Schuljahr nutzte Snape seine Vergangenheit als Todesser und spionierte als Agent für den Orden des Phönix. Außerdem unterrichtete er Harry Potter in Okklumentik (Verschließen des Geistes). Dies war aber erfolglos, da Harry im Gegenteil befürchtete, dass Snapes Unterricht seinen Geist noch weiter für Lord Voldemort öffnen würde. Der Unterricht wurde von Snape beendet, als Harry in einem unbeobachteten Moment in das Denkarium mit Snapes Gedanken eintauchte und unter anderem erfuhr, wie Snape als Schüler von seinem Vater James mit magischen Mitteln gedemütigt, aber von seiner Mutter Lily verteidigt wurde. Diese ungewollte Offenbarung führte schließlich dazu, dass Snape jeden weiteren Unterricht in Okklumentik ablehnte. Harry war allerdings geschockt von den Erinnerungen Snapes an seinen Vater und zweifelte erstmals am Bild des vorbildlichen Mannes, das er bislang stets verteidigt hatte.

\subsubsection*{Schuljahr 1996/1997}
\marginpar{13} 
Schließlich erhielt Snape im Schuljahr 1996/97 die Stelle als Lehrer für Verteidigung gegen die dunklen Künste. Gleichzeitig entwickelte sich Harry Potter zu einem Musterschüler im Brauen von Zaubertränken beim neuen Lehrer Horace Slughorn, weil ihm ein altes, mit wertvollen Hinweisen vollgeschriebenes Schulbuch des Halbblutprinzen in die Hände fiel, welcher Severus Snape selbst war. Er nannte sich selbst so aufgrund eines Wortspiels mit dem Namen seiner Mutter "Eileen Prince" und offenbarte dies Harry am Ende seines sechsten Schuljahres, als dieser versuchte, Snape mit einem Zauber anzugreifen, welchen er im Buch des Halbblutprinzen gefunden hatte.
\vspace{10pt}
\newline
\marginpar{14}  

Kurz vor Beginn des 6. Schuljahres statteten Narzissa Malfoy und ihre Schwester Bellatrix Lestrange Snape einen Besuch in seinem Haus in Spinner’s End ab. Als Bellatrix seine Loyalität gegenüber dem Dunklen Lord anzweifelte, konnte er ihre Bedenken zerstreuen, als er ihr alle Taten und Verhaltensweisen genau erklärte, die darauf hätte schließen lassen können, er sei auf Dumbledores Seite. Erst als er gegenüber Narzissa Malfoy den Unbrechbaren Schwur leistete, schien er Bellatrix von seiner Loyalität zu überzeugen. Snape schwor, Narzissas Sohn Draco zu helfen, der von Lord Voldemort den Auftrag erhalten hatte, Albus Dumbledore zu töten. Tatsächlich brachte Draco es beim Kampf im Astronomieturm nicht über sich, den Todesfluch auszusprechen, obwohl er Dumbledore bereits entwaffnet und somit besiegt hatte. Snape stieß den verunsicherten Jungen schließlich beiseite und tötete Dumbledore scheinbar kaltblütig mit dem Todesfluch "Avada Kedavra". Doch nicht nur der Unbrechbare Schwur mit Dracos Mutter ist der Grund für Snapes Handeln (wird der Schwur gebrochen, stirbt der an den Schwur Gebundene). Severus Snape unterstützte zwar Draco Malfoys Vorhaben und tötete Dumbledore. Dies war jedoch zwischen Dumbledore und Snape schon im voraus vereinbart worden, da Dumbledore wusste, dass er derart geschwächt war wegen seiner auf baldigen Sicht todbringenden, magischen Verletzung durch den Ring des Vorlost Gaunts, den Lord Voldemort verhext hatte, und den er leichtgläubig auf seinen Finger gesetzt hatte. Als Severus Snape erstmals die geschwärzte Hand Dumbledores auf dessen Bitten untersuchte, eröffnete er Dumbledore, dass er innerhalb eines Jahres trotz eines verabreichten Gegenmittels und Gegenzaubers sterben würde. Wie Harry nach Snapes Tod erfuhr, entsprach es Dumbledores Plan, der Draco nicht mit einem Mord belasten wollte und Dumbledore selbst ohnehin nicht mehr lange zu leben hatte. Es steckte jedoch noch mehr dahinter: Dumbledore wollte von Snape im gegenseitigen Einvernehmen getötet werden, damit die Macht des Elderstabs unbesiegt ins Grab genommen werden konnte. Dieser Plan ging nach hinten los, aber nicht für Harry, sondern für Voldemort. Snape war nicht dessen Meister, dafür aber Draco Malfoy, da dieser Dumbledore bereits vorher entwaffnet hatte, was zuerst keiner wusste, auch Voldemort nicht. Nach dem Todesfluch Snapes an Albus Dumbledore flüchteten Snape und Draco Malfoy aus Hogwarts zusammen mit einigen Todessern.
%SEITE2
\subsubsection*{Schuljahr 1997/1998}
\marginpar{15} 
Zum Schuljahr 1997/1998 wurde Snape vom korrumpierten Zaubereiministerium zum Schulleiter von Hogwarts ernannt. Wie von Dumbledore geplant, war Voldemort nun von Snapes Loyalität überzeugt und vertraute ihm. Snape folgte aber weiter insgeheim den Plänen Dumbledores, indem er Ratschläge von Dumbledores Porträt im Schulleiterbüro einholte. Ohne die nähren Details oder den Zweck zu wissen, half er Harry Potter mit seinem Patronus, einer Hirschkuh, damit dieser Gryffindors Schwert unter erschwerten, Mut fordernden Umständen finden konnte. Gryffindors Schwert war geeignet, Horkruxe zu zerstören. Voldemort, der bis zum Schluss nichts von Snapes Rolle ahnte, ließ Snape durch die Schlange Nagini töten, im Glauben, damit die volle Macht über den Elderstab zu erlangen. Der Legende nach steht die ganze Macht des Stabes nur demjenigen zur Verfügung, der den vorhergehenden Besitzer besiegt hat. (Da Draco Malfoy Dumbledore am Ende seines sechsten Schuljahres entwaffnete, war also dieser der "Herr" des Elderstabs.) Harry Potter und Hermine Granger fanden den sterbenden Snape. Im Sterben bat Snape Harry Potter seine Erinnerungen, die aus Tränen und Wunden flossen, aufzufangen. Harry Potter brachte diese in Dumbledores Denkarium im Schulleiterbüro und er sah die Erinnerungen Snapes. Snapes Motive und seine Vergangenheit wurden offenbart.: Der Grund für Dumbledores unerschütterliches Vertrauen in Snape lag in der unerfüllten Liebe Snapes zu Harrys Mutter Lily, für deren Tod sich Snape verantwortlich fühlte. Dies war der Grund, warum sich Snape von Voldemort abgewandt hatte. Aus diesem Grund schützte Snape als Dumbledores Doppelagent stets Harrys Leben und unterstützte ihn unter Aufrechterhaltung seiner Tarnung bis zu seinem Ende.

\subsubsection*{\large Nach dem Tode}
\marginpar{16} 
„Albus Severus Potter, du bist nach zwei Schulleitern von Hogwarts benannt. Einer von ihnen war in Slytherin und er war wahrscheinlich der mutigste Mann, den ich je kannte. “
\vspace{10pt}
\newline
\marginpar{17}  
— Harry Potter zu seinem Sohn über Severus Snape
\vspace{10pt}
\newline
\marginpar{18}  
(  Engl.  Albus Severus Potter, you were named after two headmasters of Hogwarts. One of them was a Slytherin and he was the bravest man I ever knew. )
\vspace{10pt}
\newline
\marginpar{19}  
Neunzehn Jahre nach dem Sieg über Lord Voldemort machte sich der jüngere von Harry Potters Söhnen, der den Namen Albus Severus, trägt, Sorgen darüber, dass er statt für Gryffindor vielleicht für Slytherin ausgewählt würde. Darauf erklärte ihm Harry, dass Albus Severus nach einem sehr mutigen Schulleiter aus Slytherin benannt wurde.


\subsection*{\Large Halbblutprinz}
\marginpar{20} 
"Halbblutprinz" war ein Deckname, der von Severus Snape während dessen Schulzeit verwendet wurde.
\vspace{10pt}
\newline
\marginpar{21}  
Snapes Vater war ein Muggel, welcher den Namen Tobias Snape trug. Seine Mutter war eine reinblütige Hexe namens Eileen Prince. Er war also ein Halbblut. Dies und der Mädchenname seiner Mutter - Prinz (  Engl.  Prince ) ergeben zum einen den Titel Halbblutprinz als auch einen Hinweis auf seine Abstammung als Halbblut aufgrund seiner Zugehörigkeit zur Familie Prinz / Prince. Er ist "ein halber Prinz oder Prince" (  Engl.  "Half a Prince" ).
\vspace{10pt}
\newline
\marginpar{22}  
Der Deckname Halbblutprinz trat zum ersten Mal in Harry Potters sechstem Schuljahr in Erscheinung. Da Harry kein "Ohnegleichen" in den ZAGs erhalten hatte, hätte er das Fach Zaubertrankkunde nach Vorgaben Snapes im 6. Schuljahr nicht weiter besuchen können. Harry Potter nahm an, er würde das Fach Zaubertrankkunde nicht belegen, daher kaufte er sich zum Schulbeginn auch kein Zaubertrankbuch in der Winkelgasse. Als jedoch Horace Slughorn als Lehrer für Zaubertrankkunde nach Hogwarts zurückkehrte, konnte Harry das Fach doch weiter belegen und musste ein Buch aus dem Schrank im Klassenzimmer in den Kerkern verwenden. Auf der Innenseite des Buchdeckels findet sich der Vermerk "Dieses Buch gehört dem Halblutprinz". Das Buch half Harry sehr im Zaubertrankkunde-Unterricht, da es voller nützlicher Angaben und Ergänzungen zu den Rezepten war, die zu exzellenten Resultaten führten. Snape war ein sehr talentierter und mächtiger Zauberer, auch schon in seiner Schulzeit. Seine Fähigkeiten in Zaubertrankkunde waren sehr gut und er hatte alle seine Ideen und Verbesserungen in dem Buch festgehalten.
\vspace{10pt}
\newline
\marginpar{23}  
Als Slughorn Snape von Harrys Erfolgen berichtet, wurde dieser bereits misstrauisch, jedoch spätestens, als Harry einen von Snape erfundenen Zauberspruch - "Sectumsempra" - gegen Draco Malfoy einsetzt, war er sich sicher, dass Harry sein altes Buch benutzte. Da er das aber nicht nachweisen konnte, kam es erst zum Schuljahresende, als Snape nach der Tötung von Albus Dumbledore floh, zur Auflösung. Harry benutzte wieder einen der Sprüche aus dem Buch, woraufhin Snape ihn abblockte und Harry anbrüllte, er solle nicht seine eigenen Sprüche gegen ihn verwenden und dass er der Halbblutprinz sei.
\vspace{10pt}
\newline
\marginpar{24}  
Snape ist ein sogenannter Zaubererfinder, diese Fähigkeit praktizierte er unter dem Namen Halbblutprinz.
%SEITE3
\subsection*{\Large Charakter}
\marginpar{25} 
Severus Snape vereinigte das Gute und das Böse in seinem Charakter. Einerseits schützte er das Trio insbesondere Harry Potter, aber auch die Hogwarts-Schüler, soweit es seine angebliche Loyalität zu Lord Voldemort erlaubte, andererseits warf er Harry immer wieder Steine in den Weg.

\subsection*{\Large Zitate}
\marginpar{26} 
„Ich kann euch lehren, wie man Ruhm in Flaschen füllt, Ansehen zusammenbraut, sogar den Tod verkorkt - sofern ihr kein großer Haufen Dummköpfe seid, wie ich sie sonst immer in der Klasse habe.“
\vspace{10pt}
\newline
\marginpar{27}  
— Severus Snape zu Erstklässlern aus Gryffindor und Slytherin.
\vspace{10pt}
\newline
\marginpar{28}  
Ron: "Vielleicht ist er krank."
\vspace{10pt}
\newline
\marginpar{29}  
Harry: "Vielleicht hat er gekündigt, weil er wieder nicht Verteidigung gegen die dunklen Künste unterrichten darf!"
\vspace{10pt}
\newline
\marginpar{30}  
Ron: "Oder sie haben ihn rausgeschmissen! Immerhin kann ihn ja keiner ausstehen -"
\vspace{10pt}
\newline
\marginpar{31}  
Snape: "Oder vielleicht wartet er darauf, von euch zu hören, warum ihr nicht mit dem Schulzug gekommen seid."
\vspace{10pt}
\newline
\marginpar{32}  
— Harry Potter und die Kammer des Schreckens
\vspace{10pt}
\newline
\marginpar{33}  
„Dumm, wie ein Teil dieser Klasse zweifellos ist, erwarte ich dennoch, dass Sie wenigstens noch ein 'Annehmbar' bei Ihren ZAGs schaffen, andernfalls werden Sie ... mein Missbehagen zu spüren bekommen.“ Sein Blick blieb diesmal an Neville hängen, der heftig schluckte. „Nach diesem Schuljahr werden natürlich viele von Ihnen nicht mehr bei mir studieren“, fuhr Snape fort. „In meine UTZ-Zaubertrankklasse nehme ich nur die Allerbesten auf, was heißt, dass einige von Ihnen sich mit Sicherheit verabschieden werden.“ Seine Augen ruhten nun auf Harry und seine Lippen kräuselten sich. Harry blickte finster zurück und spürte ein grimmiges Vergnügen bei dem Gedanken, dass er den Zaubertrankunterricht nach dem fünften Jahr sausen lassen konnte. „Aber bis zu diesem glücklichen Moment des Abschieds haben wir noch ein Jahr vor uns.“
\vspace{10pt}
\newline
\marginpar{34}  
— Severus Snape vor den ZAGs zu einer fünften Klasse
\vspace{10pt}
\newline
\marginpar{35}  
„So faszinierend Ihr gesellschaftliches Leben zweifellos ist, Miss Granger“, sagte eine eisige Stimme direkt hinter ihnen, „ich muss Sie doch ermahnen, es nicht im Unterricht zu erörtern. Zehn Punkte Abzug für Gryffindor.“
\vspace{10pt}
\newline
\marginpar{36}  
— Ermahnung im Unterricht an Hermine Granger
\vspace{10pt}
\newline
\marginpar{37}  
Dolores Umbridge: "Sie hatten sich, glaube ich, zuerst um die Stelle für Verteidigung gegen die dunklen Künste beworben?"
\vspace{10pt}
\newline
\marginpar{38}  
Snape: "Ja"
\vspace{10pt}
\newline
\marginpar{39}  
Dolores Umbridge: "Aber damit hatten Sie keinen Erfolg?"
\vspace{10pt}
\newline
\marginpar{40}  
Snape: "Offensichtlich."
\vspace{10pt}
\newline
\marginpar{41}  
— Harry Potter und der Orden des Phönix
\vspace{10pt}
\newline
\marginpar{42}  
„Oh, sehr gut“, unterbrach ihn Snape und seine Lippen kräuselten sich. „Ja, man kann ohne weiteres feststellen, dass annähernd sechs Jahre magischer Ausbildung bei Ihnen nicht verschwendet waren, Potter. Gespenster sind durchsichtig.“
\vspace{10pt}
\newline
\marginpar{43}  
— Severus Snape zu Harry Potter
\vspace{10pt}
\newline
\marginpar{44}  
Snape: „Das ist also Ihr Exemplar von Zaubertränke für Fortgeschrittene, Potter?“
\vspace{10pt}
\newline
\marginpar{45}  
Harry: „Ja.“
\vspace{10pt}
\newline
\marginpar{46}  
Snape: „Sie sind sich dessen wirklich sicher, Potter?“
\vspace{10pt}
\newline
\marginpar{47}  
Harry: „Ja.“
\vspace{10pt}
\newline
\marginpar{48}  
Snape: „Dies ist das Exemplar von Zaubertränke für Fortgeschrittene, das Sie bei Flourish & Blotts gekauft haben?“
\vspace{10pt}
\newline
\marginpar{49}  
Harry: „Ja.“
%SEITE4
\vspace{10pt}
\newline
\marginpar{50}  
Snape: „Warum steht dann der Name 'Runald Waschlab' innen auf dem Buchdeckel?“
\vspace{10pt}
\newline
\marginpar{51}  
Harry: „Das ist mein Spitzname.“
\vspace{10pt}
\newline
\marginpar{52}  
Snape: „Ihr Spitzname.“
\vspace{10pt}
\newline
\marginpar{53}  
Harry: „Jaah, so nennen mich meine Freunde.“
\vspace{10pt}
\newline
\marginpar{54}  
Snape: „Ich weiß, was ein Spitzname ist.“
\vspace{10pt}
\newline
\marginpar{55}  
— Severus Snape fragt Harry Potter aus
\vspace{10pt}
\newline
\marginpar{56}  
„Er hätte mich!“, sagte Bellatrix leidenschaftlich. „Mich, die für ihn viele Jahre in Askaban gesessen hat!“
\vspace{10pt}
\newline
\marginpar{57}  
„Ja, in der Tat, höchst bewundernswert“, sagte Snape und es klang gelangweilt. „Du hast ihm zwar im Gefängnis nicht sonderlich genützt, aber die Geste war zweifellos edel-“
— Severus Snape während einer feindseligen Unterhaltung mit Bellatrix Lestrange

\subsection*{\Large Hinter den Kulissen}
\marginpar{58} 
Die 2001 entdeckte Krabben-Spezies Harryplax severus wurde nach Harry Potter und Severus Snape benannt.
\vspace{10pt}
\newline
\marginpar{59}  
Es gibt einen Ort namens "Snape". Von diesem hat sich Rowling für den Nachnamen inspirieren lassen.
Vorbild der Figur
\vspace{10pt}
\newline
\marginpar{60}  
„If you'd seen my grades in Chemistry.... That's why Snape teaches Potions . . . Don't say awwww! He deserved it! We can all think of teachers we'd like revenge on.“
\vspace{10pt}
\newline
\marginpar{61}  
— JK Rowling, 18th October 2007
\vspace{10pt}
\newline
\marginpar{62}  
Vorbild der Figur "Snape" ist John Nettleship, Rowlings ehemaliger Chemielehrer, welcher Rowlings Mutter, Anne, trotz ihrer Behinderung durch beginnende Multiple Sklerose als Laborassistentin einstellte. Laut Claire Jordan, einer langjährigen guten Freundin von Nettleship, diente er als Vorlage für 70 % von dem, was wir von Snape kennen. Die Nasenform stammt übrigens nicht von Nettleship und im Gegensatz zu Snape hatte er graue Augen. Seine Stimme war ebenfalls nicht so tief wie die von Alan Rickman und er sah sich eher als der Buch-Snape denn als der Rickman-Snape. Die anderen dreißig Prozent, vor allem seine schlechte Seite sollen demnach von Rowling selbst stammen, denn, wie sie sagt, enthalten alle Charaktere etwas von ihr. Dadurch, dass er Aspergerautist war, konnte er Stimmungslagen anderer Personen nicht erkennen oder zuordnen und selbst nicht verbergen, wenn er jemanden nicht mochte. Ohne es so zu meinen, wirkte er auch manchmal gemein. Wie Snape mit seinen Schülern umgegangen ist, hat seinen Ursprung in einem Potpourri aus Erinnerungen an Lehrer, die Rowling in ihrer Schulzeit hatte, die sich ebenso verhielten, beispielsweise Sylvia Morgan, was Nettleship nicht an seiner Lehrerkollegin mochte. Auch wenn er ein sehr freundlicher Mensch gewesen ist, so teilte er mit Snape dessen politischen Aktivismus und die Kämpfernatur für das Gute.
\vspace{10pt}
\newline
\marginpar{63}  
„it wasn't much fun being the Snape prototype - some people were very nasty about it, as he was obviously regarded as a bad guy by most readers.“
— Nettleship im Jahr 2002
\vspace{10pt}
\newline
\marginpar{64}  
Lange Zeit wusste er nicht, dass er das Vorbild für Snape war und dachte, dieser sei einfach nur ein unsympathischer, seine Schüler quälender Charakter, während Familie und Freunde Nettleships dies Vermutung schon lange hegte, dass er Snape sei. Als er herausfand, dass Snape auf ihn basierte, war er zunächst ganz und gar nicht erfreut, mit diesem verglichen zu werden, da er das damit gleichsetzte, dass man ihm unterstellte, ein schlechter Lehrer zu sein, aber später wurde er erst recht Fan der Harry Potter-Reihe und schrieb auch selbst Harry Potter-Fanfictions.

\subsection*{\Large Auftritte}
\vspace{10pt}
\marginpar{65} 
\begin{itemize}
    \item Harry Potter und der Stein der Weisen (Erstes Auftreten)
    \item Harry Potter und der Stein der Weisen (Film)
    \item Harry Potter und der Stein der Weisen (Videospiel)
    \item Harry Potter und die Kammer des Schreckens
    \item Harry Potter und die Kammer des Schreckens (Film)
    \item Harry Potter und die Kammer des Schreckens (Videospiel)
    \item Harry Potter und der Gefangene von Askaban
    \item Harry Potter und der Gefangene von Askaban (Film)
    \item Harry Potter und der Gefangene von Askaban (Videospiel) (Nicht in der PC-Version)
    \item Harry Potter und der Feuerkelch
    \item Harry Potter und der Feuerkelch (Film)
    \item Harry Potter und der Orden des Phönix
%SEITE5
    \item Harry Potter und der Orden des Phönix (Film)
    \item Harry Potter und der Orden des Phönix (Videospiel)
    \item Harry Potter und der Halbblutprinz
    \item Harry Potter und der Halbblutprinz (Film)
    \item Harry Potter und der Halbblutprinz (Videospiel)
    \item Harry Potter und die Heiligtümer des Todes
    \item Harry Potter und die Heiligtümer des Todes (Film 1)
    \item Harry Potter und die Heiligtümer des Todes
    \item Harry Potter und die Heiligtümer des Todes (Film 2)
    \item Harry Potter und die Heiligtümer des Todes
    \item Harry Potter und das verwunschene Kind (Erscheint in einer alternativen Realität)
    \item Harry Potter und das verwunschene Kind (Theaterstück) (Erscheint in einer alternativen Realität)
    \item Phantastische Tierwesen und wo sie zu finden sind (Nur erwähnt)
    \item LEGO Harry Potter
    \item LEGO Harry Potter: Charaktere der Magischen Welt
    \item Harry Potter: Quidditch-Weltmeisterschaft
    \item LEGO Harry Potter: Jahre 1 - 4
    \item LEGO Harry Potter: Jahre 5 - 7
    \item LEGO Creator: Harry Potter
    \item LEGO Harry Potter
    \item Harry Potter Sammelkartenspiel
    \item The Making of Harry Potter
    \item Harry Potter für Kinect
    \item Pottermore
    \item Harry Potter: Die Welt der Magischen Figuren
    \item Harry Potter: Die Welt der Magischen Wesen (Nur erwähnt)
\end{itemize}
%SEITE6
\end{document}


\end{document}
